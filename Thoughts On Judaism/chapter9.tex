\chapter{}

{
  \centering\larger
  \textsc{summary and conclusion}

}

\vspace*{2\baselineskip}

We have tried in the preceding pages to
give our conception of the vital principles
of Judaism. We have affirmed our belief
that any man or woman may claim to
belong to our brotherhood who is convinced
that:—

\begin{enumerate}
  \item There is one sole Creator or God.

  \item The God of the world has relations with
each individual soul, and each soul,
being an emanation from Him, must
be, like Him, immortal.

\item We are responsible to God for our
conduct, and if we sin, must bear the
consequences of our sin. No
intercessor is possible or necessary, between
man and God. The divine love
enters into the hearts of those who
seek it with prayer and contrition.

\item The love of our neighbours is a
necessary development of the love
of God.

\item The Jewish brotherhood exists for a
definite religious purpose, and this
purpose involves the highest efforts
of self-sacrifice and self-realisation.
\end{enumerate}

We have shown how these principles can
affect our daily conduct, and how their
influence may be strengthened by ceremonial
observance and by the study of the
Bible. We have emphasised the duty, which
is incumbent on all believing Jews, of making
the conduct of their lives, in its religious,
as well as in its secular phases, consistent
with the highest thought known to their
generation, and inspired by the dictates of
their conscience. In developing these
conceptions we have, for the purpose of discussion
separated religious duties from secular
duties, but we have, nevertheless, emphasised
our belief that Judaism affects every relation
of life, that it should hallow our conduct on
week days as well as on Sabbaths, and holy
days. A Jewish life is consecrated to God
by the very conditions of its existence.
Finally, we made an appeal to our generation
to realise themselves, and to become
conscious of their own inheritance. Some
of the old religious landmarks have been
shifted or destroyed by the flux of time.
Some ceremonials have lost their significance,
and therefore their vitalising power is dead.
Superstition has here and there been interwoven
with dogma, and the materialistic
influences of our age have degraded us. The
time has come for us to reconstruct our
doctrines on the old foundations of love and trust.
We dare not slide along from indifference
to negation. A life, unhallowed by religious
aspiration is necessarily a sordid life. If
we shut out God from our midst, we shall
sink into ignorance and extreme degradation.
Our children need the faith by which
our fathers lived. The continuity of testimony
is demanded by humanity. It is part
of God’s plan to visit the sins of the fathers
upon the children unto the third or fourth
generation. Do we, in the face of this
terrible warning, dare to remain indifferent
to the claims of our children? Seeing that
God has shown us, through His holy men,
that light is given to those who seek it
steadfastly and earnestly, we cannot allow
our children to live with us in darkness, and
pretend to be satisfied and at peace.

The teaching of Judaism inspires us to
seek the best in life. We cannot be content
with spiritual stagnation. Around us, are
many signs of disintegration. Men and
women, professing different creeds marry in
an irresponsible spirit, and the work of
transmission is arrested; men and women,
Jews by race, marry, while still unconscious
of any living religion whatever. Perhaps
their wedding is celebrated in synagogue,
but their lives remain unconsecrated to God,
and their children grow up indifferent to any
claim beyond that of self-advancement. By
certain sections of our fellow-countrymen
we are unloved. Much of the so-called
anti-Semitism is ignorant and unjustifiable.
It is rooted in prejudice and in jealousy, but
a little of it surely is not undeserved. If
men and women live and die as mere earth
worms, if they seek to get rich by any
means within their power, indifferent to the
presence of God, and to the duties of
citizenship, anxious merely to enjoy
themselves, to eat and drink as much as possible,
to wear fine clothes and to look smart — can
we wonder that they win the hatred and
scorn of the general community? When
these same people call themselves Jews — 
pretending to be members of the brotherhood
appointed to testify to the existence of
a God of love, truth and beauty — do we
not ourselves feel utterly ashamed? God
has commanded us to seek Him with love
and self-sacrifice, and humility of heart. He
has taught us to recognise the vanities of
life, by comparing them with His reality.
He has bidden us to draw nigh unto Him — 
to seek peace from his love, to reflect His
ineffable beauty in our feeble efforts after
righteousness. What are we doing? How
are we living? Are we not often false
witnesses ourselves? Are we not also responsible
for those who take His name in
vain by calling on Him with their lips and
denying Him in their hearts. We can
commune with the living God by prayer.
Do we pray? When do we pray? Do we
pray at home before work in the morning,
and before sleep in the evening? Do we
pray in our synagogues? Do we pray to
\textsl{God}? Do we \textsl{think} when we pray? Do
we realise God’s presence? Do we put our
best into our worship? Does it exact self-denying
effort from our souls? Is it indeed
communion ?

Then, again, how about our children?
Our fathers gave us bread to eat; this
bread will not exactly suit our children’s
palates. Is it right then that we should
give them stones? Can we not make some
effort to bake the bread anew and add a
few modern ingredients, so as to render it
more acceptable to the next generation? If
we can do this, our children may live,
otherwise they will surely die — the worst of all
deaths, for their bodies will live. How shall
we account for our negligence to God? How
shall we justify ourselves before His throne?
These questions probe us to the depths
of our souls. It is good that we should ask
them. We cannot live in a fool’s paradise,
and say all is well. It is not well with our
brotherhood. We cannot see the best life
among us, joining itself with the life of other
communities and say, “Alas! we cannot
influence this desertion.” We dare not see
the worst life slipping into the sloughs of
materialism and degradation and say, “What
can we do?” and pass on. We ourselves
are responsible, for we have not made
sufficient effort to make our conduct reveal
our faith, to make that faith more real and
vital, to test the power of prayer as a
living force in our lives. We must rouse
ourselves now, immediately. Mere
acquiescence is cowardice; it means spiritual
death. What can we do?

\begin{enumerate}
  \item We can try to lead better lives, by
realising our responsibility to God
and to our brotherhood.

\item We can pray, and allow God’s love to
affect our lives.

\item We can study the Bible and all the
beautiful and pure works of the best
men of all ages.

\item We can work among the members of
our community, and show them the
love of God as revealed in our lives,
and by our friendship with them, we
can lead them to God.

\item We can examine our religious ceremonials,
  and faithfully observe all
those, which can stimulate righteousness
in our lives.

\item We can help to organise, and then take
part in the public worship which
satisfies our spiritual needs.

\item We can, by example and by precept, by
sympathy and exhortation, transmit
to our children a living religion,
based on a pure conception of the
reality of God and His laws of
righteousness.
\end{enumerate}

It may be urged that the tendency of lax
Jews is to join the larger Christian community,
and that the Christian ideal of righteousness
is as noble as our own. Why therefore
should we strive to prevent defections, which
can in no way affect the progress of the
human race? The burdens and responsibilities
of Jews are so heavy; why should we
fret ourselves if some members of our
brotherhood choose a lighter religious
discipline, in order to arrive at the same end.
But we cannot console ourselves so easily.
Men and women do not \textsl{drift} into the
realisation of a new faith. By mere indifference
to Judaism, they do not become Christians.
By self-denying, strenuous spiritual
effort alone, can we realise any religion at all,
and certainly no conscientious \textsl{change} of faith
is possible without it. We know ourselves
with what painful anxiety, we, who have been
trained in the orthodox school of Jewish
thought, pass at the dictates of conscience to
liberal Judaism. It seems at first as if our
faith must altogether crumble away, when
some of our old convictions become in any
degree modified. It is a serious and painful
duty to refuse homage to observances which
certainly jar upon our sense of truth. We do
not perform this duty in a careless or irresponsible
spirit when these observances are interwoven
with some of the happiest memories
of our childhood. When we do refuse to stifle
our conscientious questionings, and to profess
a creed to which we are really indifferent, the
change must cause much pain and sorrow to
ourselves. For a time, at least, we feel as if
we were adrift on the vast sea of scepticism,
with no rudder and no anchor. Perhaps we
also cause pain to those we love, and would
give our lives to please. But the search for
truth is God’s work, it must be accomplished
in the teeth of every conflicting consideration.
It is only when we are embarked on this
search, when we have rejected that which
appears false to our intellectual conceptions,
and have refused to conform outwardly, when
our spirit is unmoved, it is only then, that we
can feel at one with our God. This transition
from different schools \textsl{within} our
brotherhood is then accompanied by sad
and difficult experiences. A transition from
one creed to another must be infinitely
more difficult and painful.

Most Jews who drift from Judaism drift
into nothingness, whether their faith has
the name of any existing creed, or is too
indefinite even to be named. Moreover,
just as we cannot become Christians merely
by ceasing to be Jews, so we are not Jews
merely because we are not Christians. We
have to realise our inheritance and let it
influence our lives, otherwise only the noblest
souls among us can steer clear of materialism.
And the materialism of Jews is of the lowest
and most gross order, perhaps because the
height from which they descended, is so
glorious in its possibilities. We can only
arrest this descent, by ourselves climbing
nearer the heights, and proving by the
joyousness of our lives, that we realise the
blessings of Judaism. Thus, too, and thus
only, can we arrest the departure of those
truly religious members of our brotherhood,
who leave our community, because its forms
and ceremonies offer them so little spiritual
satisfaction. We have tried to show in
previous chapters that, with a little readjustment,
the highest spiritual lessons can be
gathered from the ancient observances and
practices. We must by our efforts re-trim the
lamp of Judaism and cause it to shine with
a beautiful, pure light, which cannot be
extinguished.

As Jews, we believe our religion to be
based on irrefutable principles. Any defection
from our community, we regard not only as a
loss to ourselves, but as an injury to the
proselyte. It is well with us as Jews. We
are conscious of the Omnipresence of God.
We feel the influence of His love. We
obtain strength from our direct communion
with Him.

It is our mission to draw men within our
brotherhood. We dare not let them pass
away, without making an effort to reclaim
them. Moreover, at this moment it would
seem that our mission is drawing nearer to
its accomplishment.

For, passing from other faiths, we believe
that men are gradually coming to worship the
God of Israel, and to recognise the unity of
His being and the law of righteousness,
which He has established. Even now we
see a gradual approximation of men of all
creeds. The Trinitarian idea is accepted
with intellectual reservations by believing
Christians. The conception of three Entities,
seems to be merging into the recognition of
different attributes in the one Divine Being.
Christian divines insist more and more on
personal responsibility in the conduct of life.
The universal Fatherhood is being so much
better understood that the doctrine of everlasting
punishment for the unbaptised, is
being discredited. Then, again, other communities
are coming into existence on
purpose to minister to the one God, and to
worship Him simply and directly by prayer,
and by works of righteousness. These new
Churches recognise most of our “principles,”
and we consequently feel in close sympathy
with them. All these signs of the times
awaken our gratitude and stimulate our
trust in the God of truth; they affect
our religious obligations by strengthening
them, for the faith which inspires us is
now being quickened by hope. It sometimes
occurs, in the history of scientific discoveries,
that two men, working under different
conditions, in opposite parts of the globe,
alight on the same truth by different methods.
The truth of the discovery is not for this
reason less valued; rather is it doubly proved.

Similarly, our devotion as Jews to Judaism
is strengthened, when we find that some of
the constituent elements of our faith, are
being received more and more favourably by
sister religions. The general approximation
of different communities can be facilitated in
two ways, and both are surely desirable,
because universal religious brotherhood will
put an end to religious strife, the most bitter
of all forms of human strife. In the first
place, we can study the doctrines of other
faiths with reverence and respect, and we
shall find among them some developments of
Jewish dogma, which will help us in our
search after God. We can gratefully adopt
such teaching, as is consistent with the
principles of Judaism to which we subscribe.
For example, we shall find, in the
New Testament,\footnote{\textsl{Bible
  for Home Reading}, Vol.\ II.\ p.\ 779.}
important and suggestive
modifications of the doctrines of retribution
and of the relations of suffering to sin, a fresh
and noble restatement of the old prophetic
doctrine, “I desire love and not sacrifice,”
among doctrines which to the Jewish mind are
narrow and harmful, a passionate enthusiasm
for the moral and religious regeneration of the
outcast and the sinner, fine teaching about the
nature and power of love and the duty of
forgiveness, fresh contributions to the
conception of self-sacrifice, suffering and religious
inwardness ... a striking presentment of the
true and intimate relation of the human child
to the divine father, and last not least, a
clear and emphatic recognition that this
divine Fatherhood extends equally to
the Gentile and the Jew. The second
method of approximation is by increased
loyalty to the fundamentals of our own faith,
for thus we shall draw other communities
nearer to ourselves. After all, the new
theistic communities and the developments
of old communities are new, and we as
Jews have for our faith the most precious of
all testimonies — the unbroken testimony of
past generations. Our religion possesses all
the picturesqueness, warmth, colour, poetry
and romance which belongs to antiquity.
Conduct based on the teaching of Judaism
may attain to the sublime, and our lapses are
due not to inherent defects in our faith,
but to inherent defects in ourselves. The
new organisations look to us for spiritual
light. That light must be found burning
with ever-increasing brightness in our own
lives, and in the corporate life of our community.
By loyalty to our own faith, and by
reverent appreciation of the faith of other
men, we shall help to establish the dominion
of the God of love throughout the world.

\vspace*{2\baselineskip}

{
  \centering\larger[2]
  THE END

}
