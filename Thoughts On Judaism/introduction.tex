\numberlesschapter{Introduction}

This little book purposes to explain my
conception of Judaism as a living religion.
In endeavouring to answer the questions —
What are the vital principles of Judaism?
Why are they vital? How can they be
applied to modern life? — I have ventured to
reveal my own faith, for the experience of
one soul, however unimportant in itself, may
serve as a testimony to the living faith
which is among us. Clearly there can be,
on my part, no claim to any authority whatever;
nor do I pretend that my conception
of Judaism is novel. It owes a great deal to
Mr Montefiore’s essay on “Liberal Judaism,”
though the point of view is not everywhere
the same. But, like Mr Montefiore, I too
have ventured to work on constructive lines,
and to give, however briefly and imperfectly,
a personal presentment of Judaism. I have
written in a dogmatic strain, not assuredly
because I am not painfully conscious of my
own limitations, but because there is a large
body of Jews who require the construction,
at any rate in outline, of a definite theory of
their faith, They are anxious to realise and
to transmit Judaism as a living faith, but
have no time or inclination to work out the
principles and deductions of such a faith for
themselves. This class includes busy men
and women who “have enough to do
already without thinking very much about
their religion.” There are others who think
Judaism all right in its proper place, but do
not believe it affects \textsl{them} more often,
perhaps, than two or three times a year.
They cherish certain prejudices which
belonged to their parents, and when they
attend synagogue, are glad that it should
recall memories of their infancy. Therefore
they resist the bogey of “reform,” but their
religion has merely an impersonal interest.
It makes no demands on their lives; it is
no real help to them. Then there are the
parents who want their children to be faithful
to Judaism, but cannot see how they can
attach them to a doctrine, which appears to
them to be obsolete. There are the conscientious
teachers who long to make their
lessons alive and interesting, but who themselves
have not yet \textsl{quite} assimilated the
spiritual strength which they would transmit.
All these people seem to feel that Judaism,
without dogma, is too shadowy a faith to be
really acceptable to them. There is also
that large section of Jews who, like myself,
are seeking to understand the value
of their spiritual inheritance, and who
may feel sympathy with some of my conclusions.

I have tried to remember the point of
view of these various classes, and in a
practical manner to satisfy some of their
needs. My effort may perhaps stimulate
others in the same direction, and with better
results. Thus points of religious agreement
rather than differences are emphasised,
and it is proved that the same Ideal of
Righteousness inspires all sections of our
community. The variety of conceptions
held by believing Jews, are at once a peril
and a blessing to Judaism. For what are
the reasons for this variety? In the first
place, since the authority for our creed rests
in human conscience, its phases must be as
varied as individuality itself. Secondly,
Judaism has always been closely connected
with life, and life becomes more complex as
civilisation develops.

Judaism is the hallowing of existing ideals,
and ideals shift from generation to generation.
A religion which rests on conscience is a
robust religion, and makes a supreme
demand on all human faculties. It claims
the highest life from its devotees. The
close connection between religion and life is
clearly the ideal which all cults emphasise.
How then is the variety a peril? It gives
an excuse to the indifferent to devote their
minds to other causes, instead of attempting
to realise the principles of Judaism. They
argue that a religion which depends on the
conscience of each individual, is the concern
of each individual, and if he chooses to
neglect it, his apathy need not trouble his
neighbours. If he wishes, he can adopt a
more convenient faith, or, if he is thoroughly
indolent, he can say, “Since there are so
many conceptions none can be entirely true.
I will not trouble myself but will drift on to
the end of my life and be comfortable.” I
have tried to show that indifference is a
malignant growth which leads to spiritual
destruction, and that its influence spreads
far beyond the life of any individual sufferer.
It is dangerous to feel too comfortable about
religious matters, for this sort of comfort
generally prevents aspiration. We are here
to struggle nearer to the divine truth and
goodness. We shall not get very far if our
ideal is \textsl{comfort}, if we merely want to cover
up our indifference instead of fighting and
overcoming it. The “building-up time”
has arrived, and I venture to appeal to all
who sympathise with my religious conception,
to help in the work of reconstruction.
We must rouse the indifferent from their
lethargy and get them to realise their
religious obligations. Each community
must contribute some vitality to the religious
ideal of its own generation.

The beauty of Judaism is useless unless
we can consciously assimilate it in our lives.
Before it can be assimilated it must be
understood. This book attempts to explain,
as definitely and clearly as possible, the
meaning of our faith as it appears to one
Jewish believer.\footnote{In this connection
I wish to express my sincere thanks
to Mr and Mrs C.\,G.\,Montefiore for the sympathy and
encouragement they have given me throughout the production
of my book, and for their practical suggestions for
its improvement. Had it not been for their help I should
have been overwhelmed by the difficulty of making myself
articulate. While acknowledging most gratefully my indebtedness
to these friends, I would remind my readers
that I alone am responsible for the many limitations and
imperfections of my work.}
