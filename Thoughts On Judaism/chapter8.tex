\chapter{}

I have now given a brief exposition of the
fundamental principles of Judaism, and have
shown that their realisation, depends primarily
on their being applied to the conduct of
every-day life. Such application presents, as
we have seen, great difficulties to the average
man, who needs the help of religious observance
and Biblical study, in order that he may
understand his responsibilities as a Jew.
Hence the justified appeal to all Jews, to
prove their faithfulness to Jewish ideals.
Earnestly and prayerfully, we must begin
the work of adaptation and reconstruction.
The old truths live for ever. They must
be rendered comprehensible through their
symbols; they must be revealed in daily
conduct and in ceremonial observance. The
non-religious Jew is a menace to Judaism;
his ideals are often a travesty on the ideals
of our faith. Yet we should be unfaithful to
our mission, if we ignored those, who only
claim to be Jews, in so far, as they possess a
certain pride of race, but who give no heed
to religion. We must not deny them the
privileges of our Brotherhood, but we must
rather seek to win them to religion, and to a
more religious conception of Judaism. As
things now are, many of them help to
strengthen the materialistic tendencies of
our age, against which, we have undertaken
to labour.

It is clear that Jews, who live from day to
day, indifferent to the claims of religion and
moral aspiration, make our brotherhood
hateful in the eyes of our neighbours.
Generalising from a few instances, of gross
materialism, our critics affirm that we are a
degenerate people, existing only to advance
our own interests. It is for us, who \textsl{care}
about Judaism to try to show in our own lives
its power of inspiration. Thus we may win
the most indifferent back to their allegiance,
and prove that our religion can satisfy the
needs of posterity.

We must try to show the efficacy of
prayer and faith in our lives by seeking
God’s guidance in the ordering of our daily
pursuits. If we labour, so that every night
we may venture to ask God to bless the
work of our hands; if we seek in prayer
every morning, strength to accomplish
satisfactorily our daily tasks, we shall surely
endeavour to work from pure, unsordid motives
and to do all we have to do as well as we
can. We pride ourselves justly on being a
\textsl{practical} people, but even practicality can be
over developed and so leave too little room
for ideals. Some of us are rather ready to
denounce our neighbours as mere
visionaries, because their lives are uninfluenced by
utilitarian instincts. We forget that our
history is glorious through the record of
lives, devoted to study, and this devotion is a
form of idealism.

When their political importance and the
outward symbols of their greatness had
vanished altogether, our ancestors turned
their attention to learning. Their schools
were to be the source of their glory. They
spent their lives, in attempting to unravel the
difficult problems of religion and of life. In
the volumes of philosophical and theological
literature, which have come down to us,
there may be some hair-splitting and confusion
of thought. But the spirit of unselfish
devotion and of reverence for intellectual
work, which animated the writers, is surely
not without its inspiration to-day. The
men who gave themselves up to study, were
heroes in the sight of their contemporaries.
They were followed and loved, and their
most trivial utterances were recorded by
their disciples with absolute fidelity. Even
the men, who devoted themselves to transcribing
the scrolls of the law, recognised
their work as \textsl{holy}, and devoted to it a
patient courage which is in every sense
admirable. We cannot dwell on this page
of our history, without being profoundly
moved by its pathos. The people had lost
their temple, which, in spite of the warnings
of their prophets, they had believed
indestructible. The interest of their lives had
to be now altogether changed, for it had
centred round Jerusalem, At the moment of
their degradation and misery, God revealed to
our ancestors their glorious mission. It was
not merely to build and to preserve a
magnificent temple, that God had kept them
alive, and had led them by the rays of His
own light through the darkness of the ages.
The Temple was merely the symbol of an
eternal truth, and it was as witnesses to this
truth, that the brotherhood must exist.
Their conduct, their holiness \textsl{mattered} to
God. Dispersed among all races, despised
and even hated by men of other creeds,
they were to carry out the glorious charge
which had been laid upon them. We can
imagine the tremendous uplifting, which
such a revelation must have given to people,
bowed down by the burden of misery and
defeat. Much was expected from them.
They were not to waste their time in
miserable, useless lamentation. They were to
readjust all their ideas and aims, and cease
to care about material prosperity, and political
glory. As the guardians and depositories of
a great religious trust, they learned to rejoice
in their responsibilities, and obtained, by
intellectual and spiritual striving, a happiness,
which was destined to be real and lasting.

\begin{tp}{256}
We have no space in which to follow the
gradual development of the commercial talent
of the Jews. That talent received a powerful
stimulus under the new conditions of their
life in the lands of their dispersion. In
addition to learning, commerce began to
flourish among them. Indeed, the former
depended in a great measure upon the
latter, for scholars need bread on which
to live, and the results of profound intellectual
research, do not always prove of
material value. Throughout the dark ages
of mediæval superstition, the Jewish traders
and scholars were not necessarily two
distinct classes, but there were always
enough men willing to devote their attention
entirely to study. The synagogues
worked in connection with schools.
Unfortunately, here and there, the thirst for
gold got possession of some trader’s soul,
and he became engrossed in his work, and
indifferent to the spiritual claims of his
brotherhood. He became, perhaps, rather
unscrupulous, when he found honest careers
closed to him, and had recourse to questionable
means of obtaining self-advance\-ment.
The light of learning, however, was never
quenched among our people. The
enthusiasm for God’s work flourished among
them, as a community, in spite of the
frequent lapses of individuals into disgraceful
undertakings.
\end{tp}

To-day we must remind ourselves, that
we are descended from the People of the
Book, as well as from those, whose
commercial sagacity brought honour to their
race. We should recognise that through any
honest work, we can testify to our faith in God,
and that no shame can attach to careers, which
are conducted on honourable lines, for they
give opportunity for the realisation of the
highest ideals in conduct. In dealing with
our fellow-workers, and with the public
through our trades or professions, “We can
labour with clean hands and a pure heart,”
and observe Hillel’s golden rule, “What is
hateful to thee, do not unto thy neighbour.”
But we must remember the lessons of the
past. The thirst for gold does grow with
success, and since our people are so clever
at getting on, they must beware of the
temptations, which are so often connected
with material triumphs. In order that our
community should be true to its trust,
material success must not be its distinguishing
glory. Jews must show to the world
that material comfort is useful, as a means
to an end. People cannot feel the claims of
the higher life satisfactorily, while they are
hungry and ill-clothed and badly housed.
These physical needs \textsl{must} absorb their
attention. The spirit acts through the body
while we are on earth, and it is absurd to
ignore the claims of the body. But the
joys of study, of complete self-surrender to
philanthropic ends, must also not be forgotten
by our generation. There are careers
open to men and women, which can bring
no wealth and very little worldly fame. But
they are glorious in the sight of God, and
should therefore appeal to those, who are
summoned by their faith to minister to Him.
There are unpopular causes to be won by
our generation. There is work to be undertaken,
of which the results will belong to
posterity. There are trades and professions
open to us, which demand perseverance,
self-sacrifice and self-denial, and offer no
allurement of great personal profit. As Jews, we
must remember all these possibilities for
self-devotion, and seek to claim some for
ourselves and for our children. A father
once said of his daughter of six, “I don’t
want her to go to Sabbath School, I want
her to learn how to earn money!” This
man was a Jew by race, but he knew
nothing of communion with God. He
lived in a narrow, cheerless world, guarded
by the idols of gold, which he worshipped.
He denied to his child the inheritance of
Jewish womanhood. She was to be a
money-grubber like himself, to find pleasure
merely in getting wealth. She was to be
shut out from the kingdom of pure joy.

It is by prayer that we learn to
sanctify the claims of the body, and make
them subservient to a higher life. Morning
and evening, we remind ourselves, that there
is a God above us, who expects the \textsl{best} from
us. Faith teaches us, that we must not live
for the pleasures of the hour. “We leave
\textsl{now} for dogs and apes, we have \textsl{for ever}.”
Therefore we should not be afraid to allow
ourselves to be inspired by the lives of our
ancestors in the early centuries of the
Christian era. We, too, must devote much
time to the development of our religious
ideal. Indeed, if, as we profess, we really
believe that our religion is based on progress,
we, as a brotherhood, must endeavour, by
strenuous, self-denying effort, to receive some
new particle of knowledge from God, and to
transmit it to the next generation.

Faith should not only help us in the
choice and conduct of our active lives, but
should also make us strong in the power of
endurance. We remember that Job, when
he was suffering every conceivable misery
known to man, when he was bereft of all his
children and his possessions, when he was
being sorely tried by physical disease,
became gradually conscious of the mystery
of God’s love and the power of faith was
kindled within him. He had been rather a
self-righteous man, unaware of his own
spiritual needs and limitations. God,
through His chastening, taught him to realise
His presence. The problem of suffering
and evil, continues as in the days of Job, and
we have to reconcile it with the existence of
an Omnipresent and perfect God. Evil
exists. Therefore God allows it to exist
for He is all-powerful. We cannot solve
the mystery of evil. Our faith can only
suggest palliatives, which render its existence
more endurable. We admit that some evil
is the result of wrong-doing. If we indulge
in frequent uncontrolled tempers, we gradually
alienate our relations and friends; if we
have recourse to gambling or drinking our
moral sense becomes weaker. We neglect
our duties, and misery falls on ourselves and
our homes. Then again we may commit
some deed of treachery or impurity,
beyond the reach of civil or criminal law, and
conceal it so well, that the world knows
nothing of it. Yet this deed will sooner or
later make us suffer. We cannot escape its
results. “Be sure,” says the Bible, “that
your sin will find you out.” Some people
may refuse to be deterred from evil by the
fear of punishment, but they cannot be
altogether unaffected by the knowledge, that
their children will suffer for their sakes.
Surely no stronger incentive can induce men
and women to lead steady, pure lives, than
the knowledge that, if they sin, the
consequences of shame and guilt, must be shared
by the beings, whom they love most in the
world. Punishment which follows sin, is just
and comprehensible, even to our limited
human understanding. But much evil exists,
which is by no means the result of sin.
“Some forms of suffering can be shot
through with explanatory and ennobling
light, which makes them bearable and even
good; but other forms remain dark and
inexplicable. The sufferings of sentient
animals, and more especially the sufferings
inflicted upon them by thoughtless and cruel
men, continue to be a hopeless puzzle.
Among mankind there are evils such as
idiocy, madness and moral degradation
which seem beyond explanation. There are
problems respecting the relation of civilised
to uncivilised races; there are problems
respecting the endless individuals, who have
lived and died without any approach to that
mental and moral stature, of which mankind
is capable. There is not merely the strange
difference, which oppressed the mind of Job,
between the happiness of this man and that;
but we ask, and ask in vain, what can be the
meaning of that suffering and squalor which
do not ennoble or purify, but lead in many
cases almost inevitably to sin and depravity?
To these, and many similar problems no
answer can be given; we, no less than Job,
must simply trust in the infinite wisdom and
righteousness of God.

“On the other hand, for certain aspects of
suffering there are ennobling
alleviations.”\footnote{\textsl{Liberal Judaism}, p.\ 66,
by Mr C.\,G.\,Montefiore (Macmillan).}

Were we not acquainted in some measure
with pain, misery and sin, we should hardly
be able to appreciate goodness, happiness
and virtue. We recognise that, as physical
life is strengthened by the surmounting of
obstacles, so moral life is purified by the
struggle against sin.

“We all of us have seen how in times of
trial and trouble, people are frequently at
their best. Unexpected reserves of goodness
and self-sacrifice, are then displayed.
The brave endurance of misery at home, the
ardent struggle to relieve it abroad, and the
good fight against degradation and sin, have
provided, and still provide, the noblest
opportunities for the exhibition of human
patience, pity and human love.”\footnote{\textsl{Liberal
  Judaism}, p.\ 67.}

\begin{tp}{256}
In spite, however, of all these alleviations
we must admit that misery and pain are
awful while they last. The righteous suffer
with the wicked; the helpless and innocent
with the guilty. Faith alone can help us to
face these facts courageously and patiently.
I doubt whether a man who, in the
midst of an honourable, independent life
is suddenly afflicted with some horrible
disease, which renders him for an indefinite
period of time a burden to himself, and to his
family, can derive much comfort from the
hope of compensation in another world.
The only real comfort in such cases must lie
in the belief that there \textsl{is} some explanation
for the existence of evil, for God is good.
We therefore cling gratefully to our faith in
immortality and believe that “beyond the
veil,” in God’s own good time, we shall
know, why the hitherto unexplained misery
was allowed to exist on earth. Let us
then be at peace and trust in God. Evil
is no little thing; its presence is hateful to
us. God bids us fight against evil and
misery with all our strength, but when we
can struggle no more, we have the sublime
comfort of faith. God knows best, we say,
and, through our tears, we look at the world
and think it good.
\end{tp}

We cannot claim that faith, however deep
and sincere, can remove pain altogether.
But the recognition of an omnipresent God
of love gives us power to bow our heads, and
to endure courageously what we cannot
overcome. God loves us. So long as we
live, He has work for us to do. We must
take up our burdens in the spirit of David,
who, when, he had vainly endeavoured to
save his child, ceased to mourn, and went
about the work of his life. We cannot in this
world understand the mystery of pain; we
must believe in the God of love. He can
give us peace. We must seek it from Him.
Job was helped by his suffering to realise
God. Our periods of suffering also, must
be sent to us for our good, although in the
moment of agony, we cannot help sometimes
wishing that some other method of
purification could have been chosen.
Gradually, however, in answer to our
prayers, the power of submission is
vouchsafed to us. Here again, faith is
justified by experience. Those of us who
have suffered and have prayed, who have
put our grief behind us, and let it inspire us
to further effort in the cause of God, know
that the divine help was not withheld from
us. We have issued from the fire, scarred,
perhaps, but stronger, nevertheless, in our
love and in our faith.

If we can only believe in the vital principle
of Judaism, in the omnipresence of a God
of love and of righteousness, no incident in
life can be intolerable. Every experience
must have its meaning and its purpose.
When we rejoice, we shall rejoice more
completely, if we see God’s love and care
revealed in that joy. It is not a mere
chance, that we are happy. Our happiness
is sent to us for a purpose, for we must use
it in God’s service. When we are sad, God
knows about our sorrow and pities us. He
will give us peace.

We believe in immortality. We need
another life, in order that we should have
more time in which to grow good. God is
Perfection, and towards Perfection we are
bidden to strive. Every sorrow and every
joy — everything, indeed, that happens to us
in our lives — can be used as a means, by
which we may reach higher and higher on
the upward road. But we shall not get
very far, for we are, even the best of us,
very foolish and weak. We must accept
that fact at once. “We do not want the
future life for punishment, still less do we
want it for reward; we do not even so
greatly want it for the redress of this life’s
inequalities in outward prosperity; we do
want it for the progress of men towards
Perfection.”\footnote{\textsl{Bible for
  Home Reading}, Vol.\ II.\ p.\ 207.} With
the hope of immortality in
our hearts, let us as Jews live our lives, for
how \textsl{can} we live as if to-day were the end
of everything, seeing that we believe every
soul to belong to God and to have emanated
from Him? What do the little things
matter, the pleasures of gain, the petty
cares, the trivial disappointments? Our
God knows us from afar. In His love, let
us rejoice. Through His light let us see
light.
 

