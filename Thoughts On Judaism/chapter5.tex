\chapter{}

{
  \centering\larger
  \textsc{Religious observances are needed\\ as aids
    to holiness}

}

\vspace*{2\baselineskip}

Even in the simplest period of human existence,
men and women could never have
found it \textsl{easy} to lead righteous lives.
But to-day, when there is so much to do,
that we must all necessarily be in a hurry,
there seems less time than formerly to
think about God and about goodness. Life
has become very complicated now that we
no longer live huddled together in a ghetto,
allowed only to follow certain trades and
professions, and obliged to wear badges
to distinguish us from our emancipated
fellow-citizens.

In England, to-day, every variety of occupation
is open to us, and we need
guidance in selection. Honest work, well
accomplished, is a form of service which
we may offer to God for acceptance. It
is therefore of supreme importance that we
choose our work wisely. The Jew, who
recognises the omnipresence of God must
lead a consecrated life. He can hardly
expect the divine blessing to rest on him,
if he spends his time in acquiring wealth
by unfair methods. In our seasons of
prayer, we seek God’s judgment on our
work and in the light of His perfection we
realise our many failures and ask for help
that we may in future act more worthily.
Further, the craving for wealth may overwhelm
us and spoil our lives by absorbing
them, if we do not set days apart for the
study of God’s laws. Of course, we all
know that any day is God’s day, that He
is ready at all times to hear our prayers,
but the interests of the world are often so
powerful as to crush out from our minds
the memory of His “very present help.”
By religious observance, we are reminded
of God's presence — of the possibility of
drawing a little nearer to the ideal of
Truth, Beauty and Goodness which surrounds
our lives.

We are well provided with pleasures of all
sorts in modern England. But these pleasures
cease to make life delightful when they are
used not to sweeten labour, but as ends and
objects in themselves. Moreover, some of
these so-called pleasures are degrading, for
they can only be enjoyed by those whose
self-respect is dead. God means His servants
to be happy, to rejoice in His presence.
Judaism is altogether misconceived by those
who imagine its influence depressing and
gloomy. We serve God when we seek
pure joy. But it is during our religious
observances, that we have time to question
ourselves as to our choice of amusements.
Are they innocent? we ask. Do they afford
us \textsl{true recreation}? In the rush and whirl
of life, we are inclined to deceive ourselves,
and unless we pause every now and then
to consider the tendency and motives of
our conduct, we may rush into amusements
which we can only value as an excuse for
vicious self-indulgence. We must hallow
the joys of living, of learning, and of doing,
by using them in the service of God. Our
bodies and minds need occasional refreshment,
and we rejoice that the English
national conscience is beginning to recognise
and to provide for this need. When in our
hours of prayer we seek communion with
God, we realise that our capacity for pure
happiness must be used in the search after
the best in life, and the very fact that this
capacity can never be completely satisfied,
stimulates our faith in immortality.
Religious observances help us to be at
peace in the midst of the anxiety of everyday
life. We clear a space in our hearts for
the love of God to rest there, and in trying
to cherish this love, we are able to resist the
temptations to greed and self-indulgence,
which may assail us.

We cannot fail to be affected by the religious
doubts and controversies which rage
on all sides of us. In our own souls, we
experience our periods of conflict when we
question the meaning of the struggle against
evil which flourishes in spite of all human
effort. We, too, ask ourselves sometimes,
“Where is thy God?” Our observances
carry us back to the days of our childhood,
when with joy in our hearts we went
into the courts of God and praised His
name. Once more, something of the child's\footnote{1st ed.: child}
trust steals into our hearts and we are
satisfied to rest in God and to do His
will.

The trust of thinking men and women
is different from that of children. It is
strengthened by the doubts which have
been overcome, and the sacrifices which have
been made for its sake. Nevertheless, the
consolations of faith are most easily experienced,
when we adopt the receptive
attitude of children, when we recognise how
little we know, and how much we want to
know, how small we are, and what great
things we should like to do. These moods
come most easily to those who are trained in
the habit of observance.

There is no pain more troublesome than
the pain of monotony, when day follows
day with dreary sameness, when we know
our work so well that it makes no demand on
our imaginations.

It is our observances which help to bring
variety into our lives. They suggest possibilities
of self-development and of service.
They give us time to \textsl{think}, and to plan, and
to hope. Realities alone can oppress us; in
the kingdom of fancy there is joy.

Religious observances strengthen the bonds
uniting the members of our brotherhood.
When we remember that on certain days, at
certain times, Jews all over the world are
engaged in the same religious exercise, we
feel the stimulus of the corporate ideal. We
become more conscious of the mission to
which it is our privilege to be called. We
read in the Book of Kings how Elijah, after
his great triumph over the prophets of Baal,
felt overwhelmed by a feeling of loneliness.
He had proved himself a faithful servant of
God, he had caused his Master’s dominion
to be acknowledged by those who hitherto
had been led astray by false teaching; yet,
as he wandered through the wilderness
and sank down under a tree to rest, he
cried, “It is enough; now, O Lord, take
away my life, for I am not better than
my fathers.” But God bade him arise and
gave him more work to do, and reminded
him that there were many other men in
Israel who, like himself, had not bowed
their knees to false gods. For us, too, in
our humble lives it is an immense comfort,
either in our times of joy or of sadness, to
know that we are not alone. Other men are
experiencing the same hopes and fears as
ourselves; others are seeking to speak their
word to God. The feeling of sympathy
which binds us as a religious brotherhood
is emphasised, when we come together for
religious observance.

Prayer is an effort to reach to a higher
idea of life; as we strain upwards, we are
sustained by the thought that a common
purpose inspires us and our  fellow-worshippers.
On holy days, when we
engage in public worship, we become conscious
of a desire to serve our brotherhood.
Our hearts are kindled on the altars of
God, and we become “unashamed of love.”
No conventionality or artificial distinction
can separate us from our brethren in the
hour of prayer. Souls rush together in
their effort to praise God, the Father of
all.

We can hardly over-estimate the importance
of ceremonials as educational instruments.
Our children cannot realise abstract
ideas. In order that Judaism should have a
meaning to them, it must appeal to their
imaginations. It must also make a demand
upon them. All observances should be connected
with prayer in the child’s mind — prayer
in which he must take part, which he must
thoroughly understand. In the daily ordering
of our children’s lives, we naturally set aside
certain times for certain duties, and no other
claims are allowed to interfere with the allotment
of these hours. For example, the hours
of school, and of sleep, and of meals are in a
measure sacred for our children. We make
many sacrifices in order that they should not
be disturbed. Surely we are acting most
unwisely if we neglect to set aside some
time also for worship and general religious
training.

We want our children to grow into good
men and women, strong enough to accomplish
deeds of virtue. At our peril, we neglect
to give them the discipline which will lead
them to the realisation of God’s presence, for
God is the source of the highest virtue. If
children once acquire the habit of worship,
it is never likely to leave them, even when
their lives become full of pressing cares and
harassing duties and bewildering ambitions.
Indeed, as years pass, they will grow more
and more \textsl{dependent} on the power of prayer
to create joy in their lives and to give them
courage to overcome every difficulty and
danger, which presents itself. The development
of life should include the strengthening
of our faith. Conduct, let us remind
ourselves, is three-fourths of human life.
We want our children’s conduct to be
influenced by the highest ideals; we want
them to walk humbly with their God from
their earliest years. If they can once feel
the influence of God’s love in their lives,
they will hate sin, for sin prevents them from
realising God.

In order that Judaism should be a living
religion to our children, its precepts must be
transmitted to them with intelligence and
loving care.\footnote{This passage is taken
from a paper on “Home Worship
and its Influence on Social Work,”
read at the Conference
of Jewish Women, May 1902.}
We can, if we will, create an
atmosphere in our homes which shall be
conducive to prayer and aspiration. If we
venture simply and genuinely to admit our
conscious dependence on God for strength
and guidance in everyday life, we may inspire
all the members of our household
with that reverence which alone makes
sincere worship possible. If we ourselves
perform perfunctorily the religious obligations
which we recognise in our home life, their
inspiring power will disappear. They will
be accomplished as tasks irksome in themselves
and unrelated to other phases of our
daily lives... Children hunger for sympathy,
and we cannot secure their love and
respect more readily than by convincing
them that we, as they, are subject to temptations
and determined to overcome them; that
we too have knowledge of great weakness in
the presence of the difficulties which our
lives continually present to us, but that we
have supreme faith in God’s pity and loving
kindness. How can we assure them of these
facts more forcibly than by inviting them to
pray with us? Family worship should be
the most powerful link by which children
may be bound to their parents and to one
another... By asking God in the presence
\textsl{of our children} to bless the work of our
lives, we can testify to our conception of the
sacredness of work, as the duty we owe to
man in the service of God.

By cherishing a knowledge of Hebrew in
our homes, we are encouraging our children
to appreciate their religious inheritance, for
they can through Hebrew better understand
the inward meaning of their sacred
literature. Also the knowledge of Hebrew
strengthens the bonds which unite English
Jews with their co-religionists in all parts of
the world. But, while recognising the bond
of language as an important factor in the
religious development of the Jews, we must
remember that a knowledge of Hebrew is
not Judaism. It is, of course, very satisfactory
when our children are good Hebrew
scholars. Their learning is likely to lead
them to the most useful of all studies — the
study of the Bible. But, unless they have
acquired the habit of prayer, unless their
conduct reveals a devotion to Jewish
principles, they will not be equal to the
responsibilities which they have received
from God. In our home services, then, we
must emphasise above all things the necessity
of real intelligent communion with God,
and our worship must therefore include
some “made-up prayer” spoken in all
simplicity, sincerity and reverence in the
language most familiar to the worshippers.
We would desire to teach our children to
love religious observances and to recognise
their relation to modern life. This teaching
means sacrifice on the part of the parents
themselves. Not only have they to be
careful to perform their observances in
the spirit of prayer, but they must give up
time for patient teaching, for answering
questions, for making explanations. Children
become indifferent to observances
which have no meaning to them. When they
are told, in answer to their questions, “Read
this,” “Be quiet,” “Go to synagogue,” they
lose interest in the apparently meaningless
observances, and contempt creeps into
their hearts. The “throwing off” later is
easy enough. If we let our children adrift
in the world without giving them the
anchors they need on their passage through
life, we incur a terrible responsibility. They
will have \textsl{us} to thank for their purposeless,
indifferent lives, for their weakness in times
of temptation, for their degradation. We
have received a great religious inheritance,
and, unless we pass it on in its beauty, we
are untrue to our trust. Indeed, it is right
to remind ourselves every now and again of
the sacrifices which our fathers made in the
cause of religious education. In times of
persecution, they suffered poverty and
every sort of ignominy, in order that they
should hand the lamp of the Lord to us in
all its brightness. We have to trim that
lamp somewhat in order, that its light
should be seen by our generation.

If we refuse to give the lamp this
attention — if, instead, we place it in a
neglected corner, whence its brightness
cannot fall on our lives — our children will
live in darkness and see evil all their days.

Some of our religious observances have
a historical significance which adds to their
beauty, for it emphasises the idea of our
religious mission. Moreover, in studying the
manner in which our fathers celebrated festivals
and holy days, we can draw spiritual
lessons for ourselves to-day. For example,
in Biblical times, \textsl{sacrifices} were the important
feature of religious celebrations. It seems
strange to us, how in any age men should
have imagined the destruction of life to
be pleasing unto the God of Love, but, in
all our wonder, we must not forget to note
the spirit which animated the worshippers
of ancient days. They chose their most
valued possessions, and they gave them up
willingly to the service of God. They felt
confident of His presence and of His
power to answer their prayers. All their
ceremonial rites were accomplished with a
reverence and dignity suitable to the
occasion. To-day, our forms of service are
spiritualised, and therefore more in harmony
with the views of our generation. But we
cannot improve very much on the spirit of
reverence and trust and of sacrifice which
inspired our fathers. We may even question
whether to-day we are sufficiently eager to
give of our best in the service of God; whether,
when we enter our synagogues, we are really
so conscious of the Divine presence as to
speak our prayers in the full intensity of
faith; whether we do endeavour to reveal
in our ceremonies our highest conception of
beauty. In some parts of the world, even
to-day, our co-religionists celebrate the
holy days with trembling. The sword of
persecution is still hanging over them, and
they fear lest their large assemblies may
rouse the superstitious fury of the ignorant
populace among whom they dwell. This
fear seems only to strengthen their faithfulness.
It affords them new opportunities
for self-sacrifice. In England we can
assemble, confident that our worship will
not be molested by our neighbours. This
fact should add a new meaning to our songs
of thanksgiving and give a new reason for
our faithfulness. Our less fortunate co-religionists
must be excused, if occasionally
the darkness of their surroundings enters
into their souls and shows itself by some
form of superstition in their services. Any
shortcoming on their part should render
our duty more obvious to ourselves. Our
worship must reveal the most enlightened
thought known te our generation. If it is
full of meaning and inspiration for the
guidance of conduct, its brightness will not
only irradiate our own lives and crown
them with the most beautiful possibilities,
but will also serve to compensate those
who suffer for their faith, A meaningless
relic of a past civilisation would hardly be
worth the sacrifices made in its name. Our
religion belongs not only to the past — it
is part of the actual life which we are
leading to-day, and we believe that there is
no finality to its glorious possibilities, which
may be realised by the generations who
will follow us.
