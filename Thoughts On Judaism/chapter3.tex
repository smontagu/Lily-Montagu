\chapter{}

In the two previous chapters we have discussed
some of the vital principles of Judaism.
If the principles are vital, then they must
belong to all time. If devotion to the faith
made Moses lead a good life in the wilderness
thousands of years ago, this same
devotion must also help us in England to-day.
Life is certainly changed: our work
and responsibilities are different, our
pleasures and pains are different, our hopes
and aims are no longer the same. But,
nevertheless, truth \textsl{is} eternal, and we are
convinced that allegiance to Judaism can
make modern life beautiful and good, Indeed,
if we are to preserve Judaism as a
definite religion, we must show forth its
beauty in our lives. Let us by a few examples
see how the principles of Judaism
can affect and ennoble the conduct of the
ordinary everyday life with which we are
familiar.

We believe God to be One. Therefore,
He is Omnipresent. There is then  surrounding
us, near us, \textsl{in our hearts}, a Being
perfectly true, beautiful and good. We
know this Being to be God. We cannot
see Him; we can only see evidences of His
presence when we ourselves are in certain
receptive moods. But He is ever present,
ever the same. We have by personal experience
discovered that He possesses the
attributes of love and mercy. But we do
not know \textsl{what He is}. When we were
children we made fancy pictures of Him as
a strong and kind and tender man. But as
“grown-ups” we have learnt that He is not
\textsl{Man}. Inevitably we think of God as a
very fine ether or air all-pervading and penetrating,
and we have a little bit of this ether
inside us, “in our souls,” which we also regard
as a little something inside our bodies.
But this conception is not satisfactory. It is
difficult to pray to “air” or “ether.” We
must try to think of God as a living spirit
incalculably more noble and pure than any
form of life with which we are familiar. He
is spirit, but we know not what “spirit”
means. Seeing that He has no bodily form,
He has none of the limitations which belong
to human life. Possibly He has many attributes
of which as human beings we have no
conception. The cleverest and best people
cannot tell us what God is. They advise
us to do reverence to a mystery which we
cannot understand, and to thank God for the
faith which gives joyousness to our lives and
which could not exist without the mystery.
This faith is the direct gift of God, and it
satisfies a want in our lives by inducing us
to pray. If we admit that we cannot understand
what God is because He is perfect,
and we cannot understand perfection, we
shall still be able to realise His presence.
We feel a living something within us which
is good and makes for goodness if we allow
it to control our lives. This something can
commune with a Power outside itself. We
know by personal experience that this communion
is possible and no other evidence of
the existence of God is necessary. At any
time and in any place we can speak our
hearts to God. Therefore we believe Him
to be Omnipresent. By communion with
God we discover some of His attributes.
Because we find God perfectly loving, and
merciful and true, we prove by experience
the truth of the faith which He has given us.
When we pray we experience His help.
Through communion with God, our eyes are
opened to see the perfectly beautiful elements
in His work outside our own lives, and these
elements are evidences of His being. They
cannot be created by man. They belong to
God and reveal His purity. We cannot
measure the degree of God’s holiness, but
we can \textsl{believe} it to be immeasurable. We
can derive ever new sustenance from the
source of life and believe that the supply
can never be exhausted. It is best for us
to think of God’s \textsl{attributes} and not endeavour
to penetrate further into the mystery
of His being. Enough for us to believe
that He works in righteousness. Let us
imitate the Psalmist’s example and say,—

\begin{quote}
  “Lord, my heart is not haughty nor mine eyes lofty,\\
Neither do I exercise myself in great matters\\
Or in things too wonderful for me.\\
Surely I have stilled and quieted my soul;\\
Like a weaned child with his mother\\
My soul is with me like a weaned child.\\
O Israel, hope in the Lord\\
From this time forth and for evermore.”
\end{quote}

What difference does the presence of God
— with Whom we can have communion —
make to us? It makes us care for the
right things; it gives us a standard with
which to compare our human conceptions;
it gives us an ideal. Let us again illustrate
our meaning from our conception of
love — the best conception we know in life.
The existence of perfection outside us, makes
us seek the \textsl{best} form of love. In marriage
it helps us to distinguish between animal
passion and spiritual affinity. We seek to
make our home life pure and beautiful, free
from jarring strife and vicious habits, so that
it may be in harmony with the nature of
God. God has made us in His image. At
the moment of temptation or of anger, we
may be saved, if we remember the ideal
towards which we strive, and endeavour to
let perfect love, existing without, be reflected
in our hearts. We cling to this ideal of
love, and control ourselves to resist the
momentary self-indulgence, which may drive
it from our homes. We are ready to make
many sacrifices in order to preserve it.
The ideal of perfect truth, as well as the
ideal of love, may help us in our difficulties.
We know how easy it is to lie. Were it not
for our belief in the Ideal, we should be
sometimes tempted to think that truthfulness
was not a really necessary virtue. It seems
occasionally so useful to deceive, it helps us
to get on. Sometimes our country’s laws
are irksome and prevent us from doing what
we like. For example, the housing laws
prevent us from getting the rooms we wish;
the education laws force us to send our
children to school when they would be useful
at home; industrial laws forbid us to employ
people under the improper conditions which
suit our pockets; anti-gambling laws prevent
us from making money in ways convenient
to ourselves. Most of these laws can be
evaded by skilful deceit. But fortunately
such deceit is impossible to people, who
realise their responsibilities, when they dare
to call themselves Jews. There is a God of
truth, and we declare ourselves His servants.
We can only serve Him by truth, for no
other form of service is acceptable to
Him. However difficult the struggle, however
unpleasant, we must seek to approach
nearer to the Ideal of Truth which surrounds
our lives. It is near us in our
homes and in our workshops. If we want
to make our lives at one with God, they
must be free from deceit, which is hateful
unto Him.

There is another way in which the idea of
God’s presence can help to raise the standard
of our lives. We all know how we feel
when we meet a person whom we love and
respect very much. We want to be at our
best. If ugly thoughts come into our minds
we chase them away; we try to do and say
the things which would please him. We try
to let nothing jar on his standard of good.
Now, does it not seem clear that, if in our
own lives we could realise at all times God’s
presence, we should try as hard as possible
to be better? The ideal of perfection would
induce us to make efforts ourselves to
approach nearer to God. We should try to
conquer the habits which separate us from
Him. Let us just fancy what would happen
if one morning all men were to realise the
idea of God’s presence and cling to it
throughout the day. In London the working
men and women rushing along in tubes,
trains and buses, the women going about
their household avocations, the children in
the schools, the business men in their offices,
the professional men at their desks, the
idlers, the workers, rich and poor, learned
and unlearned, all these knowing themselves
in the presence of God would seek good and
not evil. Thoughts, words and deeds would
be sanctified; God’s rule would be recognised
on earth; His creatures would praise
Him in righteousness.

The idea of God’s omnipresence increases
our reverence for life. Life must be beautiful,
since God is revealed in life. When
we find good people we must respect them,
whatever their race or creed or social
position. \textsl{Their} goodness reflects \textsl{God's}
goodness. We pay it the homage which is
its due. Moreover, no man can be entirely
bad, since all men are the children of God.
It should then be our effort to discover the
influence of the divine, even in characters
otherwise brutal. In beautiful works of art,
too, we can find God. Sometimes these
works of art do not appeal to us at first.
Perhaps we have not studied enough to
understand them; it takes time to recognise
their power and we are too busy to
devote this time. Yet when we are told
that these artistic creations are the efforts
of men and women, who saw God’s beauty
in the world and tried to reproduce it
in their work, whether in music, painting
or books, we feel reverence for the artist.
We even make an effort to understand his
work.

The presence of God is perhaps still better
realised, when we are fortunate enough to
go into the country, and see God’s beauty
revealed in nature. When, for example, we
lie on the top of a hill covered with heather
— lovely in colour and in scent — and we
look up to a sky which is perfect in its
cloudless beauty, pure joy enters into our
soul. The world seems absolutely beautiful
— God’s presence pervades all, and every
flower and blade of grass seems to rejoice
in His glory.

\begin{tp}{216}
There is a tendency in modern times to
dwell on the ugly and evil side of life.
This attitude of mind sometimes leads
to coarseness. We revel in things brutal,
until we ourselves become less delicate in
our sensibility. Perhaps we think that ugly
sights and sounds and thoughts cannot harm
us, since we can distinguish between good
and evil. “Knowledge is good,” we say.
“Why should we fear it?” This kind of
argument often leads young men and women
to dull their senses with the study of impurities,
and in spite of their self-reliance,
they gradually find it more and more difficult
to “wash themselves and make themselves
clean.” Life is short, and while we busy
ourselves with the contemplation of vice,
the years slip past. Then we have no time
to see the glory of the Lord, of which the
whole earth is full. We do not seek to
approach the Perfect Love and Truth and
Beauty which is by our side. We are too
busy peering into the mud which lies
beneath us.\footnote{The problem of
the existence of evil is referred to on
page \pageref{evil}.}
\end{tp}

According to the \textsl{second} vital principle
of Judaism, the God of the Universe has
relations with each individual soul. This
belief must certainly increase our self-respect.
God dwells, we may venture to say, within
us. He blesses our lives with a Spirit
emanating from Himself. He requires us
to keep that spirit pure and strong with
an increasing strength. The body, which
conceals, and at the same time reveals that
spirit, must be kept healthy and clean.
Any impure act, or word, or thought renders
us less conscious of the God within us.
God has endowed us with the gifts of body,
mind and heart, and since we are responsible
to Him for our lives, we are responsible for
the manner in which we use His gifts. We
cannot excuse ourselves by crying that life
is short, goodness is difficult. We eat and
sleep, work and play, love and die, but is
that the end of all? God has given us
Eternity in which to complete our lives.
He has enriched us with aims and longings
which we cannot satisfy on earth. He has
bidden us cultivate a learning spirit, and
approach with humility and hope the kingdom
of the Unknowable.

God has relations with each human soul.
He cares about each of us, even the smallest
and humblest of us. He will help us in
our hour of difficulty; if we will seek His
help, He will strengthen us. Our joy is
pleasing unto Him. He pities us in our
times of sorrow. He is ever ready to help
us. We need have no fear. “Seek the
Lord at all times, call upon Him while He
is near.” Some of us are apt to think that
our lives are of little consequence. It cannot
matter much, what we do. In the industrial
world, we are not much regarded. We do
our work and receive our wages. If we fail
to satisfy our employer, he will send us
away, and a hundred other people will be
ready to take our place. We are cheap
articles. Why should we trouble? Judaism
teaches that no soul is cheap. It has dignity,
for it emanates from God — its destiny is
with God. He cares what becomes of us.
He expects us to be good. He will help
us to do what is right. If we do our work
for the sake of our wages — just well enough
to be paid — we are working in the spirit of
slaves. We are obeying the law of force.
If, instead, we put our best into our work,
and do it as well as we possibly can, we
are serving God as free men and women
should, and our work is pleasing in His
sight. We are obeying the law of Love.
God loves us. He accepts our efforts to do
right, even as in ancient days, He accepted
the sacrifices of our fathers which were
made in the spirit of devotion. We need
never be afraid to acknowledge ourselves
the children and servants of God, “Who
brings every work into judgment, whether
it be good or whether it be evil.” We
may venture to ask God to bless our
work, in spite of its many imperfections,
if we try strenuously to labour in His
name.

Judaism teaches\footnote{See page \pageref{responsibility}, concerning the third vital principle of Judaism.}
that we are directly responsible
to God for our lives, that if we
sin, we must bear the consequences of our
sin. We know that, however much we try,
our weakness is so great that our lives must
necessarily be imperfect. We can only rely
upon God’s mercy and love and pity. We
must live so that at the end of our lives
we may say, as we commit our spirits to
God, “We tried to do our best; we remembered
Thy trust.” It will surely not be
enough for us to say, “We never meant to
do any harm. We lived from day to day
and did our work, and enjoyed ourselves
and interfered with nobody.” We are put
into the world for some higher purpose than
to “do no harm.” We have to try to do
a little good, and to leave a small corner
of the world rather better than we found
it. There is a meaning and a purpose
in our life, since we serve God through
our actions. We can never \textsl{fully} develop
our powers, for they reach towards perfection,
and we can never be perfect on
earth, We can never do enough in the
service of man, for God requires that we
should love our neighbours as ourselves,
and we can never do enough for ourselves.

The individual life, then, is important because
it comes from God and must be returned
to Him, because God loves all His
creatures and gives them their strength.
Further, no human being can live alone.
Each \textsl{affects} his surroundings. If his life is
impure, he injures those, with whom he
comes in contact; he sullies God’s world; he
increases the amount of evil and ugliness
which help to conceal the vision of Perfection
from those who need it most.
Therefore, the love of our neighbours is a
necessary development of the love of God.\footnote{See
page \pageref{love}, IV.}
We must labour to give our fellow-citizens
the opportunities for self-realisation which
we ourselves desire. The progress of the
State depends on the progress of the individuals
who make up the State. No failure
on the part of our neighbours can leave us
untouched. If we dare to ignore the need
of any human being, on the ground that we
are not related to him, we do so at our
peril. The God who has fashioned all
races of men has bound them together
in their dependence on Him. If our
duties towards the State are religious in
character, the conduct of our home life
should surely also be consecrated to God.
God sanctifies the bonds which unite
husbands to wives and parents to children.
In this sense we may believe that marriages
are made in heaven. Men and
women, who disregard their obligations to
one another, forget their responsibilities to
the Omnipresent God. When we hear today
of Jewish husbands deserting their
wives, of wives neglecting their duties,
we wonder whether our community is
going to lose one of its chief glories. In
the past, the sanctity of home life was
zealously cherished among all sections of
Jews. \textsl{Mezuzoth}\footnote{The \textsl{Mezuzah}
is a piece of parchment on which are inscribed
the verses in Deut. vi. 4-9 and xi. 13-20. The parchment
is rolled together, put into a small case, and fixed on
the right-hand doorpost. A small opening is left in the case,
through which the Hebrew word for Almighty, written on the
back of the scroll, is visible.}
on the doorposts symbolised
a truth which was recognised and
obeyed. The \textsl{Mezuzoth} are now not always
discarded from homes where unfaithfulness
has banished love. What a mockery ceremonial
observance becomes, when it is
disassociated from a moral life! We inevitably
degrade our inheritance, when we
are faithful to ceremonials and forget the
ethical teaching of our faith, A man
cannot be a good Jew, if he neglects the
duties which he has taken upon himself
as husband and father. To his children
the inheritance of Judaism passes.

We dare not neglect the principles
which should inspire our lives and be transmitted
to our children. We are charged
by God to speak unto the generations of
His love and of His goodness. If we are
silent we sin against God.

As individuals, if not as parents, we
must endeavour to render our lives acceptable
to God. We have to remember
that we are members of a brotherhood
that exists for a definite religious
purpose.\footnote{See page \pageref{brotherhood}, V.}
If we bring shame on ourselves, we bring
shame on our community. A Jew, who
is dishonest in commerce, who engages in
degrading pursuits, injures not only himself
and his family but also his community.
By greed and ostentation we betray our
co-religionists, we cause our enemies to
rejoice in our discomfiture. Even as our
mission is noble, so ought our conduct to
be beyond reproach. Judaism cannot influence
the world, unless its followers earn
the world’s respect by reason of their
virtue. Not by riches, nor by knowledge,
can we cause God’s name to be glorified,
but by “doing justice, loving mercy,
and walking humbly with our God.” The
charge, which God has laid upon our
brotherhood, is a heavy charge. We
cannot escape our responsibilities. We
would fit ourselves more earnestly for their
faithful discharge.
