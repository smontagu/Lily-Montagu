\chapter{}

Most of us are agreed that certain principles
are vital to Judaism. By this we mean that
Judaism as a religion could not exist if any
one of these principles were refuted. We
believe that they cannot be refuted, and we
endeavour, as far as we can, to reveal this
faith through our lives. Quite apart from
the accident of our birth, quite apart from
our fidelity to ceremonials, we claim to belong
to the Jewish brotherhood, because we
accept the following principles as eternal
truths:—

I. \textsl{There is one sole Creator or God.}

This great central fact dominates all our
religious conceptions. Biblical prophets and
teachers did not speculate very frequently
on the nature of God. They were surrounded
by nations who put their trust in many gods,
and made material representations of them
for purposes of worship. This idolatry,
originating as it did in a variety of causes,
may have been partly stimulated by a deep
sense of reverence. The world seemed so
wonderful to these primitive worshippers,
that it was impossible for them to believe
that it could all have been the work of one
God. So they divided up the dominion of
nature and placed it under many rulers.
Gradually they attributed all sorts of coarse
human passions to these various gods, and
consequently the worship of them became
degraded and impure. Then was the idea
of God’s unity revealed to the Jewish prophets
and thinkers, and at the same time they
recognised their own human limitations.
\textsl{They were not meant} to understand His
being, they were only called upon to recognise
and pay homage to the attributes by
which He makes Himself manifest to His
creatures. These teachers were filled with
awe at the greatness and power of God, and
with gratitude for His love. They denounced
with all their strength the creation of idols,
since idolatry degraded worship. The one
God was manifest in all His works; any
effort to symbolise His power could only
limit His greatness, which was infinite. Today
we have not any temptation to make
idols. Common sense shows us the absurdity
of such worship as belonged to the child-period
of the world’s history. When we
declare our faith in the Unity of God, we
mean primarily that the Ruler of the Universe
is one, and that His very nature forms a unity.
In Him there is no clashing of wills or
varieties of purpose. As He was, so He is
and so He will be, and by His high and
changeless will the universe is governed and
controlled. Secondly, we mean that God is
“pure” spirit, for singleness of nature is
implied in unity, and we can only conceive
what we call “soul,” or spirit as absolutely one
and changeless. Thirdly, we mean that, in
our belief, the one spirit is revealed in all
forms of creation; that the one Spiritual
Being is omnipresent. To Him belong
perfect love, truth and beauty, and these
attributes are manifest in His works.

\label{unity}

II. From the belief in God's existence
and unity important consequences follow.
The second vital principle embodied in
Judaism is ‘‘\textsl{That the God of the World has
relations with each human soul, and that each
soul, being an emanation from Him, must be,
like Him, immortal.}” Our faith in the perfect
oneness of God involves our faith that the
human spirit, in however infinitesimal a
degree, shares His attributes. Because God
is immortal, the spirit with which He has
animated us cannot be liable to decay.

We can recognise two important channels,
through which the divine life works in the
spiritual world. The all-powerful, all-loving
God, who has called all creation into being,
can influence and sustain every form of life.
The human spirit can, through communion
with God, renew its strength from the divine
source whence it came. God in His love
has given us the supreme gift of aspiration.
In seeking to lead a higher life, we open our
hearts to receive strength from God. Our
Father in His pity and love reaches down
to us and helps us. We, in our efforts to
lead better lives, move a little nearer to
God.

The power of communion between man
and God is revealed in the influence of love
on our lives, We are conscious of God’s
love, when we cease to vex our souls with
harassing questions and miserable self-absorption,
when we stand still and look up. Then
we are at peace and we feel God’s presence.
Then again, pure, unselfish human love
spiritualises our lives, inasmuch as it is
inspired by the God who is the source of love.

In its purity and beauty it reflects, however
remotely, the glory of God. Life under the
influence of love becomes bright with
possibilities which stretch beyond and above
the world of passion and sordid struggle.
Love unites us to the God of life. In experiencing
love we know ourselves immortal.

\label{responsibility}
III. The Unity of God involves the existence of law. God governs the world by
law. When the leaves fall off the trees in
autumn, we are sure that the vital sap is
being secreted and that the joyous beauty of
spring will follow the dreary barrenness of
winter. That is God’s law. When at the
seaside, we see the tide ebb and the sand
appear, we know that this condition will not
last. There will be high tide again at the
exact moment when we expect it. We know
God’s immutable law. These physical laws,
which belong to a group known as the laws
of nature, are not the only laws which, because
of their immutability, we can call God’s laws.
There is also the moral law, upon which
another Jewish principle is founded. 
\textsl{We are responsible to God for our conduct, and if
we sin we must bear the consequences of our
sin. No intercessor ts possible or necessary
between man and God. The Divine love
enters into the hearts of those who seek it
with prayer and contrition.} Every created
being is meant to develop the law of its
existence, to develop all its powers and to
live a full live. Human beings are endowed
with certain powers of mind, and heart, and
body, which, being \textsl{good}, must belong to the
one Divine Being, Who in His oneness
includes all things good. We have the
power to be good — to realise the spiritual
life which is the best in us. We are endowed
with the power to know evil and to reject it.
When we turn away from goodness and
choose evil, we slip away from God, and life
becomes difficult and harassing. Only
through repentance and a changed life can
the soul which sin has separated from God
feel near to Him again. We bear the pain
of isolation when we sin. There can be no
union with God except through righteousness,
for the nature of God is entirely good.
\label{evil}

But while conscious of our sin, we are also
conscious of the power of reuniting ourselves
with God. If we will only repent, and by
continuous effort improve our lives, we can
atone for our sin and realise again the peace
which comes from God. Sin cannot be linked
to goodness. The two are distinct — apart,
eternally separate. No intercessory power
can obtain for us remission of our sins. We
have the power to make atonement for ourselves.
We can turn from our sin and again
live at one with God. When we sin we
separate ourselves from God. But He, being
eternal in His wisdom and His love, does
not lose sight of us. He knows us still, even
when we sin. He knows our weakness and
our temptation. His love \textsl{must} be beautified
by pity — for how otherwise could He in His
perfection love us? When we turn from our
sin, when we recognise it and hate it, and
allow ourselves to suffer the pain of remorse,
our nature is purified and spiritualised. The
divine love enters into our hearts. We have
atoned; we are at one. The more frequent
the sin, the more terrible the separation, the
more difficult the return. We can imagine
people who form the habit of evil-doing, lose
consciousness of the power of this Divine love.
They slip further and further from the source
of pure happiness, and in this separateness
they experience the consequences of sin.

\label{love}
IV. When we become conscious of the love
of God, as revealed in our own lives, we feel
instinctively drawn to our neighbours, who
share with us the spiritual life which is
divine. \textsl{The love of our neighbours is then a
necessary development of our love of God.}
His unity is revealed in the oneness of the
human family, in their common need—the
need for love. We dare not shrink from
any fellow-being, seeing that we are all
the children of God. Obviously, then, we
can only fulfil the law of our being and
realise a full life if we develop the power of
service. In helping our neighbours we are
revealing our love for God; we are doing
homage to His unity.

These four vital principles of Judaism are
embodied in our “Shema” — the prayer
which should be the inspiration of our lives.
In this prayer we declare the Unity of God
and proclaim our allegiance to the law of
love — that love which should purify our conduct
in all its various phases. It is by love
we reach God; it is through love that we
avoid sin; it is through love that we seek
to accomplish our duty to our neighbours
and to posterity.

\label{brotherhood}
V. If we reflect on the spiritual possibilities
of life, inspired by the principles of Judaism,
we cannot doubt \textsl{that the Jewish brotherhood
exists for a definite religious purpose.} We
are the guardians of a perfectly pure religious
idea, for we are the direct descendants of
those men who, in an age of idolatry and
degradation, bore witness to the Unity of
God. We have been taught by generations of
believers that God is the God of
righteousness, and that by righteousness
alone can He be served. If we are to be
true to the charge which our fathers have
laid upon us, we must hand down to our
children this pure faith, And we must
transmit it not only by the declaration of our
lips, but also by the example of our lives.
God has allowed Israel to survive all the
terrors of ignorance, persecution, self-indulgence
and superstition, in order that we may
bear witness to the power of faith as a
hallowing of life. If we can only realise the
privilege and joy of this work, we shall be
equal to the efforts of self-realisation and
self-sacrifice which it demands. It is because
we are often such unwilling and unfaithful
witnesses, that the vitalising power of Judaism
is so little recognised by the world. When
Israel knows its God and allows His
love to glorify its life, other nations will
join with it in a common worship. On
that day will the Lord be one and His
name one.

