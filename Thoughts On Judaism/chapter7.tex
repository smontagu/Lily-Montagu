\chapter{}

We can find in each of the “Five” appointed
Holy days a deep, ethical meaning, if we
would seek it in a reverent spirit.

The \textsl{Passover} commemorates the deliverance
of our fathers from slavery in Egypt.
Some of the details of this deliverance, as
recorded in Exodus, are probably, to a certain
extent, legendary. Yet we may rightly believe
that the descendants of Jacob were working in
ignorance and pain for the Egyptian task-masters,
when they were led forth to the wilderness
where Judaism was founded as a national
religion. Through this religion, thus founded,
all the nations of the earth were ultimately to
be blessed. The hurried departure of the
Israelites is symbolised in our eating of unleavened
bread on the Festival of Passover;
many other incidents of their deliverance
are commemorated in the “Seder” service.
These concrete and picturesque symbols
appeal to our imagination, and their observance
greatly interests our children and
encourages them to study the history of
their race. But we do not “keep” Passover
by merely refraining from eating leaven in
any form whatever, throughout a week, or
in forbidding it to pass the threshold of our
homes. We must also try to realise the
lessons which Passover suggests, and allow
them to influence our conduct. On the
Festival of Passover we must not forget to
thank God for the privileges of our appointment,
as witnesses to His goodness and
unity. We are heirs to that inheritance
which our fathers founded in the wilderness.
But we understand more clearly than they
possibly could, that this inheritance is one
involving work and self-sacrifice. They
were told to obey the precepts of the law,
in order that their days might be prolonged
in a land of abundance. We have learned
that our highest good is to be found in
works of righteousness, through which the
spirit of God may be revealed to the world.
Our religion has passed to a broader and
more universal stage.

\begin{tp}{1248}
Through studying the history of the
Exodus, we see how the thought of God
can refine human life. The Jews were
persecuted slaves, living merely to escape
punishment. Their chief pleasures seem to
have been connected with eating, and even
when the blessing of liberty was conferred
on them, they were willing to sacrifice it if
only they could taste again the delicacies of
Egypt. This same people were taught to
acknowledge the tender care and love of
God, and gradually they and their children
became susceptible to the higher beauties of
life. The gentle consideration revealed in
some of the laws recorded in Leviticus
testifies to the fact, that the Jews had
emerged from barbarism. When they came
to be surrounded by savage tribes they
remained susceptible to high ideals, and gradually
evolved the religion to which we are
devoted to-day. This transition from barbarism
to civilisation was wrought by the
gradual recognition of the Divine Father's
omnipresence. To-day some of us lead somewhat
sordid lives, caring mainly for good
food and smart clothes and getting rich. We
have not realised God. If \textsl{once} the habit of
prayer is introduced into our lives, our coarse
pleasures will cease to absorb us and we
shall experience higher joys. The festival
of Passover should remind us that as a
nation was led out of its barbarous ignorance
by the knowledge of God, to recognise the
highest refinements of life, so may we, by
communion with Him, attain the blessings
of culture, even though we may be of humble
birth and means, and have few opportunities
for scholastic learning.
\end{tp}

Passover is also the festival of liberty.
Again and again throughout the Bible we
are told to be considerate to the oppressed,
because our fathers were oppressed in the
land of Egypt. Their sad experiences
should inspire Jews of all times to be on the
side of justice and humanity in every struggle.
The week of Passover gives us the opportunity
for self-examination, and we should
ask ourselves particularly whether in the
conduct of our own lives, in our workshops
and in our homes, we are as kind as possible
to those who toil for us. An earnest woman
who is devoting her whole life to the cause
of industrial freedom, tells how her mother
worked in the mines in the days before the
passing of factory laws, and bore to the day
of her death the mark of the overseer’s whip
on her shoulder. The pain of that blow has
inspired a noble life of unselfishness and
devotion to the cause of the oppressed. We
Jews are the heirs of pain; across those
two thousand years which separate us from
the slaves in Egypt the sounds of lamentation
echo in our ears and inspire us to feel
sympathy for all who suffer the misery of
persecution, or even the minor pain of loneliness.
Thus the festival of Passover rouses
our indignation against Russian or Turkish
misrule, and also our sympathy for the little
servant girl who drudges in our home, or
for the shop assistant who ministers to our
needs from behind the counter. And this
indignation and sympathy should be genuine
and far-reaching. If the opportunity arises
for us to relieve the persecuted, we dare
not hesitate, lest the suffering of our fathers
should testify against us. We must also be
careful so to order our lives that no profit
or pleasure can come to us at the cost of
another human being’s pain.\footnote{\textsl{Bible
for Home Reading}, Vol.\ I.\ p.\ 74.} “Let, at any
rate, the season of the festival not pass away,
without our doing something in it, during the
very week while it lasts, to make somebody
or other a little happier, and to lessen for
a little while, or in a small degree, the load
of care or sorrow which so many people
around us have quietly and patiently to
bear.

“The Passover is therefore a festival of
hope and consecration, of thanksgiving and
gladness, of freedom and charity. It urges
us to look forward and strive to be grateful
to God the Giver and the Saviour, to bear
in mind the claims upon us of the stranger,
the fatherless and the widow. Remember
the past and work for the future; hope and
help; think and thank; be strong and
strengthen; rejoice and make rejoice; these
and such as these are the watchwords of the
Passover.”

The second of the great festivals of rejoicing,
the festival of Pentecost (the name
of Pentecost means the fiftieth day, from the
Greek \textsl{Pentikonta}, meaning fifty), is celebrated
seven weeks after Passover. Its meaning
has changed since Biblical times, when in
Palestine it was celebrated as a purely
agricultural festival. The Passover ritual
observances included the offering of a sheaf
of barley. On the feast of Pentecost the Jews
were commanded to bring two wave loaves
out of their habitations and in holy convocation
to give thanks for the harvest blessings.
Since early post-Biblical times Pentecost
has, however, been mainly regarded as a
festival to commemorate the giving of the
ten words. But we decorate our synagogues
with flowers in order that we may be
reminded of the old agricultural meaning.
These flowers should quicken us to a sense
of gratitude to God for the beauties of nature,
which belong to all men alike, both small and
great. “The festival [of Pentecost] year by
year celebrates the promulgation and excellence
of the ten fundamental words of religion
and morality. It is the festival, which celebrates
the great cardinal dogma of Judaism,
namely, the necessary union of religion and
morality with each other, that is, that God
is for ever associated with goodness, and
that goodness must for ever be associated
with God. One God, and He the God
of righteousness, that is the keynote of
Pentecost. Goodness for ever rooted in
God, even as God is goodness. The love
of God shown in the love of man, and the
love of man based upon, and culminating
in the love of God. Again, Pentecost is
the festival of the family, for it declares
that the basis of social well-being is
the honour of parents and the sanctity of
the home. Then, too, Pentecost is the
festival of law, and law is a great and noble
element in human life, which will always play
its part and maintain its worth. Lastly,
Pentecost is the festival, which, through
law, bids us in a sense get beyond law...

“The tenth word bids us quench the source
of evil which is within, cut down desire and
lust at their roots within the soul, and,
leaving the negative commands of prohibitory
law, we advance to the positive
commands of morality and religion — Thou
shalt love thy neighbour as thyself, thou
shalt love the Lord thy God with all thy
soul and with all thy might. Pentecost is
therefore a great festival of religion and
morality, a day, moreover, be it well remembered,
suited for the worship not of one
people only, but of anybody of whatever
race who chooses to join us in its
celebration.”\footnote{\textsl{Bible for Home
Reading}, Vol.\ I.\ p.\ 143.}

We see how Pentecost, if properly understood,
can teach us the ultimate meaning of
all religious observance, for it is intended
to stimulate our moral ideal. We are
reminded of the worthlessness of Judaism,
\textsl{unless} it includes a high conception of
morality. We are not Jews, unless we try
consciously and steadfastly to be good, and to
consecrate our lives to the Omnipresent God.

It is important that we should not let the
festival slip by, without devoting some
thought to the study of the Ten Words.
Year by year we may, with God’s help, see
more meaning in these commandments, and
thus each Pentecost should mark some little
advance in our conception of the purpose of
life, and the sanctity of its responsibilities.
The first commandment bids us dwell on the
oneness of God, on His eternal unvarying
goodness and love. The second and third
demand complete, single-hearted and reverent
service. We ask, whether we ourselves are
entirely free from idolatry, whether the cult
of riches and honour, does not sometimes
replace the true worship of God in our “holy
of holies,” which only the Father’s eye can
pierce. Are we careful enough in our
speech and in our thoughts not to take the
name of the Lord our God in vain? The
fourth commandment proclaims, to all time,
the value of the Sabbath, as the means of
uniting man with God in a holy covenant.
The fifth summons us to honour our parents.
Surely this exhortation is not superfluous in
our day, for is it not to-day that men and
women incline so persistently to be over-wise
in their own eyes, and to underrate
the sacrifices made by their parents in the
cause of truth?

The sixth, seventh, eighth and ninth
commandments admonish us not to transgress
that moral code, upon which civilised
society is based. They emphasise the
sanctity of human life and honour. The
tenth commandment, as we have seen, bids
us look within, and destroy the root of moral
evil, which is envy and lust.

The festival of Tabernacles is still a
festival of nature.\footnote{The following
passage, excepting those portions which
are bracketed, is taken almost \textsl{verbatim}
from the \textsl{Bible for Home Reading}.}
“It is the festival of
gratitude to God, the Giver of our daily
bread. It bids us remember all, that in the
last resort, we owe to the soil. Just as the
essence of character is goodness, and not
wisdom, so the basis of our life is not the
work of brain but the work of muscle and
hand. Life in cities depends upon life in the
fields. It was once said that man made the
town but God made the country. The saying
is not quite accurate, but there is some
truth in it.” (In the country the surroundings
are more beautiful than in towns; there
is more regularity and order. This beauty
and this order are revelations of God’s oneness.
We often see in towns, buildings
which are inspired by a high conception of
beauty, but these works may be spoiled by
the cupidity or meanness of the builder.
When men become fully conscious of God’s
omnipresent love and truth, then will their
work reveal Him as beautifully as do the
rivers, trees, plains and mountains in His
open country.) “Now that we have quite got
over the danger of worshipping any part of
creation, instead of creation’s Creator, we
must not run into the opposite extreme of
error and forget to remember the divine
Creator Himself. We must not empty
nature of God because we no longer believe
that any part of nature is itself divine...
More especially for the Jews, who have been
so long, and are many of them still, forced to
live in cities, and to gain their livelihood by
barter, and trade, and commerce, the festival
of Tabernacles is not the least important of
the three. It should not only awaken in us
gratitude to God the Giver, not merely
remind us that we owe our daily bread in a
hundred ways rather to God, than to ourselves,
not merely exhort us to the virtues of
modest simplicity and cleanly strength, which
are associated with the tilling of the soil, but
it should induce us to remember that the
primal and fundamental daily labour of man
is labour in the fields. Agriculture is the
first and the greatest of the arts of man.
No people is in a healthy state of which a
certain proportion is not tillers of the
soil.” ... (There is a natural tendency
for men in every community to follow certain
trades and professions. It is well,
however, for Jews to beware of this sort of
concentration. Their peculiar power of
adaptation and their wonderful vitality
should encourage them to attempt various
forms of useful activity. Agriculture makes
considerable demand on men’s power of
judgment and endurance; it also feeds their
love of speculation and excitement. It
cannot be altogether ill-adapted to the
Jewish character.) “But if there are, at
any rate in Western Europe, so few Jews”
(to-day) “who are agriculturists, it is the
more necessary for us all to learn to love
nature, and to teach our children to love
nature and to know a little, even if it be only
a very little, about her ways and her laws
and her creatures. An out-of-door life is a
good foundation for goodness and religion.
We must learn, if we can, to love nature
religiously, looking upon her, as the creation
of God, and seeking from, and finding in her
all the comfort and the strength which we
can.”\footnote{So far the \textsl{Bible for Home Reading.}}
If children grow up “streety,” if
they feel lonely and miserable in the country,
without the noise and excitement of city life,
we feel that they have lost something for
which no material comfort can compensate
them. Parents should not grudge any
sacrifice which would enable their children
to go into the country during the summer
holidays, for children may be induced by
the influence of nature’s beauties to realise
better the existence of God. Adult workers
also need the rest and peace of country
life some time in the year, in order that
their lives may be as complete as possible.

The festivals of Passover, Pentecost and
Tabernacles are pre-emi\-nent\-ly festivals of
rejoicing. We are glad, because of God’s
goodness. We are conscious of His care and
love. Prayers of thankfulness should rise
to His throne on these festivals, from every
Jewish heart. For past deliverances, for
present blessings, and for the power of hope,
we should thank God and sing songs of
praise to Him.

In addition to the three festivals of
rejoicing, the latest code added two others of a
totally different kind to the yearly cycle.
The New Year, which owes its name to an
arrangement of the Calendar, with which we
are no longer familiar, is a day of reflection
and preparation. Its value as an “aid to
holiness” can hardly be over-estimated. As we
assemble in prayer on the solemn day, we
think over the year which is at an end, and
realise its many shortcomings. We examine
our hopes and aims, and we decide whether
they may be used in God’s service, or had
better be discarded on the threshold of the
New Year. The day of New Year prepares
us for the most solemn of all days — the Day
of Atonement — and the days which divide
these two holy days should be used by us for
penitent thought and earnest heart-searchings.
In Biblical times, the Day of Atonement
was a day of \textsl{national} purification,
for the sins of individual Jews, whether moral
or ceremonial, were felt to degrade the whole
nation. Our Fathers therefore endeavoured,
by priestly rites, and by symbolic
self-purification, to remove the stains from their
national shield. The nation, as a nation,
must be clean, for it was believed to be
God’s peculiar treasure. To-day, our
Atonement ceremony has a more spiritual
and personal meaning for us. Each soul is
felt to be responsible to his Creator, and
his confessions must be made direct to
Him. By prayer and penitence, by kind and
charitable resolutions, we seek to feel again
at one with God. We realise that the great
Day of Atonement cannot help us, unless it
follows a succession of daily efforts to reach
nearer to God, and unless it gives us a new
start on a better life, which we strenuously
endeavour to lead. “He who says ‘I will sin
and the Day of Atonement will bring me
pardon,’ for him the Day of Atonement
will bring no pardon,” taught the Rabbis
seventeen hundred years ago, and it is well
for us to remind ourselves that sin cannot be
easily wiped out. Atonement can only be
accomplished by those, who, persistently and
continuously, strive to seek good and not evil
all the days of their lives.

We would not escape the consequences of
our sins; we could not, if we would. But
we endeavour on this most holy day to understand
ourselves, and to recognise our
weaknesses. Our strength of character, even as
the strength of some great work of mechanism,
depends on the strength of our weakest part.
As we really are bad-tempered, greedy,
licentious, proud, selfish or conceited, so we
stand bare before our God. His love and
pity keep us from despair. We ask His
help in prayer. The Day of Atonement is
still to us a Day of Judgment, but it is a day
of self-judgment. We dare not be tender to
ourselves. We tear open our heart, in order
to see its full weakness. We are sorry,
terribly sorry for our many faults and imperfections,
but we must not stop at futile
regrets. So long as we are alive, we shall
have the opportunity of being good. Year by
year, these opportunities should become more
clear to us. On the Day of Atonement, we
ask in prayer for courage, and strength to
devote our lives to God. As the service
draws to a close, we make the solemn
declaration of our faith in God. “God, He
is one!” we cry with solemn iteration. In
this cry, we concentrate all the strength
derived from a day of thinking and prayer.
Its meaning is impressed upon us, and we
go forth from the house of prayer resolved
to testify by our conduct, to the truth of
our faith.\footnote{In order to preserve the
joyousness of the day, our Sabbath ritual
contains hardly any reference to sin. Consequently,
there may be a danger that the thought of repentance,
which is really so closely interwoven with the idea of
prayer, should, except on the Day of Atonement, be forgotten
in our lives. It is important, therefore, that we should remember
the great teaching of \textsl{that} Day in our private daily
prayers.}

It has always been the custom to fast on the
Day of Atonement. In fasting, we give our interpretation
of the Biblical precept, “Ye shall
afflict your souls.” This custom is valuable,
because it concentrates the interest of the day
on things spiritual. We are not distracted
by the pleasures of the table, from our work
of prayer and praise. It should always be
remembered, however, that this fasting does
not comprehend the full duty, belonging to
the Day of Atonement. It is merely a
means to the accomplishment of that duty.
On this holiest of holy days, we prepare our
hearts until they are attuned to deeds of
righteousness. Mere ceremonials cannot
avail us, as ends in themselves. They can
only stimulate us to a higher life. This thought
emphasised in the beautiful lesson chosen
from the Prophets for the Day of Atonement:
“Is not this the fast that I have chosen? To
loose the fetters of wickedness, to undo the
thongs of the yoke, and to let the oppressed
go free, and that ye break every yoke? Is it
not to deal thy bread to the hungry, and that
thou bring the poor that are cast out to thy
house? When thou seest the naked, that
thou cover him, and that thou hide not
thyself from thine own flesh? Then shall
thy light break forth as the morning and thy
health shall spring forth speedily, and thy
righteousness shall go before thee; the glory
of God will be thy reward.”

\begin{tp}{256}
All the Jewish holy days begin at sunset,
and this fact suggests a beautiful spiritual
lesson, for the mystery of birth must always
be shrouded in darkness. These
festivals are full of life, which it is for
us to absorb and make our own. It is
right that their beginning should be in
darkness. The birth of the soul is also
hidden from us.
\end{tp}

Besides these five appointed days, to
which we have referred, there are other
feasts and fasts not ordained in the
Pentateuch, but observed by many of our
brethren. Want of space prevents me from
discussing them in detail. But I cannot
pass over the festival of Chanukah, which
has its origin in the post-Biblical history
of the Maccabees, without paying a tribute to
its \textsl{religious} value. “The mere national
aspect of the matter is very small and trivial;
whether a petty tribe of folk called Judæians
preserved their separate national existence
and constitution, or became assimilated with
the Hellenistic Syrian subjects of the motley
kingdom of Antiochus was unimportant,
when looked at from a merely political or
national point of view. But it so happened,
that this small race possessed at that time
the purest and truest conception of God and
of the manner of serving Him among all the
races of the earth, and if therefore, this race
had then been destroyed or absorbed in the
mass of Greeks and Syrians, this religion
would also have perished. The work of the
Prophets would have been in vain. It would,
as it were, have had to be begun all over
again. The Maccabeæan victories insured
the continuance of the teachings and writings
of Amos and of the Isaiahs. Therefore, the
festival of Chanukah is a \textsl{religious} festival,
and as such is worthy of our high regard. We
are not specially concerned with the defeats
of the Syrians. The details of the fightings,
subsequent to the dedication of the Temple
are of smaller interest to us. The Maccabæan
family itself, suffers from the results of
conquest and victory. But the preservation
of Judaism at a time of imminent and critical
danger, remains a permanent fact of supreme
importance. If Judæa had been overcome
and absorbed, the Jewish congregations
outside it, would very probably have been
unable to outlive the shock. Therefore we
owe our gratitude to the martyrs and soldiers
whose festival we celebrate in the days of
Chanukah. Let Chanukah be also a festival
of courage, a fourth part of all virtue, as the
Greeks of old believed. The courage which
Judaism demands of us now, is not the
courage of soldiers upon the battlefield, but
it is often courage none the less. Let the
deeds of martyrs and soldiers in the age of
Antiochus inspire us from year to year
anew.”\footnote{\textsl{Bible for Home Reading},
Vol.\ II, p.\ 740.}
We need this inspiration to-day,
when it requires courage to show allegiance
to the teachings of Judaism. Again and again
we are tempted to let evil pass, when it
seems not to concern us directly, and so we
are faithless to our ideal of righteousness.
Often Freethinkers seem to be more popular
than Jews, and we are sometimes inclined
to conceal our faith in order to share their
popularity. The Chanukah lessons should
make us ashamed of such cowardice. We
bow our heads in memory of the heroism of
Judas Maccabæus, and pray that it may
inspire our lives.

Besides the historical narrative and the
precepts concerning ceremonials and festivals,
the Pentateuch contains a number of moral
laws which deserve our attention. These
enactments are mostly concerned with
brotherly love and charity. They also
formulate a high moral standard in business
and home life. Some of these laws can no
longer affect our lives, for they refer to
conditions which have passed away. In a
few instances, the ethical code has been
superseded by more enlightened conceptions,
and there are many phases of modern life, for
which the Pentateuchal laws provide no
guidance. These facts should only serve to
stimulate our interest in these Biblical books.
In the midst of verses, which give no inspiration
to modern life, we find passages of
inexhaustible spiritual strength. The Pentateuchal
laws also include a series of dietary
laws, which are valuable, both on sanitary
and on ethical grounds. These laws have
been observed in post-Biblical days with
a remarkable devotion, and even to-day,
they are respected in homes in which all
other ceremonial laws are broken. To
the clear Biblical precepts concerning forbidden
food, the Rabbis have added a
series of ordinances, which have been
accorded almost equal respect. Many of
the so-called Mosaic dietary laws are in
harmony with modern hygienic principles,
and we can only marvel at their antiquity.
Moreover, the self-control which their proper
observance requires, has been essentially
useful to our community, and has trained
them in habits of temperance. Unfortunately,
the legal minutiæ added by the Rabbis,
have here and there somewhat distorted the
vision of believers, who have been so
misled as to call themselves Jews merely
because they kept “Koscher” homes. The
effect of such exaggeration, has been
disastrous to the spiritual life of our
community. It is for us Jews, who aim at
making our Judaism a living, ethical influence
in our lives, to reveal a sense of proportion
in our observance. We must
reverently examine these laws, and, where
they are in accordance with hygienic truth,
and secure the most humane treatment
of animals, we should give them our
allegiance. In seeking truth, we are testifying
to our faith in God. As education
improves, we are happily less and less
affected by the discipline of our appetites.
There are so many pleasures, which appeal
to us more strongly than the pleasures of the
table; nevertheless, we are not so impervious
to temptation, that we can afford to undervalue
the lessons in self-control which sound
dietary laws enforce. If regarded as a means
of purification, they are in harmony with a
strenuous religious life, and should therefore be
observed in a kingdom of priests.\footnote{Compare
\textsl{Judaism as Creed and Life}, by Mr Morris
Joseph, p.\ 185, par.\ 1:— “These Dietary Laws ... may
help to maintain Jewish separateness; they may preserve
the idea of Israel’s consecration, they may exercise a powerful
influence upon personal purity. The last two objects are
obviously desirable in themselves. They are more even
than this, they are vital objects. The consciousness of
being an elect people, and a power of setting an example to
the world of personal holiness, are alike essential to the
fulfilment of our divinely-appointed errand. Every law that
strengthens these qualities, merits respect and obedience. It
is a law which still fulfils a great purpose. It is a living
law, and therefore a law that deserves to live.”

Mr Joseph rests the value of the dietary laws to Judaism,
on purely religious grounds.}
Moreover, in so far as they are consistent with the best
scientific principles, known to our generation,
they form a valuable part of that inheritance
to which we must remain faithful, if through
us, the whole family of the “earth is to be
blessed.” Such fidelity can only strengthen
our conception of the innumerable sacrifices,
which Judaism demands of us in the cause
of truth and righteousness.
