\chapter{}

We have in the previous chapters enumerated
the vital principles of Judaism, and
discussed their influence on modern life.
We have now to ask ourselves — On what
authority do we base our belief? Where
do we find these principles established which
we have ventured to formulate?

The so-called orthodox section of Jews
would reply that these principles do not
comprehend Judaism. To them, Judaism
means the observance of the Pentateuchal
and Rabbinical law, and through obedience
to that law they attain to righteousness.
The formulating of principles is to them
a matter of secondary importance. The
supreme duty is to obey the law, which
has been handed down from generation to
generation, and this obedience teaches them
self-restraint and self-sacrifice. This book
is not addressed to men and women belonging
to this school of thought. To those
who accept the verbal inspiration of the
Bible and its miraculous divine revelation,
religious duty is too clear to require comment.
No consideration of ease, self-advancement
or parental indulgence can
justify the law-breaker, who regards the law,
as the embodiment of God's eternal word.

We can feel little sympathy with those,
who shirk a duty on the grounds of its irksomeness,
who are sceptical merely through
selfishness.

These Apostates perhaps deserve some of
the anathemas, which are flung indiscriminately
at the unobservant — although conversion
to religious observance is seldom
accomplished by abuse. But in our community
to-day, there is a large class of Jews
who are unobservant, because their Judaism
no longer rests on the authority of the
Pentateuch. \textsl{They find it instead in human
conscience}, in experience and in history.

We have only to formulate this changed
conception, in order to recognise its difficulties
and dangers. Indeed at the first shock
we fancy that a religion based on human
conscience, must be a religion of conflicting,
chaotic principles. It is only after careful
consideration that we are led to a different
conclusion, and to realise that a religion
based on the authority of conscience, makes
a supreme demand on the noblest faculties
with which man is endowed. The experience
of prayer shews that there \textsl{is} communion
between man and God, and therefore
in the language of our childhood, we
may still venture to define conscience as the
Voice of God within man, and we need not
be afraid to be guided by its authority. It
leads us to recognise the existence of the
Good, the True and the Beautiful as revealed
in all forms of spiritual life, and to
find the noblest ethical lessons in the Bible,
and in the lives and works of the best men
of all ages. Seeing that God is true, we
admit that He can only be served by truth,
and therefore we are induced to make the
conduct of our lives conformable to the
highest conceptions of truth, to which, with
the help of the thinkers and teachers of all
ages, we are able to attain. The best minds
devoted to the study of religious history and
of the Bible guide us in our search after
truth. We dare not be afraid of their conclusions.
Judaism must be able to survive
the scrutiny of the keenest human intellect,
directed towards the sacred literature. It
is a sort of blasphemy to withhold mind from
the study of God’s word.

Earnest, reverent study induces us to
believe only in the partial inspiration of the
Bible, and in the diverse ethical value of its
component parts. We find in the Bible the
noblest conception of God and goodness ever
given to the world. The Book contains the
finest ideals of conduct ever formulated by
men, and we dare not disregard its teaching.
We dare not neglect the noble, ethical
precepts contained in the Bible, on the
ground that they are sometimes followed by
contributions from less-inspired souls. Let
us seek the \textsl{best} in the Bible and when we
find it, let us admit that God — the Perfect
God — has allowed His spirit to rest upon
His servants and they have spoken His
will. Then let us do homage to their
teaching.

\begin{tp}{512}
There is a terrible danger in evading the
duty of seeking God’s Word in the Bible.
Of course we \textsl{can} excuse ourselves in
numbers of ways. We can even pretend
that we are not certain enough of selecting
wisely; therefore we will escape altogether
the duty of selection. Let us remind ourselves
again, that we are responsible to God
for the use of our powers. The consciousness
of our imperfections, is no justification for
those of us, who are backward in God’s
service. Moses was slow of speech, but God
chose him as His messenger. He gave him
the help he needed to do the work which was
demanded of him. We need not then be
discouraged by the knowledge, that our best
intellectual efforts must necessarily be
terribly imperfect. We will grope after truth
although our sight is dim; our own faults
often shut the light from us, \textsl{but} God is
merciful. When we seek His word with
reverence, zeal and humility, God in His
unspeakable love allows a ray of light to fall
over our lives and to guide our conduct.
No single human being can expect during his
short life on earth to learn everything about
goodness and God. His own limitations and
imperfections limit the possible rewards of
his search. His want of success convinces
him of the existence of heaven. May we not
believe that to the gathering of all God’s
servants from all ages, countries and creeds
beyond the veil, there will be revealed
complete truth?
\end{tp}

We have tried to show in the previous
chapters that, without reference to the Bible,
man may, by communion, derive from God
the principles which should guide a Jewish
life, and that by experience he may prove
their truth. If thoroughly realised, they
stimulate righteous conduct in all who proclaim
allegiance to them. But we are a
religious brotherhood, and devotion to the
Bible is necessary, if we are to perform our
mission to humanity. We are guardians of
the spiritual treasure which the world has
received; we are the descendants of those
who bore testimony to the unity of God. Our
existence by that unbroken descent is part
of that testimony. The record of lives
illuminated by the principles of Judaism, is
needed by humanity. The gradual development
of these principles by Jewish
teachers, prophets and seers, helps men
of all creeds \textsl{to-day} to seek God and
to serve Him in righteousness. We are
the guardians of these records. Moreover,
we need the Bible teaching for ourselves,
for it affords us instruction, refreshment,
consolation and encouragement. After communion
with God in worship, our conscience
testifies to the truth of the noblest revelations
contained in the Bible. The study of the
Bible requires from us self-sacrifice and
perseverance. If we would really receive
its inspiration, we must seek it in a humble,
reverent, learning spirit; we must be willing
to \textsl{think}, before we hope to understand. If
we would sift the best from the less good, we
must attune ourselves to the right mood for
study. We are not always in the mood for
Bible study, any more than we are at all times
fit to hear beautiful music or to read exquisite
poetry. But it is good to train
ourselves to study the Bible for a few quiet
moments every day. Thus we may not
only become so familiar with its beautiful
teaching, that it may gradually affect our
minds and characters, but the habit of study
may also imbue us with the spirit of reverence.

Through trying to know God through the
Bible, we may gradually learn to seek Him
in all things good and beautiful.

The Bible narrative records the lives of
men and women whose weaknesses and
virtues were very much like our own. They
felt their dependence on God — that dependence
which we all experience. The Psalms
also contain the reflections of almost every
human mood. In these poems of sadness and
of joy, of repentance and of anger, of trustfulness
and thanksgiving, there is always the
same note of yearning. While speaking their
simple word to God — the psalmists yearn to
be at one with Him, to let His love enter their
souls, As we read of the submission of the
human will to the will of God, we begin to
understand and to share the longings of a
broken and a contrite heart; we realise the
possibilities of worship and of the union of
human beings in the service of God. Some
of the Psalms reflect a higher religious tone
than others. Even in the same psalm, we
often find verses of different ethical value —
the products of several grades of civilisation.
But we are conscious that no single psalm is
insincere. Every word rings true. We
reverence it as the expression of a man’s
soul. Face to face with God, the Psalmist is
conscious of his sins; he strains towards perfection
and the ideal seems to move higher
and higher above his plane, as he struggles
upward. As we read, we hear in reply to the
Psalmist’s passionate cry of disappointment
and despair, the whisper of God’s love — the
whisper which is caught up by the ages and
echoed and re-echoed in triumphant sounds
of hopefulness and trust. The perfect communion
between man and God described in
the Psalms gives us courage. We too will
say our word — we too will cry to God — we
too will shout for joy, since we are alive and
have the power to learn and to love. God
hears us. God answers us.

For us in these days of moral laxity, self-indulgence
and materialism, the writings of
the Prophets are full of rousing exhortation.
These old teachers are stern in their simplicity.
They insist on unselfishness and uprightness,
on \textsl{effort}. It is no use, they tell us, for us to
fill our temples with images of self, and say
we cannot see God. God is here in our lives,
crying to us to make ourselves clean, to turn
to Him and to live.

These prophets made sacrifices for the
sake of truth. Again and again God forced
His revelation upon them, His truth entered
their hearts; they dared not be silent.
Sometimes they had to give up their comfort
and ease and throw themselves completely
into the struggle against evil. Frequently
they had to incur the anger of their
contemporaries. In the cause of truth
they had to speak the word which was
nighest to them. They could not flee
from God’s presence — they could not
renounce the charge which He placed upon
them.

As we read of these strong men of old
we pray God to give courage to our generation.
The old struggle is still raging around
us, the struggle against religious indifference
and negation, against moral weakness and
deceit. The sadness of isolation is on
God’s people; His voice cries and is not
heard by men.

The Bible tells us how Jonah was loth
to warn the people of Nineveh of the
punishment that was overtaking them.
They did not belong to his school of
religious thought, and he was therefore indifferent
to their doom. But the unwilling
servant was shewn his error and was made
to recognise the universal fatherhood of
God. His words of warning caused the
people to repent, and, before his wondering
eyes, God’s mercy was revealed.

A modern \textsl{Jonah} would also be forced
to warn men of the misery of sin, and draw
them by words of love and sympathy to
experience the joys of divine communion.
We cannot imagine that he would be allowed
to leave many members of his brotherhood
in indifference or apathy, because they could
not, or would not believe in and follow all
the words of the law. He would have,
nevertheless, to admit that they too were
precious in the eyes of the Lord and might
deserve to share the joys of religion.

God knows the hearts of men and will
surely not judge those as wicked, who endeavour
to live honestly according to the
light which He has given them.

A modern Jeremiah could not plead
pressure of duties, and the pleasures of
home life as an excuse for silence, when
his eyes were opened to see the materialistic
tendencies of his age. The thought of God’s
righteousness would overwhelm him. He
would risk the pain of misunderstanding
and invectives. If necessary, he would
sacrifice all the joys that sweeten life and
go forth among his brethren and force
them to come into the light of truth. If
he feared the disintegration of his community,
the degradation of their faith, he
would not cry “Peace! peace!” for to him
there would be no peace.

Throughout the Bible, we find the highest
precepts for our guidance in every relation
of life. We also find many ceremonial
ordinances, which we are unwise to disregard.
Observances are needed by us as
aids to holiness, as reminders of God’s
goodness. They serve as the best possible
links for binding our religious brotherhood
together, and as the most helpful of all
educational instruments.

We need hardly remind ourselves that
the ideal Jewish life consciously led, in the
presence of God is a high ideal, too difficult
for most of us to attain. To whatever
section of the community we belong, we
would assuredly make use of all the aids
to righteousness which we can find in our
Bible. Conduct can never be a matter of
indifference to the believing Jew, and he
can never be satisfied that his conduct has
attained the highest plane of rectitude, for
the Ideal of Perfection inspires his life.
Observances are necessary to emphasise
the bond which unites us with God. Otherwise,
with our limited powers of vision,
we may so easily become chained to the
actual interests of the moment and forget
the “Better beyond,” which just touches
our horizon, and lights it with a beautiful pure
light. While seeking an ethical meaning
in all our observances, we should remember
that the usefulness of ceremonials is immeasurably
increased by the devotion of
our fathers. The impress of their sacrifice
makes these observances more lovable in
our eyes. But they have no ethical value,
when regarded merely as survivals of an
age that has entirely passed away; if they
are worth preserving, they must make a
direct appeal to the conduct of life with
which we are familiar. They should remind
us of God’s presence and lead us nearer
to His throne; they should give the
necessary discipline to those who exercise
themselves in works of holiness.

