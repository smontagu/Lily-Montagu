\chapter{}

In a previous chapter it was said that religious
observances must be tested by their
ethical value. If they suggest no moral
lesson applicable to modern life, it is our
obvious duty to discard them, for their
presence is likely to spoil our vision of God.
This duty of selection is incumbent on all
those Jews who do not believe in the verbal
inspiration of the Bible, and who endeavour
to devote their reason to the service of
God.

But if we study with reverence the Biblical
observances and conscientiously test their
ethical value, we shall generally be able to
derive from them some teaching applicable to
the spiritual needs of modern life. Let us remind
ourselves that if these observances do
satisfy these needs, their age gives them
special interest, and should inspire us with
peculiar reverence. For long life in itself
claims our homage, when it represents the
accumulated strength of years.

Thus, when we recall the life of Moses
several examples of heroism and self-sacrifice
inflame our imagination, but the
last incident surpasses all in spiritual
grandeur. We see him standing up before.
the Egyptian king and demanding the
deliverance of his people. Through a period
of anxiety and disappointment we note how
he learns something of God’s love and
power of forgiveness. Stimulated by this
conception, he is able later, in spite of his
personal vexation, to pray that his people
may be forgiven for their faithlessness and
discontent. Finally, we see him on Mount
Nebo yielding up his spirit to God in
perfect trustfulness. Before his human
vision, stretches the Promised Land. To his
spirit is revealed the ideal of Perfect Love,
and he is at rest. He has done that part
of the work to which he had been called.
In quiet confidence he leaves to his successor
the realisation of his own earthly hopes,
having lost none of his interest in his
people.

We admire the courage and self-sacrifice
of the leader in the presence of the enemy,
and still more, perhaps, when he is able to
forget the base ingratitude of those whom
he has served. But the full measure of our
reverence is given to him when, on the
eve of his death, he exhorts his people “to
be strong and of good courage in obeying
the behests of their God.” So it is with
our ancient observances. The devotion
of ages kindles them into life; they
yield to us the accumulated strength of
the past.

I have spoken in previous chapters of
the importance of home services, as an
educational influence in the lives of our
children. But I am well aware that the
tenement dwellings in which so many of our
co-religionists live, by their want of space
make daily family services almost impossible.\footnote{Taken
from the paper on “Home Worship.”}
There is one phase of family life, however,
which in every home can and should be sanctified
by prayer. I refer to the Sabbath eve
celebrations which should bind families close
together in the bond of holy fellowship.
This observance has never lost the blessings
of peace and hope with which it was
endowed by the devotion of our fathers, who
found in it the expression of God’s promise
to all who struggle and suffer in the world.
As the children draw round the Sabbath
lights and sing hymns of thanksgiving to
their God, even the poorest, saddest home
is made beautiful. A holy peace rests on
each tired worker. They all remember
that the God of love understands their need
and has pity on them, when they try
courageously to bear their burdens. Parents
and children become conscious of God’s
purpose in their lives. They realise their
responsibilities to Him, and together they
enter His courts and reverently ask Him
for strength to work out their lives in His
service. The beautiful story of Jacob's
dream suggests a Friday night lesson. The
tired traveller, conscious of the guilt which
is upon him on account of his mean conduct
to Isaac, lies down by the roadside with a
stone for his pillow. It is only then when
he is in trouble, when he is cut off from his
family and his friends through his own
sin, that the idea of God’s nearness is
revealed to him for the first time and his
religion acquires a meaning, which is to
influence his life. “And he dreamed and
behold a ladder set on the earth, and the
top of it reached to heaven, and behold the
angels of God ascending and descending on
it.” Then he hears the voice of God telling
him of the work which he will have to
accomplish, in order that through him and
his seed all the families of the world shall
be blessed. “Then Jacob awaked out of
his sleep, and he said, ‘Surely God is in
this place and I knew it not.’\,” The same
desolate surroundings were visible to Jacob
when he awoke as on the previous night,
the same hard pillow was under his head,
but nevertheless the world had changed
for him. It was glorified, for he had
begun to feel God’s presence. So he went
forth to learn more about goodness and
God through hard work and self-sacrifice
and the sweet consolations of love. God
chose him as His Prince, and he spent his
life in trying to realise the full meaning of
that lesson, of which the first line was learned
when he exclaimed, “Surely God is in this
place and I knew it not.” God is everywhere,
and He loves righteousness. This
is the lesson, too, which the Sabbath eve
teaches, and thus the Sabbath eve observance
may serve as a ladder by which
we may reach through prayer from earth
to heaven. It reminds us of God’s presence
and beautifies and ennobles our homes.
It fills our minds with visions which
strengthen us to go about our work and
trust to the help of God. It beautifies
the darkest corners of our lives with the
light of hope. A little girl once said,
“If you \textsl{are} naughty all the week, you
must at least be good on Friday night.”
The child did not mean that Friday night’s
goodness made up for the week’s misdoings,
but she had been influenced by the spirit
of aspiration, which belongs to Friday night,
and felt that naughty words and thoughts
must not be allowed to spoil its holy
beauty.

There are Jews influenced by Oriental
conceptions, who still seem to think that
Judaism is less concerned with women than
with men. But the tendency of the Sabbath
eve observance is to broaden our conception
of Judaism and its ceremonials. We
realise their connection with life and their
general usefulness. Upon the mother
devolves the duty of lighting the Sabbath
light, the symbol of the home, hallowed
by service. Upon her virtue and godliness,
does the purity of the home ultimately
rest.\footnote{When we hear that some of our co-religionists spend
their Friday nights in going to theatres and music-halls, or
parties, or in card-playing, we feel that they are making a
terrible mistake — a mistake which may spoil their whole lives
and the lives of their children. By choosing the wrong time
for their amusements they cut themselves off from the
highest influences of Judaism. They are even desecrating
the sanctity of their homes.}

Life would be indeed \textsl{earthy} without its
Sabbaths and holy days. They give us the
time so necessary for the tightening of those
links, which bind the soul to the God who
gave it. Nobody can desecrate the Sabbath
with work without being conscious of a
serious loss. Only absolute necessity should
drive us to Sabbath work. But if the
necessity \textsl{is} there, it can give us no excuse
to sever ourselves from the community.
Rather we must make more strenuous efforts
to create opportunities of public worship,
since through no fault of our own we
may be unable to attend the synagogues
at times when the authorised services are
being held there. If work is honest and
hallowed by the conception of God’s
omnipresence, it will not need any difficult
adjustment of ideas, in order that we
may pass from our workshop to the house
of prayer.

It is possible that existing synagogue
forms of service may fail to appeal to some
of us. But this indifference should not be
an excuse for us to remain away from public
worship altogether. The fact that men
and women come together in prayer, in
itself gives us some useful food for
thought, and as we mingle our word with
that of other worshippers, our zeal is
strengthened by their fervour. We can
only hope to influence the form of service
authorised by our ecclesiastical authorities,
if we can prove it to be unsatisfying
after a long and regular attendance. It
is no use our saying that we dislike a
particular form, and therefore are indifferent
to all Jewish worship. We must care
sufficiently to realise what we lack, and
keep so in touch with our community,
that when the opportunity arises we may
formulate our needs, and be assured of
a sympathetic hearing. It has been said
that the Sabbath should be a day of rest
and of worship. We need rest for our
bodies and \textsl{change} for our minds; we need
prayer to strengthen us for the labours of
the week. If we \textsl{must} work, we can and
should still pray, but the necessity of work
should only be admitted when some real
sacrifice has been vainly made to prevent
it. The Sabbath, besides being a day of
rest and of prayer, should be a joyful day — 
a day when we can find time to rejoice in
the midst of our family and friends and
realise the message of kindliness which the
Sabbath proclaims. It should be a day of
pure recreation, when we can have
recourse to all sorts of quiet and refreshing
pleasures.

Some of us are inclined to point to very
observant Jews, whose religious professions
are not in harmony with their daily conduct,
and to pretend that their insincerity justifies
our complete indifference. But this attitude
is clearly illogical. Because some men hide
their ugly deeds behind the dazzling light
of specious holiness, we need not refuse to
seek \textsl{true} holiness ourselves. It is wiser
to put our own lives right and to recognise
our own shortcomings than to
worry about the failings of our neighbours.

In the next chapter we will consider the
meaning of the Jewish holy days and ask
how their observance can stimulate right
conduct. The general ethical value of
holy days, we have already attempted to
establish in discussing the helpfulness of
ceremonials.

