\chapter{}

In the previous chapter I have given my
conception of the vital principles of Judaism.
In the following pages I propose to explain
this conception more fully.

It is obviously impossible to prove the
existence of God. We ask ourselves — On
what evidence do we base our faith? How,
in the face of so much misery and evil, can
we believe in an all-powerful, all-merciful
and all-just God? Our replies will only
satisfy our fellow-believers. We do not
pretend to satisfactorily “explain” God by
processes of reasoning and by argument, for
we know the limitations of the human
intellect.

We believe that “God’s thoughts are not
our thoughts and our ways are not His
ways.” We see evidence around us of the
existence of law, and we worship God as
the author of law. In man there is abundant
evidence of the spiritual life. Acts of pure
self-sacrifice and of noble heroism cannot be
explained on physical grounds. The very
incompleteness of the noblest human lives,
the suffering of the “finite heart that
yearns,” the endless striving after unattainable
ideals, all bear testimony to the existence of a God Whose perfection inspires
mortals with a “divine discontent.”

We feel God’s presence within ourselves
and in the good desires which sometimes
obtain a mastery over our lives and force us
to accomplish deeds of love. The finite
mind cannot comprehend the Infinite — the
imperfect spirit fails utterly, when it seeks
to measure itself with Perfection. But God
can satisfy the souls of those who seek Him.
He can make himself felt. We can solve the
secret of the Lord when we fear Him. No
amount of logical exposition can explain the
certainty or the intensity of our faith, We
feel God and are at rest. We seek not to
understand Him for we realise that none
of us shall “see His face and live.”

It seems clear that experience alone can
convince us of the existence of God, of His
love for righteousness, of His relations to
each human soul. To the sceptic, who
cannot admit the possibility of God’s Fatherhood,
we can only say, “Pray, ask God
to reveal Himself to you, accept the
limitations of your understanding, throw
yourself on God’s mercy, speak to Him
your doubts, and then ‘stand in awe and
sin not, commune with your heart upon
your bed and be still.’\,” This silent waiting
is difficult to achieve in our age of disquietude
and of restless activity. We toss
about one philosophical theory after another
and can get no rest. But, if we will only
be still, we shall hear the word, “very nigh
to us in our minds and in our hearts, that
we may do it.” When we peer into the
future and consider certain troubles which
may overtake us, we are sometimes inclined
to believe that such troubles will be quite
intolerable; we shall succumb under their
burden. But God reveals Himself in many
ways, and sometimes the whisper of His love
is most clearly heard in the midst of tribulation.
To Hagar, as she watched in the
wilderness, came the voice of God bidding
her arise and shake off her agony and take
up her child and live. The mother-love
revealed in Hagar lives to-day in all its
passionate intensity, and noble purity, and
reflects, in spite of human frailty, some of
the brightest rays of the divine love — the
rays of pity, tenderness, unselfishness and
forgiveness. Yet how often in our own
experience do we see this mother-love over
taken by the most overwhelming trials. A
child is snatched away without warning by
some swift malady; another is seen to linger
in suffering, and the remedies which would
relieve the pain are beyond the mother’s
means — she must watch the suffering and
cry aloud in her impotence. Nothing avails
 — God’s will is done. Another child, full
of bright promise, is chained to a life of
misery and temptation. The mother-love
is in conflict with conditions which it cannot
overthrow, even though through its intensity
it survives — in their despite. In all these
instances, we bow our heads in awe before
the mystery of God’s love. According to
the beautiful Maccabæan legend, the oil
which seemed only sufficient for the one
night’s ritual celebrations in the Temple
fed the sacred lamp for a whole week.
Similarly, our power, which seems so limited,
is in times of trial strengthened by God's
love. He never sends us trouble without
supplying us at the same time with the
courage to endure. But we must train
ourselves to seek His help — to look up in
prayer to His throne. Then when the
moment of our trouble comes, our faith will
not fail us. The glorious light of hope and
love will burst through the darkest clouds
and irradiate once more our lives. We
cannot go through life without learning to
know and to admire men and women, who
bear their troubles with splendid fortitude,
who live saintly lives, but who nevertheless
deny that they are in any way conscious of
the existence of God. Our acquaintance
with these heroic men and women sometimes
affects us uncomfortably. We are
mystified by their courage, and their
scepticism suggests doubts to ourselves.
But there is no question that many of these
sceptics love and worship God under a name
which they create for themselves. Perhaps
they believe in goodness, or in law, or in
nature, or in a spiritual essence, and
consciously or unconsciously endow these
abstractions with many of the attributes
which, according to Jewish teaching, belong
to God. But we must admit that there are
others who serve God by their righteousness,
while yet unable to acknowledge His
sovereignty. They are strong enough to
live good lives without the aids to holiness,
which religion supplies. But while reverencing
these courageous folk, and admitting
that their righteousness makes our lapses
all the more grievous and shameful, we
venture, nevertheless, to believe that the
possibilities of virtue must be greater to
the believer than to the unbeliever. “Life’s
ideals are hallowed by religion,” and if we
refuse to recognise the existence of
perfection outside our lives, we must admit
limitations to the degree of our own
endeavour. Moreover, the strong man who
relies solely on his strength cannot live free
from peril. Hillel taught us never to be
sure of ourselves till the day of our death.
Further, none of us can live for ourselves
alone. The sceptic, like every other man
since the days of Cain, is destined to be
his brother’s keeper. He is responsible for
his children, and even for his neighbours,
whom he may have infected with his
scepticism. Who knows whether their
strength will be equal to his own? It is
the task of the believing Jew to wage a
crusade against religious indifference, and
negation, with all their deadening tendencies.
His effective weapon will be the testimony
of a holy life illuminated by joy and hope
and dignified by responsibility and purpose.

Our belief in the immortality of the soul
rests primarily on our faith in God’s Unity\footnote{see above,
page \pageref{unity}.}
and can be further explained by our
consciousness of His love. We have been
endowed with powers of mind and heart
which we cannot fully realise in this world.
Human love would be indeed an irony if
it ended with death. Our desire to learn
wisdom, to work righteousness, to attain
pure joy, can never be completely satisfied
on earth. These desires are good; they
come from God; they testify to our immortality,
for the God who sent them loves
us and will not suffer any good thing to
be lost.

Unless we can accept as a vital principle
of our faith the fact that the love of our
fellow-men is a necessary development of
our love for God, domestic and social
life lose their sanctity. If the service
of man is a form of divine service,
passion and self-interest cannot tempt us
to deny our domestic and civic obligations.
Moreover, in reverencing the unity of God
as revealed in His creation, we are ready
to work without reward to brighten lives
yet unborn.

When Frederic William of Prussia ordered
his chaplain to prove in one sentence the
truth of religion, he answered, ‘‘The Jews,
Your Majesty!” This story cannot fail to
gratify our vanity, but it should also quicken
our sense of responsibility. The Jews continued
to exist through ages of sorrow,
ignorance and persecution. They preserved
the holy purity of their faith in spite of all
temptations and misfortunes. It was difficult
enough to remain faithful, to resist the
temptation of perjury. Again and again
they could have bought comfort and advancement
for themselves and for their
children by denying the faith which they
had inherited. But they remained true.
They declared their allegiance to God and
their faith in His love and in the claims of
personal service. But they did not stop at
mere verbal declaration. If they had not
shown that their faith in God inspired
righteous conduct, that it affected their
common everyday life, then no inspiring
lesson could be drawn from their survival.
But throughout the ages the Jews believed
in God and this belief affected their conduct.
They were at peace although they were
surrounded by deadly foes; God satisfied all
their highest longings, although they lived
in penury; and they were free in the midst
of their bondage, because they believed
themselves to be both the servants and the
children of God.

To-day, in England, we are surrounded by
different temptations. Life is easier; the
roads to prosperity and success are open to
us. Our needs are less obvious and crude
than those of past generations ; nevertheless,
the obligation to praise God is less easily
remembered than that of petitioning Him.
Prosperity seems to have dulled our sense of
gratitude instead of quickening it, and to
have increased our greed. Perhaps, also,
the importance of declaring our faith in the
one God is less apparent since other
communities have proclaimed their allegiance to
Him. But surely life can never be easy to
live \textsl{well}. We stumble forward and new
rocks are in our path. We look into the
future and new opportunities of service
influence our imagination, As we go forward
we need light and yet more light, and this
light can be supplied by our faith. As we,
with our improved opportunities, grow in
intellectual and moral power, our faith
should grow in intensity. A religion is
dead unless it can satisfy the needs of a
progressing civilisation.

We claim that Judaism embodies vital
and eternal principles which can in all ages
lead men to righteousness. Unless we
believe ourselves to be the appointed
guardians of these truths, we shall not be
able to resist the temptation to merge our
life with the life of the majority.
Separateness involves self-sacrifice; the continuance
of our brotherhood is not possible without it.
The pain of this sacrifice disappears when
the privilege of service is recognised.

