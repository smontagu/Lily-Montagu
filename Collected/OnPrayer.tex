\chapter{On Prayer}

``Ho every one that thirsteth, come ye to the waters.''
This call, taken from Isaiah 55, is a call to prayer. Some
of you ask: ``Why pray? What is the good of it?''

I think we need to pray. We are incomplete without
contact with the living God -- and that is what prayer
means. We are beings created by the living Spirit of
Goodness, Truth, Love and Justice, who wish to return and
draw from our source renewed energy with which to carry on
our lives. We pray, then, for “the increase'' of our
spiritual power. We need sustenance and exercise for our
spirits, quite as much as we need food and gymnastics for
our body. We meditate on God's law of righteousness, and
the desire to be better and to do better fills our hearts.
Why pray? \textsl{We pray in the first place that we may live more
fully}.

We are all conscious of some wrong doing in our lives
which separates us from God. We pray for the power to
overcome evil in ourselves, and to become one with God. We
are each of us directly responsible for the conduct of our
lives. We have to destroy the wrong by our own efforts.
The consciousness that God is real -- that something of His
spirit is in our hearts, even while, in its perfection, it
is the supreme life force in the universe, this faith gives
us the power to overcome sin, for it makes our will strong
and directed towards good. \textsl{We pray, therefore in the
second place for self advancement in righteousness}.

As we pray, we feel ourselves united with every
aspiring human being; his life and well being are part of
our own lives. We understand his needs, for we actually
share them, and so \textsl{the third blessing which we discover in
  praying is the unity of mankind}.

But we would like to alter the world; we would wish to
see some of the evils and misery, cruelty and injustice
swept away. \textsl{So we pray for the removal of evil}, though it
still persists and sometimes seems to grow in magnitude.
What then is the good of praying?

The good lies in another direction. In prayer, we
realise the value and power of human personality. We pray,
and the possibility of achievement is unveiled before our
eyes. We can choose good and reject evil. That is our
human prerogative. The same privilege belongs to the men
and women who are in positions of great power and responsibility.
We would not surrender the freedom of human
personality. \textsl{When we pray we show the connection between
belief and conduct}.

Freedom is part of the human inheritance. Prayer has
revealed the power of man to create the better world
through obedience and loyalty to the laws of God. It is
\textsl{for us} to help in establishing the kingdom. We must not ask
God to do it for us, and so surrender our human independence.
He has offered us the power to work with Him.
Arise, shine!\footnote{Isaiah 60:1}

We love to pray for our dear ones, and sometimes we
pray and the calamity we wish to avert comes just as if we
had not prayed at all. Why then, we ask, does not God
hear? I believe that God \textsl{does} hear, and it is well for us
to think of our dear ones when we are considering the
reality of God in prayer. Let us seek from His revelation
ways to increase our wisdom and our power of loving.
Perhaps we shall lose some of our selfishness and our
ability to give pain. But when things don't go right, as
we believe, for our beloved, we must remember the limitation
of our vision and that what seems evil to us may in
the end be good. \textsl{Through prayer we learn to trust in the
supreme love of God, and that He acts only through love}.

Life is sweet for you all. In spite of its sad,
gloomy passages, you have the power of learning and loving.
You can see some beauty in the world, even if the glimpses
are few and far between. You can sometimes see the wonders
of nature and hear glorious music. Most of you have behind
you the security of home life and the trust of those who
love you. You, because you are young, can experience the
pleasures of the body and the mind. You feel grateful for
life. \textsl{In prayer you can give thanks}.

May I appeal to you to pray daily -- to feel yourself
consciously in the presence of God? If doubts assail you,
face them and wrestle with them. In the end, they will, I
believe, add strength to your faith. Don't give up prayer
because it is difficult. Learn to create the right atmosphere
for prayer, the atmosphere of reverence and humility.
Clear your hearts before you pray from selfishness and
insincerity, and your mind from impure thoughts. Then
throw yourself into the ``Everlasting arms''\footnote{Deuteronomy 33:22} of God, and in
the depths of your heart you will hear Him speak. ``Speak
Lord, for thy servant heareth.''\footnote{1 Samuel 3:9}
I beg of you not to delay
praying until you are too faint and weak for want of
spiritual nourishment to pray at all. \textsl{Pray now}! Tomorrow
it may be too late.

\attrib{Club Letter no. 5, March 1939}
