\chapter{Can We Possibly Be Mistaken?}

We are all familiar with the phrase: "I may be
mistaken, but," and the conviction that the speaker is
certain he is right rings through his words. But the
history of nations, as well as the history of individuals,
is a record of mistakes, and very frequently it is just out
of these mistakes that progress is made. "Every good, that
is worth possessing," says Professor James, "must be paid
for by strokes of daily effort."\footnote{William James, Talks to Teachers (1908).}
It is pathetic to see
how much error has been perpetrated in good faith by men
and women prompted by the best motives. Witness all the
terrible industrial laws which grew out of the error that
cheap labour, child labour, was essential to national well-being.
So children worked inordinate hours in mines and
factories, and even risked their lives in cleaning
chimneys, and good people shook their heads over the
sadness of these children's lot, but felt it belonged to an
unchangeable order of society. It required the illuminated
conscience of a Shaftesbury to reveal error, and to demand
that this error should no longer be persisted in.

Individuals often fall to their lowest standard of
conduct through smugness, or lack of courage, for, having
recognised error, they refuse to correct it. Perhaps it
looks bad to own ourselves mistaken, and it is only the
courageous who can struggle against error, especially if it
has the support of the majority. Unless we can accept an
absolute standard of goodness, which progresses from age to
age, it is almost impossible for us to recognise our
errors. Through our natural indolence, and through our
inelination to live from hour to hour, without weighing the
consequences of our actions, we cherish expediency and
revel in compromise.

But it is about the absolute standard that I should
like mainly to speak this morning. It happens to all of
us, who think at all, to have moments when we wonder how we
can continue to believe in God in the face of evil. We ask
in misery, and, sometimes almost in despair, whether
perhaps after all we are not building up our lives on
popular myths and fairy tales. There \textsl{can} be no certainty
in the field of religion.

In 1914, a war, it was believed, could only last a few
weeks, months at most. We could trust the German people
never to harbour hatred against England. The two countries
were so intimately associated in friendship and national
interest. They could not long be separated. Then, when
these prophecies ware falsified, we believed that a passion
for peace was being developed between 1914 and 1918 which
could never again be disturbed, With the fraternising of
all classes in wartime had come a new era in the life of
the English nation. Class distinction was destroyed for
ever. Men and women had discovered the true values in
life, and could never again fritter it in vanity, or in the
pursuit of selfish aims. We ourselves spoke these or
similar words only yesterday - 20 years or so ago - and now
we know how false all these opinions were. We were
terribly, sadly mistaken. Can we possibly be mistaken in
our convictions again? We can endure to be mistaken in our
political and social theories, but we cannot let our religious
certainties totter. Not those! Surely not those!
How can we \textsl{know} there is a God? Is it not possible that in
this, as well as in so much else, we are mistaken?

If you were Orthodox Jews and believed in the literal
and eternal truth of every word contained in the
Pentateuch, you could not accept the possibility of being
mistaken. God revealed Himself on Mount Sinai. The
Israelites actually heard the thunder and saw the lightning
which accompanied the giving of the Ten Commandments. They
saw the light on Moses’ face when he descended from the
mountain, having spoken to God, and been in close contact
with His spirit for forty days and forty nights, needing
nought else but spiritual sustenance. The eye-witnesses of
all these wonders told of them to their children, and those
in their turn spoke of them to those who came after them,
and so on throughout the generations, even to this day,
when we stand together in this place. ``Ye are my
witnesses,'' said the Lord.

How is it that even those Jews who call themselves
Orthodox begin to hesitate in their convictions, and wonder
if they, even \textsl{they}, may be mistaken? They do not realise
that their doubts are inconsistent with their orthodoxy.
Perhaps they do not often think about religion at all.
They are Jews, and it is easier to call oneself Orthodox
than otherwise. The Orthodox umbrella covers people and
makes them comfortable. They need not think any more about
their Judaism. But, unfortunately, for themselves, they
are not protected against the disintegrating scepticism of
the age, if they only give lip service to their Orthodoxy.
Unless they have dared to use their minds in the service of
religion, their convictions cannot be real and lasting for
all time, as are those of the consistently Orthodox.

You and I who have imbibed the principles of Liberal
Judaism, and are determined to give them allegiance, cannot
place ourselves, even for our own comfort and protection,
under the Orthodox umbrella. We have lost faith in
miracles, The Bible, we think, reveals God, and is full of
His inspiration, but we cannot believe such parts to be
eternally true which clash with our sense of truth, and our
conception of righteousness, justice and love. We have
learned that the authority for our own belief lies in our
own conscience nourished by communion with God, and trained
by the scholars of the past. We have to examine our religion
for ourselves, anew, today, in this year 1939, in the
midst of human upheavals and miseries and desperate
conflicts and doubts, Will our religion survive? Can we
be mistaken?

Friends, we have each to do our searching for
ourselves. Our religion to be worth anything must be our
own, the result of our own searching and sifting and
striving and doubting and triumphant affirmation. I can
only with great deference, and in all humility, explain to
you something of my own in case it may in some measure be
acceptable to you.

I start out deeply influenced by the devotion of the
past. I have read my Bible, compiled by men and women who
called themselves, as I do, Jews, and have found in the
Bible some teaching which appears to me so perfectly good
that my mind cannot conceive of anything better. It seems
to me that this teaching is perfect, and I do it homage.

``Seek peace and pursue it.'' ``The just shall live by his
faith.'' ``Love thy neighbour as thyself.'' ``Love mercy.''
``Thou shalt not kill.'' ``Thou shalt not steal.'' ``Thou
shalt not commit adultery.'' ``Walk humbly with Thy God''
These are only a few of the sayings which, to my mind,
represent absolute truth,

I read my Bible history, and see the progress of a
small group of people. I see how they survived persecution
and suffering. I see a purpose running right through their
story. I see why they were preserved alive. I recognise
the treasure which they have been holding throughout the
ages, and how this treasure must serve as a blessed
influence in the world for all time. I see the succession
of men and women, and how with their aid more and more
spiritual truth has been found to serve the whole universe,
Then I realise that I am part of that succession of men and
women, and I know the purpose of my life here,

Now friends, \textsl{if} men and women had obeyed God's law,
not intermittently but continually, not in certain spots on
the globe, but throughout the whole world, would not our
present miseries have to a large extent disappeared? But
\textsl{why} should men have been so wicked for so long and so
frequently? Why should they have acted in defiance of love
and justice? Why were certain individuals allowed to set
aside the law of goodness and truth which should have
enthralled the world, and in their insolence and wickedness
have been allowed to dominate instead? Why were not their
evil machinations frustrated?

Did God reveal Himself, and then withdraw from our
world? I don't think so, but I think that God gave freedom
of choice to humanity, and He also gave them the assurance,
which is being verified every day that he would strengthen
the will of the righteous who seek advance through
righteousness. If we obey, we will be allowed to live in
peace and security. If we disobey, we lose quiet and peace
— see the world today! In quietness and confidence shall
be your strength. I dare not call this teaching out of
date. I see around me the chaos which results from
defiance. I feel sure it was God who set before humanity a
blessing and a curse, and said: Choose thou! How can I be
mistaken, when I see what disobedience brings?

My faith is supported by personal experience. I have
learned to pray, and when I pray I can make an effort to
adjust my life in harmony with the spirit of goodness. I
\textsl{can} feel the revelation of God, even as my fathers could.
I see the wonders of creation, and the uniformity of the
laws by which they hang together. I recognise the one mind
of the God Creator. I see the marvels of harmony in
natural colours and the harmony of sound in the songs of
the birds. Again, I bring homage to God the Creator. I
see goodness in my fellow men. I experience love. I
recognise discoveries based upon the law of truth. I see
works of art. I listen to sublime music - God, God everywhere!
I hear within myself the voice of conscience
bidding me seek to be good and just and merciful. I feel
the stings of remorse when I lack courage, when I fail in
obedience. My God remains within me when I would ignore
Him, I experience pain. But all this is so difficult.
the existence of the perfect God, revealed in the Bible,
and in the world, in the lives of others, and in my own
soul, makes my own imperfection so overwhelming. I find it
so hard to live up to my religion. Can it not be all a
mistake, seeing it is so often beyond our attainment?

But how about love? Do you doubt its existence
between parents and children, husband and wife, and
friends, because it is sometimes absent when it should
hever fail? How about light? Do you doubt the existence
of the sun because it is often hidden? Is there no truth
in the world because falsehood is so inclined to vaunt
herself? Have you never seen beauty? Ugliness exists, but
beauty must prevail.

No friends, I don't think we can be mistaken in the
absolute and eternal God idea. When we have failed and
uttered such incalculable absurdities, and acted with such
unfathomable stupidity, our eyes and hearts were not sufficiently
trained, We had not got far enough on the mountain
of life to get the right perspective. The light was still
shaded; our eyes were still weak, And even today, this is
our state - weak, imperfect, in definite and unquestionable
need of guidance. It is through our mistakes that we can
attain to God, But we must dare to struggle if we would
attain, We must use our mistakes if we would climb nearer
to truth, In God's light we see light, and away from God
there is darkness now and for evermore. God's light must
grow more and more until we can reach the Perfect Day.


\attrib{A sermon preached at the Liberal Jewish Synagogue, November 11, 1939}
