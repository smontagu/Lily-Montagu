\chapter{Kinship With God}

Many of you will remember a book published in 1907,
called \textsl{Father and Son}.\footnote{An autobiographical novel by Edmund Gosse} It is the record of a struggle between two temperaments,
two consciences and almost two
epochs. After years of pathetic endeavor to produce reconciliation,
the father charged the son with evading the
Inspiration of the Holy Scriptures and with explaining away
any particular Oracle of God, which pressed upon him. He
felt that the son was sailing down the rapid tide of time
towards Eternity, without a single authoritative guide
excepting what he might forge on his own anvil, excepting
what he might \textsl{guess} in fact. The son rebelled against this
description of his spiritual state. He refused to regard
this world any longer merely as the uncomfortable
antechamber to a Palace, which no one had explored. He was
told that he must cease to think for himself; defiance was
offered to the intelligence of a thoughtful and honest
youth, with the normal impulses of his twenty-one years,
and disruption followed. Religious independence had to be
emphasised, and two beings who loved one another intensely
had to separate outwardly even as, through the action of
the time force, they had for many years separated inwardly.

I doubt whether such cases of spiritual conflict in
their extreme form could be instanced in our Jewish
Brotherhood, even in the last century. The fathers in our
midst did not consider themselves responsible to God for
their sons' secret thoughts and most intimate convictions.
It sufficed that outward conformity was maintained, that no
schism was introduced into the Community as a whole. In
some instances, indeed, much agonising sorrow could have
been averted if the logical connection between religious
thought and religious observance had been more quickly
recognised. Obedience which, in the father's life, led to
holiness, seemed to be a good thing in itself, which should
be adopted by the child with loyalty and fidelity. It was
the discovery of a very obvious inconsistency between
thought and ceremonial, which gave impetus to the Liberal
movement in this country. The spirit of the age gave a new
importance to the value of truth -- sought after for its own
sake. Conformity could no longer be tolerated, if it
involved the sacrifice of truth. Our leaders guided us in
the work of adaptation. The relation between Jewish principles
and modern life was proved anew and congregational
life became possible.

Today the perspective has shifted slightly and we
stand, each one of us, more keenly conscious than ever
before of our own ``Soul hanging in immensity.'' Together
with the readiness, so nobly prevalent today, to sacrifice
the individual for the sake of a cause, there comes an
intense respect for personality. We \textsl{know} today, as we have
never known before, that God has put eternity into our
hearts; and we protest with all the sincerity which is
derived from intuition, unexplained and inexplicable, that
life is not cheap, but rather precious beyond measure.
Today we are far less concerned with the emancipation of
the brotherhood from the oppression of arid legalism, than
with the emancipation of the individual soul from the
torture of arid negation. As a Community our shackles have
fallen and we are free to enter into our religious inheritance.
Our form of service is in harmony with our sense of
fitness; we are no longer worried by the claims of tradition
when these clash with our conception of truth. We
have boldly enunciated our belief in progressive revelation,
and this faith has quickened our hope for the future
and intensified our reverence for the past. The congregation
of today is free, and if it is to seize the fruits of
its own emancipation it must gather together the thoughts
and aspiration, the character and will, even the doubts and
perplexities of each individual member.

It appears to me that the ideal of congregational life
can only be approached when each individual member becomes
conscious of his religious life, or, in other words, of his
``Kinship with God.'' Upon this consciousness depends the
realisation of our origin and destiny, our faculty for
prayer, our impulse to surrender ourselves to the highest
conception of truth and righteousness. ``Ye shall be holy,
for I the Lord Thy God am Holy.''\footnote{Leviticus 19:2.} In all the agony of
unrest which we are today experiencing, we cling with
intense longing to the idea of an Eternal God, with Whom we
can have communion, \textsl{because} we share His life and His
nature. Perhaps we have occasionally chafed at the formula
reiterated in our ancient forms of public prayer, the
formula: ``Our God and the God of our Fathers.'' Today we
understand better, and chafe less. \textsl{Our God Idea} varies
from generation to generation; we believe that it expands
and develops, but we need to feel sure today of the existence
of God and of our kinship with Him. We want to know
Him as the Unchanging Spirit of absolute Love and Truth and
Righteousness. The appeals of our fathers echo for us
across the ages; their sense of kinship with God encourages
us today. The unity of \textsl{longing} among God's children is one
of the evidences of His Fatherhood; our spiritual need,
shared, as it is, by the whole of humanity, can only be
explained by reference to our common origin. I suggest that
the sense of our kinship with God, based primarily on
intuition, is supported by the testimony of the past and
the character of our ancient worship. But our conception
is also assisted by the actual experiences of every day
life.

The prophet appeals to us to be holy, \textsl{because} God is
Holy. The appeal is based on the possibility of men to
imitate God. The pure soul shall see God. It is holiness,
it is purity, which creates the kinship. What then is the
divine holiness? Where is the point of contact between
human holiness and divine holiness?

I can only touch on the fringe of this vast problem.
The holiness of God implies an absolute standard of love,
truth and righteousness. May not man's holiness be
attained in the effort he makes to approach this standard,
disciplining himself to obedience, even at the cost of
material self-advancement and convenience? Jewish teaching
gives us a Holy God transcending, and immeasurably
excelling the human ideal; but it also suggests kinship
with that ideal. ``Be holy for the Lord your God is holy.''
The power of direct communion constitutes one of the
Glories of our faith; but without the sense of kinship the
language of prayer, whether articulate or unexpressed,
would fail us and, what is more important, there would be
little room for human aspiration. Indeed, much of the joy
and hope in human life lies in the ``infinite pain of finite
hearts which yearn.''\footnote{Robert Browning, \textsl{Two in the Campagna}.} This yearning is conditioned by faith
in the kinship of God. In one of his sermons, Professor
Jowett explains that God's holiness means the
Spirit which is altogether above the world and yet has an
affinity with goodness and truth in the world. It implies
separation as well as elevation, dignity as well as innocence.

Belief in our kinship with God explains some of the
aspects of human life which rouse our greatest veneration.
Indeed, it accounts for the unconquerable optimism which
should belong to every normal believer. No evil should be
tolerated as characteristic of human nature -- for kinship
with God implies the ultimate perfectibility of man. It
justifies the Jewish doctrine of the possible annihilation
of evil by the substitution of good.

The universality of God's love includes the sinner as
well as the Saint. Human love in its finest aspects
reflects the divine in its inclusiveness and power of
forgiveness. ``Be sure his Mother loved him,'' was said of
an evil doer, of whom nothing else that was good could have
been spoken. Today a little child asks as he goes to
rest, ``Has God been happy with me today?'' Generation after
generation have willed that their children shall be saved
for the highest. In our educational plans we affirm anew,
with every child that is born, our faith in his kinship
with God. It is this faith which impels us to grudge no
sacrifice to secure for the child the glories of his
inheritance. Again, it is belief in man's kinship with God
which rouses our indignation when we see innocent lives
crushed and thwarted by the degrading struggle against
unfair social conditions. Bad economic conditions may rob
man of his freedom, and freedom is part of human holiness,
which is akin to the divine. The thought of our kinship
with God should make us disregard all that is fleeting and
ephemeral in our appreciation of that which is eternal.
Our hope is in God, as we endeavour humbly to walk with
Him, doing justice and loving mercy when we catch a Ray of
His divine Light. It is upon His rod and His staff that we
rest, for we feel that we share His life as we pass through
the valley, which men call the valley of death.

So far we have dwelt on man’s claims to kinship with
God as a state which, if consciously experienced, brings
joy and peace and hope, and impels us to the highest
effort. But we must remember that God, according to the
teaching of Jewish prophets and psalmists, in His infinite
mercy reveals His kinship with His human children.
Throughout the Old Testament God the Ruler is also God the
Father. ``As a father pitieth his children, so does the
Lord pity them who fear Him.''\footnote{Psalm 103:13.} God's extreme tenderness is
further expressed: ``As one whom his Mother comforteth so
will I comfort thee.''\footnote{Isaiah 66:13.} In the psalmist's view, surely, were
there no kinship between the divine and the human spirit
there could be little sympathy. Would pity then be acceptable?
Would comfort then be possible? Still bolder,
perhaps, is Isaiah's conception: ``In all their affliction
He was afflicted and He bore them and carried them all the
days of old.''\footnote{Isaiah 63:9.}

This view of a suffering God brings Him into close
relationship with human life, and it is this sense of close
relationship which, I venture to think, we are needing
today. God in His freedom voluntarily suffers with His
suffering children. Our belief in the Supreme Spirit, who
rules by law, is by no means shaken. Through intuition,
through the sense of our own insufficiency, which causes us
to reach out to Perfection beyond ourselves, through the
actual experiences of every day life, we have learned that
it is possible to establish our sense of kinship with the
Holy Spirit, Who makes for righteousness. The God with
whom we claim kinship, in spite of our weakness and our
imperfection, is a Living God, and our kinship with Him
suggests the possibility of a complete life. Permanent
alienation is impossible because of that kinship: the
hardened sinner has within his soul the power of saintliness.

The realisation of our kinship with God banishes the
feeling of loneliness, which so often creates despair. The
world may look, as it does today, terrible and even
horrible. We may witness every hour the denial of God and
of Goodness; but within man himself, in the fact of his
kinship with God, is the promise of the ultimate triumph of
love. The Ideal of God's love is imitable, though unattainable,
on earth; the conviction that we are allied to it
impels us to labour and to hope. God works through
righteousness alone, and it is only by the establishment of
Good that we can cast out evil. ``Be ye holy, for I the
Lord Thy God am Holy.'' It is as if the Father gently and
tenderly bids us throw away all unworthiness and become
conscious of our alliance with Himself. He bids us renew
our faith in all that is good and pure, just and holy,
because our nature is capable of this faith. He bids us
love without stint, because we can draw love from the
sources of love. He bids us be truthful, even in our
inward parts, because no discipline is too hard for us
whose lives are linked with the divine, He gives us the
freedom to attain - even while He sets before us the Ideal
of Righteousness. ``Be ye holy, for I the Lord Thy God am
Holy.''

It has been well said that ``God does not inhabit one
world and man another, the \textsl{creation}, of which he is a part,
is the incarnation of the life of God. God's nature is not
radically distinct from man's nature. God’s life and his
are not mutually exclusive; if the man's life is part of
the life of creation and the life of creation is the incarnated
Life of God, there must at any rate be the
possibility of conscious relation between God and man. The
great central fact in human life, in your life and in mine,
is the coming into a conscious vital realisation of our
Oneness with the Infinite Life, and the opening of ourselves
fully to the divine inflow. In the degree that we open
ourselves to the divine inflow are we changed from mere men
into God-men.''\footnote{from Ralph Waldo Trine, \textsl{In Tune with the Infinite}.}

I would plead with you today that this glorious self-consciousness,
implying, as it does, a sense of kinship
with God, is derived from the actual experience of prayer,
and from the effort after righteousness; that it alone
explains the slow and often painful, but nevertheless
steady, upward trend of human life. As we become conscious
of our relationship with God, we become less and less
attracted by evil associations. Is not this that which is
meant by the command: ``Ye that love God, hate evil.''\footnote{Psalm 97:10.} The
self-conciousness which we have been describing stimulates
man to the highest effort after righteousness and the
strongest belief in the potentiality of that effort,
\textsl{inasmuch} as it can be linked with the divine. But we
remind ourselves that this sense of human dignity is dependent
on our faith in God as the Universal Father. Because
we have the power to attain holiness we are kin to the High
and Lofty One ``that inhabiteth eternity, whose name is
holy,''\footnote{Isaiah 57:15.} but this power is allied in us with the human power
to sin. Our kinship with God is inevitable, but we share
it with the blackest sinner in our midst. We cannot deny
the glory of this sinner's humanity without denying our
own, for the God who conditions that glory is the Universal
God, who dwells in \textsl{every} human soul to redeem and to save
it for good. The point of contact between man and God is
holiness - with the increase of human holiness does the
relationship between man and God become closer. Man is
united to man in his weakness as well as in his strength,
in his failure as well as in his moral excellence. Let him
deny both forms of kinship at his peril. The \textsl{consciousness}
of our kinship with God is no passive state. It comes with
our effort after righteousness. We have only to formulate
our ideal to realise our deplorable short-comings. Belief
in our kinship with God -- in our human possibilites -- fills
our hearts with a sense of utter unworthiness, even while
it stimulates us to a fuller hope.

We pray for the day on which ``the Lord shall be known
as one and His name one.''\footnote{Zechariah 14:9.} That day will become
perceptibly nearer when the unity of human life will be
recognised through the recognition of divine unity in the
individual soul. The call is clear to us today. Let us
hear it and obey; let us be conscious of our kinship with
God through our reverence for absolute truth, through our
humbly associating ourselves with Him in His work of
redemption and salvation, through our power of loving
without end, through our faith in the continuity of human
life, which is linked with the divine for the purpose of
creating righteousness.

\attrib{A sermon preached on 15th June, 1918 at the Liberal Jewish Synagogue}

