\chapter{Spiritual Possibilities of Judaism Today}

The history of a community, like the history of an 
individual, is marked by the recurrence of periods of
self-consciousness and self-analysis. At such times its members 
consider their aggregate achievements and failures, and 
mark the tendencies of their corporate life. Perhaps even, 
the sudden recognition of facts which have been unconsciously
suppressed may lead to regeneration. For many 
years self-consciousness has been growing among English 
Jews, and they have expressed, in whispers to one another, 
dissatisfaction with their spiritual state. It requires, 
however, some stirring accident like the Conference on 
Religious Education held in June last, and the East End 
meetings resulting from it, to cause an effective diagnosis to 
be made. Until Jews are honest enough to recognize that 
the majority of them are either devoted to ceremonialism at 
the expense of religion, or indifferent both to ceremonialism 
and to religion; until they have energy to examine their 
religious needs and courage to formulate them, they are 
courting comfort at the expense of truth, and they must fail 
to restore to Judaism its life and the endless possibilities 
inherent in life. 

It is not enough for us to give a frightened glance of 
recognition at our materialism and spiritual lethargy, and 
then seek to draw the veil in all speed, hoping impotently 
that grim facts will grow less grim if left alone. We have 
ultimately to confess that facts cannot be thus set aside by 
mere desire. Moreover, these facts prove on examination 
to be stimulating rather than terrifying, fraught with hope 
rather than with negation. If I appear dogmatic in my 
efforts to prove my contention that Judaism has been 
allowed by the timid and the indifferent to lose much 
of its inspiring force, I can only plead in excuse the 
sincerity of my convictions. 

I take as the objects of my criticism the two most 
comprehensive types of English Jews, and for purposes of 
convenience call them ``East End Jews'' and ``West End 
Jews'' respectively. It must, however, be clearly understood 
that these two forms of religion do not prevail exclusively
in any particular district of London. Representatives 
of both classes may sometimes live in the same house, 
and may, conceivably, belong to the same family. Again, 
a considerable district in West Central London is largely 
inhabited by ``East End Jews,'' and many ``West End 
Jews,'' with their vague ineffective aspirations, crowd the 
neighbourhood of Bishopsgate. But these epithets are 
intended to convey the idea of two sets of people, differing 
less in dogmatic belief than in the tone and temper of 
their minds, and especially in their view of the proper 
relations between religion and life. It will also be shown 
that, although these two classes are to-day quite unsympathetic
to one another there are many signs of a better 
mutual understanding among members of the younger 
generation; and it is chiefly upon this reunion that I 
base my own belief in the possibilities of the reanimation 
of Judaism as a religious force. In my endeavour to 
rouse the lethargic I may perhaps have dwelt more fully 
on the defects of my two ``types'' than on their qualities, 
but I have little doubt that there will be many both able 
and willing to make the balance more even. 

The ``East End Jew,'' whose religion is vigorous in spite 
of its deformities, has no confidence in the shadowy faith of 
the ``West End Jew,'' and refuses to be taught by ``West 
End'' methods. Examining this distrust, I find that it 
arises from the recognition of the dissimilarity in the two 
religions. The ``East End Jew'' is determined to follow 
the worship of his fathers, and spurns the flaccid religion 
of his ``West End'' brothers. To the pious ``East End 
Jew'' religion is obedience glorified into a cult; for him, 
God exists as a just Law-giver, ready to forgive and help 
those who obey the Law, delivered by him to his people 
through his servant Moses, and having misfortune and 
failure in reserve for the rebellious and indifferent. He is 
continuously conscious of the ``God without,'' whom he 
seeks to approach at prescribed times and seasons. Every 
act of obedience tends to increase the sum of his righteousness; no evil can touch him while pursuing the divine 
mandate. He does not consciously strive to realize the 
``God within,'' and to develop it by communion with the 
divine Ideal of Truth and Love existing without, for the 
idea of an immanent divine presence does not seem to 
affect his creed. When he repeats the prayers ordered by 
his fathers, he is less stirred by the effort of the soul to 
hold communion with the Infinite than by a sense of 
righteousness resulting from unquestioning obedience. The 
glow which this obedience produces, suffuses his daily life, 
and encourages him to persevere in his rigid observances, 
and to face all earthly difficulties with courage and hope. 
Can we wonder that the ``East End Jew'' regards with 
half-scornful fear the man who, while still calling himself 
``Jew,'' ventures to neglect the ordinances prescribed of old, 
and makes no apparent sacrifice in the cause of his faith? 
For him, prosperity seems to authorize self-indulgence and 
laxity of conduct. 

Admitting the possibility that prejudice and ignorance 
render the ``East End'' observer unappreciative in his 
criticism, can we substantiate for the ``West End Jew'' 
any claim to a deeply religious life? Can we deny that 
in many ``West End'' homes, callousness takes the place 
which religion should occupy? Having been born Jews, 
and believing it more respectable to be identified with 
some religion, the members of the class under consideration 
generally belong to some synagogue, and perhaps attend the 
services more or less regularly. But their religion is seldom 
interesting, never absorbing to them. They are far more 
concerned in the length of the service than in its adequacy 
to satisfy their spiritual needs. They make no demand on 
their Judaism; it has no real influence over them. They 
either sink into materialism or create a religion of their 
own, based on a vague belief in the existence of a higher 
law, and nourished by an exacting moral sense which 
requires self-restraint and self-development. This religion, 
without an historical past and admitting of no outward 
embodiment, is helpful only to those individuals whose 
moral strength is great enough to call it into existence. 
The vast majority of men and women need a more definite 
cult to draw them to their God. The strict system of 
religious discipline adopted in the ``East End'' has much 
definite and salutary influence. It awakens veneration 
and instigates self-sacrifice; it leads to morality, sobriety 
and strength of purpose. It encourages kindly intercourse 
between men, inducing often heroic acts of charity. I only 
venture to criticize it, because I see it worshipped as God 
alone should be worshipped; because true communion with 
God is being shut off from man by the observances which 
were intended to lead man to God. Even the recent 
remarkable gatherings of working Jews, bent on Sabbath 
observance, do not allay our apprehension for the future 
of Judaism. While admiring the earnestness which inspired 
these meetings, it is to be feared that the Sabbath, instead 
of being desired as a day for the renewal of spiritual life, or 
as a stimulus to moral progress, is now required for mere 
physical rest and idleness, and for the complete equipment 
of the ritual-god, which has been fashioned so curiously 
and is so generally worshipped. Indeed I tremble for 
the future of Judaism, as I recall the words that Isaiah 
addresses, in the name of the Lord, to the idolaters of all 
ages: ``Your new moons and your appointed feasts my 
soul hateth : they are a trouble unto me; I am weary to 
bear them'' (Isa. i. 14). 

Between the worship of ritual prevalent in the ``East 
End,'' inspiring by its intense fervour, but repelling 
through the materialism and intolerance which it produces, 
and the vague religion of the ``West End,'' existing apparently but to satisfy a convention, there seems indeed little 
affinity. Yet the children of both types of Jews are united 
by a common need which neither form of religion is able to 
satisfy. For the sons of the pious ``East End Jew'' are also 
beginning to question the meaning and value of the laws 
which bind their fathers' lives so closely. We see them 
shocked by their inconsistency, and disappointed by their 
inadequacy; we see them drifting away from the worship 
which, at least in its origin, was inspiring, and, for want of 
some better object, devoting themselves to ``self.'' The 
daughters of the pious do not even attend the synagogue 
services, which have begun to weary their brothers. Through 
force of habit they cling to the domestic side of religion, but 
they do not attempt to ennoble the sordid elements in their 
lives by trying to introduce the ideal. 

Similarly, when the children of the indifferent ``West 
End Jews'' have passed the period when the example 
of parents is followed without question, when they begin 
to think for themselves, they realize that they have no 
religion. The majority accept, after a period of uncertainty, the conventional pretences of their parents, and 
adhere to them until they become inconvenient, when they 
cast tbem off altogether. Many of the young men and 
women have periods of intense craving for some definite 
faith, and would even return to the Ghetto-worship if 
their minds could admit its principles. Perhaps they have 
glimpses, which fill them with extreme joy and hope, of 
a revived and ennobling Judaism, which might become the 
guiding inspiration of their lives. A few retain these 
visions, and are continually cheered by them; a very few 
seek to realize them more closely; the many prefer to 
banish disquieting dreams; and in choosing for the 
hour peace of mind, cut off for ever real happiness — 
spiritual joy, the best of the gifts which God offers to his 
children. 

In what way does the community attempt to meet the 
needs of its younger members, in whom the hope of 
Judaism in the future rests, and to stay the current of 
indifference which, both in the ``East End'' and in the 
``West End,'' is threatening its foundations? Among some 
of the better educated parents who are conscious of their responsibilities, there is noticeable an ominous bewilderment, 
when they consider what form of religious instruction they 
are to give to their children. If they send them to religious 
classes at the synagogues, or if they arrange for masters to 
give private instruction, the success of their efforts depends 
generally on the personal force and influence of the teachers. 
The children are further required to attend the synagogue 
services, but these have no hold over their growing life. 
As a rule, they are inattentive; and if they pray at 
all, it is that the prayers may speedily end. The 
occasional introduction of a children's sermon does not 
for them materially relieve the tedium of the service. 
A preacher who speaks to a mixed congregation of adults 
and children is generally self-conscious, and his words are 
often addressed to unresponsive minds. The children are 
humiliated by the seeming publicity of their faults, and 
irritated by the silent or whispered delight of their elders, 
when some pulpit rebuke is especially applicable to them. 

Yet children are naturally religious. 

On the other hand, the ``East End'' parents are not 
satisfied that the instruction given by the Religious Education Board and the voluntary Sabbath-class teachers is 
adequate to satisfy the spiritual needs of their children, 
and at the educational conference the fact was generally 
deplored that these children were still further instructed 
out of school hours. It is still more deplorable that these 
very children, who have been subjected to this elaborate 
religious training, who have attended school and cheder, 
are found, when they have ceased to be students, and even 
during the period of their preparation, to be conscious 
of no anxiety to pray, of no sense that religion renders 
truthfulness and self-sacrifice obligatory. With some 
signal exceptions, the voluntary Sabbath-class teachers 
consider that they satisfy the claims of religion by insisting 
on extreme decorum in their class, and instructing it in the 
recital of prayers which are not felt or understood. The 
children are not awed by the grandeur of God, nor drawn 
to him by his love. They merely repeat a weekly lesson, 
which has become easy through iteration, and which can 
awaken no spiritual joy. 

The members of the Religious Education Board are, 
doubtless, inspired by the highest motives. They would 
like to adopt the ``East End'' religion, and to teach it 
in an intelligent and enlightened manner, showing that 
the greater part of the observances are estimable only as 
methods, never as objects of worship. But the members 
of the Board belong to the leisured classes, and seem 
unable to carry out their scheme by personal effort; their 
own religion is not the ``East End'' religion, and they 
cannot impart a fervour for what they do not feel. They 
depute the work to others, who conscientiously teach religion as geography and history are taught. Consequently, 
although their pupils come to answer intelligently, the 
lessons they learn are not assimilated in a manner to 
influence their everyday life. Religion remains a subject 
to be noticed at certain specified times and seasons, but 
has no intimate connexion with life's joys and cares. 

The conditions of modern Judaism, then, from every point 
of view, present a grievous aspect to honest observers. In 
vain we seek to gloss over facts; in vain we point triumphantly to our charity-lists, to our learning, to our position 
in the front of every rank and profession. We yet have 
to confess ourselves unable to impart to our children a 
strengthening faith. Are we not also becoming every 
year more self-indulgent, more ostentatious, less reverent? 
Why is there a growing tendency among all classes of 
Jewish youth to forget the serious purposes of life, and 
to set the pleasures of gain, of dress, of food, of dancing, 
and of acting, above all else that is desirable on earth? 
Why do we gamble so much? Why do we grudge personal 
service in combating the moral evils of our day? Why is 
complete personal sacrifice to the needs of our poor so rare 
among us? Why do our philanthropists, even our ministers, 
forbear to introduce religion into their visiting work, 
unless to or about those who are about to leave this world? 
Why cannot we suppress the lying and deceit which flourish 
in our midst? Why do our friends and relatives marry out 
of the faith, passing among the Gentiles as freethinkers, 
upon whom religion has no claim? Why are the old laws, 
which kept the minds of our fathers in pious subservience, 
still preserved, seeing that here and there they require a 
sermon to justify their existence, and a sacrifice of truth 
to facilitate their observance? 

The answer to these questions must be, that the highest 
Jewish influences are for the time being dormant, and have 
ceased to inspire our lives; that our belief in a supernatural law is only a verbal one, and that in spite of our 
professions we are stirred by no desire to prepare ourselves 
for a better spiritual state. Indeed, without some strong 
spiritual awakening, how can we hope to arrest our 
degenerate tendencies? 

Yet, in spite of all these depressing facts, in spite of 
our present callousness and inertia, there is every reason 
for hope — for hope, glorious and infinite. If we examine 
our Judaism with a trusting spirit, we find that it still contains the germs of life; we find that its abiding essence is 
simplicity and truth. At present our thinkers are oppressed by the religious lethargy from which our age is just 
emerging. Only now and again a true believer appears 
in our midst, one who clings to his religion, and derives 
from it spiritual joy and a stimulus to moral progress; who 
sacrifices his own pleasure constantly in order to serve his 
fellows; who draws inspiration at all times from God and 
from his creations, because his Judaism impels it. The 
problem before us is how to restore confidence to our 
thinkers, and to encourage them to free our religion from 
the earth which is clogging it, and to allow it to spread 
and to stimulate the lives of all generations. There is only 
one method by which we can hope to achieve these ends. 
\textsl{That method is association}.

When about the year 1840 the Italians became conscious of their state of subjection and determined to 
issue from it, Giuseppe Mazzini appealed to them to 
associate together in the service of God and of their 
country. He saw that only after self-regeneration could 
his countrymen hope to frustrate tyranny, and that only 
by association could they obtain the needful strength 
to execute the tasks before them. Whether it is left for 
generations yet unborn to inherit the glorious future which 
Mazzini predicted for his country, or whether the associated 
bands of young Italy are gone for ever and have left no 
trace, I still believe that the great Italian's teachings are 
fundamentally true. We Jews are suffering at home from 
the tyranny of spiritual sloth, and abroad from the tyranny 
of persecution. If we are to be free, Mazzini's powerful 
appeal for association should echo and re-echo in our midst. 
For his words may be applied to communities as well as to 
nations, to religious brotherhoods as well as to political 
states. He says, ``Association is a security for progress. 
The State represents a certain sum or mass of \textsl{principles} in 
which the universality of the citizens are agreed at the 
time of its foundation. Suppose that a new and true 
principle, a new and rational development of the truths 
that have given vitality to the State, should be discovered 
by a few among its citizens. How should they diffuse the 
knowledge of the principle except by association? ...
Inertia and a disposition to rest satisfied with the order 
of things long existing, and sanctioned by the common 
consent, are habits too powerful over the minds of most 
men to allow a single individual to overcome them by 
his solitary word. The association of a daily increasing 
minority can do this. Association is the method of the 
future'' (Mazzini, \textsl{Duties of Man}). 

It is only by association that we can effectually enunciate 
the principle, that we are required to use in God's service all 
the gifts of mind and heart which he has granted to us, since it 
is a form of blasphemy to conceal or to pervert truth, in order 
to render our service of God acceptable to him. We, who 
are conscious of our great needs, must organize ourselves into 
an association to rediscover our Judaism, encouraging one 
another to reformulate our ideal. We shall be able to rally 
round us the discontented and weary, and together we may 
hope to lift Judaism from its desolate position and absorb 
it into our lives. Together we must sift with all reverence 
the pure from the impure in the laws which our ancestors 
formulated in order to satisfy the needs of their age, and 
refuse to resort to hair-splitting argument in order to re-establish
a religion which was originally founded on a basis 
of truth, dignity and beauty. We must no longer grimly 
reiterate the fact that Judaism has ceased to appeal to us, 
and lack the energy to inquire into the cause of its degeneration. We must boldly follow Isaiah, Jeremiah and 
Ezekiel, and allow a place to progress in religious thought. 
Yet, at the outset of our search, we shall be persuaded that 
only the elect among us can worship at the ``Fount of 
Inspiration'' without some assistance in the form of a ritualistic system, and that the perpetuation of Judaism therefore 
requires the maintenance of certain ceremonial observances. 
For the essence of a religion cannot be transmitted in all 
its simplicity to a child, whose mind cannot conceive an 
abstraction, and a certain discipline of observance is 
essential to character-training. We can only combat 
our tendency to self-indulgence and to spiritual sloth by 
having fasts and holydays reserved for communion with 
God. Inspired by a natural desire to examine with all 
tenderness the possessions which our fathers preserved 
with so much courage and devotion, we shall probably find 
treasures of beauty and truth where we had expected deformity and deception. We shall then be able to assign to 
observances, which had been worshipped as the end, their 
proper place and function as means for the attainment of 
holiness. 

Judaism once rediscovered, and our faith in its utility 
revived, we shall be able to undertake with better heart 
the instruction of our children. One of our first duties 
as an organized association must be to arrange children's 
services throughout the kingdom. We have to teach our 
children first to ``seek the Lord, while he may be found, 
to call upon him while he is near''; and, secondly, that 
``our thoughts are not his thoughts, nor our ways his 
ways'' (Isa. lv. 6, 8). We must make them realize that 
God is Love, and that human love which bridges life and 
death, is only a reflection of divine love, which reaches 
from heaven to earth. The lesson of God's omnipresence 
may be best enforced by a constant variety of service, and 
by the introduction of passing events and the incidents of 
daily life as themes for prayer. It might certainly be 
urged that a constantly varying service during childhood 
would render any fixed ritual irksome in after life. But 
with the growth of judgment the necessity for some uniformity in worship will be felt. Our children have to 
learn that prayer involves \textsl{effort}. If they could see their 
leader moved by spiritual need, struggling to approach his 
God, they would unconsciously join in the search, and 
experience veneration in the presence of God. From the 
beginning, the value of prayer in combating vicious 
pleasures and the neglect of truth must be enforced. The 
children must learn that the active, conscious search after 
God cannot be confined to morning and evening prayer, 
nor begun and ended on Sabbaths and festivals. The 
believing Jew and Jewess must seek guidance from God in 
the morning, be conscious of his presence throughout the 
day, and pray for a renewed inspiration at night. Then 
Judaism will have gained through fervent prayer far more 
than it can have lost through less regard for form; and its 
professing followers will look to it once more to satisfy 
some definite need in their lives. 

As an association, we must prove the utility of our 
religion by showing that it admits of endless development. 
We must prove that we are not a destructive body, and 
that we did not chafe because we required more ease; for 
while waging a crusade against deceit and impurity, we 
are only seeking to restore to Judaism its power over 
our lives. We must avoid all boasting and ostentation; 
even as our aim is high, so should our self-distrust be 
great. It is obvious how inadequate is our strength to 
achieve even a small part of the purpose we have in view, 
seeing that our generation, however united and zealous it 
may become, can only indicate the road which posterity 
may think it right to follow. 

I have suggested the organization of an associated band 
of worshippers, bound together by the tenets of a living 
Judaism. It is possible to attempt a slight forecast of the 
lines on which such an association may work and its more 
immediate results. We may hope for the gradual abandonment of gambling and other vicious pleasures, the desire for 
a more simple life animated by love of truth and of piety, 
and an increase in the number and devotion of those who 
are ready to devote themselves to preventive rather than 
curative social work, and who would attempt relief by 
moral stimuli, as well as by material props. The present 
is the right hour, and England is the fit place for the 
initiation of this movement, which may restore to Judaism 
its glory. 

In England Jews can freely develop all their powers, 
and follow, unquestioned, their ideals. If, then, the 
English Jews are better able than most of their continental 
brothers to recognize the potentiality of their spiritual 
inheritance, the obligation rests imperatively upon them to 
formulate its meaning and render it intelligible. By continuing to follow mechanically a religion which they have 
not the energy to revive, by maintaining tenets which 
jar on their sense of truth, they are neglecting their most 
urgent duties, and rendering themselves for ever unfit to 
serve their brothers. For we English Jews owe a duty 
to our less fortunate co-religionists, who are still suffering 
from the effects of persecution. In some countries we find 
Jews who have been denied the advantages of education, 
for whom persecution has tightened the spiritual bonds by 
causing them to build up a wall of observance effectually 
shutting out God's light. They are dimly conscious of a 
glorious inheritance transmitted to them by their fathers, 
and threatened by cruel and impious strangers. Too 
fearful to examine the nature of this inheritance, and to 
discover that its qualities defy the art of thieves, they 
fence it with rows of bulwarks constructed with pious 
ingenuity. Holy is the aim of the persecuted; there 
is no schism or rebellion in their midst; they do not understand that their service has been gradually diverted from 
the Giver of all good to the ritual gods, who were originally 
raised on high for the purpose of his defence. But when 
persecution ceases, and men are freed from its effects, they 
will examine the nature of the ritual gods they have served 
so conscientiously. A revulsion of feeling, a horror at 
their long idolatry may follow; tradition may lose its 
purifying hold over their minds, and they may yield themselves up to licence, and call it intellectual emancipation. 
It is possible for us in England to avert this catastrophe. 
At the moment when our persecuted brethren are in their 
greatest need, when they realize the hollowness of their 
worship, they may be saved from spiritual anarchy if they 
see among us a religion comprising all that was valuable 
and lovely in the ancient faith, embodied in forms acceptable to emancipated minds. 

In other countries we see Jews who, having once known 
intellectual freedom, are now denied the privilege of 
developing all their intellectual, social and material possessions, unless they submit to conditions which will 
rob them of their Judaism. For a long time they may 
refuse to accept these conditions, although, like their 
less enlightened brothers, they have probably not examined the purport of their religious inheritance. They 
are at present restrained from doing so by vanity rather 
than by piety, for they vaguely believe that their ancient 
faith, the source of so much of their fathers' glory, will not 
survive a severe scrutiny. Their attempts to suppress 
intelligence result in a lifeless form of worship, and, in all 
probability, the scruples which made the preservation of a 
religious system at all possible will gradually melt away 
before the claims of self-advancement. The religion which 
has long ceased to inspire love will be at last critically 
examined, and contempt for it will fill the minds of its 
former devotees; its merits over other religions will 
appear doubtful, and men will resolve no longer to cramp 
their own and their children's lives in its cause. Consideration for the children's happiness will probably be 
most potent in inducing the change to be made, and even if 
the state religion is not formally adopted, the old religion 
will no longer block the road to success. 

These gloomy prophecies are warranted by many precedents in the
history of continental Judaism; but I 
believe that, if in England we associate to maintain Jewish 
ideals, we shall be able to show by the gladness and the 
holiness of our lives that Judaism is worth \textsl{any} sacrifice. 
Then the persecuted will renew their courage, and be saved 
from deserting the religion whose value is proved before 
their eyes. 

Surely we English Jews can have no excuse for continued
indifference and waiting. To us the call is clear 
and unmistakable. For our own sakes we must revive 
Judaism, and having reconciled its dogma with our highest 
conception of truth and beauty, allow it again to bind us 
to the God who cares for us. In order to answer the 
challenge of the ``East End Jew,'' we must prove that our 
faith is no longer comatose, that we are truly striving 
after an ideal, and that we are ready to make any sacrifice 
that our religion may claim. For the sake of our foreign 
brothers, whose eyes are blinded by present misery from 
seeing the light which is within their reach, we English 
Jews must unite to strengthen our faith and proclaim the 
infinite hope contained within it. 

There is everything to fear for the future of Judaism, 
until it can be accepted by the most enlightened among 
us. Better to have died in the Ghetto than to have outlived the possibilities of our religion. But surely there 
is no need for despair seeing that a broader and more 
beautiful worship, which will grow in intensity, as the 
needs of a more developed civilization become greater, can 
even now be dimly foreshadowed. 

Some critics are fond of noticing the popularity of Ghetto, 
or hard-shelled, Jews among the Gentiles, and comparing 
it with the odium suffered by the unobservant Jews. We 
cannot hide from ourselves one reason at any rate for such 
a preference. The Ghetto Jews need not be feared as rivals, 
since their development is checked by laws of their own 
making, while the emancipated Jews are without binding 
laws, and therefore uncramped in their competition with 
their neighbours. The racial Jew, devoted to self-seeking 
and ostentation, and arrogant of his race, although destitute of spiritual faith, is indeed deserving of every scorn. 
His Judaism is not of his own seeking, and he consequently 
makes no sacrifice to follow it; he cherishes a materialistic 
ideal, which threatens the highest good of our age. It is 
the Jew who is a racial Jew only who must be helped 
to religious Judaism once more by being induced to join 
an association intent on proving the value of the religion 
for which his fathers lived and died. And if such a band, 
doing such glorious work, should reawaken intolerance 
among our neighbours, we are prepared to welcome martyrdom and to call it a joyous deliverance, seeing that 
it will have freed us from the lethargy which is at present 
oppressing our spirits. If, as is far more probable, we are 
able by a strongly organized religious movement to arrest 
our own spiritual degeneration and to revive our faith, that 
mission of the Lord's Servant unto the nations, which was 
the highest aspiration of the Second Isaiah, may even yet 
be turned from a vision into reality. 

But before we Jews can claim to be a religious brotherhood, before we can pretend to possess a faith through 
which we can speak tidings of salvation and peace to all 
nations, we must be able to rest our title on our own 
efforts rather than on the accident of our birth. Whatever 
the creed of his father, whether Roman Catholic, Protestant 
or Jewish, a religious man must seek and discover God for 
himself. I believe that in Judaism will be found the 
methods by which God can be most surely approached, and 
that these methods are certain ultimately to prevail universally. But no fresh discovery can be made exactly on 
the lines of the past; the temperament of one generation 
differs from that of another, and life is only possible when 
it can adapt itself to environment. Let us dare to speak 
with courage to our brothers and sisters, and to our sons 
and daughters; let us bid them not hesitate in their search 
after the divine, because they use data and methods not 
already tried by their ancestors. Judaism is strong enough 
and wide enough to inspire them and their children for 
ever; let us ask them to make progressive demands upon 
it. Let us tell them indeed that they can \textsl{only} be Jews 
and Jewesses if they \textsl{do} live up to the ideals of truth and 
morality expounded by the best teachers of their age. 

\attrib{From ``The Jewish Quarterly Review'', Vol 11, No. 2 (January 1899) pp. 216--231}
