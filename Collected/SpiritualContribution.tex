\chapter[The Spiritual Contribution of Women as Women]{The Spiritual Contribution\\of Women as Women}

I am attempting to discuss a subject with you today
which is fraught with particular difficulty. To begin
with, it is very likely that you as feminists will disagree
with my premises. My central thought is based on the idea
that although women should cooperate with men in all the
activities open to human beings, and that they should
recognise no limit except that of incapacity, and that sex
cannot disqualify anybody from doing that for which they
feel themselves fitted, women have certain qualifications
which are different from those possessed by men. Indeed, I
am a feminist too and believe in complete equality between
men and women in the social, political, economic and religious
spheres, but I think that humanity is enriched by the
diversity between the two sexes. It follows that they must
develop their special qualifications, and they must give as
complete a contribution to the world's spiritual treasury
as possible, but that it must have a character of its own,
and not be an imitation or a replica of the contribution
made by men.

Men and women have the creative faculty jointly, but
men can work more objectively than women. Let me take a
simple illustration. Father and mother want their small
son or daughter aged 4 to learn to pray. They want to
stimulate the capacity for prayer. Father takes his child
to Synagogue while no service is in progress and carries
him round and shows him the various features, including his
special seat, and the Ark, and explains that the books of
the Bible are contained in the Scroll which is in the Ark.
He is informative and interests his child. The Sabbath
comes and the boy or girl is dressed in his best clothes
and accompanies his Daddy to the Synagogue and carries his
prayer book. The child sits between him and Mummy. Daddy
occasionally shows him the place in the book, and he stops
fidgeting for a moment. He is happy and prepared to repeat
the experiment on subsequent Sabbaths. Gradually, after a
long time, the atmosphere of the Synagogue impresses itself
on the child and he feels the inclination to worship. The
realisation of that possibility will depend in a great
measure on the father's own reaction to the service and
what he says about it when he gets home.

The mother has another method. At bedtime she can do
many things. The element of thanksgiving is a good
preparation for prayer. She remembers a beautiful fungus
which she and Johnny admired together while on their walk.
The imaginative faculty is strong in her. ``Johnny'' she
says, ``do you remember our walk today and that fungus we
saw, and the lovely streaks of colour, the red and yellow
bits, and the little bits of green and brown, and how we
wished we could have found a mushroom, but then we said it
would not have been half as pretty? Shall we thank God
for making that fungus?'' ``Yes, let's, Mummy.'' ``Thank you
God for making that lovely fungus, with the red and yellow
bits, and the bits of brown and green'' added Johnny, ``and
next time please make it into a mushroom so that we can eat
it.'' ``Amen'' says Mummy. Another night Mummy and Johnny
make a list of the people they both love and ask God to
bless them. They make their prayers together and they are
their own special prayers.

A woman creator gives her own spiritual experience of
pain and joy to the progressive conceptions of Judaism. A
man analyses and sifts and reasons while with the woman he
climbs the mountain of God. I have heard it said how a man
climbs step by step until he reaches the level within his
reach. His path has been sure but rather slow. He looks
round and sees a woman by his side. He did not see her
while he was climbing because she sprang from ledge to
ledge taking many risks.

In religious discussions on the source of authority in
Progressive Judaism, you will find men more interested in
external authority than women. A man says: ``We shall have
chaos unless the men of scholarship and experience get
together and decide on specific observances. Does Sunday
observance lead to disloyalty? Should we have more Hebrew
in our liturgy? Is the Cantor's assistance essential or
even desirable?'' ``Well, I don't see that it matters what
the big people think'' says the woman. ``I know I cannot get
my young people to service on any day but Sunday. Hebrew
may be all right for some people, but it is no use for
those who do not understand it. If I want the best
singing, I go to the Opera. When I am at a service, I want
to sing and join in, and I know my children do. If they
can sing, all the better, and if they can't, let the others
sing louder and drown my children's voices, but they too
are singing.'' There is an element of practicality in all
this, perhaps a little less feeling of responsibility, a
longing that her own young people should be satisfied.

\tolerance 512
We women must approach the question of international
peace from, I think, a rather different angle from that
adopted by men. In the first place, I think we must lift
the problem out of the sphere of politics into the sphere
of religion, or, as I would greatly prefer, bring a strong
religious influence to bear upon the political issue. Men
and women of course know equally well the misery and
futility of war, the demoralisation which is part of its
aftermath. But women surely realise more fully the affect
of war on home life. The responsibility seems to me to
rest on them to overcome the sense of defeatism and
frustration which is surging over the world. Men are in
their outlook more realistic than women. Their vision is
blocked by the existence of the atom bomb and the threat of
totalitarianism. They have given so much to the cause of
liberty, and they see all forms of tyranny flourishing and
becoming ever more threatening. They are weary but
resolved not to show any weakness. So they turn to war
preparations as the only way to secure peace. Here is the
woman's part. We have to affirm with all the strength at
our disposal that because God is, the reign of peace and
righteousness must triumph in the world. If we firmly
believe in the wickedness of war, we must turn away from it
and find other ways of settling our differences, however
difficult the search may be. If we fail now, our civilisation
perishes and with it all that is precious in home
life. It is only on the plane of religion that we can find
the way to restore our faith in man and in ourselves. We
can help because we are outside the actual fighting arena.
Even though we are in the war unit, our methods are not
confined to physical force.


Before I close, I would ask you to consider whether
you think you can resist the present drift away from the
consecration of home life. Men may make a greater effort
even than women to keep up appearances, but you know as
well as I do that a rottenness has set in, and the moment
has come for you to arise and shine forth. On the stage,
in novels and in the newspapers, the idea of the divided
home is accepted as inevitable. The children are being
sacrificed to the general feeling of inevitability. The
natural happy and chaste home life is regarded as
exceptional and unexpected. The marriage vow has become
loosened. The marriage ceremony is only religious in form,
its significance is forgotten. Men think you feel that
this phase is inevitable, that the new code of morality or
immorality is likely to prevail; it is in the trend of the
times. Business is absorbing. The survival of the fittest
is the law of the day, and no other consideration can count
while the struggle for existence is so fierce. Here again
we must make our faith felt and before we can do this, we
must revere it, each for herself. If home and the children
are our most precious possessions, if the exaltation of
their worth is anything more than mere phrases, we must
hurry to the work of salvation.

Yes, friends, in the field of religious education, at
home, in the work for peace and chastity, we must work with
new zeal and new faith. Our contribution is needed. Our
courage must be added to that of our men. We must restore
to them some of their lost confidence and hope. We allowed
them to lose much through our apathy and inertia. Now we
must know that we stand before God and must either perish
or be prepared to obey His word. Each must say for
herself: ``Here am I, send me.''\footnote{Isaiah 6:9}

\attrib{Chicago: Jewish Education Building, Friday, November 26th, 1948}
