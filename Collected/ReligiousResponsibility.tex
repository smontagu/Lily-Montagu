\chapter{Religious Responsibility in Public Life}

I am grateful for the privilege of addressing you, my
sisters of the Jewish and Christian faiths. My subject is
women's religious responsibility in public life, and I am
speaking as a member of the Liberal Jewish community. I do
not represent the Orthodox section of our community
although we hold the Orthodox in the qreatest respect as it
is the progressives whose system of religion fits in, to
quote our leader Dr. Mattuck, with the thoughts and ideals
of Western culture.

We believe in certain eternal truths which we seek to
apply to the life of the state. God, through His prophets
and teachers, has revealed to us an absolute standard of
righteousness. Through our belief in the Oneness of God,
we conceive the idea of the Brotherhood of man. All men
and women have the right then to live, and we must, therefore,
see to it that they are supplied with the opportunity
to obtain the essentials of Life. No man or woman, as a
child of God, can be denied the right to a home with air
and light; as also the possibility of developing his
physical and intellectual and spiritual life. In the light
of our faith in One God, body, mind and soul must be
reverenced as holy.

There was a time when people no less good than
ourselves deemed it necessary to let little children work
in mines and factories in order that industrial profits
could be increased. God knows what sins we are committing
today for the sake of what we believe to be equitable gain
while a Shaftesbury is preparing to show us our crimes and
bid us change our ways. We shall be judged by future
generations even as we judge our predecessors.

Today, since we are aware that children are made in
the Image of God, it is necessary for us to reverence them
as personalities. This recognition must impel us to try as
soon as we possibly can to disallow overcrowded classes in
schools, so that each child's possibilities and Limitations
should be known. Moreover, in order that our children
should realise themselves, they must have occasional opportunity
to see large stretches of country and sea; that sea
which was once described by a child as the one thing in the
world of which there seemed to be plenty for everybody, and
as she said it, she laughed aloud and rejoiced exceedingly.

It is because we reverence human personality that we
also stress the necessity for self-realisation among
adults. We know that this can only be secured if a certain
amount of leisure is possible for rest, recreation, for
study and thought. Woman assists in the function of creation,
and it is therefore easy for her to recognise the
responsibility of maintaining the value of human faculties.
She must throw herself into the arena in which evil is
combated until it is overcome. She can no longer find
protection by insisting on retirement while her men folk
fight in the cause of righteousness. Woman's emancipation,
if it means anything, must mean that she is called to
cooperate with man in every sphere of life. Because, as I
think, her endowments though equal to, are different from
those of men, her distinctive gifts are needed for human
advancement. She must fulfil her responsibility as a
citizen and elector, and since she believes in the
omnipresence of God, she cannot allow any activity to appear to
be outside the sphere of His influence. If there is
corruption in the political or civic life with which she is
familiar, she is responsible as far as her individual
influence goes for securing clear administration. Where
oppression or unfairness is observable between employer and
employed, between class and class, she must become aware of
her own devotion to freedom and work to remove the injustice.
The tyrant, whoever and wherever he is, must let his
people go that they may serve God, the Highest. Woman
recognises no other domination.

It is part of woman's responsibility to look after the
health of the community. We have grown out of the era when
certain classes were supposed to enjoy unsanitary conditions,
and so were left to them without protest or relief.
We heard it said that if they had baths, they would not use
them, or that a section of the population was so ignorant
that a certain measure of infantile mortality was
inevitable. No good advice, it followed, was acceptable
even if given with the utmost altruism. Today we know that
ignorance is not valued as a class privilege. Every
section of the population seeks the knowledge which will
save them and their homes from disease and deterioration so
long as it is given in an acceptable way. We must stress
the idea that men and women must be treated as human
beings, as striving, aspiring people and not as machines
capable only of mechanical advance. The woman voter must
introduce into the state policy plenty of love, much
imagination, a large amount of understanding sympathy.

I have recently read the story of Bermondsey in the
life of Alfred Salter by F. Brockway. Without accepting
his extreme political ideals, there is one portion of his
activities which must rouse the admiration of women of
every political shade of opinion. It is his faith in
beauty as a means of redemption. Not only were the
overcrowded and unhealthy citizens of Bermondsey to be carried
into the country and allowed to soak in the influences of
its beauty spots, but flowers and exquisite shrubs must be
brought to the streets and alleys of the slum area and
stimulate the spiritual life of the inhabitants. Salter
succeeded in his aims to a surprising degree. The
appearance of the district was transformed, and the
interest of the people in culture and religion was
quickened. In the recognition of beauty as a revelation of
God, it seemed possible to accept his nearness, to strive
to harmonise life a little more closely with the reality ol
God's being. As citizens we have the opportunity to
increase the beauty of everyday Life, to insist on legislation
which can give opportunity to ordinary men and women
to increase their knowledge of art and to have greater
opportunities for enjoying it. Certainly, the response to
the beauty of art and music is not universal or uniform.
But the response does come sometimes from unexpected
quarters and everybody must have his chance. I remember
years ago giving a talk on the pleasures of life to a group
of very poor men and wondering whether they were being
bored, and whether politeness alone prevented them from
saying so. After the lecture one man got up and said: ``I
am a great lover of ornithology, are you?'' I was not quite
sure at the time of the meaning of the word, but I felt
humble indeed when he began to express his feeling in
picturesque language.

We women must be unafraid to enter public life by
whatever door we think fit so long as the light from Heaven
illumines our work, and we venture to seek God's guidance
in dealing with the problems which confront us. We are
hearing on all sides that the standard of morality is being
lowered for small immoralities are increasing greatly and
are being tolerated. People are hesitant in their
resistance to moral deterioration. Even while in positions
of responsibility, we women allow such evil to be brought
in the holy name of charity. We bribe people by the love
of excitement of all kinds to give money to important
causes. We allow small personal prejudices to imperil the
progress of great efforts in social service. The time has
come, I venture to think, when we must remember that our
social service can be of little use if its achievements
entail the weakening of the social conscience. We who have
entered public life know the encouragement to evil given in
the three words: Everybody does it. One of the great
contributions which we women can make to public life lies
in our insistence that our work is worthless if it
flourished through cheating or gambling or
self-advertisement. Our methods of organisation must be as
admirable as the cause for which we are labouring. We are
pressing for the better care of children in their own homes
and in institutions. We know and are proclaiming from
every kind of platform that much of children's delinquency
and indeed of their unhappiness can be traced to the broken
homes to which many of them belong.

Lady Stansgate has stressed the importance of our
influence in home life. The recognition of the moral law
must, I think, be carried from the home sanctuary into
public life. We must not be afraid either by example or by
precept to show our reverence for the underlying principles
of cleanliness and chastity. If our principles are
old-fashioned, we salute them on that account. They have
survived for 3,000 years and their failure will involve the
failure of truth and of love.

Perhaps the time has actually come when we should
apply our religion, be it Christian or Jewish, to the life
with which we are familiar --- our everyday life. We do
certainly do this up to a point, but generally in such a
vague unreal way. People blame religion for the immorality
existing in the world. ``What has been the good of it all?''
they say. ``Much unhappiness is caused by religion, but
when does it really prevent evil? In spite of religion
people go their own way and try to escape from God.''

People blame religion, but should we not blame
ourselves for not heeding religion? Certainly, as a
Jewess, I know that we have been taught Judaism for
thousands of years, but we are often unreceptive and so our
standards are still low in actual life. We can proclaim
great truths about honesty and cheat the customs officers
and smile at our cleverness. We may listen to the command:
Thou shalt not commit adultery, but subsequently forget it
and our lapses may be condoned, since desire is allowed
dominion. We are sometimes unafraid to tell lies because
we deem their colour ``white.'' The Psalm which says that
God desires truth in the inward parts is very beautiful,
but it can be made to seem a little out of date. Are we
justified in these small reservations? Who has granted us
our licences?

I think that we must really carry our faith into
public life and affirm that it is still applicable. Shall
we no longer deem the law breaker here who, as a Committee
member, sanctions irregularities, and the man who sticks
out for the sake of principle a prig? After all, if we are
privileged to engage in social service, it is because we
want to share the best we know with those whom we are
trying to serve. Their moral gift strengthened by grim
fights with reality is often greater than ours. Surely
religious sincerity must be one of our best qualifications
for usefulness; a working religion must express our creed.

It was a practical faith in God's command: ``Let my
people go that they may serve me'' which made liberty for
the oppressed true freedom -- not licence -- one of
humanity's most precious spiritual possessions, and it was
conceived by a slave people at the hour of their emancipation
and applied to life as it progresses through the ages.

There is a tendency in public life today to belittle
the possibility of peace by our cynicism and want of faith.
Dame May Curwen will deal with our international
responsibility, but I should like to suggest to you that here in
our own country, we must stress the ideal of cooperation
rather than that of rivalry. Her many women's
organisations start with high objectives and are broken up into
sections because of petty jealousies and rivalries. Such
catastrophies can only be averted by the courage and faith
of those who value peace as a revelation of God.

May I summarise the ideas I have tried to express in
this paper by a quotation by Ramsay in Lord Samuel's book
of quotations:

\begin{quote}
A nation cannot permanently remain on a level
above the level of its women.\footnote{From Viscount Samuel's \textsl{Book of
  Quotations} (London, 1947), quoting from \textsl{A Historical Commentary on St. Paul's Epistle to the Galatians} (London, 1899) by William M.\,Ramsay.}
\end{quote}

I believe, therefore, that there is no sphere of
public life in which women are not needed, but the success
of their entry depends on their faithfulness to the
standards of morality and spirituality which must be interpreted
by them according to the teaching of their ancestral
faith. This teaching together with its interpretation is
progressive in character, and will ultimately bring all
seekers nearer to the God of the whole world.

\attrib{Interfaith Meeting: June 28th 1950}
