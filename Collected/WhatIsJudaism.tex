\chapter{What is Judaism?}

I was present at a conference a short while ago when a
young mother asked that a pamphlet should be produced in
answer to this question as it affects the modern boy and
girl. It occurred to me that our young people might care
for me to deal with this question. I cannot promise to
offer a reply in one letter, but perhaps I might use a
series of letters in the new year to deal with so important
a subject.

There will be an immediate reaction among some of our
young people to my suggestion that the question has a
particular application to themselves as representing the
youth of the day. They will probably protest that Judaism
is the same for all time, since the days of the giving of
the commandments on Mount Sinai until the present day.
Judaism must be the same religion to be upheld at all costs
without any relation to the time factor, past, present and
future.

From one point of view this reply may be true, from
another, it is essentially false, and I am going to try and
explain in this and subsequent letters why this is the
case.

When I was young, Judaism meant observance. A man was
a good Jew if he kept his religion, and keeping religion
implied the loyal observance of ancient laws and ceremonies.
It was believed that God Himself was the authority
for the importance of observances, that His will was
revealed in the Pentateuchal code through a series of laws
and commandments.

This point of view is no longer upheld today by our
young people, but there are some eternal truths which Jews
must and will believe in for all time. It is no use your
saying: ``Oh yes, I do believe that Judaism is in no way
changed. Its obligations are for us what they were for our
ancestors, but I just can't obey them in this country at
the present day.'' If you really believed after the manner
of your father, and you persist in your mode of life, you
are living in sin. But you do not so believe. You say to
yourself: ``I like the old laws and customs. They meant so
much to my dad when he was young, but the Eternal God who
knows our ways could not have meant the laws for all time
and in all countries, He knows that I just can't keep
them. I keep three days a year and do what I can, and
nobody will say I am not a good Jew, as far as is
possible.''

Now I want you all to use your minds in connection
with these problems because the argument used above, which
I hear over and over again, is not at all satisfactory. If
Judaism is to be what it should be, the central fact in
your life, you must find its meaning in quite another
direction.

Judaism does not mean keeping laws. It means a way of
life. Judaism is based on the belief in God, and \textsl{that}
belief is good for all time. A man who professes Judaism
should lead a life which he should attempt to harmonise
with the God idea, The first principle in Judaism is the
unity of God, and it is upon this principle that I want to
dwell in this letter.

If you believe in God, you believe that God, the
spirit of perfect righteousness, love, justice, beauty and
truth, is in the universe and above it, that He lives in
man and above him. The world, we are told by the Psalmist,
reveals the glory of God, the earth declares His majesty.
We read in Genesis that man was created in the image of
God, and in Deuteronomy that His word is very nigh to us,
in our heart and in our mind that we may do it.

Because God is one, the knowledge of God should fill
the whole world and all parts of life can be made holy.
All peoples are His children; there should be no conflict
or divisions. Don't you think that the message of Judaism
to the world concerns the brotherhood of man, and if it
were properly delivered - through the example of our lives
- there would be peace and cooperation between all men?

As regards your personal life, is it not clear that if
you believe in God, you must have a standard in your work
life and recreational activity, in your thoughts and in
your feelings? God is one and indivisible. You cannot
switch Him off your life when His presence seems inconvenient.
If all of us who profess Judaism were to become
conscious of the presence of God in our lives, and in our
power to make contact with Him, don't you think that we
should become better men and women?

The Jewish teachers of the past told us to seek to
imitate God. They said that we needed no intercessor; we
were responsible to God for the conduct of our lives and
that we should go direct to Him for guidance and assistance.
They spoke from their own experience, and it is
your privilege and duty, if you profess Judaism, to verify
the truth of their statements by your own efforts in
testing their method. ``Seek the Lord at all times: call
upon Him while He is near.''\footnote{Isaiah 55:6}

If we try to imitate God, we seek to act righteously.
Everybody is tempted to fall below the best he knows, and
it is only with God's help that we can keep ourselves free
from the domination of selfishness or passion or greed.
But the God of the universe is in our hearts, and we must
give heed to Him. The God of truth demands truth from us
even in the inward parts. Subterfuges, deceits and
swindling won't do, nor will a pretence at sincerity, for
God is one, and we have in our weak imperfect way to accept
the ideal of truth which is another name for God.

Our God is the God of love, and He is ever present.
Just think how often we try to act as if love did not exist
in the world, We are unkind and thoughtless and indifferent
and \textsl{forget} all about love. Yet we say the Shema
once or twice a day and believe that God asks us to give
love with our entire being, so as to leave no room for
ugly, jealous, sordid feeling. We can serve our God only
with our best. The second best won't do.

If you profess Judaism, you must seek to be just to
yourself by self discipline and self control and self
development. You dare not spoil the good that is in you,
for through being a Jew you dedicate that good to the
service of God. You must use in God's cause your power of
body, mind and soul. Then you must seek the good of other
people, and see that they have the best opportunities for a
good life, those indeed which you desire for yourself. The
one God, being just, requires you to practise justice
yourself, if you would imitate Him, not only for the people
you love, but for those from whom you are distant by accident
or by choice. There is room in God's world for every
man and his right to be here is as good as yours.

God has made His world beautiful, and we have found it
possible to spoil it. Through our wilful blindness, even
the glory of nature may be wasted on us. Unless we train
the eyes of children to see and their ears to hear, beauty
makes little appeal to them.

We have all an artistic sense which can be cultivated
because it is one of God's gifts which is universal. Let
us seek beauty and reverence it, because it contains an
element of the divine spirit.

I have spoken about the unity of God as the first
principle of Judaism, and shown how it can affect our life
as a community and as individuals. In my next letter, I
will try to show more fully man's relation to God, and the
possible effect of our loyalty to God in making the world
progress towards goodness.

As I said at the beginning of my letter, in my view
Judaism does not consist in keeping observances, though
observances have an important place nevertheless in Jewish
life. They will be put in their right place when we find
the way of life which is Judaism. In conclusion I would
only ask you to consider the difference between the way of
life with God and the way of life without Him. Most of us
do fear to be unworthy of the best we know. Some Jews
today are trying to live in defiance of their Judaism.
Nevertheless, Judaism is blamed for their shortcomings.

We would try to witness to our faith, to show that it
is good to know that God is the Lord. Holding that faith
we say: ``The Lord is with us, we shall not fear.''\footnote{Psalm 118:6}

\attrib{Club Letter No. 33, January 1942}
\vspace{2\baselineskip}

This is my second letter in the series: ``What is Judaism?''.
In my first letter, I wrote of faith in God,
and what it implies. In this letter, I want to write of
our relations with one another, Perhaps it would be
simpler to call this part of our discussion ``Faith in Man'',
for if we accept the suggestion that there is something
divine in human personality, we must feel reverence for
every man and trust that he is ‘making for righteousness’.
If, however, we can, as we want to do, consider real life,
we claim that we cannot be expected to reverence our
enemies when they have proved themselves cruel and
treacherous, mean and crafty. As Jews, we can, I think,
nevertheless, affirm that if we had, from the beginning,
shown proper faith in man, the sadistic tendencies, which
lead to moral degradation, and which we notice particularly
in our enemies, would never have developed.

War is a large factor in creating the evils which we
deplore most at the present time, but we cannot hold our
enemies altogether responsible for the war conditions.
Before there was any idea or expectation of this present
war, the moral and religious standard of life in every
country, in every city, in every home, was far below that
which belongs to those whose faith in God has led them to
have faith in man. If we reverence the divine in man, we
must again give him the ``large place'' in which that divine
element can grow. I recall with love and appreciation
those words of the Psalmist: ``God has set me in a large
place.''\footnote{Psalm 118:5} This verse does not refer to good housing and
open spaces, although we know that the child's satisfactory
spiritual growth does require space and privacy, light and
air; the words mean, I think, that every man requires
freedom. He must not be bullied or oppressed; he must be
given his opportunity to develop his personality; he must
not be allowed to suffer through neglect from any curable
form of ignorance or disease. He must be trained to show
love and kindness to his fellow men, to seek truth, and to
show mercy and justice to all with whom he comes in
contact.

I say that none of us has been sufficiently alive to
these obligations and war is one of the evils which has
ensued. How often in our prayers, we ask for the power to
distinguish between good and evil. We need God's light, in
order that we may see light. Nevertheless, we allowed the
economic misery of the Germans to become so intense, that
they could find no true deliverance, and stumbled into
hailing the advent of Hitler, and thought he was their
saviour.

We must believe in the divine spark in every man, and
yet with shame and anxiety we must say that it may be altogether
hidden by human unkindness, if evil is allowed to
increase without restraint. A magistrate told me the other
day that an old man of 79 was brought before him for
stealing a coat and put on probation. It was found on
enquiry that the old man had been charged 55 times
beginning when he was quite a young man and sent to prison
for 6 weeks. When at long last, the man was told that the
method of probation was to be tried, he looked surprised to
the degree of bewilderment. Every week the probation
officer lunched with him and was deeply interested in his
account of himself, interested, you see, in him as a
personality. Friends obtained for him the old age pension
for which, till then, he had been considered ineligible,
and his self respect rose wonderfully with the assurance of
a reliable income. That man has made good because another
man believed that in spite of his record there must be good
in him, and that good must always be worth finding.

Will you consider for a few moments how your own life
will be affected by faith in yourself as a being possessing
something of God? The effect on yourself will be tremendous.
Do you want to be something really worthwhile, a
witness to the reality of God's existence, a very humble
partner with God in creating righteousness and joy in a
very small and very restricted corner of the world, that
corner in which you have some small influence? You can
achieve this high privilege, if you make yourself really
fit in body, mind and soul. Sometimes, you say you are fed
up, and it is no use your going on trying for the best; you
are misunderstood, or circumstances are unfair and
altogether hard. You feel you must give up trying.
Judaism teaches that you have the power within you which
makes all good things possible, and makes you feel ashamed
to give up or even to fail. Take courage and remember that
more than 2,000 years ago a religious genius wrote from
experience: ``The Lord is with me, I shall not fear.'' Take
courage and test the power within you by formulating your
faith in prayer, telling God about your hopes and
difficulties, and asking for renewed faith in yourself.
You will make good through your Judaism.

Faith in humanity makes a high standard of home life
possible, for love between husband and wife, parents and
children, sisters and brothers contains a great element of
faith. As you are anxious for the well being of your
beloved, you can easily believe that those who are bound to
you by family ties reciprocate the same feeling. We are
inclined to trust our home folk more than anyone else in
the world, because through believing in the ideal of God's
unity, we come to believe that the different elements in
our home can be united by Him into one perfect whole. Our
conception of friendship is based on faith. If any feeling
of suspicion once gets into our hearts, our relation with
our friend is spoiled. It is best to admit this fact and
to let each go her own way. But if we have faith in man
through faith in God, we do not suspect easily. We forgive
readily, even as God does, although ours is the weak,
imperfect way. We give opportunity for clearing up
mistakes and misunderstandings in our human way through
frank talk with one another, even as God invites us to lay
our doubts and difficulties before Him in prayer.

We know that there is room in God's world for every
type of man and woman, and they are our sisters and
brothers, for we are all the children of God. We must not
make life harder through our conduct for any individual we
find in the world, for our faith in man which arises from
our faith in God convinces us that his right to live is the
same as our own. We all believe that we are working to
build up a better world. That is our great hope and consolation
in these days of suffering. The world can only be
better when we can trust one another in our business life,
our recreational life, and in our international life. It
is, however, no good to try to live in a fool's paradise.
We \textsl{know} that there are crooks in business, men and women
who have no appreciation of fairness in sport, governments
which cannot be trusted. The time will certainly come when
faith in man will be established, for God Himself, who
created man in His image is a guarantee for the possibility
that man can attain to trustworthiness, and if individuals
can reach this stage in moral development, so assuredly
must it be possible for nations.

It seems to me that the progress of the individual and
of the nation depends in a large measure on man being able
to complete satisfactorily a term of probation. As individuals,
we must prove that we merit the confidence of God:
``Ye shall be holy, for I the Lord Thy God am holy.'' In
contact with God we can raise our standard of life, and
this process leads to holiness. God will know if we ring
true even to the inward parts. We cannot deceive Him. We
must fit ourselves to become His servants and accept the
fact that the process of probation will be hard and last a
long time. If we deserve to be trusted by God to do His
work, I believe we shall merit the confidence of man. In
the world of the future, men will have the governments
which reveal the stage of morality which they themselves
have reached. In the past, some nations have been backward
in intelligence, and had to struggle for ages before they
could attain the standard of their generation. Some have
been hopeless defectives and have been allowed to disappear.
Perhaps the same thing must happen on the moral
plane. Nations must prove their ability to reach the
average standard of their time, and be willing with other
nations to attain an infinitely higher stage of moral and
spiritual progress. Our faith in man, based on faith in
the Fatherhood of God, forbids us to despair of human
progress.

Again and again, I say that you and I must begin our
probation at once and seek God at all times, for Judaism
teaches us that He is near and ready to assist us in our
forward journey. In my next letter, I will write of the
Jewish belief in progress both for the individual and for
the community. So far we have seen that Judaism asks for
faith in God and faith in man.

\attrib{Club Letter No. 34, February 1942}

\section*{Belief in Spiritual Progress}

We think of Judaism as a spiritual influence binding
us to the living God. Through Him our lives must progress.
He has commanded us to be holy because He is holy. We have
to establish our kinship with Him and to endeavor to
imitate Him. God has put eternity into our lives, and so
the time is limitless in which we can approach Him.

There is a recurrent note of optimism throughout the
Hebrew Bible. Suffering must be overcome, if possible. It
should never be considered as a desirable end in itself, to
be sought as a condition pleasing to God. If we read the
Psalms, we shall find the spirit of progress clearly
expressed. Man is expected to go from ``strength to
strength,'' overcoming difficulties, and be prepared to
serve his God through his happiness.

The Old Testament has no definite teaching on immortality.
In a few Psalms there are phrases which, taken
from their historical setting, seem to indicate a clear
belief in an infinite future. Take for example the verse
in Psalm 16. ``In Thy presence is fullness of joy; at Thy
right hand are pleasures for evermore.'' The general trend
of the Old Testament is to ignore dogma on the theme of
immortality. Perhaps abstractions did not greatly interest
our child race. They were gathered to their fathers by
their God, and their fate could surely be left to His
wisdom and love. Nevertheless, we find a well-developed
faith in immortality among the teachers and writers of the
early post-biblical period. Let us consider the words in
the Wisdom of Solomon, Chapter 3: ``The souls of the
righteous are in the hand of God, and there shall no
torment touch them. In the sight of the unwise, they
seemed to die, and their departure is taken for misery
... but they are at peace.'' Such definite affirmations
can easily be multiplied. The biblical characters show no
fear or rebellion in the face of death. Even Moses denied
the entry into the promised land asserts the triumph of the
human soul when confronted with apparent defeat. The Jew's
season of fasting is succeeded by his joyous festival of
ingathering. Even though he walks through the valley of
death, he fears no evil.

The Jews' mission in the world is the message of hope.
``O Zion, that bringest good tidings ... lift up thy voice
with strength.''\footnote{Isaiah 40:9} We are here to speak to the world of an
ever present God of love and justice who will cause
righteousness to triumph in the end. He will bring every
deed to judgment, whether it be good or whether it be evil.
Atonement is possible to every sinner, if he turns himself
to God, resolved to live anew in His presence. There need
not be finality even in sin. Because God is the ruler of
the world, liberty and security must be achieved by human
effort made in cooperation with God.

Do we come to find progress in the community as well
as in the individual, and understand why our messianic age
is ever before us? We can no longer believe in a personal
Messiah, for we cannot conceive of any human being with
power to lead the world, with its multitude, of conflicting
tendencies, to abiding peace and happiness. It is nevertheless
a principle of Judaism that each individual can be
a Messiah or messenger of God, working for righteousness
and together bringing nearer the dawn of the perfect day
when the highest human aspirations will be satisfied. The
testimony of the Jews to the reality of the All-Father, and
the standards of human conduct revealed by Him for the sake
of His children of every race and creed - this testimony
will be a factor in building up the Kingdom of God.

Judaism requires the whole of man in the service of
God. ``Thou shalt love the Lord thy God with all thy heart,
with all thy soul and with all thy might.'' There is an
intellectual, as well as an emotional element in our religion.
God is the God of truth, and we must seek truth in
order to find God. We are sometimes rather too ready to
excuse ourselves from making opportunities for study. We
hear people say that they are unable to concentrate their
minds on problems in these difficult days. They have no
time for reading, they have too little leisure, and are
altogether too tired to study. Nobody blames us for making
these ``good'' excuses. They are valid and proper, but
unless we are very strict with ourselves, we may lost the
best that is in the world. ``Seek ye the Lord at all times.
Call ye upon Him while He is near.'' (Isaiah 55, Verse 6)

Even more dangerous than want of study is want of
thinking. People accept what their fathers give them in
the way of religion and do not assimilate their possessions
through the power of thought. Even the belief in God which
may be considered as the first principle of Judaism cannot
influence our lives unless we think about it. It is not a
matter of lip service. We have to turn this thought over
and over in our minds every day of our lives. Is is a
tremendous thought! Then there are the simpler questions
which give evidence of progress or stagnation, active
loyalty to our Brotherhood, or mechanical acceptance of
tradition, in order that we may go on in our way with as
little trouble as possible. Our community stands for
certain great truths which have been handed down from the
past, and must be rejected, or revitalised by our devotion.
How far were our fathers justified in making their faith
our faith, and giving us the responsibility of carrying on?
Our ancestors forged a chain of testimony, binding them to
God. They used the best material available to them, and
for doing that work so well, we offer them our homage and
our gratitude.

But supposing we honestly know that in the course of
ages new and better material has been found to increase the
human store of truth? Are we to make no use of these discoveries
for the strengthening of our relations with God?
How are we to tell if the new discovery is good? God has
given us prophets and teachers to help us, and however
small our minds are, they have been given to us to use in
His service. I will illustrate this point of view in my
letter on the relation of man to God, which I have delayed
sending out, but I think the way is now prepared by my
letters on faith in God, faith in man, and the present one
on spiritual progress. Some of you may agree more readily
that Judaism holds for us an element of emotional interest
than that it has for us a strong intellectual appeal. So
often in our religious experience, we are up against
insoluble problems and inexplicable conditions. Our
intellects will not help us to explain fully the existence
of evil. We shall not penetrate the mystery of life or of
death. We cannot define God by thought alone. Nevertheless,
we are prepared to stake our lives that God is within
us, and His spiritual influence can be felt in all parts of
the universe; that we derive our spirits from Him, and
because He is perfect in truth, beauty, love and righteousness,
we are charged to imitate Him and progress towards
Him, so far as our faulty human nature will allow. We find
by experience that the more love we give to humanity, the
more love we feel coming to us, and we believe God is the
source of love, and that love increases the more itis made
use of. I should like you particularly to take away this
thought from this letter.

The Jewish Brotherhood is called to the service of
God. Other religions are called to similar service, but we
have our own method of service, our own history, our own
divinely appointed purposes. There are many men and women
who call themselves by the name of Jew, but degrade the
faith because they are indifferent to the teaching of
Judaism, and to their sacred charge. They prefer to live
for material advancement, and neglect spiritual progress.
They ignore their call to service, and therefore refuse to
prepare themselves for their great destiny. You can belong
to the better set; then you will be sure of your advance,
and even if you have to endure some suffering and exert
yourself in an extreme measure, you will all the time be
glad of your high office and determine to make yourself as
worthy as possible. Remember, we are called to be a
kingdom of priests ministering to the well-being of
humanity. We have a long way to go before we deserve to be
recognised as useful priests, but we may get there if we
begin at once. ``Tell the children of Israel,'' says God to
Moses, ``that they go forward.''\footnote{Exodus 14:15}

\attrib{Club Letter No. 35, March 1942}
