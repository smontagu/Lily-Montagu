\documentclass[14pt, article, extrafontsizes, twopage, a4paper]{memoir}

\settypeblocksize{24cm}{17cm}{*}
\setlrmargins{2cm}{*}{*}
\setulmargins{2cm}{*}{*}
\setheadfoot{\onelineskip}{\onelineskip}
%\setlength{\topskip}{1.6\topskip}
\checkandfixthelayout
%\sloppybottom
\fixpdflayout

\parskip 0.25em
\parindent 1em

\usepackage{relsize}
\usepackage[all]{nowidow}
\usepackage[dvipsnames]{xcolor}
\usepackage{bidihl}
\usepackage{graphicx}
\usepackage{tikz}
\usepackage{caption}

\usepackage{polyglossia}
\setmainlanguage{english}
\setmainfont{Brill Roman}
\setotherlanguage{hebrew}
\newfontfamily\hebrewfont[Script=Hebrew, Scale=1.1]{David CLM}

\setlength{\footmarkwidth}{1em}
\setlength{\footmarksep}{0em}
\footmarkstyle{#1\hfill}

\newcommand{\attr}[1]{
  {\raggedleft\smaller#1

  }
}
\newcommand{\hlc}[2][1,1,0]{{\definecolor{bidihlcolor}{rgb}{#1}\bidihl{#2}}}
\newcommand{\todo}[1]{\hlc{#1}}

\begin{document}
{
  \centering
  \Large Selections from the writings of\\
  \LARGE Lilian H.\ Montagu (1873–1963)

}

\chapter{A brief biography}

Lilian (Lily) Montagu, was born on December 22, 1873 in a well-off family of British Jews. Her father, Samuel Montagu, was a banker who worked unceasingly for the Jewish community, in particular for the immigrants who came from Russia and Eastern Europe in large numbers towards the end of the nineteenth century. He became one of the first Jews to be elected to the British Parliament.

Lily was involved all her life in communal work for both the Jewish and the wider community. She worked as a social worker and founded a Club with the objective of enriching the lives of working-class Jewish women intellectually, socially, and spiritually. She was active in the cause of women's right to vote at the beginning of the twentieth century. After the First World War she was one of the first women to act as a magistrate.

In 1902 she was among the founders of the Jewish Religious Union, whose initial aim was to provide services on Saturday afternoons as a supplement to the conventional synagogue services. Eventually it developed into the Liberal Jewish movement. In 1926 Lily was one of the founders of the World Union for Progressive Judaism, still active as an umbrella organization for Liberal and Reform Jewish communities around the world.

She authored many books including novels, books on Judaism, a weekly letter to the member of her Club and anthologies of prayers for contemporary Jews.

She never married and for most of her life lived with her sister Marian. Lily died on January 22, 1963, aged 89.

{
  \centering
\includegraphics[width=5cm]{lilyolder.png}\\
\textsmaller{Lily Montagu}

}

\clearpage
\chapter{Lily describes her early life}

My childhood was a very happy one. I was one of a
family of ten children, six girls and four boys. I came in
the middle of the family, having four sisters and one
brother older, and one sister and three brothers younger
than myself.

Our parents gave us by example rather than precept
an insight into Orthodox Judaism as a great and wonderful
inheritance, which must be held in the utmost reverence,
and which would exist for all time. In all their
piety, Judaism was, in my parents’ view, not directly connected
with the problems of everyday life. The observances
upon which, to them, Judaism was based must be
observed with the utmost rigidity, and obedience was a
supreme act of religion.

I can trace my first questioning of the utility of
observances if pursued as ends in themselves to experiences
connected with Passover and the Day of Atonement ... 
the Day of Atonement ritual was so full of
repetition. People came in masses, but they were not,
apparently, interested in the long service. They thought
it quite right that it should begin early and continue for
the prescribed time, but they really seemed unaffected by
the contents of the prayers. They lounged back in their
seats in Synagogue. They walked in and out. They
chatted and laughed on the threshold of the synagogue.
For the closing section, they seemed all to return. The
phrases were deeply impressive. I was moved by the
closing proclamation of faith in the One God, and
believed all my fellow worshippers were similarly
affected.
But no sooner was the last word spoken than there
was a very hurried exodus ... The few
remained, for an evening service was in progress. Nobody
wanted it, but it was the right and proper thing to read
the evening prayers at that particular hour, and the
Minister had to stay and read ... The reading was accomplished
at a miraculous speed, and we too hurried home. \textsl{Then
everything began as usual}. \textsl{That} was the part which
mystified me, and set me thinking ... How could it be \textsl{finished}
so suddenly? I was approaching adolescence and beginning
to ask questions.

The first opportunity I had for taking real initiative
was when I started children’s services in the vestry room
of the New West End Synagogue, when I was between
seventeen and eighteen years of age. The organisation
was very simple at the start. I knew I had the encouragement
and sympathy of Mr.\ Singer, our Minister, but
when I began he was away on a holiday or on sick leave.
So I wrote to the Committee of Management and told them
how utterly boring the long Hebrew services were to the
children of the Congregation ... If they prayed at all, I said, it was that the service should
come to a speedy end. I asked, and quickly received,
permission to hold special services either before or after
the regular service.

My father’s deep sympathy with me, and his pleasure
in my small achievements, made him quite pleased with
my success ... The
liturgy being mostly in English, instead of Hebrew as in
Orthodox Synagogues, was completely understood by the
worshippers. I varied the service from week to week, and
informal talks instead of sermons proved acceptable. The
attraction supplied by these simple services encouraged
me later to attempt something on similar lines for grown-ups ...
My father thought I might help some
of the weaker brethren and prepare them for real services.
It was only when I was among the leaders of a
schismatic movement and proclaimed my belief that the
Bible was partly a human and not an entirely divine book
that I caused him real pain. For the last year or two
before his death, my father felt that his “Lilchen” (his
special term of endearment for me) was divided from
him by a wall of disapproval which even great love could
not break down. We both knew the love was there all the
time, and to-day I feel that his understanding sympathy
has been restored to me. Perhaps he knows that all my
work, and especially my Club work, aimed at keeping
Judaism alive among our young people.

My mother’s faith was beautifully simple ... She believed
completely in God’s fatherhood, and that observances
were ordained by Him for the good of His children, and
that He must be obeyed ... I have dwelt on the loving
understanding of my mother ... because during forty years of my life I
have been called “mother” by a vast number of girls
and women. I feel I owe much of my power to win the
confidence of “my children” from having as my model
the mother who throughout her life had no greater
pleasure than that of sharing her children’s interests, and
whose faith in God was expressed in every detail of her
life.

\attr{From \textsl{Faith of a Jewish Woman}, 1943 and \textsl{My Club and I}, 1954.}

{
  \centering
\vspace*{.5\baselineskip}
\includegraphics[width=5.6cm]{lily19.png}\\
\textsmaller{Lily, aged 19}

}

\clearpage
\chapter{The Spiritual Contribution of Women as Women}


I am attempting to discuss a subject with you today
which is fraught with particular difficulty. To begin
with, it is very likely that you as feminists will disagree
with my premises. My central thought is based on the idea
that although women should cooperate with men in all the
activities open to human beings, and that they should
recognise no limit except that of incapacity, and that sex
cannot disqualify anybody from doing that for which they
feel themselves fitted, women have certain qualifications
which are different from those possessed by men. Indeed, I
am a feminist too and believe in complete equality between
men and women in the social, political, economic and religious
spheres, but I think that humanity is enriched by the
diversity between the two sexes. It follows that they must
develop their special qualifications, and they must give as
complete a contribution to the world's spiritual treasury
as possible, but that it must have a character of its own,
and not be an imitation or a replica of the contribution
made by men.

Men and women have the creative faculty jointly, but
men can work more objectively than women. Let me take a
simple illustration. Father and mother want their small
son or daughter aged 4 to learn to pray. They want to
stimulate the capacity for prayer. Father takes his child
to Synagogue while no service is in progress and carries
him round and shows him the various features, including his
special seat, and the Ark, and explains that the books of
the Bible are contained in the Scroll which is in the Ark.
He is informative and interests his child. The Sabbath
comes and the boy or girl is dressed in his best clothes
and accompanies his Daddy to the Synagogue and carries his
prayer book. The child sits between him and Mummy. Daddy
occasionally shows him the place in the book, and he stops
fidgeting for a moment. He is happy and prepared to repeat
the experiment on subsequent Sabbaths. Gradually, after a
long time, the atmosphere of the Synagogue impresses itself
on the child and he feels the inclination to worship. The
realisation of that possibility will depend in a great
measure on the father's own reaction to the service and
what he says about it when he gets home.

The mother has another method. At bedtime she can do
many things. The element of thanksgiving is a good
preparation for prayer. She remembers a beautiful fungus
which she and Johnny admired together while on their walk.
The imaginative faculty is strong in her. ``Johnny'' she
says, ``do you remember our walk today and that fungus we
saw, and the lovely streaks of colour, the red and yellow
bits, and the little bits of green and brown, and how we
wished we could have found a mushroom, but then we said it
would not have been half as pretty? Shall we thank God
for making that fungus?'' ``Yes, let's, Mummy.'' ``Thank you
God for making that lovely fungus, with the red and yellow
bits, and the bits of brown and green'' added Johnny, ``and
next time please make it into a mushroom so that we can eat
it.'' ``Amen'' says Mummy. Another night Mummy and Johnny
make a list of the people they both love and ask God to
bless them. They make their prayers together and they are
their own special prayers.

A woman creator gives her own spiritual experience of
pain and joy to the progressive conceptions of Judaism. A
man analyses and sifts and reasons while with the woman he
climbs the mountain of God. I have heard it said how a man
climbs step by step until he reaches the level within his
reach. His path has been sure but rather slow. He looks
round and sees a woman by his side. He did not see her
while he was climbing because she sprang from ledge to
ledge taking many risks.

In religious discussions on the source of authority in
Progressive Judaism, you will find men more interested in
external authority than women. A man says: ``We shall have
chaos unless the men of scholarship and experience get
together and decide on specific observances. Does Sunday
observance lead to disloyalty? Should we have more Hebrew
in our liturgy? Is the Cantor's assistance essential or
even desirable?'' ``Well, I don't see that it matters what
the big people think'' says the woman. ``I know I cannot get
my young people to service on any day but Sunday. Hebrew
may be all right for some people, but it is no use for
those who do not understand it. If I want the best
singing, I go to the Opera. When I am at a service, I want
to sing and join in, and I know my children do. If they
can sing, all the better, and if they can't, let the others
sing louder and drown my children's voices, but they too
are singing.'' There is an element of practicality in all
this, perhaps a little less feeling of responsibility, a
longing that her own young people should be satisfied.

\tolerance 512
We women must approach the question of international
peace from, I think, a rather different angle from that
adopted by men. In the first place, I think we must lift
the problem out of the sphere of politics into the sphere
of religion, or, as I would greatly prefer, bring a strong
religious influence to bear upon the political issue. Men
and women of course know equally well the misery and
futility of war, the demoralisation which is part of its
aftermath. But women surely realise more fully the affect
of war on home life. The responsibility seems to me to
rest on them to overcome the sense of defeatism and
frustration which is surging over the world. Men are in
their outlook more realistic than women. Their vision is
blocked by the existence of the atom bomb and the threat of
totalitarianism. They have given so much to the cause of
liberty, and they see all forms of tyranny flourishing and
becoming ever more threatening. They are weary but
resolved not to show any weakness. So they turn to war
preparations as the only way to secure peace. Here is the
woman's part. We have to affirm with all the strength at
our disposal that because God is, the reign of peace and
righteousness must triumph in the world. If we firmly
believe in the wickedness of war, we must turn away from it
and find other ways of settling our differences, however
difficult the search may be. If we fail now, our civilisation
perishes and with it all that is precious in home
life. It is only on the plane of religion that we can find
the way to restore our faith in man and in ourselves. We
can help because we are outside the actual fighting arena.
Even though we are in the war unit, our methods are not
confined to physical force.


Before I close, I would ask you to consider whether
you think you can resist the present drift away from the
consecration of home life. Men may make a greater effort
even than women to keep up appearances, but you know as
well as I do that a rottenness has set in, and the moment
has come for you to arise and shine forth. On the stage,
in novels and in the newspapers, the idea of the divided
home is accepted as inevitable. The children are being
sacrificed to the general feeling of inevitability. The
natural happy and chaste home life is regarded as
exceptional and unexpected. The marriage vow has become
loosened. The marriage ceremony is only religious in form,
its significance is forgotten. Men think you feel that
this phase is inevitable, that the new code of morality or
immorality is likely to prevail; it is in the trend of the
times. Business is absorbing. The survival of the fittest
is the law of the day, and no other consideration can count
while the struggle for existence is so fierce. Here again
we must make our faith felt and before we can do this, we
must revere it, each for herself. If home and the children
are our most precious possessions, if the exaltation of
their worth is anything more than mere phrases, we must
hurry to the work of salvation.

Yes, friends, in the field of religious education, at
home, in the work for peace and chastity, we must work with
new zeal and new faith. Our contribution is needed. Our
courage must be added to that of our men. We must restore
to them some of their lost confidence and hope. We allowed
them to lose much through our apathy and inertia. Now we
must know that we stand before God and must either perish
or be prepared to obey His word. Each must say for
herself: ``Here am I, send me.''\footnote{Isaiah 6:9}

\attr{A speech given in Chicago at the Jewish Education Building, Friday, November 26th, 1948}

\chapter{On Prayer}

``Ho every one that thirsteth, come ye to the waters.''
This call, taken from Isaiah 55, is a call to prayer. Some
of you ask: ``Why pray? What is the good of it?''

I think we need to pray. We are incomplete without
contact with the living God -- and that is what prayer
means. We are beings created by the living Spirit of
Goodness, Truth, Love and Justice, who wish to return and
draw from our source renewed energy with which to carry on
our lives. We pray, then, for “the increase'' of our
spiritual power. We need sustenance and exercise for our
spirits, quite as much as we need food and gymnastics for
our body. We meditate on God's law of righteousness, and
the desire to be better and to do better fills our hearts.
Why pray? \textsl{We pray in the first place that we may live more
fully}.

We are all conscious of some wrong doing in our lives
which separates us from God. We pray for the power to
overcome evil in ourselves, and to become one with God. We
are each of us directly responsible for the conduct of our
lives. We have to destroy the wrong by our own efforts.
The consciousness that God is real -- that something of His
spirit is in our hearts, even while, in its perfection, it
is the supreme life force in the universe, this faith gives
us the power to overcome sin, for it makes our will strong
and directed towards good. \textsl{We pray, therefore in the
second place for self advancement in righteousness}.

As we pray, we feel ourselves united with every
aspiring human being; his life and well being are part of
our own lives. We understand his needs, for we actually
share them, and so \textsl{the third blessing which we discover in
  praying is the unity of mankind}.

But we would like to alter the world; we would wish to
see some of the evils and misery, cruelty and injustice
swept away. \textsl{So we pray for the removal of evil}, though it
still persists and sometimes seems to grow in magnitude.
What then is the good of praying?

The good lies in another direction. In prayer, we
realise the value and power of human personality. We pray,
and the possibility of achievement is unveiled before our
eyes. We can choose good and reject evil. That is our
human prerogative. The same privilege belongs to the men
and women who are in positions of great power and responsibility.
We would not surrender the freedom of human
personality. \textsl{When we pray we show the connection between
belief and conduct}.

Freedom is part of the human inheritance. Prayer has
revealed the power of man to create the better world
through obedience and loyalty to the laws of God. It is
\textsl{for us} to help in establishing the kingdom. We must not ask
God to do it for us, and so surrender our human independence.
He has offered us the power to work with Him.
Arise, shine!\footnote{Isaiah 60:1}

We love to pray for our dear ones, and sometimes we
pray and the calamity we wish to avert comes just as if we
had not prayed at all. Why then, we ask, does not God
hear? I believe that God \textsl{does} hear, and it is well for us
to think of our dear ones when we are considering the
reality of God in prayer. Let us seek from His revelation
ways to increase our wisdom and our power of loving.
Perhaps we shall lose some of our selfishness and our
ability to give pain. But when things don't go right, as
we believe, for our beloved, we must remember the limitation
of our vision and that what seems evil to us may in
the end be good. \textsl{Through prayer we learn to trust in the
supreme love of God, and that He acts only through love}.

Life is sweet for you all. In spite of its sad,
gloomy passages, you have the power of learning and loving.
You can see some beauty in the world, even if the glimpses
are few and far between. You can sometimes see the wonders
of nature and hear glorious music. Most of you have behind
you the security of home life and the trust of those who
love you. You, because you are young, can experience the
pleasures of the body and the mind. You feel grateful for
life. \textsl{In prayer you can give thanks}.

May I appeal to you to pray daily -- to feel yourself
consciously in the presence of God? If doubts assail you,
face them and wrestle with them. In the end, they will, I
believe, add strength to your faith. Don't give up prayer
because it is difficult. Learn to create the right atmosphere
for prayer, the atmosphere of reverence and humility.
Clear your hearts before you pray from selfishness and
insincerity, and your mind from impure thoughts. Then
throw yourself into the ``Everlasting arms''\footnote{Deuteronomy 33:22} of God, and in
the depths of your heart you will hear Him speak. ``Speak
Lord, for thy servant heareth.''\footnote{1 Samuel 3:9}
I beg of you not to delay
praying until you are too faint and weak for want of
spiritual nourishment to pray at all. \textsl{Pray now}! Tomorrow
it may be too late.

\attr{Club Letter no. 5, March 1939}

\chapter{Religion in the Club}

Miss Harris\footnote{Emily Harris (1844--1900), author and social worker, co-founder of the West Central Club with Lily Montagu.} and I had many disagreements over the
question of Sabbath breakers. She did not think these
should be allowed Club membership. I, being younger,
realised better the economic pressure of the age which
made strict Sabbath observance impossible for those who
wished to live independent lives. I had been influenced
by Liberal Jewish teaching, and believed that if we would
serve our God who is the Father of all men, we must
translate His word in the changing circumstances of life.
It was all important that we should ask ail those under
our influence to discover God’s word, and try to live in
accordance with it in their working lives. If they lived
truly by His word, they could worship Him all day long,
whatever they were doing, and not only at Sabbath services.
Moreover, they could hallow any day and every
day through prayer, and could use for worship any part
of the recognised Sabbath which was at their disposal.
In this sense, I made a very strong appeal for Friday
evening observance in the home. For me the Sabbath
observances were closely connected with some of my
happiest personal experiences of home life. We children
had Sabbath Eve prayers which we read with our
mother. We had our family gatherings on the Sabbath
Eve prefaced by the Sabbath blessing on the wine of
fellowship and the bread of sustenance, and most important
of all, each child in turn received a blessing. Both
parents blessed each child, laying their hands on the
child’s head, and speaking the Hebrew blessing as they did
so. This custom made a deep impression on me. On
Friday nights we rejoiced in being together. When some
of my sisters and brothers married, they came back with
their husbands or wives to celebrate the Sabbath at home,
and later the children of the family joined us, and the
family evening became more and more delightful. Our
lives were so much absorbed in diverse activities that the
Sabbath gave us a unique opportunity for keeping the
flame of loving interest alive among us. Even after our
dear parents passed on, the custom of Friday evenings
at home with family gatherings and family prayers
remained and was upheld by our eldest sister, Mrs.
Franklin, who gave us the opportunity for holding these
meetings in her home. Young people tell us to-day that
important engagements frequently occur on Friday
nights, and it is impossible to refuse them. For me,
because I valued the consecrated family life keenly, there
was never any difficulty at all. We just remained at home
on Friday nights, and no other suggestion was ever considered,
much less entertained. Life accommodated itself
around this established custom as the sea adjusts itself
round a rock which stands up in irresistible strength. It
is there, and nobody can move it.

Before the establishment of the West Central Liberal
Jewish Congregation some of our devoted Club members
walked to Notting Hill Gate on the Holydays to enable
me to hold a four hours’ service in a hall near the Synagogue
at which I attended for some years. On Saturday
afternoon, I walked with my beloved sister to Dean
Street, and we held our special Sabbath afternoon services
which were quite well attended. Before I could start
these services I had to overcome great difficulties at home,
for no other reason that they were supposed to give me too
much exertion after a heavy week’s work. But I believed
then, as ever afterwards, strongly in the power of worship,
and was convinced that the habit of not attending services
was rooted in the boredom which a traditional service
evoked. A year or two of workshop life seemed to wipe
out the small knowledge of Hebrew which most of our
girls had acquired as children. They could not understand
the traditional service and it bored them extremely.
Moreover, they were not accustomed to take any part or
responsibility in the service and found in the liturgy and
sermon nothing which was related to ordinary daily life.
Our services were different. They were in English and
were brightened by congregational singing. Only such
prayers were used which had a meaning for modern
Jews and Jewesses in the actual circumstances of their
lives. The sermons treated of vital subjects. Ultimately,
the West Central Liberal Jewish Congregation was
founded, and men as well as girls attended and sat together,
instead of as in the Orthodox Synagogues being
separated. Children’s services were also held for some
years and attracted large numbers. Later, the children
were invited to come with their parents and a special
address was given to them in a separate room.

For the first years we refrained from using instrumental
music as we were afraid of alienating the few worshippers
who were sincerely Orthodox. To my surprise, I found
no objection raised from this quarter when we did make
the change. As often happens, the really religious are
seldom opposed to reform even if they themselves do not
desire or value it. Our Orthodox members said on this
occasion: “Perhaps the music may bring some people
who would not otherwise come to a service.” It is only
the pseudo-Orthodox who get their souls entangled in
legalism, and who criticise us for making the service as
beautiful as possible.

Throughout the history of our Club, talks on religion,
as well as the united act of prayer, have been features of
our Club life. We have always prayed together on Club
holidays, and through wonder at Nature’s beauty our
members have been led to worship the Creator of the
world. We have felt the unity which comes from
Sabbath Eve celebrations conducted in a family atmosphere.
We have held open-air services and felt the truth
of Psalm 42, for in holiday atmosphere we easily experienced
our longing for God “as the hart panteth after the
water brooks.” Generations of Club members have
carried home from their holidays at the Green Lady
Hostel, Littlehampton, memories of “talks under the
tree” held every night for those who cared to come
rather than to listen to concerts on the front. At these
discussions, as well as at the meetings led by a variety of
people in the Club itself, the frankest questioning was
encouraged. We had the opportunity of removing curious
superstitions and of teaching the fundamentals of a living
Judaism. Again and again we have been appalled by
the wrong sense of values among our young people, and
at the sordid sex disqualifications in which the girls still
believed. Being filled with reverence for the piety of our
ancestors, we exercised very careful thought before we
tried to weaken our young people’s unquestioning allegiance
to ancient customs. We knew that the service they
paid was only lip service, but we did not want to quench
the dimly burning wick. We could not, however, support
any longer a policy of mere drift, for this drift might
easily become. a drift into materialism. We believe we
acted in the best interests of Judaism when we helped our
young men and women to use their minds in the search
for God.

In recent years I have written a monthly letter on a
religious subject to Club members, and invited them to
discuss their special problems with me. I believe that
our religious life will become really strong when we have
convinced our young people that they are under an
obligation to discover God anew each for herself and himself.
God is great enough to enter into every soul, but we
must make ourselves ready for Him and eager to receive
Him.

Our ancestors have told us of their experiences and
handed these experiences down to us. But our religion is
dead if we worship only with our fathers’ hearts. We
have to kindle our own lights. We believe that the truth
of Judaism is co-extensive with life, but it is a progressive
force, as is life itself, and its presentment cannot be
changeless if it is to fit every generation of believers.
Through our Club we must always try to stimulate faith,
and become interested in its abstractions and in its history.
But, above all, we must assimilate the idea of God as far as
is possible with our limited intelligence. God must
become so real to us that we can live under His guidance,
working for Him and with Him, and trusting that this
kinship is for ever. With this faith we can pass even the
valley of death and still fear no evil.

\attr{From \textsl{My Club and I}, 1954, Chapter 4}


\pagebreak
\chapter{Service for the Blessing of a Baby}
\parskip 0.75em
\parindent 0em

{\larger

I love the Lord, because He hath heard my voice and my supplications.
Because He hath inclined His ear unto me,
therefore will I call upon Him as long as I live.
The cords of death compassed me,
and the pains of Sheol get hold upon me:
I found trouble and sorrow.
Then I called upon the name of the Lord;
O Lord, I beseech thee, deliver my soul.

Gracious is the Lord, and righteous;
yes, our God is merciful.
The Lord preserveth the simple,
I was brought low and he saved me.
Return unto thy rest, O my soul,
for the Lord hath dealt beautifully with thee.
For Thou hast delivered my soul from death,
mine eyes from tears,
and my feet from failing.

I will walk before the Lord in the land of the living.
What shall I render unto the Lord for all His benefits towards me?
I will offer to Thee the sacrifice of thanksgiving,
and will call upon the name of the Lord.

Blessed art Thou,
O Lord our God,
King of the universe,\\
who bestowest mercies upon thy children.

Blessed art Thou,
O Lord our God,
King of the Universe,\\
who hast kept us in life,
preserved us and enabled us\\
to attain to this occasion.


Lo, children are an heritage of the Lord;
and the fruit of the body is His reward.

There is none holy as the Lord;
for there is none besides Thee,
neither is there any rock like our God.

\clearpage
\textsmaller{The mother will say:}\\
My God,
I turn unto Thee with a deeply grateful heart,
for the deliverance which Thou hast wrought for me.
By thy help,
I have come through a time of great trial,
through pain and weakness, unto strength.
Thou art our guardian.
Thou enrichest our life with many gifts.
Humbly we behold Thy goodness in the gift of children
for which we offer the fullest thanks of our hearts.
And I beseech Thee, my God,
help me and my husband
to be worthy of the blessing
Thou hast given us in this child,
and to show our gratitude to Thee
by unwearying efforts to lead him
in the way of righteousness and holiness.
Teach us so to guide and instruct him
that he shall grow up to be loyal to Judaism
and a worthy member of the Jewish community.

\textsmaller{Minister:}\\
Dear God and Father,
we entreat Thy fatherly love for this child.
May he learn more and more,
as he is more able to know,
that Thou art ever near him,
growing up in Thy presence as in the sunshine,
and keeping the sense of it all the days of his life.
Open his eyes to see the beauty and wonder of the world about him,
so that there may be a perpetual spring of freshness in his heart.
Still more,
let him learn to love the goodness that is in man and woman.
We do not ask that he shall be spared the trials and troubles
which no human being may escape.
But we humbly beg for him a strength that shall increase with his years,
and courage to face and subdue all evil.
Grant him above all a loving heart that he may live to do thy will in faithfulness.
Cause him whose name shall be [\textsl{baby’s name}]
to attain unto virtue and knowledge,
and to be a blessing.
Bless him with long life.

{\centering\larger[2]
  The Shema may be recited

  The Priestly Benediction

}

}

\end{document}
