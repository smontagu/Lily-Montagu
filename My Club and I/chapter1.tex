\chapter{Club Leader}

I want to speak of myself in relation to the Club which
I helped to found more than forty-seven years ago. It is
easy to trace in the influence of my home environment
the creation of ideals which affected my career as
a Club Leader.

My childhood was a very happy one. I was one of a
family of ten children, six girls and four boys. I came in
the middle of the family, having four sisters and one
brother older, and one sister and three brothers younger
than myself.

Our parents gave us by example rather than precept
an insight into Orthodox Judaism as a great and wonderful
inheritance, which must be held in the utmost reverence,
and which would exist for all time. In all their
piety, Judaism was, in my parents’ view, not directly connected
with the problems of everyday life. The observances
upon which, to them, Judaism was based must be
observed with the utmost rigidity, and obedience was a
supreme act of religion. This point of view was made
plain to me later on, when I visited the homes of my Club
members. It helped me to understand how it was possible
for the heads of families to “lay Tephillin,”\footnote{The Tephillin
Commandment is embodied in Israel’s manifesto (see
Deuteronomy Chapter VI, Verses 1 to 14, in particular Verse 8, where we
are ordered to bind the law of God on our eyes, that we may look on life
with love, on our hands, that we may do acts of charity, and over our
hearts, so that we may feel love for our neighbours. The observant Jew
fulfills this commandment literally with leather bindings, full of symbolic
meaning.} and mumble
prayers in a corner of the living-room, while the rest of
the family were chattering with one another, and performing
many acts connected with domestic life. These
people were quite unconscious of acting irreverently. They
were each getting on with the task in hand, and were not
affected by the deep spiritual meaning of the “Tephillin”
observance. For them, the main spiritual value was in
obedience.

I remember once, on the Day of Atonement,\footnote{The
Day of Atonement is the most solemn day in the Jewish calendar.
It is the day when by prayer and repentance we seek to harmonise our
lives with the idea of God. When we fall short of the standards which
have been given us, we sin against God, who has endowed us with the
power to choose good and reject evil, turning ourselves from the evil which
separates us from God. Our fathers prescribed fasting from sunset to
sunset, and those of us who can forget our bodies during those hours are
much helped by the opportunity to give ourselves entirely to thought
and prayer.} that my
brother Edwin (afterwards Secretary of State for India)
had a very severe headache, and, being always of a nervous
disposition, he alarmed his old nurse so much that
she sent for a doctor, who promptly ordered food. I can
still recall the note of horror in my father’s voice when he
said to the doctor: “You ordered a son of mine to eat
food on this day, and it was not a question of life and
death! How could you?” Already, at that time, this
brother of mine, who was sincerely religious in the deepest
sense of the word, and who suffered more than any of us
through being a Jew, was very unobservant and very
impatient with all forms of traditional belief. But till
that time, he had not eaten on the great Day of Atonement.
By that act, he bade fair, in his father’s view, to
cut himself off from his community.

My father’s attitude towards religious observance was
shared by the fathers of our Club members, who considered
that if men, whose religious responsibility was far
greater than that of women, acted in defiance of the
religious laws, they might still be good men, but they
could not be good Jews. It might also be possible to be
good Jews and not good men. I remember being told
while still a child that a man had deserted his young wife,
but that he was a good Jew, observing the ritual laws
faithfully. In my father’s house, however, a high moral
standard of generosity and kindness prevailed, even
though to my mind it seemed unrelated to the religious
observances practiced there.

Both my parents were invariably courteous to all those
less fortunate than themselves with whom they came in
contact. They were untiring in their industry, and had
an amazing power of concentration. When I discovered,
as all Club leaders must, how much routine drudgery is
involved in work of organisation, I was grateful for the
manner in which tasks were finished at home if once they
had been begun. No distraction was allowed to draw us
away while our work was incomplete. My father used to
let us all sit with him in his library after dinner, and
although he never interfered with our talk or occupations,
he himself worked on at his important letters and did not
allow himself to be disturbed by our conversation or
laughter.

When, at the age of fifteen, I left the private school,
Doreck College, in which we were all educated, my chief
concern was lest I should be expected to study for fewer
hours than when I attended school. I was allowed in a
great measure to control my own education, and I did
not realise that school hours should be longer than those
devoted to private study. I feared that I should lose the
educational opportunities enjoyed by my sisters who
remained at school until the age of seventeen. My parents
thought that with the fifth child it might be well to introduce
some change in the routine of education. I made
myself a timetable, and was allowed to adhere to it
strictly. I had been very happy at my school, and although
its teaching methods would not be in favour to-day, I
have always been grateful for their thoroughness. The
teaching of syntax was good, and has helped me in making
my “case summaries.” I learned to read aloud in a
manner which has pleased three generations of Club
members.

In many other ways, my school life prepared me for
Club leadership while I was quite unconscious of
the process. If the strings of names learned in
my geography lessons, and lists of exports and imports,
gave me mental indigestion, I discovered how easy it was
to get to the top of a class if you were determined to do so.
If you concentrate on being first, you can be first, and
yet know very little about the subject in which you distinguish
yourself. I acquired a multitude of red cards or
“precedences,” as they were called, throughout my school
career and, later, knew how to encourage my girls to win
stars and certificates in Club competitions, even though
these successes did not always reveal exceptional skill
or depth of knowledge. There is, I fear, an art in
succeeding in competition with others which is no real
indication of intellectual or technical ability.

I made several friends at school from among the
teachers and girls, who maintained their friendship with
me as long as life lasted: others are still in close relation
with me. One friend, Margaret Gladstone, afterwards
the wife of J. Ramsay Macdonald, first introduced me to
Club life, and revealed to me the real friendliness which
existed between her and her Club members; the kind of
friendship which is essential to successful Club life.

After I left school, I studied with Ruth Adler (afterwards
Mrs. Eichholz). We had lessons from a tutor, Mr.
T. Oldfield, who was always encouraging and never impatient
with us, but he was accustomed to prepare boys
for examination, and in consequence he talked so loudly
that the family always said that they could, if they
wished, share our lessons while they sat in rooms at the
other end of the house.

Both my parents were absolutely punctual in their
habits. I can never be grateful enough to my mother for
the very great importance she laid on the strictest punctuality.
It was almost the only thing she was strict about.
She could not bear us to be even a few minutes late for a
meal. An early riser herself, she had no patience with
people who were late for breakfast, and she was really
hurt when she discovered that she could not make all her
sons conform to her views of punctuality. Her daughters
were always in time in the mornings. We used to make
tremendous efforts to be home from the swimming bath,
or from any engagement, at the time we were expected,
and if we failed to be present at the beginning of a meal,
we knew that a very uncomfortable time under the
shadow of disapproval was in store for us. This training
in punctuality helped me to get through my crowded days
when I entered social service.

I was always quite incapable of contributing small talk
at any social gathering, and, in consequence, I was wholly
useless for entertaining guests. I was very shy and self-conscious.
Some of my happiest times were spent at my
aunt’s house, the home of Mrs. N. L. Cohen, at Round
Oak, Englefield Green. She was one of my first heroines.
I was a friend of my cousin Hetty, the eldest daughter of
the house, and was deeply impressed by the beauty of her
home life. I must have been somewhat of a burden to my
hostess since I would sit through a whole meal without
uttering a syllable, for I honestly could not think of anything
to say. However, my uncle and aunt were very good
to me, and gave me the opportunity for long private talks
on all the big questions of Jewish and general interest,
and, to my intense joy, seemed to be really interested in
what I thought. They gave me an insight into the greatness
of life. I believe hero worship is really useful as a
formative influence in a child’s life. I say this although I
don’t think hero worship is very popular to-day. It was,
however, potent in the early years of our Club, and led up
to devotion to the principles on which the hero’s life was
based. Our heroes had always to be careful not to exploit
the children’s affection, or use it for their own purposes.
The worshipper can grow beyond her heroes, but she
should do so by the natural process of adolescence which
implies a development in her sense of appreciation. Club
leaders must beware lest by their failures they shock their
young people into losing faith in life.

My parents were always glad to invite to our country
house all kinds of people; there was no class distinction.
We were proud that our father was a self-made man, and
had proved in his own life how a good education could be
secured by sheer personal effort and good manners
through natural kindness and unselfishness. One of my
heroines who influenced me considerably was discovered
through some unconventional gatherings which were held
in the house of the then Chief Rabbi (Dr. Hermann
Adler), in order to bring girls from the East End and
West End of London together for cultural purposes. My
heroine was a girl who lived in the East End and devoted
her whole life to social work. She was engaged te a young
working man, and her love story was to our minds
intensely romantic and exciting. When she stayed with
us at Swaythling, she used to read us her love letters and
tell us of her experiences among the poorest people and
in the worst slum houses. My imagination was kindled,
and I, aged about sixteen, determined to live and work
in the East End. Nothing else would satisfy me. When
I spoke of these dreams to my father, he said very gently
but with such personal conviction that I could not help
being impressed: “That’s nonsense! You can do what
you like from your own home. A Jewish girl does not
leave home unless there is something wrong with her. If
she does not marry, she waits till she is forty before leaving
her family.”

Alas, it was discovered that my heroine did not lead the
life she described to us, the life of self-sacrifice and devotion
to the people. She was all the time seeking her own
ends, and she played with the hearts of the men with
whom she worked. Mrs. Adler, the wife of the Chief
Rabbi, made the discovery, and told me very strictly that
if ever I sought the companionship of my friend again, or
had anything to do with her, she would tell my mother
of my undesirable friendship. I was much hurt, for in
my home we told our mother all about our friends, and,
when we went out, where we were going, and afterwards
all our different experiences. Annoyance at the threat
helped me, I remember, to bear the deeper pain of
disappointment, for we do not experience acutely two
emotions at the same time.

As time went on, I grew more and more determined to
do some work when the opportunity arrived. Although
we lived in a large house, and my father was a great
collector of “objets d’art” (pictures, furniture, china and
silver), I never had a feeling of being wealthy. Both my
parents were extremely simple in their tastes. My father
collected his treasures because he loved them. He taught
himself the history of art and had a most intimate knowledge
of his different treasures. He used to say that
collecting, where he was concerned, brought heaven to earth,
and he would have to go on collecting in the next world,
cherubs, perhaps, if nothing else. I never remember any
reference being made to the commercial value of these
works of art. My mother kept her own accounts, but lived
unquestioningly on the money my father allowed her. She
had all that she asked for, but she never allowed herself
nor requested any extras, and she delighted in small
economies. When she became a widow, she was rather
frightened of her great financial responsibilities and could
never persuade herself that she had enough money to
meet them. We were brought up to spend as litle as
possible on ourselves, and to make our allowance, which
was very moderate, suffice. My father was quite right
in saying that if I had a million pounds a week, I should
never have enough for all the work I wanted to do. He
advised me to make do with what I had, and to worry
as little as possible about finance. I think that has been
my policy throughout life, and perhaps I have worried
too little about the practical side of my Club work, knowing
that I was quite incompetent to make our financial
resources adequate. I did, however, always try to share
with my girls my appreciation for simple pleasures, and
my disregard for material values.

\begin{tp}{1024}
My father’s deep sympathy with me, and his pleasure
in my small achievements, made him quite pleased with
my success when I started the children’s services which
gave me my first opportunity as a “lay preacher.” These
services were held in connection with the New West End
Synagogue and attracted fair-sized congregations. They
had the approval and encouragement of our Minister and
leader, the Rev.\ S.\,Singer. They accustomed the children
to public worship, although on untraditional lines. The
liturgy being mostly in English, instead of Hebrew as in
Orthodox Synagogues, was completely understood by the
worshippers. I varied the service from week to week, and
informal talks instead of sermons proved acceptable. The
attraction supplied by these simple services encouraged
me later to attempt something on similar lines for grown-ups.
My work for Judaism among adults, as far
as it affected my Club, will be dealt with in another
chapter. I would now only dwell on the fact that my
father, influenced by tradition, did not think serious
reform could ever be effected by a woman where Judaism
was concerned. The tendency of all reform was evil in
his eyes, but he was not strongly opposed to any
unconventional, informal services which I arranged for myself
and the few young people who were interested in my
point of view. I might organise services in hotel rooms or
in halls to supplement the ordinary Synagogue services,
and, in this way, my father thought I might help some
of the weaker brethren and prepare them for real services.
It was only when I was among the leaders of a
schismatic movement and proclaimed my belief that the
Bible was partly a human and not an entirely divine book
that I caused him real pain. For the last year or two
before his death, my father felt that his “Lilchen” (his
special term of endearment for me) was divided from
him by a wall of disapproval which even great love could
not break down. We both knew the love was there all the
time, and to-day I feel that his understanding sympathy
has been restored to me. Perhaps he knows that all my
work, and especially my Club work, aimed at keeping
Judaism alive among our young people.
\end{tp}

\begin{tp}{512}
My mother’s faith was beautifully simple. When, as a
child, I had night terrors and believed on occasions that
I could not live until the morning, my mother would tell
me that my father was too good for God to let him be
hurt as badly as he would be if his little daughter were
taken from him. So she gave me courage. She believed
completely in God’s fatherhood, and that observances
were ordained by Him for the good of His children, and
that He must be obeyed. She had, of course, an exaggerated
idea of my power of speaking and preaching and
was indignant with the community for not making more
use of my services. Many a time she would tell me that
she would like to come to my services, but she could not
do so as they competed with the Synagogue services; but
she hardly ever missed one of my children’s services which
were held in the Synagogue vestry room and were
countenanced by the authorities.
\end{tp}

I have dwelt on the loving understanding of my mother
in this chapter because during forty years of my life I
have been called “mother” by a vast number of girls
and women. I feel I owe much of my power to win the
confidence of “my children” from having as my model
the mother who throughout her life had no greater
pleasure than that of sharing her children’s interests, and
whose faith in God was expressed in every detail of her
life.
