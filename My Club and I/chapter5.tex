\chapter{Education in the Club}

Our work was always carried on in close touch with the
local schools. Lists were sent to us of girls who were leaving
school at any particular time, and we made every
reasonable effort to bring them into touch with our Club.
I am afraid that our elementary schools have not so far
been able to inculcate in the majority of their pupils the
desire to pursue knowledge for its own sake. There is a
spirit of finality about the children when they come to
us. They are now old enough to go to work, and they
consider that they know enough to enable them to start,
or, at any rate, that they know as much as other people
of the same age. The utilitarian point of view has sunk
deeply into their minds. They are ready to earn money
and they are only interested in study as far as is required
by their work. It has sometimes been very difficult to
persuade the children who work eight or nine hours a
day at a certain form of industry not to go on in the evening
with the same subject. No stress has been laid in
school on the importance of a well-developed personality.
These children feel that they have to reach a good level
in the labour market; so they will do what is necessary to
render themselves more efficient. There are plenty of
girls waiting for every job, aud only the best available
workers are taken. Life is very serious and grim and
unchildlike.

In the Club we have always tried to make the teaching
as efficient as possible, but we have also suggested the cultivation
of hobbies, and even ventured to introduce new
ideas as being of great importance to learners. There is
a difference in atmosphere between school and club or
institute. The girls are attracted by the personality of the
teacher and her friendliness is important to them. Their
attendance is voluntary, and they make the most of this
fact. We have always persuaded the girls to take off their
hats and outer garments when they come into the Glub.
This apparently unimportant custom has a definite influence
in levelling the “haves” and the “have nots”
among the members, and securing a homelike atmosphere.
The girl who attends an institute belongs to a type
different from that of the average Club member. She is
more interested in the acquisition of knowledge for its
own sake. She receives perhaps more attention in her own
home where there is more leisure and more privacy. In
the Club we have to make amends on behalf of society
for defects in home circumstances.

I have often said that if the Club teacher has the gift
of imparting knowledge added to a gift for friendship, if
she can gain a grip over the heart and imagination as well
as the mind of her pupils, she can teach them anything
from Greek to lace making, from laundry to musical
appreciation. We have always tried to include in our
syllabus a great variety of subjects, sometimes numbering
more than fifty in one year. These can be classified
under the following headings:

\begin{tp}{1024}
\begin{enumerate}[(1)]
\item Classes for physical training.
\item Domestic science classes which will prepare for satisfactory home keeping.
\item Technical classes to give girls the opportunity to learn subjects which will provide them with an alternative trade.
\item Art classes which afford an outlet for creative feeling.
\item Classes in music and dramatic art which afford emotional outlet, intellectual training and also recreation.
\item Literature which supplies intellectual food.
\item Knowledge of public affairs which helps to train members as citizens.
\item Languages, including French, German, Spanish, and Russian, which help the girls in their work, and English for foreigners.
\item University tutorial  classes, conducted under the auspices of the Workers’ Educational Association.
\end{enumerate}
\end{tp}

Many of these subjects have a definite place in character
training. Quick physical movements stir the power of
hopefulness; creative work strengthens self-respect,
especially for those who are part of the great machine in
which all industrial processes are minutely sub-divided.

In the remote days of Club life classes were few. We
found early that there was no advantage in forming
classes which were not required by the students
themselves. They had to collect the right number of students
and then ask for opportunity to study their subject. It
was useless to give them what we thought they ought
to like. They had to have what they themselves actually
did like.

We were among the pioneer clubs who received a
direct grant from the Board of Education. To achieve
this, we had to keep our registers carefully and work
according to the approved programme. My sister, with
her signal accuracy, was responsible for making up the
returns on the elaborate registers which were required
before the simplification was introduced. I assisted her in
the checking, and on this account we had hilarious
experiences during the night watches (for we had not time
during the ordinary working day for these extras), while we
sought to get the entries to tally with the summaries which
were required at the end. (All through our working life
we have been able to find fun in doing the dullest of jobs.
This has been one of the reasons why, with very little
definite recreation, we have always been able to find
an adequate amount of amusement in life. Even the tasks
which involved the dullest routine, and a large amount of
definite grind, could be of use in amusing us.) For a while,
ours was among the institutions which had teachers from
the L.C.C. as well as a grant from the Board of Education.
Thus we received assistance from two public
authorities and rejoiced in our double advantages, though
these advantages were too great to last. While they did,
we were much helped financially and we could enjoy
comparing the Board of Education inspectors with those
sent to us by the L.C.C\@. The point of view was definitely
different and each was helpful in its way.

\begin{tp}{1024}
For the long years during which we have worked under
the L.C.C., we have been deeply grateful for the courtesy
and kindness and encouragement shown to us. Of course,
we have always chafed at the L.C.C. requirements in
regard to numbers. Every institution has chafed in a
similar way, but to no effect. We have recognised the
necessity of ensuring that the services of the state-paid
teachers were not wasted. We have, however, regretted
that so much greater attention seemed to be paid to the
presence of the students than to their educational progress.
Moreover, we have always felt that a period of probation
should be allowed, during which the students could be
tested. If the subject and the teacher were approved,
numbers were bound to follow, but it seemed a pity that
the few earnest girls should be penalised when they were
keen on a subject because, for the time being, they could
not draw in sufficient other students. Yet we owe an
infinite debt to the L.C.C., and they have been very patient
with us and accepted with great toleration the excuses
invariably offered when the life of a class was threatened
before they finally withdrew the teacher. We count
among our L.C.C. teachers some men and women who
have been the best possible friends to our Club. They
have known and been personally interested in our girls,
have recognised our special difficulties, and have been in
sympathy with our various aspirations. Although
peripatetic, it is wonderful. how these teachers have been
able to throw themselves into our Club life, and to give
extra time whenever required, especially to assist in
functions.
\end{tp}

Dramatic work has been a great feature in our Club
life. Jewish girls and boys seem especially gifted for
this form of art. It is amusing to note the change in
public opinion which expresses itself in the changes in
Club dramatic work. At the beginning, we were anxious
lest we should give a handle to our critics, who warned
us that we were harming our girls by making them eager
for the stage; we were therefore careful to restrict the
number of dramatic classes and the amount of time
devoted to rehearsals. Girls took the parts of boys, and
we were very particular about the choice of plays. The
emotion of our young people must not be too much
stimulated in the Club. As time went on we became more
ambitious. No good plays could be produced unless boys
and men took the male parts. We began mixed classes
with much fear and trepidation. We even had men
teachers! The result from the dramatic standard was
good, but, as always happens with the timorous, our worst
fears were realised. Many engagements followed. In
one instance, a teacher married one of his pupils. Worse
and worse, we found our classes of social as well as of
artistic value. Gradually, however, we lost our fears and
the tone of the classes became very good. To-day we
have several students who actually come up to stage
standard, and we are glad that they should take part in the
life of the real theatre. In our Club we have a good stage
and curtain and lighting effects, and we aim at being as
proficient as possible.

Our operatic company has also become a feature in
Club life. We find that after a day in workshop or office,
or in a very responsible business position, our young
people are disinclined for the studying of serious opera,
but the plays of Gilbert and Sullivan, with their gaiety
and brightness, serve as real recreation and give immense
pleasure to those taking part. The operas, when produced,
attract large audiences, and here again our members
rejoice in the opportunity of giving pleasure to others.
Our technical classes have been valued for themselves
and for their practical usefulness, but it must be confessed
that they also afford a great opportunity for conversation.
The capacity of Club members for talking is something
astonishing and it is immeasurable. Club noise is
cheerful and clean. It expresses good fellowship and a
complete oblivion to the nerves of those who are not actually
taking part. The flow of talk is quite inexhaustible. There
are no lulls as are observed among other groups engaged
in conversation. The high-pitched voices go on and
on—bless them! Well, noisy classes are very popular and
the successful teacher does not try to still conversation—indeed
she might as well try to stem the flowing stream
with a teaspoon—so long as it is compatible with progress
in education.

We have also our serious classes, for which concentration
is necessary. The teacher of literature provided by
the L.C.C. has been able to make some of our girls into
real book-lovers. She has stimulated thoughtfulness
among her pupils: she has revealed her own interest in
and enjoyment of good literature, and the girls respond
in a wonderful way.

The study of sculpture has always been a feature in
our Club among a small group who work with real feeling
and an appreciation of the beautiful. I know that in
all art work a background of culture is of incalculable
value, but we find in Club classes that our young people
do discover something which undoubtedly exists in the
depth of life itself, and is essentially beautiful. It is the
same discovery which the erudite make after fumbling
with the key which opens the door into the highways
beautiful. Some of our girls and boys plant themselves
in these places by the right of their humanity which
partakes of the divine.

As with the evolution of the dramatic class in relation
to public opinion, so we have seen extraordinary changes
in physical training. There was a time when drill tunics
were worn well below the knee. I was often so severely
criticised for allowing shorter garments that I used to go
about with a yard measure before Club displays and
myself pass the garments which were likely to give offence.
I was told that if I treated these matters lightly I should
certainly be leading my girls to immorality. I took
particular notice because my chief Club critic was an aunt
whom I loved and greatly revered. Not so my girls.
“She must have a dirty mind,” they said. I remember
at one Club exhibition a lady’s coming to me and
appraising all our work, the wonderful behaviour of the girls,
their demeanour and happiness. There followed an
eurhythmic display. She was so affronted that we had
to return her subscription of ten shillings. She could no
longer be connected with a Club which showed such
depravity. Yet by gradual changes, the tunic grew
shorter and shorter until, to-day, it has disappeared
altogether, excepting perhaps for the purpose of eurhythmics,
and the girls rejoice in the greater freedom of movement.
Indeed, their morals do not suffer. These are best
cultivated by the practice of self-control, which results from
religious training and through definite instructions in the
laws of life.

I have always approved sex mstruction in Clubs and
unrestricted questions after each lecture. I do not deny
that such instruction, however well given, is only second
best, though we try to get the finest lecturers
who approach the subject from the religious as well as
the medical angle, and give to the consideration of
childbirth all the reverence it deserves. The best is that given
by the mother to her little child long before adolescence.
The child is thus saved from associating with the
beginning of life anything sordid or unclean. But here again,
the Club must compensate the girl for the inability of
the home to supply her with adequate education for the
development of her life.

There have been years when the citizenship classes
have gone remarkably well. Here and there we have
interested girls in trade organisation, or international
problems, and this knowledge has affected their whole
life. Individual girls have represented the Club at
important conferences. We are glad that a few of our members
have held in the past and still hold important posts in
social service. As health visitors, nurses, teachers, matrons
of children’s homes, club workers, some of these girls have
won distinction, and they attribute their success to the
training they received in the Club. It must, however, be
admitted that we have not been able to raise a very
general or well sustained interest in citizenship, and we
deplore this fact. Miss Levy’s determination for ultimate
success is still strong, and is not likely to fail, especially as
the need is so great. On Club holidays, we have had
opportunities for discussing newspapers with our girls and
for awakening interest in public life and international
problems. We are responsible in a great measure for the
use made by the girls of their scanty hours of leisure.
Some of this time must undoubtedly be devoted to training
in citizenship, or we are unfaithful to the democratic
ideal of our time. We must have thinking citizens. We
see in our clubs, moreover, the bad effect of undigested
slogans. Religious belief affects our political creeds. The
two are related in the world, and our Club should be in
miniature a world at its best. All our dreams for a new
and enlightened community must find realisation on however
small a scale in our Club.

We have always experienced difficulty in our educational
work on account of the mixed ages of our Club
members. We have never disregarded the supreme needs
of the junior and adolescent members, and we know that
it is not easy to get these into the same class with the older
students, whom they are inclined to regard as “antiques.”
The junior is naturally more adept in physical movements,
and her recent connection with school makes study
less hard for her than it is for the older student.

Certain classes have to be divided according to age, but
I have always felt that the senior Club member is quite
as important as the junior. She may, in the popular
jargon, “be able to take care of herself” more successfully,
but she knows a degree of loneliness and sometimes
a bitterness against the circumstances of her life which are
unknown to the child who is starting on her life full of
hope and expectancy. Sometimes the older girl thinks
that through circumstances she has been unfairly deprived
of her personal happiness and usefulness, and this feeling
may lead to an anti-social frame of mind unless, in an
atmosphere of great sympathy and keen activity, she has
the opportunity to find an outlet for the expression of
her own personality.

I have only touched on some important features in
our Club educational curriculum. During the many years
which have passed in which my sister and I studied the
absentees in making up our registers, or analysed the ages
of our members, we noted in our returns certain important
facts which were of use in our work as a whole. So
to-day the pulling up of absentees is a great deal due to
the unremitting effort of our Club leader. To her, as to
us, each age recorded indicates a girl or boy with special
needs and aspirations, never a mere number. Our Club
leader knows and cares for each one separately. All
through the years we have found that the best pupils are
the adults, those over 21, who have learned the importance
of education through the experience of life’s values,
and the need for concentration which brings security and
self-reliance, without which life seems futile. We must
find the way to awake and retain the interest of the
younger people in our Club, for we cannot survive their
indifference. One of the joys of Club work lies in the fact
that its possibilities are without end, and we, as leaders
and workers, never finish our apprenticeship. Each new
difficulty is a challenge to greater effort. So far we have
attained some success on the educational side of Club life
by making our members want the best, and then trying
to give it to them. Because there is a fringe of apathetic
and indifferent members in our Club, as well as in all
others, we must go on ourselves learning and attempting
new ways. We thank God that we have so many unsatisfied desires, which give interest and zest to our work.
