\chapter[The Training of the Leader and Some Incidents in Early Club Life][The Training of the Leader]{The Training of the Leader\\and Some Incidents in Early Club Life}

I was young when I started the West Central Jewish
Girls’ Club, and after forty-seven and a half years I still
do not feel old. That is one of the wonders of Club work.
It keeps us young. We are strengthened by the vitality of
our young people, and their supply appears to be
inexhaustible.

At the end of the last century, it was not usual for girls
in my set to do any definite work outside the home. My
mother was a very busy woman in her own household and
family affairs, and she had also a lively interest in her
community as a whole, a very tender sympathy with
people of all classes, and a keen pleasure in making others
happy. As soon as we Icft school, my sister and I held
classes in our home for girls who wanted coaching in a
few subjects such as history, literature and French, for
which in school they had had insufficient opportunities.
We made friends with these girls, and the close personal
touch gave me my start as a Club leader. It helped me
more than the happy evenings at which we assisted once
a week at the Jews’ Free School. These evenings were
regarded more as a kind of spree for the workers than
as any very serious form of social service.

To-day, a scientific and practical training is very
properly regarded as essential to the success of the Club
worker. She must study economics and social science;
she should understand the theory of social work in many
of its aspects, have an insight into various types of Club
management, and have shared in all kinds of social
organisation.

To my shame, I must confess that I had no training
whatsoever, but I thought when I started, and that belief
has been strengthened by the years, that the basis of
Club work is friendship and mutual understanding, and I
learned much from the group of girls whom at the age
of seventeen or eighteen I tried to teach.

I also learned something from my father’s political
experiences. He was glad when my mother, my sister and
I accompanied him to his meetings. We could watch the
rows of men and women who revelled in listening to good
oratory and were swayed by it. It may be that many of
my father’s constituents came from crowded homes which
offered little sitting room. Perhaps some who attended
the political meetings were not voters at all, but people
who liked to listen and applaud, and, at times, to make
their voices heard. Occasional heckling and unimportant
rows added to the excitement and interest of these
meetings. The vast majority of people came because
they liked to listen and were interested in and sympathetic
to the point of view which was expounded. As I sat on
the platform, I watched some of these intelligent but careworn
men and women before me, and saw how their faces
brightened, and some of their burdens seemed to lift as
they became absorbed in the speeches to which they
listened. I began to understand the meaning of true
recreation which must appeal to the whole personality of
those for whom it is planned. I saw the effect of a good
story on certain types of people. The laughter began as
a trickle which went slowly along the lines: then, if the
story deserved it, it roused suddenly a loud guffaw which
was immediately followed by a burst of laughter which
would be about to fade away when the man of humour
who was slow in the uptake saw the joke. He laughed
loud by himself, but a few seconds later the applause
would be redoubled and prolonged. The slow man had
contributed greatly to the success of the story. When the
audience pulled up their chairs and settled down once
again to follow the speeches closely, they seemed more
neighbourly, and the speaker’s appeal won a better
response. The interest of the audience was more unified.

In this way I learned that people can be held by a
common interest, and that their personal sorrows can
be modified if their vision is widened. If, as at those
meetings, the humblest person seemed to count, if his
opinion is received with deference, his self-respect is
quickened. If, after our English method, a man’s cooperation
is sought, he is glad to come in and offer his
willing support. He is not to be won by an attempt to
dominate his mind, or to force his personality. His personality
must be respected and then he can be used for a
big cause. Moreover, when, at a political meeting, the
audience and the speakers have come to understand a
joke and laugh heartily together, they can hope to work
together in unity and achieve something. People like to
laugh together—at something—perhaps even at their own
selves. So long as they laugh together, the way is prepared
for joint activity of all kinds. From this laughter,
co-operation may afterwards be zealous and enthusiastic.
It gives a good start. Some of these lessons which I took
from the political meetings were not forgotten when I
came to found our Club.

When I was about nineteen, I was introduced to Miss
Emily Harris by my cousin Miss B. Franklin (now
Viscountess Samuel). Miss Harris was holding Sabbath
classes in a small room in Bloomsbury. She was a selfless
worker to whom Lady Nathaniel Rothschild and
Lady Battersea entrusted the leading of Sabbath services.
Miss Harris was chronically breathless—the word to be
taken in the literal sense—in her enthusiasm about her
work. Her kindliness made her beloved in spite of her
queer clothes and her general eccentricity. She had fantastic
pet names for all her girls, as, for example:
“The Shakespearian Beauty,” “The Adjutant,” “The
Pigeon,” and “The Cuckoo,” and was eager to help them
over any little domestic trouble. It must be said that
Miss Harris believed that non-attendance at the Sabbath
Class was synonymous with moral delinquency. She
turned a deaf ear to any suggestion that it was
economic pressure, and not moral depravity, or
even disloyalty to Judaism, which caused girls to
break the letter of the fourth commandment. Miss
Harris held that Sabbath work was indistinguishable from
sin.

She asked my cousin and me to give Sabbath class
members happy Sunday afternoons, with talks on various
subjects, Shakespeare readings, small impromptu concerts,
and such-like entertainments. The afternoons proved
interesting, and we made friends with the girls and have
been in touch with some of them—now old women—all
our lives. They numbered about twenty, and, on one
memorable Sunday afternoon, they told us that if we
would move to bigger premises in Soho we should find
large numbers of girls who would welcome the formation
of a girls’ club. Then there began a hunt for premises,
and, after many difficulties and disappointments had been
overcome, we lighted on two large rooms at 71, Dean
Street, W.1. We had space, but no other facilities for a
Club. We asked the girls if they could each bring a
friend to our first social gathering, and they responded by
overcrowding our rooms. We entertained our guests with
speeches and an impromptu concert. The most important
item in our programme was a toy symphony, for which
I had enlisted the help of many friends and relatives, I
was wildly excited over the success of the first Club concert,
for I always felt a joy in numbers, bemg impressed
with their potentiality. If so many responded to our first
invitation, we could, in my view, be certain that we should
get sustained support. A few of my relatives were somewhat
critical, and spoke of the bad manners of some of
our audience, and commented on the fact that when told
about our Club plans the girls agreed to join as members
as if they were doing me a favour and not vice-versa. For
many years, long after the Club was fairly well organised,
I had to fight this kind of criticism.

There is a type of social worker who is quite impossible
for Club work. She is never satisfied with working among
people who present a pleasing appearance, who tolerate
no favours, and refuse to be met otherwise than as
“woman to woman.” I was always most indignant when
people attempted to bring the spirit of patronage into
our Club. I used hotly to explain that as my girls were
engaged mainly in sewing trades, they could emulate their
wealthier sisters in their dress and pay only a quarter of
the cost for doing so. My girls had good taste and knew
how to wear their clothes. They were often in close contact
with models, and had facilities for obtaining remnants
of good materials for trifling sums. Moreover, these
girls felt no sense of inferiority even though they had to
experience hard conditions of life due to misfortune and
sometimes. to social injustice. A Club worker must enter
on her career in the learning spirit. She must not attempt
to foist her standards on the girls among whom she intends
to work. She must study their standards, and exchange
her points of view with theirs.

It is a fact, which I gradually realised through experience,
that you cannot get success in a social organisation
unless the people whom you wish to benefit share the
difficulties and responsibilities of its management. If you
attempt to do things for girls and boys in Clubs, instead of
showing them how to do these things for themselves, you
can be certain of failure. You may give of your best, and
wear yourself out in your unselfish desire to be of service,
but the young people will remain utterly unappreciative.
It took me many years to discover this, and I had to go
through much tribulation and disappointment before I
did so.

In generalising about the fundamental principles of
Club life, I have jumped some distance from the early
beginnings of my work and the opening concert with the
toy symphony performed by ladies and gentlemen of my
acquaintance for the benefit of the Club. I must return
now to the narrower scene of the work.

Among my personal worries was the anxiety of my
mother. She was informed by kind relatives that Soho
was a part of London which was fraught with dangers
for young girls of good family. I knew at this time nothing
concerning social evils. My girls taught me a great
deal later on. But I could not see why, because I was
better bred and better educated than the Club members,
I should be less qualified than they to resist the temptations
which were vaguely talked about in hushed whispers
in my presence. When a certain relative said forcibly
that Dean Street was a street so dangerous that even the
police did not venture there singly, I did not know that
he was confusing my Dean Street with the Flower and
Dean Street of East End notoriety, and was overwhelmed
by floods of tears. Finally, it was agreed that my sister
and I could visit the Club on certain evenings in the week,
but not more than once in the same week. It was thought
that if we were allowed to go too often to the Club we
should cease to care at all for society, and already we
were wholly indifferent to dances and such suitable
interests. The rule that we could never go in the streets
alone was adhered to with particular strictness when we
wanted to visit the danger zone of Soho.

The Club was, nevertheless, open every night, except
on the Sabbath Eve. The girls seemed interested in
singing, drill and needlework, and I sought and found
among my friends those who could teach these subjects,
for there was no London County Council in those days to
help groups of Club members attempting educational
work to rise out of the slough of inefficiency. I myself
started off with talks on some great men and women
drawn from different periods in history, although I had
only a very small class. I soon found, however, that as
the head of the Club I must know every girl, and could
not devote my attention to taking a class. But my
method was quite good, and I recommend it to Club
leaders whose members desire to study history. It may
serve as an attractive introduction.

From the beginning, we suffered from three main difficulties.
Our premises were totally unsuitable for the
holding of several classes at the same time. There was no
possibility of separating girls of different ages, and
teachers and workers, as we called them, were very
irregular. They came regularly so long as no other engagement
mterfered. The girls were disappointed if they
came on a certain evening expecting to see their special
worker and she was absent. Many of these girls felt themselves
in workshop or factory to count only as numbers,
for they could be replaced immediately they happened
to fall out. In the Club they asked for interest in themselves
as human beings: their worker must care about
them as individuals. Irregularity on the part of the
worker may permanently estrange a girl and make
satisfactory membership unlikely. I remember once in
those far off days before Club relations were properly
established between workers and girls, a child’s bringing
me a telegram tightly pressed in her hot hand, her face
wreathed in smiles. The telegram was from a worker.
It read: “Sorry cannot take the lesson to-night—another
engagement.” “Why,” asked I wonderingly, “are you
so pleased?” “She took the trouble to tell me and to
send a telegram,” the girl replied with distinct rapture.
“She knows I come on purpose.” ‘The girls too were
often irregular, because their working hours were
uncertain.

I learned very soon that the members appreciated real
instruction and were impatient with poor teaching. Moreover,
the nice teacher who could not hold her class
although she knew her subject did not remain long with
us. In another chapter I shali dwell in detail on the
evolution of our educational work. From the beginning
I saw that classes were essential. The girls told me that
they did not want to come to the Club just to laze about.
Unless some definite reason for coming was provided for
them, they were so irregular that they did not come under
Club influence at all. It was therefore necessary that the
workers should be efficient. It was no use saying: “So
and so is such a kind woman. It is so nice of her to come
such a long way. She can’t teach, but...” The poor
lady was just wasting her time by coming at all. Some
people were unfair enough to consider our girls ungrateful
if they did not thank them for teaching them badly.

But why should this be expected from them? The girls
were for the most part tailoresses, working from 8 o’clock
in the morning till 8 o’clock at night, with an hour for
dinner and half an hour for tea if they happened to work
for an employer who kept the Factory Acts conscientiously.
They knew that since their education was
“finished” when they reached fourteen, the age when
children of another class began their serious education, it
was well to use some of their leisure time in compensating
themselves for their short schooling. I cannot assert that
all the girls felt this. Some who came to us at fourteen
believed that they had learned all that they needed to
know for they had passed the school standards. It was
part of the good teacher’s function to make the adolescent
children see that they knew very little and through
ignorance, were very likely to miss some of the greatest
joys life could give. To do this, she had to be full of
sympathy and understanding and also very gentle and
tactful.

These children who had just entered the labour market
were strong individualists. They had to make their own
place in workshop or office. They were very precocious.
When they were allowed peeps into the Club worker's
life, and began to feel her friendship and recognise the
source of some of her happiness, they themselves got back
some of their natural youth and became more receptive
and amenable to suggestions about the need for work
before difficult subjects could be acquired. Our girls
gradually responded to the education facilities offered to
them. It meant an effort, however, for them to come
regularly, and they would not come if the teaching were
only make-believe.

I should like here and now in this early part of my
book to protest against a way of thought among the public
which is due to ignorance and want of sympathy. I protest at the manner in which girls are lumped together as
“Club girls,” as if that epithet gives them certain characteristics
which are inevitable, and from which they cannot
escape. Even when our Club was small, I used to be told
that so and so was’a Club girl, and another person was
just like a Club girl, and that such and such an entertainment
(which no human being could enjoy) would go
down well with Club girls. When our Club reached a
membership of several hundreds the same general epithet
was considered to be sufficiently descriptive. As a matter
of fact, Club girls represent a great variety of character
and attainment. One of the joys of our Club has always
been just this variety. From the first, we had our frivolous
and our serious-minded girls, our purposeful and our
people of light make, our strong characters and our feeble
ones, our very intelligent and our backward people, our
energetic and our lazy girls. But I need not make an
exhaustive list. Our members in the early days represented
nearly every type known in the world. Indeed,
the Club was a little world for us, and the events which
occurred there were of supreme importance to the members
and to the young leader who felt herself, quite
wrongly of course, responsible for their happiness.
After the educational work was arranged according
to some sort of timetable, and the social evenings were
becoming fairly popular, and we had gathered together
about a hundred members, I heard, while I was staying
in the country with my family, that the owner of the
house at 71, Dean Street, had decided that he needed
the whole of his premises which included our rooms. My
misery was unbounded. The new term’s programme had
been prepared. The girls had got into the habit of expecting
much from their Club. Many had no other place to
which to go outside their homes, which were uncomfortable
and overcrowded. It seemed to me that the whole
world would come to an end if the Club stopped. I was
not allowed to go to London myself and look for rooms.
That would hardly have been proper, but an old aunt
of mine, Mrs.\ Ellis Franklin, was in town, and in the kindness
of her heart undertook the task and soon found what
she considered suitable rooms in a house at 8, Frith Street.
We were very grateful indeed for her recommendation,
and established ourselves there as quickly as possible after
I returned to town. It was long afterwards that some
knowledgeable people told my elders that our rooms were
next door to, if not part of,.a bad house. Nevertheless,
they did very well for our purposes for a short while
before we took a very important step forward in our progressive
existence. Must not our guardian angels be
amused and often smile at the deeds of kindly eunts,
whose own lives have been always well protected, and
who desire only what is safest and best for their young
people whom they hold in sincere affection and regard
with tender solicitude?
