\chapter{The Future}

The story of our Club began in 1893 and has been told
as far as April 16th, 1941, when our premises disappeared
in ruins. The loss of a building rendered precious by
memories covering nearly half a century is rather overwhelming.
It is not surprising that it has affected our
sense of values. We had been so busy keeping records,
and working up a very elaborate organisation. We hoped
that we should thus serve those who came after us,
ready to take up the reins as they fell from our hands.
But in a few minutes these records were changed into
rubble.

If I had been told a few months ago that there could
ever be a new start, I should have ridiculed the
idea. Since our tragedy occurred, however, we have
found widespread encouragement and sympathy. Not
only have our present and past members rallied round us
with expressions of affection and loyalty; not only have
we received innumerable messages of love and hope, but
we have actually gathered together large sums of money
from all parts of the community. Our members and
their friends collected over £1,200; Dr.\ I. Feldman
received in response to an appeal from the general community
more than £800, and I had another £200 sent
to me from personal friends. People who were, we
thought, quite indifferent to us have shown a practical
sympathy which moved us deeply and encouraged us
greatly. Before the war began, my sister and I were
considering a graceful retirement from Club work, and
the possibility of using our small remaining strength in
furthering other work which we also cherish sincerely and
have been obliged to neglect in the course of years. The
bombs have, however, challenged us to continued effort.

We cannot leave our people without a home or plans
and means for reconstruction. Our community has
shown that they expect something more from us. We
have had more than five hundred letters of encouragement
from our own members, from friends within and
outside the community, from responsible communal
organisations, from colleagues and societies representing
all denominations. They all say: We are behind you:
You have our trust and affection and loyalty: Why not
go on? We smile in our bewilderment. It is expected of
us that we shall not stop. We have to start again although
we start from scratch. All our collected visible material
has gone.

The lives of our members and associates furnish for us
our records. Perhaps we relied too much on material
things, and did not realise sufficiently where our strength
lay. Perhaps we cared too much for our beautiful building
and equipment. God grant that as we prepare for
our new adventure we should know how to be sufficiently
grateful for the things unseen, and accord to the visible
and tangible their proper valuation.

We look to the coming of peace because of the opportunities
it may bring for a new and better life for all
humanity. We pray that we may be privileged to take
part in the work of social amelioration. We believe that
our Club may play its infinitesimal part in the grand
work of reconstruction. ‘The individuals who make up
our small group are all aware of God, and with His
help their capacity for usefulness is immeasurably
increased.

When I visualise our future activities, my thoughts
go first to the Children’s Club of the future. More work
should, I think, be put into this branch of our organisation.
Boys and girls must come to us for recreational
classes in greater mumbers after school hours, and we
must give them better opportunities to start hobbies and
to care for the worthwhile things, for music and poetry,
friendship, and religion as expressed in life.

I believe that in our new Club we shall have mixed
activities on a larger scale. We shall never be able to
compete with the purveyors of West End entertainments
in providing amusement for our young people. Housing
will be so good generally that our young women and men
will not need our premises, as they did in the past, for
light and air, for a certain degree of privacy, and for the
entertainment of their friends. But they will always, as in
the past, come to us for our specialities—for personal
interest and friendship and the opportunity for training in
social service. There will always be a number of young
people who will not be attracted by the educational facilities
provided by the L.C.C. in institutions and continuation
schools. These people will need a centre such as ours
which allows them to create for themselves and their
friends the cheerful atmosphere they need, while at the
same time compensating them through our educational
activities for the abrupt termination of their early school
life. A section of our boys and girls returning from
national service will have realised their educational
deficiencies through contact with men and women who
have had better privileges. There will always, I think,
be a desire for good dramatic, operatic and choral work.
In our sub-divided industrial world there is little opportunity
for creative work which develops a sense of beauty
and, at the same time, stimulates self-respect. Our classes
in applied art will, therefore, continue to flourish;
materialistic tendencies will be corrected by interest in
music, painting, sculpture.

We hope that in the future the training of our young
people as citizens will be better defined, and that they will
be prepared to take part in all economic, social and political
activities of a constructive kind in this district. The
Club will stimulate service in the larger community.

We shall want our Club to cater not only for isolated
individuals, but for members of family groups. In this
way, the personal lives of our members can be affected,
and Club influence strengthened in its relation to the
home background. Young married folk will be accommodated
in our Club, which will provide recreation and
education for them.

We believe that the democratic characteristics of the
Club will prevail, and outside workers will cease to be
needed except as teachers, lecturers, doctors, dentists
and lawyers, able and willing to give professional
co-operation. The tendency of English social effort seems to
be to enable young people to attain self-fulfilment through
state organisation which takes into primary account the
claims of individual personality.

The Club will always, I hope, be maintained on a
religious basis. I believe that there may be a revival of
institutional religion when people as a whole become more
conscious of God, but that these institutions will exist not
only to preserve the teaching of the past, but to focus the
activities of those who share a living religion and wish to
send it forth intensified into all the highways and byways
in which life is spent. In this work, our Club must certainly
take its share. I hope, however, that in the future,
as in the past, our members will know that their leaders
do not hold interest in the Synagogue to be the only test
of a religious life. They will understand ever better and
better that life to the Jew must be holy in all its phases,
and although the Synagogue must help in making it so,
we can also find our God by other methods, sometimes
perhaps better suited to our personality.

The Club must ever assure its members of their need
for God, and by creating for them a high standard of
education, recreation, and social service, help them to
reach Him. Conscious of its dependence on the
Fatherhood of God, it will give its members the desire to
learn and to act, to aspire and to serve in the cause of
Judaism, for the benefit of humanity.
