\chapter{The Grandchildren of the Club}

As our Club approaches its jubilee, it is of course to be
expected that we should number among our present members
a great many grandchildren. And what of their
mothers? I think that we as a Club had little to teach
our girls of the privileges of motherhood. Their
homes gave them striking and impressive examples of
family love, and especially of parental devotion and the
power of sacrifice. In Club classes however the members
had the opportunity of improving their technical knowledge
of home craft, and of studying cooking, nursing and
every branch of hygiene. In our mutual confidences,
especially on Club holidays, we had many opportunities
of talking over home ideals and sharing views on the
relations between parents and children. We liked to read
that poem “Grown Up” contained in Elizabeth Waterhouses’
little book of “Life and Death,” especially the
verse:

\begin{verse}
“He has griefs I cannot guess,\\
He has joys I cannot know:\\
I love him none the less,\\
With a man it should be so.”
\end{verse}

We knew full well that there is a time force which divides
parents and children and makes their standards and
aspirations diverge, but the bridge of love can span
this separating gulf. We used to discuss how the pain of
growth could be mitigated by love and trust.

Our Married Members’ Guild has been the means of
furthering Club ideals and strengthening them in the lives
of the grandchildren. Our members are convinced that if
the modern mother is to be able in any degree to retain
the friendship of her children, she must have a wide
outlook and be interested in a multitude of subjects.

Our grandchildren sometimes tell us how their mothers
love to talk of the old Club life and its great occasions,
and to compare the past with the present. The grandchildren
are interested in hearing of the distinctions
which West Central members have won, and realise that
the Club training has stood them in good stead. Our
Club Magazine, or “Club Link,” as it is called, serves to
give contributors an opportunity to express their views on
Club life as it is and as it may become. The grandchildren,
under the guidance of Miss Levy, learn to the
full the value of this Club training through sharing in
the responsibility of Club management, and understand
that social work can be done without expectation of
recognition or reward beyond that of knowing a piece of
work has been well done. The worker in our Club has to
give of her very best, and to give it completely; nothing
else is acceptable. The grandchildren have without question
been brought up to have a high conception of social
service.

Few girls to-day are in danger of losing their chance of
marriage by giving too much time to Club work. Before
the mixed element was introduced into Club life, we
were told that Club interests absorbed too much of a girl’s
time. I remember one married member’s telling me that
if there had not been a pause in the succession of Sunday
teas she would never have met her husband, for she would
not let anyone else officiate at the Club tea table. The
modern girl’s life is happily too full of interests and distractions,
and she is too much bent on fulfilling herself to
run the slightest risk that Club functions might interfere
with her permanent happiness.

The Club grandchildren then do not want to emulate
their mothers in concentrating on Club activities. For
them life is full of more various possibilities. They want
to cling to their Club, but its ways must be somewhat
different from the old ways. It must give new opportunities
for education and recreation. Even the small children
of ten and eleven must have every variety of amusement.
However delightful the pleasures may be in themselves,
they pall if they are the same for even a short period. The
Club grandchild will not tolerate, as her mother did, the
same form of entertainment week by week, or the same
course of study.

The future will reveal how far this hunger for change
is good, or if, while in essence tending to useful progress,
it does not make a business of pleasure and deprive youth
of its highest joys and finest human privileges. These are
passed over too quickly by those who are constantly
engaged in the search after more and better entertainment,
and who demand, above all, change for the sake
of change. Our grandchildren have little time in which
to learn to know themselves, or to collect their powers
before they try to leap forward. They are always eager
for the next leap, and are prepared to go forward without
much thought or preparation.

On more than one occasion we have attempted holidays
for Club members and their children. These have
been only moderately successful, because no holiday hostel
is big enough to contain all the perfections attributed to
the children by their mothers, perfections which are not
always recognisable by other people. Little rubs are bound
to occur, and these lead to disappointments and even to
hurt feelings. The grandchildren as small children have
had their own holiday parties, and these have been
delightfully happy. For many years the children were
entertained in the Green Lady Hostel, Littlehampton, the
home of memories for their mothers. They have also had
holidays in school houses taken for their benefit during
school holidays. These buildings were often beautifully
adapted to produce happiness for the children, and the
playing fields were a great asset. We are grateful to the
staffs of these schools who gave the chance of sharing the
advantage of their school equipment to those for whom it
would otherwise have been quite inaccessible.

The Children’s Club has played an important part in
Club history, and to-day its members include a large percentage
of Club grandchildren. Under the leadership of
Miss K. Lindo, Mrs.\ Walters, Mrs.\ Gerald Montagu,
Miss Hyman and now Miss M. Simonis, the children have
learned to play and have been taught the elements of
Club discipline and the principles of Club loyalty. The
Club offer hospitality to a troop of guides, and the Children’s
Library has been a feature of the Children’s Club.
On Open Days we have been able to show good results
as far as these can be shown, particularly in dramatic
work, elocution and dancing, but the influence of the
Children’s Club cannot be shown in exhibition form. It
has entered into the minds and hearts of the children, and
will undoubtedly bear a good harvest. My colleague
Miss O. Lazarus has been leader-in-chief of the grandchildren.
It is she who has organised for many of them
their religion-school and services, and through visiting
their homes she has been on intimate terms with many
of them since infancy. Every day our grandchildren
prove more strongly their affection for Miss Lazarus,
their mother’s friend, and show their trust in her
wisdom.

It is important that a Club should if possible influence
its members before they go out into the world as wage
earners. Once in the labour market, they are likely to
become individualists, for at the age of fourteen they have
to fight for their position in the economic and industrial
world. Before this time arrives, they should have learned
through the Children’s Club the democratic ideal. We
believe that every human being should be free to develop
her personality, and that conditions should be secured for
her to do so. The development of the individual must not,
however, interfere with the development of society. If the
member’s conduct detracts from this wider obligation, her
freedom must be restricted. In the Children’s Club, we
attempt to prove the meaning of organisation in so far as
it reflects the needs and aspirations of modern society. The
value of human personality is emphasised, and it is shown
how through service alone this personality reaches fulfilment.
Children must be taught the value of freedom, not
for itself, not as synonymous with licence, but as the
opportunity to further the best life of the community.
