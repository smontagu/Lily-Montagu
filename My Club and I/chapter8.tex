\chapter{The Mixed Element in the Club}

I have always felt that our Club should remain essentially
a girls’ club. From quite early days, I have wanted to
introduce boys and men for social purposes and for
certain educational work, which is best done from a
community angle. The dramatic and operatic classes can
certainly not do the best work unless men are allowed to
take male parts. Also, we must follow the modern
tendency towards free and complete co-operation among
men and women in all social and industrial activities.
Nevertheless, so long as clubs worked on an educational
basis exist, I think we can achieve most if we try to give
opportunity to our boys and girls to realise themselves as
separate groups in their clubs. Of course, we must allow
as much free social intercourse as possible, and
co-operation for civic purposes, but since by nature boys and girls
are not the same, they need separate training. Few clubs
are sufficiently large to allow boys and girls to have
separate rooms for physical training and for manual and
technical work, and for the many other subjects such as
natural science and domestic economy, in which the
sexes are not identically interested. Moreover, I think
that we develop the best tone and the keenest sense of
responsibility when each sex is separately responsible
whether as hosts or guests. The tendency among clubs
which are entirely mixed is, I think, to deal less
thoroughly with character training. We have to decide
whether our primary object is to give our young people
through the club the best possible time, or to train them
for the highest citizenship. The adolescent period, at
which most of our young people join our clubs, is greatly
influenced by sex emotion. Unless they have always been
educated together in well-managed co-educational
schools, most girls, during the adolescent period, are
inordinately excited in the company of boys. Our members
should have the chance to know themselves and to allow
their characters to attain a certain stability before they
share all their activities with boys. They should come
together frequently rather than invariably, by special
arrangement rather than by right. In this way, each sex
finds the other more developed and self-controlled when
they do seek closer social contact. We get the best out of
our boys and girls when they have this contact as a privilege
rather than as a right.

If we had been able to have our Children’s Club on as
big a scale as our Adult Club, I think we might have
modified the sex excitement among our adolescents
through their early training; but we have not had sufficient
workers to enable us to do this. I do believe that at
whatever age the boys are tackled, men Club leaders are
necessary for the success of boys’ clubs, and we have not
been able to find the leaders who would assist us. We
were instrumental in advancing the foundation of the
Boys’ Club in Fitzroy Square, and have always worked in
friendly relation with the leaders, our members interchanging
hospitality whenever possible. But the methods
and ideals of the two Clubs have differed fundamentally.
Close co-operation has therefore not been possible. On
several occasions in the past, our senior members have
tried to assist in the formation of a club for men and
women, but their joint efforts have not been fraught
with great success. The good tone of our Club mixed
evenings has been secured ever since we made our girls
responsible for them. Our system of associate men members
has been in operation for a great number of years,
and has joint administrative work in all Club activities.
The men workers have been remarkable for their hard
work and devotion.

The social life of our girls and boys has become freer
as with the advance of years prevailing conventions have
changed. At one time we were very strict about the
introduction of new visitors. Tickets for our dances had
to be obtained beforehand, and we were rigid about denying
admission to complete strangers. Somebody had to
vouch for the good behaviour of each of our visitors. But
to-day our young people cannot be troubled with all these
formalities. Our present Club leader has been able to
obtain the interest and keen co-operation of a number of
splendid member workers, who never spare themselves in
dealing with the details of our social functions. Our social
life has accordingly changed considerably, and the boys
and girls are glad to find in the Club the entertainment
which they desire.

\begin{tp}{256}
For very many years we have had considerable success
in dealing with the problem of men friends on Club holidays.
In the very early days we were much troubled by
the propensity—general in all clubs—of a few girls in
each party to pick up boys whose personality appealed to
them and whom they were able in varying degrees to
attract. We have gone through much perturbation over
these strangers. We used to make rules that only boys
and men who were known at home might be introduced
on Club holidays. It was easy to find such boys and to
introduce them on the flimsiest excuses. We did not like
our girls to go to dances at the seaside and to pick up
acquaintance with strangers. As time went on, we rather
encouraged our girls to invite brothers and fiancés to visit
the holiday party, and if they so desired to join in our
excursions and our Sabbath services. The openly
approved young man was not dangerous, and his
appearance was natural and desirable. Any kind of strictness or
want of sympathy with regard to boy friends produced an
uncomfortable reaction. Clandestine meetings were not
unknown; trysts were arranged after lights were out by
girls who slept on the ground floor of the hostel. But
these misdemeanours were quite exceptional and received
the strongest disapproval from the general majority in
any specified Club party. Our Club leader, Miss Levy,
has won the confidence of the girls and she is able to give
them more licence than we did in the old days. She finds
it possible to accompany young people to occasional
dances while on holiday, or to trust them to return home
on their own account at the time arranged. She urges
them in all ways to act openly and without deception
where their personal conduct is concerned.
\end{tp}

We sometimes wonder how parents, knowing as they
must some of the facts we know about holiday “pick
ups, can let their young daughters spend their holidays
in seaside “diggings” without any supervision or
restraint. With a holiday party, there is an infinite variety
of ways for girls to spend their time happily; there are
so many new and delightful mterests that the sex interest
takes its right place among the rest. The girl and boy
who go on holiday on their own may be rather bored
unless sex excitement is indulged in very freely. The
results are often quite disastrous. But even on Club
holidays, we can only feel safe about the amount of licence
we can give if we know we have the confidence of our
young people and, even more, their love. Moreover, they
must have been trained to respect themselves and to
recognise that their bodies as well as their souls are the
gifts of God. No pleasure is worth having which makes
us feel ashamed when we speak our thanks to God for
the day which is spent.

Since tennis became popular among the young people,
happy opportunity for sociability offers itself very easily.
Indeed, I think the social side of our Tennis Club has
always been one of its strongest features. In recent years,
the Tennis Committee, chiefly through the energy and
initiative of its chairman, has succeeded in acquiring its
own ground. This is a great feat, and the Tennis Club
holds much promise for the future.

In all our mixed activities, our Club leader has been
able to hold the interest of the young men, although she
would greatly welcome the assistance of a man worker
who could make his influence felt with imdividuals, and
who would, like herself, have time during the day to assist
with the general Club organisation. The men who belong
to the Club accept and approve of our attitude towards
the religious side of our work. We have been surprised
and pleased by the strong feeling they frequently express
on the importance of Jewish principles. Until the
disastrous internment policy began, we rejoiced in the
membership of a number of well-educated young refugees who
regarded the Club as a home to which they turned with
affection and appreciation. ‘They were helped by the
friendliness of the other men and by the sympathy and
care they found in the Club as a whole. We provided
English lessons for these boys and they made excellent
progress; but above all they valued the personal interest
and kindness which were shown them in the Club.

\begin{tp}{512}
We were rather surprised at the strong protest the men
associates made when we mooted the plan of having
non-Jewish members in the Club. Although they themselves
are almost completely non-observant, they express a deep-seated
Jewish loyalty in their attitude. Possibly their
protest if traced to its origin would have much to do with
fear and the fact that the men feel themselves without
any strong religious anchorage. I always feel that the
only real safeguard against mixed marriages—and it not
infrequently breaks down—is zeal for the spread of
Judaism. If we have been brought up in a home in which the
religious influence is vital, and we are imbued with a
desire to bear witness to our faith and add to its strength,
and pass it on to those who come after us, we may be
able to resist when we are tempted to yield up our loyalty
to our ancestral faith. Our boys have had no such teaching,
and have little sense of religious values, but still they
do not want to yield. They know that inter-marriage
spells the disruption of our community. They don’t want
to bring trouble to their parents. They were therefore
strongly opposed to the introduction of non-Jews as
members and did not see that religious allegiance depends on
inner conviction alone and can never be secured by the
removal of outside conflicting allurements. Our young
people opposed our suggestions so vigorously that we
withdrew them a few months ago. But the war has
softened all kinds of division, and at the present time we
open our doors to people of all religions who care to join
us. It may be that national service, throwing, as it does,
all types of believers together, may strengthen Jewish
loyalty, and our young people may be less fearful and
exclusive when they return.
\end{tp}

I wish that religious considerations would claim more
of their attention when they do choose their partners for
marriage. Again and again I find that the Jewish girl
or boy is satisfied that the partner is not a Christian, and
never thinks further or asks more. It 1s not realised what
a danger secular Judaism is to the general community,
and how it can only be overcome if a spiritual ideal is
considered to be part of the marriage endowment. Many
of our boys and girls have foreign parents and the time
force and the changed conditions of life completely
separate the point of view of the new from the old.
Parents give their children great freedom in the choice
of their friends. The old method of introducing brides
and bridegrooms through professional matchmakers has
almost disappeared in the families of Club members, and
we rejoice in this change. We are glad that a long friendship
usually precedes marriage, but we regret the daring
of some of the young people who play incessantly with
fire feeling sure they will not be burned. They do usually
escape disaster, from the physical point of view, but the
flirtations and abandonment, the light love making which
sometimes characterises “evenings in,” the want of
dignity and self-respect which tends to spoil the best manhood
and womanhood among us, must be strongly deprecated,
and through Club influence should be changed.

In the old days the restrictions and artificiality associated
with the relations between the boys and girls were
most objectionable. Deceptions were frequent because
trust was withheld. Ignorance about sex matters produced
curiosity and furtiveness. The most wonderful of Nature’s
phenomena was degraded through ugly association.
Much progress has been made, but we have also lost some
of the fine chivalry which existed between boys and girls,
and some of the beautiful self-control and respect which
prepared the way for Jewish marriages set in an
atmosphere of holiness and aspiration. Perhaps Club
influence, strengthened by sensible instruction in hygiene,
will gradually make our young people more sensible to the
real meaning of marriage, standing as it does for the perpetuation
of all that is strong and healthy and sincere in life.

Our young people will gradually learn how to value
themselves better. Self-cheapening will be recognised as
hateful. The Club will give opportunities for the best and
healthiest comradeship. Parents as the best “pals” of
their children will no longer fear them, even as the children
have long ceased to fear their parents. There will be
no playing with human life, for it will be held to be
precious in the sight of man as well as of God.

