\chapter{The Government of the Club}

We started the Club with a very sincere and, indeed,
impassioned desire to help others less fortunate than ourselves.
Towards the end of the last century there was a
keen wish among carefully brought up girls to express
themselves in work, although the term “self-express\-ion”
had not then become popular. Girls rebelled against
being mere parasites of society. Having been given a fair
education, they felt an urgent need to use their energy
objectively. Otherwise, it created havoc in their lives,
and different forms of rest cure were ordered for those
who were in reality suffering from the suppression of
their natural desire for activity. They had to submit to
the prescriptions of fashionable doctors, produced especially
for the benefit of girls in good society.

For myself I had at the beginning of my Club career
the sympathetic encouragement of my parents. They
were convinced that it was desirable and altogether right
and proper for young girls to have some outlet in social
service. It was only when the small beginning grew
until it absorbed my life, and made me give practically
whole-time service, that I came up against parental
anxiety and concern.

The accepted life’s programme for every girl in my set
was that she should go out as much as possible, know
plenty of “nice people,” and settle down in an early age
in marriage. My complete failure to conform with this
widely supported plan brought much disappointment and
anxiety to those who loved me, and the ladies of my
mother’s visiting circle, though generally kind and sympathetic,
did not all approve of my mode of life and outlook.
Because I worked very hard, dressed badly, went
out very little, was always shy and awkward at social
functions, I was held up as a warning by my mother’s
acquaintances. They said kindly, “How’s the Club?”
and, hardly waiting for a reply, kept their daughters
away from my bad influence. That question, “How’s
the Club?” though often asked, offended me greatly,
especially as the Club grew larger. It was so impossible
to say how a Club was, a Club representing several
hundreds of girls and through them influencing as many
families. It included innumerable activities, and when
one of these was going well, another was showing signs of
failure. It was impossible to report on the welfare of a
big organisation in two or three words as was expected
of me.

As I have said in another chapter, I had no training
for Club work, except that which I could find for
myself, and, as is the way with all pioneers, I had to gain
experience by practice on other people, sometimes to their
unhappiness. I did, however, very early evolve certain
principles, which, with long years of experience, I have
not found reason to change or modify to any important
extent. I learned that a Club to be of value must influence
every phase of its members’ lives, and that its keynote
must be sympathetic sharing. Before many years had
passed, we had to deal with large numbers. It was all
important that each member should count. However
attractive the Club programmes, the girl’s membership
did not depend for its value to herself on the activity
of the Club. She came to us because she knew we were
interested in her. She could talk to us. We cared.
Coming together as we did for a few hours in the evening,
we could not establish the Club on the important personal
relationship which must constitute its chief characteristic
unless we also knew and visited one another’s homes. The
visiting is such an important part of Club life that it will
need a chapter to itself. Suffice it to say here that when,
as a girl just out of her ’teens, I first deemed visiting of
essential importance in connection with my Club, I caused
great consternation and roused considerable opposition at
home. For the first year or two I had to do my visiting
in the company of a governess; on no other condition was
I allowed to undertake it. Happily, the governess in
question, who taught my youngest sister her school subjects,
was a kindly and tactfully unobtrusive woman, and
so the absurd situation was not recognised by my Club
friends; it only worried and annoyed me. My people
accepted my companion as their friend.

My parents were generously hospitable, and allowed
me to invite my friends to my home whenever I wished,
although it must be confessed that they sometimes gave
way to regrets that I wished to entertain Club members
and nobody else. Margaret Gladstone used to tell me
that I was snobbish in an unusual way: a person had to
belong to the industrial classes to be at all welcome to
my friendship! Through my invitation to Club members,
I had to combat another form of conventional
prejudice. In the days of strong class separation, we were
told that by inviting the girls to our home we roused
discontent among those who were ill-housed and lacked so
many things necessary for material and physical
well-being, to which, indeed, they were entitled as part of
their human heritage. We in our large houses had all
we needed and much to spare. Although, here and there,
it may be possible that feelings of discontent were roused,
I think, on the whole, we did much more good than harm
in establishing the right relationship between us. We
were simple people, and the spirit of our hospitality was
sincere. Our guests were brought near to us through our
showing them the beauty of our home. They were not
estranged. In my view, I could always learn much from
them, since they lived in close contact with reality, and
they had won their place in the world by their own efforts
and not by the bounty of their ancestors.

The joint country holidays did more than any other
activity to help the Club leader to understand, and to be
understood, by the Club members, but I shall have later
to speak in detail about this side of our organisation.

I felt from the beginning that through the Club I
must share with the girls the things I valued most in life:
education, friendship, faith. That was why we built our
Club on a religious and educational basis, and I tried to
make each member feel herself known and cared for.
There was no side of her life which did not interest and
concern me. I tried to learn all I could about the conditions
of each Club member’s industrial life, and recognised
that it was only with the help of sympathy and
imagination, and by building a bridge of affection that
we could unite people who were not weekly workers with
those who lived from day to day, and who might, at any
given moment, be deprived of the means of livelihood.
The girls had to be convinced that although hitherto protected,
as far as possible, from the more sordid and tragic
side of life, I was never shocked by anything I heard.
I was not interested in surface polishing, but wanted to
be drawn into my friends’ real lives and actually share
their problems and difficulties. The Club leader has
always to be a woman of big sympathies, interested
in actualities. The limitation of our Club premises made
it difficult for us to hold private conversations there with
our girls. When, after our uncomfortable sojourn at
Frith Street, we moved to two floors at No.\ 8 Dean
Street, the premises relinquished by the old Soho bazaar,
it was said that our intimate talks had to be held in the
Club passage within earshot of the Club caretaker, who
made the most of her opportunities.

Before much time had elapsed, I learned that to attain
success we must let our members share the government
of the Club; but the progress towards complete
democracy, which we have now attained, was of course
slow and painful. At first we believed that the workers
must form the Management Committee. We were told
that Jewish girls were inclined to take advantage of
authority if weakly used, and that we must assert ourselves. This fallacy was soon exploded, and we learned
that the only authority worth having was based on
affection and respect which we must earn for ourselves. There
were some workers who never attained it. They were
always taken advantage of; girls were always rude to
them, but it was their own fault, not the fault of the
girls. It is one of my very few claims to merit that only
one girl throughout my long Club career was really rude
and defiant in her behaviour towards me. Perhaps it
was with me as it was with another Club leader, a friend
of mine, who overheard the following conversation
between two of her girls. “Why don’t you tell her off
properly?” “No,” replied the other, “I know how little
she knows of bad words, and I am not going to be the
one to teach her.” Be that as it may, it was temperamentally
easy for me to win confidence and to obtain cooperation.

Since we were not continually at the Club, we were
represented by a superintendent who was responsible for
the happiness and welfare of the girls in our absence. A
completely successful Club superintendent must be God
appointed. Our ladies did not belong to that category,
and we had frequent changes and several disasters before
we improved our system. The ladies we chose were kind,
but not on the whole understanding. They worked hard,
but some saw all the things which should have escaped
their vision and left unseen the things that mattered;
they saw the girls’ little weaknesses and missed their painful
sensitiveness, and the suffering which sometimes
vaunted itself in loud behaviour, the desire to be useful
which needed stimulating, and should certainly not have
been snubbed. Our first superintendent, while never
sparing herself in doing her duty and in organising functions
excellently, would estrange some girls by saying:
“I can tell why you have come. You knew there was
something on.” If any difficulty arose, she said: “If
you don’t do what I tell you immediately, I'll report you
to Miss Montagu.” Her authoritative manner alienated
her assistant honorary Club workers, and I spent most
of my time in trying to patch up quarrels between the
superintendent and the members, and between her and
her co-workers. I recognised her kind heart and her spirit
of devotion, but I suppose I had greater opportunities for
knowing her than did the girls. The other superintendents
were of another type, but they were either rather “soft,”
or they regarded the girls as human door mats upon
whom they might trample. Their disciplinary methods
were soon on the way to depleting the Club of its members.
These good superintendents complained bitterly of
the noise of the members, of their want of consideration,
of their selfishness. The girls said they would not be
ordered about like children, and that they would much
prefer to stay away altogether from the Club than
submit to constant interference with their liberty. Before
long, we decided to do without a superintendent. We
wanted to keep our girls.

From very early days, we asked the girls to enter their
names as they came into the Club, and we kept careful
lists. Absentees were looked up, or we wrote them
postcards before many days had elapsed, and did our best to
show them that they were always missed if they stayed
away. We were told by a section of communal critics
that we enticed girls away from their homes, and indeed
destroyed family life. One leader in all kinds of benevolent
social enterprise caused me acute unhappiness by
complaining about our Club to outside committees and
in the press. She herself, though full of human beneficence,
had no idea of home life in which there was no
opportunity for privacy or room for social intercourse.
As she grew older, she seemed altogether to forget that
girls need to have some recreation, change and excitement
as much as lambs need fields in which to frolic.
She forgot that misdemeanour arises more often from
dullness than from wickedness. There was, however, a
suggestion of truth in her criticism when she said that
girls made the Club a pretext for going to undesirable
places of amusement. Occasionally there was a girl who
said she went to the Club and went elsewhere, or she
asked leave to go home early instead of waiting for the
closing assembly, and abused the privilege when it was
granted. In consequence of these failings, we were particular
about girls’ signing the attendance sheet, and, for
a while, asked them to give the time of their leaving
when they left before the Club ended. This rule was
not of much use. The hall clock was always said to be
slow, no matter what time it kept. We could only enter
the signatures in our alphabetical register and note our
absentees.

We drew up an imposing constitution before the Club
was many years old, and introduced a Club Members’
Sub-Committee, with representatives on the main Committee.
We soon found that the Club representatives
knew more about the needs of their fellow-members than
we did, and that it was a waste of time to have separate
committees of Club members. We must all be elected to
serve on the same Committee and work together. Our
constitutional changes were, of course, gradual, although
they can be described in a few words.

There was an element of fear in our hearts when the
members on the Committee asked for mixed activities.
We asked our near relations to come and help to keep
order at the first dances. Gradually we learned that good
order as well as pleasure and sociability could only be
secured if the members themselves managed their own
dances, and a mixed committee acted as hosts and
hostesses. The idea of the mixed committee has taken
root in our Club and works very smoothly in our present
Executive and in the sectional committees. The men
are appointed by the sections to which they belong, and
represent the activities in which they are interested. All
this seems so simple now, but we had to learn the basic
principles of Club management through practical experience
aided by common sense and sympathetic understanding.
We had to ask ourselves how we liked to entertain
our own personal friends in order to discover that we did
have our own way of doing so, and certainly did not want
to delegate the responsibility to outsiders.

After our procession of superintendents, I instituted a
system of Club letters, and my co-workers and I in rotation
looked after the Club on certain evenings. The Club
letter, which was prepared with extreme care, gave a list
of classes or other activities, messages to girls, teachers
and other workers. Answers and comments were filled
in, and the Club letter was returned to me when I was not
present at the Club, and upon these reports the work of
the next day and evening was based.

The Club classes, at first recognised by the Board of
Education, and also for a time by the L.C.C., and subsequently
by the L.C.C. alone, became more numerous.
As our Club premises became overcrowded we took more
accommodation in the house at No.\ 8 Dean Street, and
then, when our lease ended in 1913, not imagining that
war was impending and certainly not what it entailed,
we decided to build our own premises. It was thus that
we obtained our beautiful Club building in Alfred
Place, W.C.1. Our teachers were provided by the L.C.C.,
and that authority required to be satisfied with our timetable
and the number of students in each class.

We appointed a registrar, who helped us greatly by
keeping our class registers in good order and assisted us
in making our annual claims for the government grant.
Qur members began to help us by making themselves
responsible for every branch of our organisation. The
amount of voluntary service the Club received was remarkable,
as we gave every encouragement to the development
of this form of social service. We learned that the
aim of the Club should indeed be self-development
through service. Girls who worked in certain trades all
day took with real enjoyment an evening class in the
subject in which they were expert. The library assistants,
organisers of rambles, guild secretaries, were all Club
members. We attempted the very difficult method of
mixing girls of all ages above 14—the school-leaving age.
We thought that the protective characteristics of the
older girls would be roused, and the younger girls would
be stimulated by such contact.

Unfortunately, although we have overcome many of
the obvious difficulties arising from the mixture of ages,
it has been hard to allow sufficient privileges to the older
members without denying too much to the younger ones.
It has not been easy to give the younger girls all the
responsibilities they deserved, while the older girls grew
into women who had to satisfy their own special needs.
But there was no possibility of having a junior and senior
Club in our district, for we had not the funds nor the
workers, and we had therefore to do as well as we could
while realising the problems which faced us.

Our elections were serious and important. We divided
the Committee into two sides, workers and members, and
both sections were for many years elected by ballot by
the whole body of members. Gradually it came about
that nearly all the workers were actual members, or girls
who after many years of membership only came to the
Club to assist in the management. The workers’ side of
the Committee was elected by a restricted ballot. In
addition, each class elected its own representative, and
these members together formed a council who could
bring special matters to the attention of the Executive
Committee. The class representative was the link
between groups of members and the Executive. In recent
years a Junior Advisory Committee, with our Club Secretary
as Chairman, was formed to bring forward the
Junior point of view, and to give opportunity for training
to the younger girls.

For twenty-eight years, with the help of my sister, I
worked as Honorary Club leader, going to the Club in the
morning and regarding it as the centre for visiting work,
but doing the supplementary secretarial work from my
home. During the last twenty years, since the death of
my mother, we have founded a Day Settlement, and spent
the whole day in the building. For the last seventeen
years, it has been impossible for us to do Club work every
evening after our long day at the Settlement. For two
years, I kept the leadership in my own hands, members
working as superintendents of the evening with the aid of
the Club letter. These two years of transition were among
the least satisfactory in our Club history. Sixteen years
ago, we decided to hand over the evening management
of the Club to our friend, Miss N.\,G.\,Levy, one of the
Club workers. She had grown up in the Club, and for
many years had been the Secretary of the National
Organisation of Girls’ Clubs. Endowed with a big heart
and a very active brain, she had come in contact with
every kind of Club and Club leader. She had thus
acquired practical knowledge of a multitude of methods
and types of leadership. My sister and I had been told
that when we had no longer the strength to supervise
every detail of Club life, the Club would gradually disappear;
but our friends have been proved to be mistaken.
Miss Levy has a great gift of leadership. She
had been able to win the love and respect of the 700 to
800 girls for whom we asked her to do this work. She
has an infinite power for taking pains over details, and
never allowing small mistakes to remain unchecked. No
detail escapes her energetic control. She has been able
to win the co-operation of her fellow-members so that
she has always been most loyally supported in her work.
The old position of Club evening superintendent has
been taken over by Miss Levy and her colleagues who
are members of the Club. While I was in charge, the
Club remained pre-eminently a Girls’ Club. We had
mixed activities, but the girls were always the hostesses,
and their men friends came by invitation. In recent
years the men associates have found a good leader in
Miss Levy, and have taken an important share in Club
management. We had a guild of senior members who
acted in an advisory capacity to the general Club. Within
the Club itself, we included members of a flower guild
who undertook to grow flowers in their homes.

In latter years we have had a guild of silent friends who
pooled a small weekly contribution, giving the money to
me to assist anonymously any member overtaken by
sudden trouble. We had at one time a guild for observing
the Sabbath evening in the home. To-day, with Miss
Levy’s assistance, we have as well a mixed choral society,
a mixed dramatic section, our tennis club with its own
ground; and each section is controlled by a joint committee
of men and girls.

In the midst of all these changes, Miss Levy has
remained loyal to the old Club ideas that each girl counts
and that her presence and her absence, her personal
problems and her anxieties must be noted. She has also
endeavoured by all means in her power to stimulate the
religious life of the Club. The personal friendship existing
between Miss Levy, my sister and myself has been
such that she has never allowed us as pioneer workers to
feel unwanted while our creation goes from strength to
strength. I can never be grateful enough that she has
the power to carry on, in the fullest sense of the words,
preserving the traditions of the past and adapting and
changing old methods to meet the growing and changing
needs of the present, while all the time planning for a
greater future.
