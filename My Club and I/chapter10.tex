\chapter{The Married Members of the Club}

There is something very delightful about every meeting
of the Married Members of the West Central Club. Some
of our people are white-haired grandmothers, others are
brides who were wedded during the present war. All seem
united and all seem young when they come together.
With the improvement in economic conditions,
marriages are seldom made as an escape from monotonous
and ill-paid work. Such marriages occurred when I
began Club work. To-day, we observe among our girls a
much keener sense of responsibility, and an appreciation
of the ideals of Jewish marriage and home life. I think
the free intercourse which is possible to girls and boys
to-day leads to good comradeship and permanent friendship.
Moreover, it saves many young people from disaster.
Their minds are opened to the facts of life, and
they are less likely than were the girls of the last generation
to be deceived by sex glamour. Nevertheless, we have
to use Club influence to curb the general tendency of girls
te exalt the attraction of men just because they are men.
There is the natural dread of a lonely future. Spinsterhood
denies to girls the fulfilment of life. We try to impress
on our young girls that while marriage with the
partner of her choice offers to a woman a complete life,
the best that God can give her, any form of coerced
marriage is likely to be disastrous, and is altogether unworthy
of men and women intended by God to be helpmates
to one another. This coercion is undoubtcdly
thoroughly bad, whether it is due to interfering parents or
to meddling friends. The coercion may also be created
for a girl by the fear of loneliness, of being left in a backwater,
of being despised by her contemporaries. This self-coercion
is not always recognised by its victims. It makes
a subtle entry into human life and leads to hell on earth,
for there can be no worse life than that spent day after
day, year in and year out, with an individual you do not
love.

I wish that our young people considered a little more
definitely their attitude towards Judaism before deciding
to found a new home. It is not wise to try to found a
home on differences in essentials. There must be unity
on the vital principles of life and faith, on the acceptance
of inherited opinion and on the sense of responsibility to
the next generation.

Even before the days of physical self-protection, we
know that although many of our young people were ready
to play with fire and to take risks which might ultimately
involve health and sanity, only an infinitesimal percentage
went over the brink of conventional morality and lost
all self-respect. Our young people are full of vitality,
and religion is not always alive enough to be a restraining
influence. They have always taken risks before, and they
take risks to-day telling us that they know where to stop.
(I hope very much that there will be a reaction against
the licence prevailing to-day in all walks of life. I think
and hope that if this reaction occurs, many of life’s lost
amenities will be restored, and, what is infinitely more
important, the children of the next generation will have a
better chance to live healthy and sane lives.) Yet it is true
that their strong urge to morality has protected our people
in a remarkable way all through our Club history.

We have been accustomed once a year to give a
monster tea party for the married members and their
babies. My sister and I have tried to deliver personally
most of the invitations issued for this party and have
taken the opportunity to see our girls in their own homes.
This visiting has been among our most delightful and
encouraging experiences. On the whole, our girls have a
very high domestic standard. They have carried out
their determination, made in many cases when they
were quite young, to build up better homes than those in
which they were reared. The difference in beauty and
comfort between the homes of the present generation and
those of the last is absolutely amazing. The girls show
a keen joy in their homes, and they have invariably made
us delightfully welcome on our visits. It has been my
privilege to hold a consecration service in a fair number
of these homes, and I have not been disappointed when
I have revisited them. They are thoroughly Jewish in
character because they are full of holiness, which is made
up of peace, joy and aspiration. It is unfortunate that,
as with so many young people in these days of economic
pressure, the wives feel obliged to continue salaried work
at the beginning of married life. But before long this
practice is discontinued as the needs of the children
become paramount.

At our great annual gathering, the old girls take the
keenest possible pleasure in meeting their friends. Reunions
are very frequent among people who have not seen
one another for years. There is no doubt that our grandchildren
are among the most beautiful children on God’s
earth, and by the time the afternoon is at an end we find
that we have used to the utmost all the known terms of
admiration and endearment. We have hoped occasionally
to use the opportunity of the annual tea party to hold a
meeting and to discuss Club and Synagogue affairs, but
the children positively refuse to allow this part of the
programme to take place. The toddlers entertain us with
songs and dances from the platform; the babies express
themselves in other and noisier ways. We manage to get
a few whispered conversations with individuals, and from
every spot throughout the afternoon the company
radiates happiness. Many of our guests are experiencing
sorrows and anxieties, but they manage to suppress and
even possibly to forget these for a few hours while the
party lasts.

We try to keep in touch with the married members not
only on festive occasions, but at any time when they may
need to see us. They are among those who call on us at
the Settlement to discuss personal and family problems,
and to tell us of joyous experiences or of grievous incidents
which have occurred in their homes, Many of the
problems are naturally centred on the health and education
of the children, and we are glad to include the Club
grandchildren in the Children’s Club, the Religion
School, and, as the years pass, in the big Club itself.
Besides the Married Members’ Party, and the visiting,
and the special interviews, we keep in touch with our
married members by means of our Guild. This was established
about twenty years ago for girls who married after
a membership of at least five years. This period was later
reduced to three years, and our membership approaches
three hundred. The Hon.\ Secretary is Miss E. Bloom,
who succeeded Miss M. Pyser, and we owe much of our
Guild success to both these able helpers. The Guild meetings
are well attended. We discuss Club affairs (in so
far as they interest married members) and also matters of
general communal interest. We have addresses from outside
lecturers, and our Guild provides a responsive
audience with a great desire for information. They are
alert and keenly interested in all live subjects. As
mothers, they have become convinced of the special duty
of being ahead of their children in matters affecting the
life of to-day. We discuss all kinds of problems, but I
think the best attended Guild meetings are those in which
health problems are considered, or life in other lands,
especially when it is described by those who can speak
from personal experience or travel. Our girls tell us how
they like to be carried away at Guild meetings into something
quite new and different from everyday life. It need
hardly be added that they particularly enjoy seeing their
old friends and leaders and having personal chats with
them.

Two social functions appear in the Guild programme
in normal times and are always immensely appreciated.
These are the annual meeting held in July in Mrs.\ Franklin’s
garden in Porchester Terrace, and in September at
the Red Lodge, our own home, to start the Guild session.
Mrs.\ Franklin’s love for children makes her think of every
device for their entertainment, and while our business
meeting is in progress within doors, the children are safely
left to the rapturous enjoyment in the garden of varied
pleasures from sand castles to donkey rides. Our common
room is crowded at our autumn meeting, but nobody
minds that the chairs are very close together and the heat
of the room excessive. We are all very happy, and
although we do very little business, we produce great
amusement among ourselves.

It is agreed that no other group can outdo our West
Central Married Members’ Guild in the power of conversation.
Our Guild has in recent years arranged an
annual country excursion. Not brakes drawn by weary
horses, but motor-charabancs are used to-day, and our
people do not have to be invited to country houses for
their tea; they are able to procure this for themselves
at the seaside.at the end of their enjoyable motor-run.
Indeed, the conditions of life have changed in almost
every particular since the West Central Married
Members’ Guild was first established. But the present members,
like their predecessors, are immensely enthusiastic about
their Guild, and for the same reasons. The guilds are
founded on love and faith and perpetuate friendship and
the search after knowledge, and so they always were and
always will be thoroughly worthwhile.
