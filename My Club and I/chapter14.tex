\chapter{Red Letter Days}

It is difficult for my readers to conceive what a Club Red
Letter Day meant to us thirty years ago. Preparing for
the anniversary celebration involved a complete surrender
of our time and interest to the programme of the
day. This great event occurred every year in March,
in the days when the Jewish Girls’ Club was still something
of a novelty, and people were surprised to find that
girls whose working day was long and strenuous had real
talent in drama and choral singing.

On the anniversary Sunday afternoon, we could, as I
have said, expect the attendance of the exalted, those who
represented the important families of the community.
Although my mother, who was the President of the Club,
would on the other 364 days of the year regret the shortness
of my nights and urge me to sleep as long as possible,
she would on the eve of the Club anniversary let me drink
as much coffee as I wanted, and herself call me in time
in the morning to go over my anniversary speech just
once more, so that I might know it by heart. She herself
came to the Club early on this great day, with my sister
and myself, and with some workers and senior members
we made all the necessary preparations for our visitors.

In the days when our Club was at Dean Street, we
begged the loan of a theatre, very frequently the Royalty,
for the anniversary performance. Every detail of the programme
and stewarding was most carefully prepared, so
that we had the right number of theatre attendants and
Club stewards, and all the arrangements were as perfect
as possible. We rejoiced in the mystery of dressing-rooms
and arrangements for scene shifting. We had to be very
careful not to lose ourselves in the communicating
caverns under the arena. We made lists of the girls who
were allowed admission to the afternoon performance and
allotted their places to them.

After a very creditable performance had been given
and enjoyed, and after the speech had been made, we all
went over to the Club to urge our visitors to buy the
articles made by the classes for ornament or comfort, and.
all the time we were glowing with excitement to know
how much had been collected on the donation slips.

When our distinguished visitors left, we all breathed
easily again. It was possible to eat and to think. There
was no more anxiety. We were determined that the evening
performance should be as good as the afternoon one,
but we expected as our guests, the kindest and most
generous of visitors—our Club parents—who soon beamed
upon us with kindness, encouragement and admiration
from the boxes, stalls and dress circle seats. After the performance
the mothers and fathers in their turn crossed
the road to the Club, and the sale of goods was renewed
with very good result. We concluded our celebrations in a
state verging on collapse, but on reaching home we were
not too weary to go over all the incidents of the day and
to count up the shekels which had poured in.

One anniversary which will always stand out in our
memories was that for which a pageant was written and
produced by Mr. Sivori Levey. It was an elaborate and
very ingenious piece of work showing the history and
activities of our Club, and the weeks of preparation which
preceded it made demands on every individual member
and worker of the Club and stimulated our corporate
spirit.

Our annual excursion was for many years one of the
best appreciated events of the Old Club. It could only
have been possible in the remote days when girls had few
treats and only very occasional opportunities to visit the
country. We were in those days much less sophisticated
than we are now, and saw nothing undignified in going
out in brakes. Indeed, the days brought unalloyed joys,
but the outings had to be carefully organised. For many
years we visited Gunnersbury Park, by the kind invitation
of Mr. Leopold de Rothschild, but occasionally we went
to other generous hosts who entertained us on their estates.
There was always time for informal walks and talks, for
organised games and sports, for Club talk and even discussions
on current problems, for the inspection of the
beautiful gardens and appreciation of plants and trees,
and for the eating of sandwiches brought from home and
for the sumptuous tea provided by our host. Nowhere
else in the world could you be more certain of the company
of congenial friends, bent on enjoying themselves,
and for that day at least forgetting all forms of worry and
anxiety, and being absolutely happy. I doubt whether
any pleasure in Club history has ever come up for quality
and degree to the excursions.

Some Red Letter Days were experienced when we
joined an International Music Festival, spending a week-end
in Paris. The tickets cost 10s.\ per head, and we
received hospitality in a French Youth Hostel. We won
a beautiful trophy, but the excitement which this success
brought faded into unimportance when compared with
other experiences. I remember trying to get across the
road with the girls and being frightened of the traffic. A
gendarme suggested that we should unfurl our banner
which we were carrying, and this we did, marching comfortably
across the road behind the banner which bore
the inscription “Love thy neighbour as thyself” (embroidered
by my mother.) On another occasion we were
having tea in an unimportant restaurant and the people
suddenly rose and sang our National Anthem because we
were present. Our excitement was intense throughout the
week-end, and we did sightseeing at a feverish rate.
When we returned home to supper we fell on our chairs
through sheer exhaustion. The girls begged that we
should have a peep at Paris at night, and this we did. I
can never forget this holiday, not only because it left me
with a little physical trouble which has never altogether
disappeared, but because of its intense enjoyment.

It was quite by accident that Queen Mary became
interested in our Club when she was still Princess of
Wales. My sister-in-law, the Dowager Lady Swaythling,
happened to have in her early days of married life the
same hospital nurse as the Princess of Wales, and this
nurse became interested in our Club and talked about it
to her royal patient. We arranged an incognito visit, only
informing the police of the event, as we were bound to,
but so late that publicity was rendered altogether impossible.
Moreover, I had collected a number of workers,
without telling them the purpose of my request, and so
the element of surprise added to the interest of the event.
Queen Mary was her own simple, gracious self, and was
deeply interested in all the classes, especially, I remember,
in the elementary English classes arranged in those
days for Polish and Russian students. Queen Mary
compared our textbooks with those used in her own
nursery.

The girls recognised their visitor immediately, but
behaved in their usual, natural, pleasant way, sure of
themselves in their own Club, and happy to be hostesses
to the Princess, who was deservedly respected and beloved
by all. Permission was given for the singing of the
National Anthem, and never was it sung with greater
sincerity and enthusiasm. The Queen remembered her
visit, and with her wonderful gift for recognising faces
and remembering names, she identified me with the Club
when she saw me several years later in connection with
some war work which she inspected.

The consecration and opening of the Club premises
in Alfred Place by the Duchess of Albany, assisted by
Lord Reading and other prominent men and women, was
a day of gladness and hope. If in 1913 we could have
foreseen the war of 1914, we should never have dared to
make our wonderful change and to fasten a prodigious
debt on ourselves. But in November 1913 we were
free from premonition of any kind, and experienced a day
of supreme joy.

The new Club looked to us as if it would provide room
for every kind of class and social activity desired by our
young people. I was told that the sitting-rooms were so
attractive that we should never get the girls to leave
them to join their classes. We could not have foretold
that under the stimulating influence of the Club the girls
would become so enthusiastic about classes that they
would deem an evening wasted if they were not occupied
m some educational work. At the end of 1913, we were
all happy and excited and full of optimism. The address
made by Miss N. G. Levy on behalf of the members on
the occasion of the consecration of the new building,
explains the importance of this Red Letter Day, and indicates
the spirit in which Club functions were carried on.
(See Appendix.)

I have called this chapter Red Letter Days, and as I
concentrate on the past, such crowds of memories come
into my mind that I find it very hard to make a selection.
There was an evening on which an illuminated
letter of thanks was offered to my dear mother
for her unvarying kindness and good work as President.
She, unknown even to me, had prepared a “reprisal.”
She had beautifully embroidered a banner for us with the
words: “Love thy neighbour as thyself.” I still remember
her smile as she presented it. She was delighted that
she had been able to keep the secret so well.

Then there was the wonderful tea party when I recovered
from an operation and I received the gift of a
copper plaque, embossed with some verses from a Psalm.
I had been absent for eight weeks, the longest period
during which I had been away from the Club, and was
proud and pleased to be made a fuss of when I returned.
There was also the presentation of my portrait in 1927,
and at the same time the gift to my sister, Marian, of the
Synagogue Ark. I said then and I repeat that no
Ark since the flood has been associated with an individual,
and that may be because, since the flood, there has
never been such a wonderful friend and colleague as my
sister.

We had a grand gathering also when after the terrible
accident and long sojourn in hospital of Miss Dora Isaacs,
this selfless worker returned to us. When I received my
O.B.E. another gathering was held, and a fine album
containing the signatures of many friends was presented
to me.

There was the time when we placed the Club on a completely
democratic footing, and I resigned from my position
as honorary Club superintendent and delegated the
work to Miss N. G. Levy and her member colleagues. As
I was still to be responsible for the conduct of the Sabbath
services, the occasion was taken to present me with a
Scroll of the Pentateuchal Law for Synagogue use, and
hundreds of girls attended the service at which the
presentation was made, and I thought that the millenium
had really arrived.

In commemoration of Miss Levy’s advent as Club
leader, we assembled each year at the Red Lodge on a
Saturday evening in December until prevented by war
conditions in 1939. The gatherings always included the
hundred oldest Club members who were with us and
helped us when we made the change. Many of these girls
are now married, but their affectionate interest and devotion
to the Club are still very apparent and our commemoration
parties are full of joyous conversation. So
it is also when a few of the girls who shared our Club
holidays year after year come together in the summer for
an au revoir party in our little garden. We look “before
and after” and cherish our memories and renew our
hopes.

We celebrated six years ago the tenth anniversary of
Miss Levy’s leadership and dwelt on her wonderful work
of mind and heart, and rejoiced in the fact that the Club
had been able to carry on in new and better ways and
was certainly not dependent on our leadership and influence.
The wind of love has blown the Club spirit
far and wide so that its seeds have been fertilised in many
hearts, and especially in her whose devotion and ability
have remained an encouragement and stimulus to me.

In Club life, there are an infinite number of Red Letter
Days. We have our fine opera and dramatic nights, our
successful annual meetings and rallies, fairs and fêtes,
and other days which are recorded in my personal diary
as days of harvest, when the results of dedicated effort
can actually be recognised. Every Club leader knows that
there are and must be periods of depression. There are
days of failure and disappointment, comfort often coming,
however, from unexpected sources. People who
have seemed indifferent suddenly show that they care. A
Club leader’s depression can never last long. She has to
deal with life, and cannot expect to see progress continued
without frequent lapses. But she must look forward hopefully
and feel convinced that “the best is yet to be.”
Every failure is a challenge to greater effort and
quickened faith.
