\chapter{Religion in the Club}

I have mentioned that our Club grew out of a Sabbath
class led by Miss Emily Harris. When this class was
conducted in our new premises in Dean Street, it attracted
large gatherings of people. Alas, some of these, drawn
from various districts of London, did not come only to
worship. Miss Harris had many wealthy friends who
helped her in her benefactions, and she loved distributing
little gifts among the faithful. Some chaffering used to
take place outside the building as a few old ladies compared
their takings. Miss Harris was, however, delightfully
oblivious of the effect of her little presents on some
of her congregants. She gave unstintingly of herself and
her strong faith in her desire to serve. The Club received
this influence from its esteemed founder, and its subsequent
life was developed in the direction to which she
pointed.

Miss Harris and I had many disagreements over the
question of Sabbath breakers. She did not think these
should be allowed Club membership. I, being younger,
realised better the economic pressure of the age which
made strict Sabbath observance impossible for those who
wished to live independent lives. I had been influenced
by Liberal Jewish teaching, and believed that if we would
serve our God who is the Father of all men, we must
translate His word in the changing circumstances of life.
It was all important that we should ask ail those under
our influence to discover God’s word, and try to live in
accordance with it in their working lives. If they lived
truly by His word, they could worship Him all day long,
whatever they were doing, and not only at Sabbath services.
Moreover, they could hallow any day and every
day through prayer, and could use for worship any part
of the recognised Sabbath which was at their disposal.
In this sense, I made a very strong appeal for Friday
evening observance in the home. For me the Sabbath
observances were closely connected with some of my
happiest personal experiences of home life. We children
had Sabbath Eve prayers which we read with our
mother. We had our family gatherings on the Sabbath
Eve prefaced by the Sabbath blessing on the wine of
fellowship and the bread of sustenance, and most important
of all, each child in turn received a blessing. Both
parents blessed each child, laying their hands on the
child’s head, and speaking the Hebrew blessing as they did
so. This custom made a deep impression on me. On
Friday nights we rejoiced in being together. When some
of my sisters and brothers married, they came back with
their husbands or wives to celebrate the Sabbath at home,
and later the children of the family joined us, and the
family evening became more and more delightful. Our
lives were so much absorbed in diverse activities that the
Sabbath gave us a unique opportunity for keeping the
flame of loving interest alive among us. Even after our
dear parents passed on, the custom of Friday evenings
at home with family gatherings and family prayers
remained and was upheld by our eldest sister,
Mrs.\ Franklin, who gave us the opportunity for holding these
meetings in her home. Young people tell us to-day that
important engagements frequently occur on Friday
nights, and it is impossible to refuse them. For me,
because I valued the consecrated family life keenly, there
was never any difficulty at all. We just remained at home
on Friday nights, and no other suggestion was ever considered,
much less entertained. Life accommodated itself
around this established custom as the sea adjusts itself
round a rock which stands up in irresistible strength. It
is there, and nobody can move it.

When I began Club work, I was distressed to find that
the Sabbath Eve observance was neglected in very many
of the homes of our members. The majority of Club
fathers had domestic workshops, and they worked late on
the Sabbath as well as on other days. It is true that the
Sabbath candles were kindled as the emblem of home
piety. These could burn away all little unkindnesses and
family misunderstandings, but, instead of lighting up
family life, the candles often remained beacons of loneliness
for the older folk, while the younger ones hurried
out to their amusements with a rather perfunctory farewell
to their parents. Where family gatherings did exist,
it was too often only to give opportunities for the whole
family to go to the cinema together.

In extenuation of the falling off in Sabbath Eve observance,
I was told that neglect of a custom, however good,
was not as serious as the breaking of a religious
law. I adhered to my Liberal Jewish point of view that.
ceremonials which are aids to holiness, which, in fact,
assist ordinary people to render ordinary life holy, were
worth preserving even at the cost of personal sacrifice.
Legalism which, alas, had usurped the place of life-giving
religion, I felt to be unacceptable. I remember one
devastating example of this legalism, in its extreme form.
A girl who had until this time faithfully observed the
Sabbath evenings at home was given tickets for a good
concert held on a Friday night, and she went with her
friend. In the popular language of exaggeration, she
said to me the next day: “I thought my father would
kill me, as, of course, I never told him before I went.
But when he opened the door and I told him where I
had been, he said: ‘Well, you did not pay for the tickets.
They were given to you.’” It was against the law to
buy or sell on the Sabbath.

In spite of general neglect, I rejoice to remember that
in many humble homes among the Orthodox remnant in
my district, as well as in all other parts of the Kingdom,
the Sabbath evening at home remained a really beautiful
institution. The Sabbath lights were blessed by the
mother, whose love for her home rendered her face quite
spiritual as she pronounced the simple blessing for the
Sabbath. The children were always dressed in their best.
For the Sabbath the tablecloth was of the whitest; the
food was always something special. The homes were full
of love.

For many years I struggled to preserve the observance
of the Sabbath Eve among Club members. I still make
my appeal. Through a guild fellowship, some of our
members pledged themselves to remain at home on
Friday nights. I prepared a Friday evening service which
was widely circulated. Pretty badges representing the
Shield of David were made and given to members. I can
claim that we did attain a limited degree of success, and,
to-day in many homes, young mothers who were once
Junior Guild members and others who came strongly
under Club influence hold their home Sabbath services
and rejoice in their allegiance.

There are vast numbers whom we have not been able
to win to Sabbath observance. Indeed, we recognised,
before the Club had been long established, that in the
face of the complete indifference prevailing in most
homes, we could never interest the average member in
institutional Judaism at all. We could only win the few,
and \textsl{that} we were determined to do through organising
Sabbath services on Saturday afternoons and providing
services on the Holy days. We chose Saturday afternoons
because our members usually ceased work at
1 o'clock, and could attend a service after dinner. Services
on Sundays would have been considered
Christian by the unthinking. The parents of many of our
members had been the victims of persecution in their own
countries. The iron of cruelty had entered their souls.
They feared Christianity, and did not understand it at all.
Their religion was unfortunately often so negative as to
be chiefly “not Christianity.” In late years, it has become
more positive in its Jewishness, and as Jews advance
in interest in their own faith, they can respect and understand
better the religion of their neighbours. Recently
there has been formed a Fellowship of Jews and
Christians to understand better each other’s religion, while
each group retains its own creed and religious practices.

Already, in Miss Harris’s time, Holyday services were
organised by our Club in this connection. I remember
that Miss Harris asked Dr.\ Montefiore, leader of the
Liberal Jewish Movement in England, to come and read
to her congregation for an hour or two on the Day of
Atonement. With his usual kindness, he agreed to do this,
and penetrated to our premises on the first floor in Dean
Street. On his way he met several girls in. gymnastic
costume who were occupying the lower part of the
premises. Ever inclined as he was to make excuses for
people’s weaknesses, he said: “ Queer to dress like that
for the service on the Day of Atonement, but after all I
know it is hot and I expect Miss Harris does not mind.”
He never imagined that the lower rooms were let to a
gymnastic school.

Before the establishment of the West Central Liberal
Jewish Congregation some of our devoted Club members
walked to Notting Hill Gate on the Holydays to enable
me to hold a four hours’ service in a hall near the Synagogue
at which I attended for some years. On Saturday
afternoon, I walked with my beloved sister to Dean
Street, and we held our special Sabbath afternoon services
which were quite well attended. Before I could start
these services I had to overcome great difficulties at home,
for no other reason that they were supposed to give me too
much exertion after a heavy week’s work. But I believed
then, as ever afterwards, strongly in the power of worship,
and was convinced that the habit of not attending services
was rooted in the boredom which a traditional service
evoked. A year or two of workshop life seemed to wipe
out the small knowledge of Hebrew which most of our
girls had acquired as children. They could not understand
the traditional service and it bored them extremely.
Moreover, they were not accustomed to take any part or
responsibility in the service and found in the liturgy and
sermon nothing which was related to ordinary daily life.
Our services were different. They were in English and
were brightened by congregational singing. Only such
prayers were used which had a meaning for modern
Jews and Jewesses in the actual circumstances of their
lives. The sermons treated of vital subjects. Ultimately,
the West Central Liberal Jewish Congregation was
founded, and men as well as girls attended and sat together,
instead of as in the Orthodox Synagogues being
separated. Children’s services were also held for some
years and attracted large numbers. Later, the children
were invited to come with their parents and a special
address was given to them in a separate room.

For the first years we refrained from using instrumental
music as we were afraid of alienating the few worshippers
who were sincerely Orthodox. To my surprise, I found
no objection raised from this quarter when we did make
the change. As often happens, the really religious are
seldom opposed to reform even if they themselves do not
desire or value it. Our Orthodox members said on this
occasion: “Perhaps the music may bring some people
who would not otherwise come to a service.” It is only
the pseudo-Orthodox who get their souls entangled in
legalism, and who criticise us for making the service as
beautiful as possible.

Although our congregation has made fairly good headway
in the West Central district, it has not attracted a
multitude of Club members. They have, as I have said
before, no interest whatever in services of any kind. If
attendance involves the slightest effort, the vast majority
refuse to make it. The reasons for this complete indifference
are numerous. The most important is that the
parents are utterly uninterested. In the Polish and
Russian villages from which most of these parents came,
it was not the custom for girls to attend services at all.
Again when work has to be done on the Sabbath, there is
no urgent feeling for hallowmg the remaining part of the
day. The family meet for Saturday dinner, and that is
the first meal of the week they can take together. They do
not want to separate immediately after, even for a service.
After a tired morning’s work, which is the climax of a long
working week, people are more inclined for walking or for
sittmg in front of a cosy fire, or even for a weekly shopping
expedition, than for a Sabbath service. Then, London
weather is seldom the right weather for Synagogue attendance.
It is too hot or too cold, too fine or too
rainy. The inclination is not there, and I do not attribute
the phenomenon to the indifference of the generation to
which our young people belong. I believe that given the
right background, our Jewish youth would respond in
England to Liberal services, as they undoubtedly do in
the United States of America. I do not admit the
work-difficulty is insuperable, for even in those cases where
people do not work on the Sabbath, they prefer to walk
about the streets rather than go to Synagogue. I believe
we must not expect to succeed until there is a general
realisation of God’s presence, and this will come sooner
or later. It is possible that even when it comes it will not
express itself in services. But I think that the religious
feeling, when it is once produced, will be so strong that
it will have to find expression in life. It will give fresh
interest to ordinary daily routine. Corporate feeling is
such a powerful force that men and women will one day
want to come together for worship even as they do for
every other human activity. The Jewish community must
at last discover its inheritance. It will know that it has a
message for all the world, and it will have to testify to its
rediscovered faith. For the moment, we are weighed down
by apathy, convention and misunderstanding, and by the
prejudices which are shown against us. We think that we
shall some day shake ourselves free and then we shall want
to speak to God in our Synagogue groups as well as in the
privacy of our homes.

Even though we have not drawn our Club members as
a whole to attend Synagogue, we have not entirely failed
in living by our religion in the Club. Not only have the
few responded with zeal and enthusiasm and moulded
their lives accordingly; not only have homes been established
on a religious basis and the faith passed on to the
next generation, but we have had our harvest in many
other ways. It has been generally understood that the
Club leaders and their colleagues care about Judaism.
We have been able to introduce a certain religious interest
in our Club. It has fallen woefully short of the ideal we
set ourselves; it does exist, nevertheless, and it is real.
Girls and boys throughout the years have often sought
God on their own account, having been stimulated by the
Club. They have brought their confidence to us and
asked for help, not merely because we were their friends
and they could trust us, but also because they recognised
that we were in contact with some power greater than
ourselves and this contact revealed itself in our lives and
gave them security. To put the matter simply: our
young people have discovered that our faith means a great
deal to us and is the main source of our strength and
happiness. We are ready and anxious to share it with
the Club people who are bound to us by the closest ties of
hurnan friendship.

The religious feeling in the Club has been shown in
significant ways. Our members pay Judaism homage
when they profess allegiance to our Club. Whether
actively sharing in our direct religious work or not,
whether or not they come to the Synagogue services and
cultural activities, or just passively take cognisance of
their existence, our members, if challenged, jealously
guard the Jewish character of our Club. Although giving
a hearty welcome to non-Jewish workers and visitors,
they recently opposed, when consulted, the unrestricted
introduction of non-Jewish members, even in war-time.
They are opposed to encouraging between Jews and
non-Jews friendship which might lead to inter-marriage.
If individually they rebel sometimes at having to
comply with our Club rule that all members should at
the conclusion of the Club evening join in the fellowship
of prayer, they, as a whole, uphold the retention of the
rule. They have often stoutly maintained that they
would not alter the religious character of the Club as
expressed in the assembly. I believe that our simple
words of extempore prayer have brought a multitude of
young people to feel the meaning of prayer, and to recognise
the effort entailed when we try to bring our imperfect
spirits into contact with God’s perfection. They have
been glad to add their supporting faith to the proclamation
of the Shema, by which we conclude our brief evening
service.

Throughout the history of our Club, talks on religion,
as well as the united act of prayer, have been features of
our Club life. We have always prayed together on Club
holidays, and through wonder at Nature’s beauty our
members have been led to worship the Creator of the
world. We have felt the unity which comes from
Sabbath Eve celebrations conducted in a family atmosphere.
We have held open-air services and felt the truth
of Psalm 42, for in holiday atmosphere we easily experienced
our longing for God “as the hart panteth after the
water brooks.” Generations of Club members have
carried home from their holidays at the Green Lady
Hostel, Littlehampton, memories of “talks under the
tree” held every night for those who cared to come
rather than to listen to concerts on the front. At these
discussions, as well as at the meetings led by a variety of
people in the Club itself, the frankest questioning was
encouraged. We had the opportunity of removing curious
superstitions and of teaching the fundamentals of a living
Judaism. Again and again we have been appalled by
the wrong sense of values among our young people, and
at the sordid sex disqualifications in which the girls still
believed. Being filled with reverence for the piety of our
ancestors, we exercised very careful thought before we
tried to weaken our young people’s unquestioning allegiance
to ancient customs. We knew that the service they
paid was only lip service, but we did not want to quench
the dimly burning wick. We could not, however, support
any longer a policy of mere drift, for this drift might
easily become. a drift into materialism. We believe we
acted in the best interests of Judaism when we helped our
young men and women to use their minds in the search
for God.

In recent years I have written a monthly letter on a
religious subject to Club members, and invited them to
discuss their special problems with me. I believe that
our religious life will become really strong when we have
convinced our young people that they are under an
obligation to discover God anew each for herself and himself.
God is great enough to enter into every soul, but we
must make ourselves ready for Him and eager to receive
Him.

Our ancestors have told us of their experiences and
handed these experiences down to us. But our religion is
dead if we worship only with our fathers’ hearts. We
have to kindle our own lights. We believe that the truth
of Judaism is co-extensive with life, but it is a progressive
force, as is life itself, and its presentment cannot be
changeless if it is to fit every generation of believers.
Through our Club we must always try to stimulate faith,
and become interested in its abstractions and in its history.
But, above all, we must assimilate the idea of God as far as
is possible with our limited intelligence. God must
become so real to us that we can live under His guidance,
working for Him and with Him, and trusting that this
kinship is for ever. With this faith we can pass even the
valley of death and still fear no evil.
