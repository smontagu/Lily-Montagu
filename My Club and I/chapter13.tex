\chapter{The Exchequer of the Club}

Our Club expenses have always been far greater than our
resources. We could never really afford to run the Club
on the costly scale on which we were embarked in 1919,
but we seemed to have no choice. It was necessary to
have a building in the West Central district, where the
rents and rates were terribly high. We had to pay or we
could not go on. We had to go on.

Partly because of the novelty of girls’ clubs when ours
was started—and I was among the pioneers in the girls’
club movement—and partly because my mother took such
a deep personal interest in the work, our Club did for
many years attract a considerable amount of attention.
It was our custom while in Dean Street to borrow a large
West End theatre and let the girls give an annual performance,
at which I made an appeal for funds. To this
entertainment, of which more will be told in another
chapter, the pillars of the Jewish community used to come
in quite large numbers. It was a social affair as well as
being interesting and novel in itself. I have been rather
surprised in reading through letters written thirty to forty
years ago to see the attention which our Club received.
Considerable sums were contributed in those days for its
maintenance.

When the premises at 8 Dean Street became impossible
because our activities overflowed their restricted space, an
influential committee of business men assisted us, and we
made a collection for building our premises in Alfred
Place. At the speech I made in March 1913 at the
Royalty Theatre to launch the big appeal for £8,000,
slips were distributed for completion by the visitors. I
shall never forget the thrill of excitement when a slip
was handed in by the naptha king, Mr. Joseph Fels. It
was a slip for £1,000, given on condition that we had a
roof garden on our building. I thought the figure must
be a mistake. Never had such a large sum been written
on a slip for any charity anywhere; at least, so it seemed
to me. I showed it to my mother and to a number of
workers. I felt almost ill with excitement, and then suggested
giving it back. The slip could not be real. At last
we settled down. and found that the money was really
intended for us, and should be used for the roof garden.

A home had already been built and was in use for a
number of girls. In the large basement room we catered
for a number of business girls and supplied a fair meal of
meat and vegetables and pudding for 5d. The Club had
still to be built, and Mr. Ernest Joseph kindly accepted
responsibility for this work. The Men’s Committee continued
to function and to collect as much as they could,
but the unexpected catastrophe of the Great War frustrated
their efforts. It was obviously no use trying to
collect funds for a sectarian club, when the whole community
was straining every nerve to raise money for all
sorts of war necessities and war charities. The committee
had to dissolve itself when still £4,000 were uncollected.

For several years we had the debt as a dead weight
round our necks. We had to pay interest which restricted
the amount for the upkeep of the Club. We spoke of the
debt frequently, but what was the good of talking and
thinking about it? It was there, and we could do nothing
about it. At last my sister and I paid it off, having discovered
that we could control some reserve fund from our
family legacies. Nobody took much notice of this little
transaction. The debt was no longer talked about: it
was removed. I often smile over this incident in our
careers. It was so simple, and so swiftly done. We visited
our lawyer, and then we did not have to think or worry
any more about the debt. That was all. But our resources
were exhausted. We could use for our work as
much as we could spare from our simple personal requirements,
but our remaining capital was in trust, to produce
our income. We could never again do a mighty deal, and
clear away a weight of worry! That privilege had been
ours once, and we are grateful.

We had now our building, and the Club gradually
entered a new phase of its existence, becoming more and
more democratic in character. Clubs for Jewish girls
and boys grew up everywhere. Our anniversary celebrations
were no longer unique. Several other big clubs had
buildings that were even more suitable than ours for club
purposes. To us, our Club must always overshadow in
importance all else that the world contains. But to the
community it became just a club among many. The Day
Settlement was an interesting addition which grew up to
satisfy the needs of Club families. The community did
not understand that the Settlement was a thing apart;
they were not interested. To them, the Club was always
the Club, and it was a “Montagu show” which the
family might well keep up if they wished to. No doubt,
they said, it did quite a lot of good in amusing the girls
and in keeping them out of the streets. The anniversary
entertainments were, they supposed, like other Club
affairs, dull and uninteresting, except to those who
knew the Club from within. The performance must be
amateurish and third-rate at that. If we sent invitations,
these were put in the waste-paper-basket forthwith, or
occasionally given to the maids, who might go if they
liked. Such was the ignorance which existed about our
work, until, in the lurid light of recent calamity, its value
was at least realised and the loss of our building and
possessions has been somewhat mitigated by the wonderful
sympathy shown to us by all sections of the community.

From about 1920, our financial position became more
and more serious. Nobody seemed to be interested in the
fact that our work in its many-sided aspects deserved support
on a generous scale. The financial difficulty has
continued up to the present time, and, although, thanks
to our leader, the Club has made good progress in organisation,
there is apparently no solution for our financial
problems. We realise that the Settlement work must
always by its very nature be supported by public subscription.
The activities themselves, except for a Board
of Education grant for the children’s work, have to be
sponsored entirely by those actually engaged in the work.
The few subscriptions which have been maintained help
to support the framework of the undertaking.

We were obliged in 1919, as I have explained, to give
up the home established for business girls as a part of our
Settlement activities, and to use the building of No.\ 32
Alfred Place for business tenants who have, on the whole,
been most elusive. Many vanished leaving an alarming
overdraft on our hands. We should have liked to use the
premises of No.\ 31 (which were serving for our Settlement
during the day and for the Club in the evening and weekends)
completely for Settlement purposes during the daytime,
but this was unfortunately not possible as the ground
rent had to be paid, and the letting of the halls and
various rooms assisted us towards covering this very
serious item in our budget.

We had the hope at one time that the young people’s
subscriptions would cover the club expenses and that no
outside assistance would be necessary for that part of our
work. The Club would, we thought, be self-supporting,
paying rent to the landlord, the West Central Settlement.
We did not, however, fully appreciate the difficulty of
making a club self-supporting in the West Central district
of London. The weekly subscription was raised by common
consent to 3d.\ or 5d.\ a week, according to the age
of the members. Our system of collecting Club subscriptions
is business-like and carried out by a rota of experienced
worker members. The vast majority of girls and
boys pay their membership and associate fees conscientiously,
and a few give extra donations. There are
some whom we value as members who not unreasonably
regard ours as an expensive Club. The normal subscription
has had to be modified for several members who can
only pay less than the usual amount. We ask from
our refugees the modest contribution of 1d.\ a week.
On the other hand, we are aware that there are a few
members who are capable of paying more in fees than
the Club demands, but who don’t like paying. Some of
these have belonged to the Club for many years, and any
suggestion of terminating their membership meets with
unmitigated and horrified astonishment. “No, Miss Levy,
you can't mean me! Don’t you remember what an old
member I am? Do you think I am going to run away ?
Your money is all right, believe me! Of course, I can’t
leave the Club, that’s absurd.” But they don’t like to
pay, and they could afford to. We don’t want to lose
them. In most ways they are excellent people, and they
need the Club more than they may suspect.

Miss Levy feels very keenly the responsibility of covering
expenses. She supervises every item of class expenditure
and tries to equalise it with the collection of class
fees. She keeps the secretarial expenses extremely low,
and scrupulously saves money on every possible occasion.
As many activities as possible are rendered profit making,
and her fertile brain is always trying to think of new ways
of raising funds. Our secretary has the assistance of the
Dance Committee in making the dances pay, and of the
Committees of the Operatic and Dramatic Societies in
making their entertainments yield a profit. Mainly by
her own exertions she has organised a canteen which
supplies a very useful income to the Club. By means of
sports days, fun fairs, garden fêtes, the sale of “stamp”
books, and other activities, we have raised money wherever
possible. We have managed to do without whist
drives and, in latter years, without jumble sales as
money-making activities. It seems to me inconsistent to invite
people to play cards for high stakes in the holy name of
charity, while we discourage gambling among our young
people and realise its power for evil in the community.
Much money, after arduous toil, can be made in jumble
sales. These sales are full of articles given away by people
because they don’t want them, and by so doing are falsely
encouraged to think they are doing social service. Moreover,
those who buy at a jumble are dominated by the
search for cheapness. They are out for bargains. They
buy what they don’t want, something perhaps faulty or
altogether shoddy, simply because it is cheap. Here again
there is an error. Such errors should not shadow Club
life even in order to improve its finances.
Partly on account of general depression, our funds were
very low right up to the war period. In vain has our
Honorary Treasurer, Mrs.\ Waley, worked and sifted and
considered. The costs of rent, rates, light and fuel are
staggering. Profitable entertainments on a large scale are
impossible except for well-known charities. Our deficit
has crept up. We can only plan and hope that things will
improve after the war, though the future of the Club is
shrouded in mystery. The majority of our members
realise the difficulty of the financial problem and the
anxiety it causes us, and are always eager to augment the
income in any way possible to them. Nothing hurts them
more than to be told that activities over which they have
taken a great amount of trouble cannot be made a paying
proposition as the members’ power of giving is exhausted.
Their strong desire is always to make money for the
Club. In a vague way, however, all are invincible optimists,
and believe that somehow the way will be found to
keep our Club going; for our Club is good, and radiates
goodness and happiness, and it cannot die until it has no
longer any work to do. Its capital consists mainly of the
good will, the energy, the affection and the faith of our
young people.
