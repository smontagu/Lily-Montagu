\chapter{Club Holidays}

I have found that Club holidays provide the most important
factor in the development of Club life. It is on holiday
that the leader and her girls really learn to know one
another and to like one another. The leader realises the
quality of the material with which she has to work, and
the girls in their turn will recognise the qualities of their
leader. The important values of life are seen on holiday,
and it becomes clear how far a Club has helped its members
to self realisation, and where and how and why it
has failed. It is then that Club friendships are made, or
renewed and rendered permanent.

As I was a pioneer in Club work, I had to discover the
best methods and to find the facilities for organising holidays.
I could not expect help from anyone with similar
experience, because there were no such people. Before we,
as workers and members, actually went for our first joint
holiday, I arranged for small cottage parties near the
estate of one of my friends who gave additional
hospitality. I soon realised that this form of holiday was
inadequate, and with immense trouble we found a large
farm a few miles from Bexhill to accommodate girls and
leaders.

The idea of my leading such a party with my sister was
so unpopular at home that we simply could not speak of
it. All sorts of calamities were predicted for me, and I
was asked again and again why I did not send the girls to
the country with a capable superintendent, who would
have some chance at least of managing them, and who,
unlike myself, had some rudimentary idea of housekeeping.
When after our two weeks away my sister and I
returned home, we had for the first time to refrain from
telling our mother of our joyful, albeit somewhat
hazardous, experiences, for she disapproved entirely of
our undertaking. We were bubbling over with enthusiasm
and were glad indeed to be able to pour out our story into
the ears of our old nurse Rosie, our lifelong friend.

On our first holiday, the majority of the girls slept in
a great barn, which served as a dormitory for about
thirty members. My sister and I had to use a lantern
as we climbed the rickety stairs to say good-night to the
barn folk. We ourselves shared a cottage a few yards
away with some of the senior girls. That dormitory
seemed often full of grievances when we visited it. The
light was dim; there was no proper accommodation for
clothes. The fact that so many girls slept in one room
naturally tended to excitement. The few who wanted
to rest were disturbed by the majority who wanted fun.
I was told that again and again the threat was sounded,
“I shall tell Miss Montagu in the morning,” but when
morning came all grievances were forgotten, and we
revelled in all kinds of enjoyment.

It is true that our hostess was eccentric and had no
idea of punctuality. Our meat during that holiday came
ready cooked from London, and our housekeeper would
examine it for live stock with a microscope and then
pronounce the meat to be uneatable. She saved us
from starvation by producing an infinite number of eggs
of  all kinds, including turkeys’ eggs, which we enjoyed.
But Miss C.\ had a nasty habit of going off to buy the
bread and the few other necessities which were not grown
on her farm just at the time when we were expecting to
begin our meal. She would drive off in her high dog
cart and smile benignantly on us. The girls knew how
impotent my sister and I were and how distressed, and
they invented cheery songs about our landlady and
comforted us.

Our chief amusement on the holiday was to spend
time by the sea, and I always wondered at the ease with
which my girls obtained lifts to and from Bexhill. It
was when they were with us that they seemed to
walk.

We had a wonderful time together that first holiday
in spite of all the domestic shortcomings. We were all
young together, and my sister and I learned much about
the lives of our girls, and they came to know more of us,
and Club work took on a different and more interesting
aspect when we returned home. Other leaders took the
second holiday party, and, by that time, our landlady’s
idiosyncracies becoming more accentuated, we decided on
a different scheme for the next year.

We determined to borrow a country house which had
been vacated, and heard of one belonging to a certain
Jewish philanthropist. With great hesitation, I got over
my natural shyness and called on this gentleman. I still
remember his question when I had managed to drag out
my outrageous request. “Why did you come to me? Did
your father send you because he thought me the greatest
fool he knew in the community?” In spite of this unpromising
beginning, it was through this gentleman that
we obtained a house in Lowestoft where we spent our
second Club holiday, and found great happiness there.

The difficulty of securing the right accommodation for
Club parties induced us in or about the year 1900 with
Sister Mary Neal and Sister Emmeline Pethick (afterwards
Mrs.\ Pethick Lawrence) and other social workers
to establish a joint holiday home in Littlehampton. Clubs
of all denominations and other organisations rented the
Green Lady Hostel in turn, and enjoyed the advantage
of being looked after by an efficient hostess, while organising
the holiday as their leaders thought best. The walls
of the hostel, if they could speak, would tell of scenes of
the greatest happiness. It did not matter that in some
years the catering was curious and unsatisfying, that the
comforts of the house were few, that the windows were
somewhat inadequate, and that the mattresses could
never have been described as soft; that the piano could
never have been new, and the cups and saucers must
always have been chipped. No house in the world could
ever have entertained happier groups of girls and women.
Mr.\ Pethick Lawrence and my father gave generous donations
which enabled us to build an additional storey to the
house, so that we could receive fifty-two guests at a time.
As the years went by, our original committee separated,
our parties became more varied and included boys,
deficient children, school journeys, a party from a Borstal
Institute, mothers and children, and always in the
summer, girls.

The hostel had quite a large and pretty garden at the
back of which was a field, and the countryside stretched
beyond it. Pleasant country walks were within reach, and
the glorious excursions which were possible afforded us
infinite pleasure, especially the river excursions, which
were made the occasions for much singing. The sea was
only ten minutes’ walk away.

The work of holiday planning was arduous. We had
to collect the girls’ money, and assist those who could
not meet all their own expenses. We gave immense
thought and discussion to the arrangement of the bedrooms,
and I must admit that we found it difficult to
secure reasonable quiet at night. Of course, we were not
strict enough, and I think the girls rather enjoyed seeing
us walk round again and again, expostulating mildly at
first, and then with greater and greater severity. The
good-night visits to dormitories were a feature of the
holiday. Girls, who in ordinary life had no time to fuss
over small ailments, needed every kind of attention from
us, as we went from bed to bed carrying our medicaments.

We arranged beforehand the pleasure for each day,
finding a great relief in having the catering done by our
hostess, and thus being free to further the other sides of
holiday life. There are club leaders, I have discovered,
who send the girls out to amuse themselves and do not
arrange activities for them all to share in. I cannot think
that these leaders ever realise the possibilities of club
holidays. We prided ourselves on finding new walks and
excursions every year, although the same girls accompanied
us again and again. A middle-aged woman said
to me not long ago: “We didn’t want anything but
Littlehampton, and when one year we did go to Herne
Bay, it was like a jubilee!”

Every morning I gave out the programme for the day,
which generally included many hours on the beach. Our
people were deeply refreshed by the sight of the sea and
loved to bathe. They appreciated the point of view of a
small girl belonging to another club who said that the
sea was the only thing in God’s world of which there
was enough for everybody, and she laughed and laughed
as she said it. I have never been good at sport, but, like
my sister, was a fair swimmer, and enjoyed giving swimming
lessons. We encouraged our girls to take mild walks,
but on account of their sedentary lives they were not good
walkers. It was so easy to talk while we were strolling
along, and we changed partners so as to get to know
well as many people as possible. I have very little sense
of direction, and I remember once losing the party on
one of the rare occasions when my sister, an excellent
guide, was absent. They were very tired, but they knew
I was worried, and so they assured me that they were
quite fresh and did not want to rest. “You are dears,”
I said, “but you do tell whoppers!”

During our holidays we did a great deal of reading
aloud, and we took story books and poetry books with
us on our so-called country walks, which were marked by
frequent floppings in fields or under trees. Here and there
we inculcated a real love for poetry which was afterwards
developed. I remember too the deep interest shown in
our books, which seemed always to be unlike other books,
being much more interesting when read aloud. A mother
who was listening to the reading of a simple love story
asked me when I closed the book at an exciting page,
“Excuse me, but \textsl{did} they marry?” “You must wait,”
I replied firmly.

I have always advocated as comfortable holidays as
possible for our Club members in preference to camping
out. It is a treat to those who live comfortably throughout the year to rough it on holiday. Most of our girls had,
at least in the early Club years, plenty of opportunity for
roughing it, quite apart from holiday experiences. We
all liked sitting together at table and having meals
arranged as beautifully as possible. This was particularly
important on the Sabbath Eve, when everybody tried to
look her best, and we made the evening meal as festive
and homely as we could.

Perhaps, because happiness is the best of all preventive
medicines, we had very little serious illness to cope
with. I remember one case of pneumonia brought on by
a delicate girl’s lending an undergarment to a friend, and
then catching a very severe cold. We had to take her to
hospital at night, and on our return found her bedroom
emptied of its other inmates. These girls, all very young,
did not think we could take a girl out at night except
for fever; so they had crawled into other people’s beds
away from infection. We soon had the room filled again
by more sensible people.

When, after a few years, Miss Lewis joined our Club
organisation, she came with us on holiday and taught the
girls a large variety of fancy handwork as they sat in
the garden or on the beach. Every girl who has been on
a Club holiday will remember the joy of making lavender
sachets by intertwining ribbon round the stems which
were folded to enclose the flowers. Another favourite
occupation was to passe-partout reproductions of well-known
pictures or snaps. It was very entertaining to the
girls to make little presents for their friends and to profit
from Miss Lewis’s ingenuity.

Other features of our Club holidays were the garden
swing which was never empty, our orchard, a constant
source of temptation to our weaker sisters who suffered
in consequence, sports day and regular games of cricket
taught by my sister, newspaper reading, and, above all,
the open-air services and discussion groups. When we
had first tried to obtain the hostel for a holiday home, the
owner, a retired army man, refused to sell it if Jews were
to make use of it. I went to Canon Wilberforce, as a
dignitary of the church and also a personal friend of
mine, and asked him to help us. “Don’t touch the place,”
he said, “it is stained with prejudice.” “But I need it
badly,” I told him. Then we agreed that our new
religious life might clear away the prejudice of the past,
and I think it did, for our services were vital and had
real influence. Our people who shied away from institutional
religion in London liked the holiday services. They
were always deeply impressed with the Sabbath Eve
observance and often resolved to introduce it in their own
homes. Concert parties attracted large numbers to the
front in the evenings, but a fair proportion came to sit
with us “under the tree” in the garden to discuss all
aspects of Judaism, and to learn about and clarify their
faith in themselves. We were often amazed at the girls’
ignorance of the Bible and of Judaism itself, but they were
responsive, affectionate and loyal. We were absurdly
anxious about the punctuality of the girls in returning
home at night, and stood at the gate of the hostel garden
with watches in our hands to greet them or to admonish
them severely. When we realise the temptations of city
girls and their independence of all parental control, we
must smile at the tyranny we exercised, albeit prompted
by the tenderest solicitude. On looking back, I wonder
that our girls, as a whole, hardly resented our attitude
at all. They knew before they came on our joint holidays
what they were in for, and I suppose they felt themselves
compensated for a few restrictions by the amount of fun
they derived from the holiday at the hostel. Indeed, I
think they enjoyed the strict mothering, even though they
probably smiled at it when we were not present. The
caring and fussing was not all on our side. We in our
turn were always well taken care of; such a wealth of
affection was showered on us that these holidays were
certainly the best we have ever had in our lives. Although
we were busy from early morning to late at night, and
were sorry to find at the end of the fortnight or month
that we had not had time to get sufficiently near every
girl to gain her confidence, we found the Club holidays
wonderfully refreshing and invigorating. Before many
years had passed, our mother rejoiced with us. She came
to see us off, and became interested in all the details of
our hostel life. She in her own life had always found the
keenest enjoyment in providing joy for others. In the
hostel we had the best possible opportunities.

After the lapse of many years, we felt that special
holidays for the older girls should be arranged; otherwise
there would be no chance of their ever seeing the beauty
of England. For many years my sister, Mrs.\ Franklin,
invited four to six girls to spend a fortnight on her
estate in Ireland, and arranged expeditions for them in
glorious country and gave them delightful entertainment.
Moreover, as people’s wages increased more ambitious
holidays could be paid for. The Club grandchildren had
their organised holidays also. So for one fortnight of the
holiday month we took special parties to Devonshire and
Somerset and Wales and had wonderful times together.
We were so intensely happy that we actually wept when
the holiday came to an end. I remember after one holiday
coming home in a saloon carriage which had been
reserved for us and feeling positively limp with emotion
at the end of the journey. We might have been separating
for life, or travelling together into some danger zone.
But when we returned to the C major of ordinary daily
life, the happiness of the holiday weeks remained with us
as a beautiful possession which was to endure.

\begin{tp}{512}
Our present Club leader has continued and developed
the tradition of successful West Central holiday parties.
She has arranged mothers’ parties, has rented schools in
different parts of England for holidays, and has taken
groups abroad. The girls delight in her leadership.
Nevertheless, we have seen signs that, with the improved
economic position of workers, Club holidays, as we have
known them, will not in future play an important part
in Club life. The modern girl prefers complete freedom
on holiday; she likes mixed parties if she does not go
away with her special girl friend. She is no longer so
sentimental in her attachments. She is able without the
help of her Club leader to obtain all the attractions a
holiday can offer her. She no longer cares to be part of a
crowd, even though her personality is closely considered
all the time. She desires on her holiday
something altogether different from the life with
which she is familiar during the year. She is more a
creature of change than her mother was, and the same
kind of treat makes no appeal to her. Indeed, she does
not need to be given any kind of treat; she can find what
she wants for herself. The modern girl would be resentful
where we were once intensely amused, when, after
being invited by a great landowner to visit his estate, we
were given tea in cups without saucers on a stony beach
skirting his wall. What did it matter? We brought our
own fun with us. We loved to be together. Our girls
deserved the description of human flowers moving in
God’s garden, a description given to us by one of our
hostesses. The modern girl would not wish to give up a
whole day of her holiday to making a fancy dress for a
garden fête arranged for her own Club party. She needs
more exciting events. She would not get the excitement
we did from providing the villages where we stayed with
a wonderful amateur concert which gave an opportunity
for the use of Club talents. I am not sure that she knows
the depth of happiness which we had on the old-fashioned
Club holiday, but she thinks that she knows something
better. Who can tell?
\end{tp}
