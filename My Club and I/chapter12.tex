\chapter{The Settlement as it Grew Out of the Club}

I have always felt that every individual member must be
regarded not only as a club unit, but as a member of her
family. We must seek the sympathy and co-operation of
the parents of our members and convince them that we
are interested in all their family problems. It is not possible
to know girls sufficiently to influence them if we
see them only on the nights when they come to the Club
for education or recreation. We must know their background
and understand the influences which are helping
us in our work, and those which are militating against
us.

I have found visiting to be an essential part of Club
work, and we visit in order to establish friendly
relations with the families of our members. Our
visiting is different from ordinary district work. It is
altogether removed from any kind of means enquiry. The
Club has been established to enable girls and women of
different education and circumstances to come together
and to work for a high standard of home and civic life.
We have had delightful experiences in our home visiting.
Hardly ever has the purpose of our visits been misunderstood.
Only once can I remember being received in a
churlish spirit, because the visit was looked upon as an
intrusion. I remember a visit in a snowstorm at considerable
personal inconvenience, and was met by the
ejaculations: “Why, whatever made you come on a day
like this? Couldn’t you have waited?” The girl who gave
me this welcome was in urgent need of hospital treatment.
Her rudeness, as I fully understood, only expressed
concern on my account.

My sister and I visit the home when a new member
joins us, in order to impress on her the responsibility and
privileges of membership, and to assure ourselves that this
membership is desired for the girl by her parents. A list
of children leaving school is sent to us by the local school
officials, and through our home visits we can discuss more
easily than at a conference the kind of work which is
desired by the parents and the child. We are able, of
course, in our visiting to eliminate altogether any idea of
patronage. We easily win the confidence of the girls’
parents, and can convince them that we are glad to be
of service whenever we are needed, and then we are consulted
as friends when our advice is thought to be of use.
We work through and with all local and central agencies,
and when material benefit is required, we do not give it
directly from the Club, but are able to make recommendations
to the agencies from which help may be
sought. The majority of our people like to discuss personal
and family problems with us, and to ask advice on
questions of social, industrial, educational and religious
interest.

We are among the first to hear when a young man
appears on the horizon. The association may be light and
easily terminated, or it may lead to a long and satisfactory
friendship resulting in marriage. Our view of the girls
is sometimes a little different from that held in the
home. We can chat with parents and compare notes;
sometimes we can correct our own estimate of a girl’s
character by hearing her mother’s point of view; occasionally
we can increase the sympathy shown to her at
home.

Before we were able to give the whole of our time to
Club work, visiting was undertaken only in a limited
degree. When our own home was broken up in 1919,
through the death of our mother, we started a Day Settlement
at Alfred Place, and spent all our time there and in
the district for which we worked. That district has
changed greatly and is no longer mainly residential.

Many of the houses with which we were familiar have
been pulled down and in their places business premises
have been erected. Our people lived mostly in badly constructed,
ill-ventilated houses in small streets in Soho, St.
Marylebone and St. Pancras. They were within a stone’s
throw of some of the luxurious houses and places of entertainment
existing in London. It is small wonder then that
we had to cope with a certain amount of restlessness and
a sense of injustice in girls who came from drab and
wretched homes. As the district changed in character,
our families moved to various parts of London, and home
conditions were immeasurably improved, but the need for
visiting was undiminished as its purpose was not related
to merely physical and material conditions. Distance difficulties
were easily overcome, and the West Central Club
and Settlement at 31, Alfred Place, Tottenham Court
Road, was thought to be conveniently near to all parts of
the city and suburbs. Of course, a fair proportion of
friends still lived, and many more worked, in the immediate
neighbourhood of our Settlement and were able
to come in at any time. It is a fact that we always found
someone waiting for us at whatever hour we came to the
Settlement.

We have found that medical problems naturally loom
very large in the life of our people. Much illness is
caused by adverse conditions of life, and when outside
interests are few a woman is encouraged to focus her
attention on her ailments. We were told a great deal
about the health symptoms of the various members of
the family whenever we did home visiting. Sometimes we
were saddened by the knowledge that a little opportunity
for rest and change, especially where the mother was
concerned, might have saved her from serious suffering
which had become chronic. The Club worker is
depressed during her visiting by the many “might have
beens” of which she obtains knowledge. There is the
mother who cannot go to hospital on account of her
babies, or who has to resume work far too quickly after
her confinement. There is the father who, although his
heart is not strong, has to lift heavy irons as there is
no one to do it for him. There are the nervous children
rendered more nervous by spoiling and bad management.
There are the uncontrolled young people who have never
been taught obedience. There are the parents who slave
in order that their children should have an easy life, and
who pride themselves that their girls never have to wash
up a cup and saucer. Yes, we were depressed by all this,
but we were also stimulated to make life better through
our Club for those who came under its influence.

\begin{tp}{256}
From very early days a Club doctor was available, and
her services were much appreciated. Gradually we established
day and evening clinics for medical and dental
treatment. These were particularly useful to those who
had no time for out-patient treatment, who had
no faith in panel doctors or were uninsured, who
would not bother about a doctor if it meant taking time
of from work, or who waited until they had toothache
before visiting a dentist. All these people availed themselves
of Club facilities because they were offered as Club
privileges by friends who really cared about their people's
well being. Knowing, as we do, how much illness is
caused by defective teeth, we allowed a system of weekly
payments for dentures. We also secured the services of
a chiropodist, as many of our girls have their feet painfully
affected by long hours of standing. At times we have
had a masseuse in attendance. Indeed, in all our Settlement
work, we have advanced step by step as the definite
needs of our people proclaimed themselves and we sought
to satisfy them.
\end{tp}

The following-up of patients after medical and dental
treatment would not have been possible without the assistance
for the last thirty years of Miss O. Lazarus. She
has undertaken the care of hundreds of families who have
been associated with our Settlement, although they have
no Club affiliation. Miss Lazarus, who is altogether single-minded
and devoted, has always been able to convince
our Settlement associates that their personal problems
were our great concern. Generous to a fault, Miss Lazarus
gives all her time and all her strength to her work, and
wins the confidence of those who come under her influence.
When she started her work, she was prepared to
accept everybody at his own valuation, being entirely
without guile herself. I remember Miss Lazarus’s trying
to cure a gambler who professed a desire to reform by
giving him a race game which he could play without
money! Our friend’s knowledge of life has naturally
advanced with her years of experience, and she is able to
draw the best from others by her own simplicity of purpose,
her unselfishness and her faith in human nature.

She has organised a branch of the Hospital Savings
Association which has over 600 members. One of the
attractions offered to our members through their affiliation
with our group is that when they come to pay their
contributions they can discuss their personal problems and
receive help or advice if they need them. We are naturally
in touch with all the local hospitals, and Miss
Lazarus has a flair for getting patients admitted into
hospital with the least possible formality.

I have tried to show that the health work undertaken
by our Settlement is not the least important of its activities.
Our people gain obvious advantages through having
easy access to expert advice, and their ills melt away in
the friendly atmosphere of our organisation. A Jewish
Settlement must feel responsibility for all the different
aspects of life, physical, intellectual and spiritual; for
the whole of life is holy and reveals the unity of the
Divine Father. A large part of our Settlement work is
concerned with the Children’s Club described in the
previous chapter.

We have always been very careful to keep records, and
to note every visit paid to our Settlement associates,
the kind of advice given, and the developments resulting
from it. We summarise these preliminary records on large
cards for permanent retention. The rough-chart cards are
destroyed. In early Club years, I made all the entries
myself in big volumes, and laboriously summarised them during
my annual holiday when I had opportunity to
study the records. Those were the days before the card
index was in vogue. I thought our successors would be
assisted in carrying on if they knew what we believed to
be the tendencies of our people, and the spiritual resources
which helped to make up their personalities. My friends
and relations expostulated in vain at the amount of
drudgery involved. My father would watch me sitting
with my large volumes on the lawn of our country home.
I think he considered that this grind insulted the sunshine
and cast a gloom over the beauty of the summer days.
But I worked on with a sense of duty and also with a consciousness
of great happiness, feeling like a mother
absorbed in the life-story of her children, in their various
illnesses and joyous experiences, and refusing to be
affected by the gibes of her relations. They could not
understand and appreciate the satisfaction which I was
privileged to enjoy. In the course of years, the records
were dictated to my secretary who wrote them on cards.
For the last ten years, Miss W. I. Paynter has been
responsible for this part of our work. Actuated by human
understanding and sympathy she has been able to do her
work methodically while never allowing it to become
mechanical. The task of record keeping can never be
an easy one.

We Settlement workers naturally receive many confidences
from those among whom we work. In our
records, we are careful to confine ourselves to facts and to
our own observations on these facts. We respect the
sacred nature of confidences and do nothing to weaken
the trust of friends. It often occurs to me that social service
organised on a scientific basis may spoil the chances
of those who wish to start life anew after perhaps making
a mistake which they would like to conceal. We are sometimes
asked by a sister organisation for a report about a
certain associate of our Settlement. Our information may
prevent a fresh opportunity’s being given by a new
source; yet the suppression of information may involve
injustice to others. The person concerned may have
wished certain incidents to be buried and forgotten; but
we know, and, if asked, must say what we know. If we
had no records and worked in isolation, separating ourselves
from the rest of the community, there would be a
serious overlapping. The wrong people would prosper
and the right people would lose their chance. I must say
that I have always had a particular affection for the so-called
“wrong people.” They often possess a fund of
goodness unrecognised by those who are bound by convention
in examining their neighbours’ lives. I often
advocate a little blindness on certain occasions, and believe
that by refusing to be literally correct in our dispensation
of assistance, we may become more just and
certainly more loving.

We had for many years a girls’ home in connection
with our Settlement, but in the end we found that the
average girl preferred to live in lodgings with a friend
rather than tolerate any form of restraint or supervision.
She did not want to be looked after if she could not live
with her parents or natural guardians. Moreover, a girls’
home in the West Central district became for us an impossible
proposition owing to the cost of rent and rates,
and it therefore became necessary for us to close the
home. We have found great difficulty in recommending
safe lodgings to girls who needed them. In our district
few families have spare rooms. During the time the home
was in existence, we had the opportunity of coming in
close contact with many types of girls and young women,
and some of our most valued friendships had their origin
in the Emily Harris Home. It was our good fortune
before the last war to receive as our guests many French,
German and Russian girls of considerable intellectual
ability and strength of character who came here to study
the English language and conditions. We retained contact
with many of our German girls and knew of their
sufferings under Nazi rule. Some few we have welcomed
again as refugees. Some of our home girls have emigrated
to America, and have helped to carry on the Montagu
Club which was founded when my sister and I visited
the States in 1930.

A group of social workers advertised our visit for the
sake of the old West Centralites. A party was arranged,
and the kind gentleman who took us to the meeting place
was somewhat amazed to find a number of young women
throwing themselves upon us in affectionate embrace as
we entered the room. The kind gentleman withdrew,
not because the noise was so great that he thought the
ceiling might descend, but because he feared he was intruding
while such an outburst of emotion was in progress.
We had to hear a great deal about the new country,
and to give all the news about the friends in England.
The minutes flew. Club memories were revived.
When country holidays in Littlehampton were mentioned,
squeaks of delight came from all parts of the room.
Sounds of laughter accompanied references to jolly Club
incidents and the echo of this noise remains with us, and
we should be sorry to lose it. Subsequently the Montagu
Club was founded, and is admirably run by some of our
married members. They are keen citizens and discuss
national as well as social and religious problems.
They keep in close touch with us and exchange reports
with our Married Members’ Guild. They are conscious
of our financial difficulties and send us generous support.
They give their Mother Club very real assistance in her
old age and their support is rendered beautiful by
affection and loyalty.

We have organised all the guilds and societies which
are associated with the name of Settlement. Besides these
activities, we have our small loan fund, and our wardrobe
which supplies clothes to convalescent people, and to
those who want to brighten their appearance when they
seek new work, and to others who are in need. Throughout
the day interviews take place with all and sundry who
call upon us. We work on a small scale as we are an inconsiderable
group of workers, but we all work as hard as
we can, and try to form a useful background for our Club
and its large membership of young people with their fine
aspirations and great expectations.

The Settlement and Club workers co-operate as closely
as possible. Each day messages from the Club are brought
to the Settlement and appropriate advice is given. We
believe that it helps our people to know that we are
available all day long in case we are needed, and that
even in holiday time some of us can be found on the spot
and the others can be reached by letter. This, I think,
is one of the secrets of success in settlement and club
work. We are always there because we care so much.
There is nothing official about us. We are not working
according to routine, capable with a big C\@. We just
want our people to trust us. I am often criticised as
being incapable of taking a holiday. I don’t think this
is true. I can enjoy idleness with the best, but I hold
that holidays are impossible without peace of mind, and
there can be no peace of mind if we do not respond to
the calls that are made to us when we have pledged
ourselves to receive them. Human beings cannot be dropped
at any season of the year. Our life is closely interwoven
with the life of the district. We cannot be separated.
We have no “cases” among our people. Some we like;
some we love; some we don’t care for very much. We
want to serve them all.

Before I close my chapter on Settlement work, I must
write something about R. P\@. She belongs to our Settlement,
and I don’t think there is anyone like her in the
world or that she will have a successor. When I first met
her she lived with her old mother in a tiny room in a
small house in Soho. Her mother smiled a great deal
and wore the biggest and the most decorated bonnet that
I have ever seen. R. was kept by her mother to the
strictest path of rectitude, and she was a devoted daughter.
Both mother and daughter were scrupulously clean;
indeed, their room shone in its cleanliness. They had two
little cupboards; one for meat dishes and the other for
butter and milk dishes. The curtains were provided
every Passover\footnote{Passover is the Jewish spring
festival which commemorates the exodus
of the Hebrews from Egyptian slavery. It supplies for all time the
distinction between licence and true freedom, for it exalts the commandment:
“Let My people go that they may serve Me.”}
with new ribbon bows, and new artificial
flowers were placed on the table. The mother’s feather
bed was reverently looked after by the daughter, and
piles of fine linen, part of the mother’s trousseau, were
in the drawers. R. has had extremely bad eyesight ever
since childhood; so much so that she could not go regularly
to school or be trained to earn her living. Her
mother and she had a small allowance from the Jewish
Board of Guardians. This allowance was continued on
a small scale when the mother died. The same room was
still occupied by the daughter, and the allowance was
supplemented by private charity. The food was still
cooked with the same care and placed in the same
larder—on the window sill. R. said she had enough for “her
poor living.”

R. has worked for the Settlement for many years. She
discovered that many local people wrote to us when they
required small services rendered, or to make appointments
with us. R. saved them this trouble. She brought
me the messages and asked for the money equivalent for
the stamp which was saved. That’s how her penny collection
began. She has a marvellous memory and will string
off with complete accuracy her list of messages. Every
now and then she demands from us small loans for people
she can recommend to our charity, and then collects the
repayments. If there is a defaulter, woe betide that person!
R. takes up a place in the market and collects for
the Settlement. In a year in which her health was good
she has collected as much as £70. She is passionately
devoted as a friend. Occasionally, she imagines I have incurred
somebody’s enmity and rushes to warn me. Her
voice is loud and her words are voluble. When I have
been ill she has been most solicitous and insisted on visiting
me at the earliest possible moment, and has poured
small goodies over me as an incentive to recovery. Sometimes,
she is angry with me, and when her temper is up
she is very difficult to manage. Once, during a season
of estrangement from me for some imaginary offence, she
appeared in the office of a large West End hospital and
offered to collect for them instead of for the Settlement.
The secretary rang me up and told me that a woman,
obviously deranged, declared that she collected funds
every year for me and had offered to help them. “We
could never use such an individual,” he said. He was
mistaken in his judgment. Like most deaf people, R.
does speak very loudly always, bless her, and when she
is angry, she shouts, but, as she says herself, she 1s an
“ehrliches Kind” (an honourable child). She has layers
of clothing which never wear out, but she is very fond of
being smartly dressed, and insists on a bright hat and coat
when she goes for her annual holiday. She makes a
striking picture, since her hair, which is so fair as to be
almost white, is beautifully dressed. R. occasionally
dresses other people’s hair for one shilling, which she gives
to the Settlement.

Our friend gets on very well in convalescent homes,
where doubtless she is regarded as a typical Jewess,
although she is really unique. She regards herself as an
Englishwoman of the Jewish faith. When she goes to the
seaside she is invariably photographed. Her table
manners may be a little strange, but she is scrupulously
clean and uncomplaining among strangers. She is very
generous and has on frequent occasions shared her food
and even her tiny room with someone in need. She has
had one or two offers of marriage, but after consultation
has refused the offers with maidenly dignity. They were
not disinterested.

R. visits the Jewish cemeteries on appropriate occasions
and being a “Zaddik” (righteous woman), her
prayers accompanied by lamentation are regarded as very
effective. She goes to the graves of those who are unvisited
and prays on behalf of their friends. She always
visits the graves of our dear parents once a year and says
a prayer over them.

When the raids began, she came trembling and clung
to me saying: “I’m frightened.” I assured her that I
was also, but that we must say nothing about our fears.
She evacuated herself for a short time, but soon returned
to her little room.

R. is beginning to look very old and frail, but her
energy is not exhausted. She still collects when she 1s in
the mood, and loudly scolds those who offend her. She
comes to us often and gives us her wet kisses and the
pennies she has collected, and the instalments of the “haloans,”
as she calls the small sums of money lent to
people whom she knows. She curses Hitler and all the
enemies of England and of the Jews.

R. does talk loudly, and uses guile to get people to give
their pennies to the Settlement, but she has a great fund
of common sense, and sterling qualities including a heart
of gold, and anybody who wins her affection has a very
precious possession indeed.

