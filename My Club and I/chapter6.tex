\chapter[Industrial Life as Represented in the Club][Industry in the Club]{Industrial Life as Represented in the Club}

When we started the Club, the average earnings of its
adult members was 12s.\ a week. Children leaving school
began their working life by earning 1s.\ a week, but
this starting wage was soon raised to 2s.\ 6d., which was
the average wage paid to children until about fifteen
years ago. It is true that the buying value of money was
considerably higher then than it is to-day, but the wages
of women and girls were so low that it is not surprising
that they regarded their working life as a temporary evil
rather than as a career in which it was desirable to be as
efficient as possible. Girls, or, at any rate, Jewish girls,
were expected to marry, and they often married for the
sake of deliverance from the crushing circumstances of
their lives. Most of my girls in the early years of the Club
were tailoresses in domestic workshops. The employers
had to carry on their work near the big, fashionable West
End retail shops. Rents were inordinately high, and it
was necessary to live near the “shop,” so that rooms had
to be used both for living and working purposes. The
average working hours were 8 a.m.\ to 8 p.m., with an
hour for dinner and half an hour for tea. In hot weather
the men often worked stripped to the waist, and the
atmosphere created by overheated bodies was anything
but pleasant. Many of our elegantly clothed men friends
would have been surprised and shocked if they had seen
some of the places in which their clothes were produced.
They were under the impression that the workrooms in
which the workers sat were somewhere at the back of the
pleasant shops in which they gave their orders and
received assurances that their garments would be faithfully
delivered on a given day. It would seem, indeed, that the
outfitters regarded it as a joy and privilege to have the
garments produced as required by the customer. The
purchaser knew nothing about the real conditions and
did not enquire. Many of our girls worked for very long
hours as dressmakers and milliners. It seemed remarkable
that children could survive the strain of such
protracted sedentary occupation. It is not surprising that
they were glad of the chance to make a noise in their
Club in the evening, or that they talked loudly in the
streets. They needed an outlet after long days of repression.
To-day we don’t like to contemplate the average
eight to ten hours during which many of our girls of
fourteen work, but although sanction for the twelve-hour
day still disfigures our factory laws, the working day has
been considerably shortened.

All through our Club history, we have been familiar
with the fact that in many workrooms and workshops bad
language is used and smutty stories are told.
The tone of the workroom depends mainly on the
personality of the manager. As in most human activities
character tells, and it is not only the standard of the
manager which counts. Every individual worker may
assist or help to frustrate the will for good in any workshop
or place of business. One of my friends told me how
she was chased round the room in which she worked, and
her ears stuffed with cotton wool, when, as a very young
girl, she had ventured to protest against the discussion of
some obscene subject. Her courage proved to be not
without influence.

Workshop life has immeasurably improved with the
increased wage-carning capacity of women and with their
political enfranchisement. Many of our girls through
skill and industry attain very responsible and well paid
positions in workrooms and business houses. In the clerical
world they have shown very great proficiency. Although
as in every other group of girls and women, we have had
and still have some of the “light make,” who do exactly
what they are paid for, and a little less if possible, the
great majority have always shown themselves eager to
take an important part in the industrial concern which
they enter. It is often true of the girl, as it is of the boy,
that on the occasion of her first interview with her future
employer she is considering how she will run the business
when she becomes a partner. The vast majority of the
girls are interested in their work. They like to show
initiative and to give of their best. This was particularly
noticeable during the 1914 to 1918 War when girls turned
from uncongenial work and won for themselves good and
responsible positions in new trades through sheer
initiative and hard work and industry. A very high percentage
prefer clerical work to any other, because they think it
more varied and interesting than the sewing trades, and
they are often too well acquainted with the harsh
conditions of the sewing trades in their own homes. They
have seen there the long period of slackness which causes
such insecurity among working tailors and renders the
objectionable credit system necessary.

I could say much about the credit system as it is created
by the well-to-do customers. They do not understand
how their procrastination in paying their bills makes it
necessary for the young girl in the workroom to endure
much slackness, during which she hardly knows what to
do with herself, and sometimes seeks excitement at a
dangerous cost, in sheer boredom. They do not know that
because they do not pay for their clothes when they
receive them the work-girl must purchase her own clothes
through clothing clubs, and often wears out her garments
before they are paid for, and so lives in debt week after
week.

In the domestic workshop there is demoralising
unemployment to be borne in the early part of the week,
and a terrible pressure of work as the week advances.
It is not surprising that the young people seek work away
from home. Moreover, home life necessarily suffers when
the best room is required as a workshop, and this arrangement
often causes overcrowding in the other rooms. It is
therefore to be expected that when a girl tells us of her
engagement, she is pleased to be able to say of her fiancé:
“He is not a tailor.”

Quite a number of our Club members have elected to
take up social service, for though the remuneration may
be restricted, they have come to love the work through
doing it in the Club, and they are eager to make use of
their powers as far as they possibly can.

An interesting experiment in industry has grown out
of one of our own Club activities. I have always felt
that a well organised business which offers opportunity
to people for self-development under good and happy
conditions is one of the best forms of social service. In
consequence of this, my sister and I, assisted by the great
interest of our mother, decided many years ago to teach
our girls artificial flower making. The trade was mostly
in the hands of French, German and Swiss makers. We
saw no reason why our English girls should not make the
flowers equally well. With the help of the L.C.C. we
had a successful class. We brought over a teacher from
France to give the girls special training. Gradually, we
founded a small co-operative business in which the girls
might find outlet for their creative talent and for their
business powers. We had no desire to reserve this business
for our own Jewish girls. It employed in the ordinary
way workers of any creed who did not object to Sabbath
observance, and a week of five days was established. My
sister has served as treasurer to this business and has given
much time and thought to its interests. Many have been
its vicissitudes as it emerged from a semi-philanthropic
industry to a limited liability company. As the honorary
chairman, I am deeply interested in the work, and especially
in the welfare of the workers and the idealistic
element which the enterprise embodies. We have had
wonderful times when, perhaps too rashly, we were able
to divide a large sum in the form of bonus among the
workers; we have had bad times when we have lost
heavily. There is no profit-making outside the workers’
own wages. When the general community has been slack
in all industries, we have been slack also, but we have
carried on. Our workers have had hardly any period
of unemployment, although after the last war we had to
reduce our staff, all our workers finding good openings
elsewhere. The happy conditions are much appreciated.
The present directors who started the enterprise with us
are remarkable for their skill and altruistic devotion to
the interests of the firm. We work for the best houses,
wholesale and retail, in the West End.

Outside the West Central Flower Company, I must
admit to my deep regret that with a few exceptions our
girls have not identified themselves closely with trade
organisations. Certainly, one girl did much useful trade
union work here and then went to the U.S.A., where she
has attained considerable success as the head of a
communal centre in which organised labour plays an
important part. Another did much lecturing on industrial
subjects. Perhaps, however, because of persecution, most
of our parents are individualistic. They have had to
fight for their places in the world of industry ; they have
had to win for themselves the right to work for their
living; so their strength has spent itself, and they have been
inclined to teach their children to get on with their work,
mind their own business and leave other people to get on
with theirs, saying: “What’s it to do with you?” Again
and again we have tried to inculcate a wider point of
view, and to explain the advantages of belonging to
unions. The results have been rather sporadic and not
very successful.

Quite early in our Club history, we realised the
importance of securing for our young girls at least the
minimum legal standard of working conditions. We knew that
bad employers, by evasion of the law, could get an unfair
advantage over those who benefited their employees by
keeping the law. The first necessity for us leaders was to
know the law, we had the assistance on our Council of
Miss Clementina Black. At one time she made a skilful
rhyme of the Factory Act. She knew that the law, as
printed on paper, and hung on the walls of factory and
workshop, was most unattractive reading to the average
worker whom it concerned. Moreover, the girl would
draw considerable suspicion on herself if she stood gazing
at the Factory Act in her leisure time and tried to
disentangle its enactments. I don’t know whether the rhyme
ever proved very instructive, but I know it roused
considerable mirth among the members of our Club.

In order to obtain the confidence of the workers and
convince them that it was their responsibility to report
evasions of the Factory Act, we founded the Clubs’
Industrial Association. Club leaders and delegates from
affiliated clubs met on Saturday nights once in two
months and listened to lectures and joined in lively
discussion afterwards. We had lectures from factory
inspectors, especially the senior inspector, Miss Anderson
(afterwards Dame), and Mrs.\ S. Webb, and since we had
the girls’ confidence we could easily discover illegal
overtime, and other evasions of the Factory Acts. We knew
the injured-looking worker who always looked as if she
were lifeless. Her spirit had been suppressed by bad working
conditions. When, after considerable effort, we could
get that girl to tell us some of her troubles, we might
discover that she was among the sufferers whose hardship
was caused by illegal employment. If we had knowledge,
we could take appropriate action, and, without risk to the
girl, report to the Factory Inspector, who, through her
visits of investigation, could right the wrongs for our girls
and also for many of their fellow-workers.

\begin{tp}{512}
Under the presidency of Mrs.\ Sidney Webb, we formed
the Industrial Law Association, in connection with the
Women’s Industrial Council, and lectures were organised
for club leaders who would take counsel together on
general industrial conditions, and they in their turn would
instruct their girls. Among other distinguished colleagues,
were Miss Margaret Bondfield and Miss M. Symonds.
These meetings were often held at Alfred Place, when our
Club was established there, and we were delighted to see
evidence of unity and co-operation existing among the
girls of various creeds who together studied their own
industrial conditions, and sought ways of improving them.
Incidentally, the girls got to know and like one another,
and the National Organisation of Girls’ Clubs, in which
the Clubs’ Industrial Association was ultimately absorbed,
benefited from the basis of friendliness on which the
preparatory work was founded. Girls became unafraid to
tell of the many ruses which bad employers used to outwit
the inspector. I remember hearing how a girl's young
man was called in to assist the inspector. An official had
often called at a certain workshop, suspecting that work
was being carried on in the upper floors, and would
mount the stairs to the first landing, but did not dare to
proceed further because of the complete darkness which
prevailed. The innocent caretaker had no knowledge of
anything beyond the passage immediately connected with
the front door which she opened. The young man heard
from his girl that on a certain night some late work would
be called for, three knocks would be given, and from the
dark building would emerge girls delivering the promised
work. The boy, having made his entry in the prescribed
way, watched with the inspector in the passage. The
work was brought down, and the illegality was then easy
to prove.
\end{tp}

From the Industrial Law Committee, which was a
separate organisation for work among club leaders and
other social workers, and which generally met at
Mrs.\ H. J. Tenant’s house, there was formed the Industrial
Law Indemnity Fund, which still functions under the
presidency of Lord Lytton, the funds being in the hands
of four or five trustees. Miss Towers acts as honorary
secretary. If it is proved that a girl or young person loses
employment through fulfilling her duty as a citizen in
reporting infringements of the Factory Act, she is paid
out of the fund until she is in employment again. It 1s
clear that in the work of industrial amelioration, club
leaders need not be handicapped by the fear of causing
their girls to lose their means of support through speaking
the truth. Besides saving good employers from unfair competition
with unscrupulous people, we often gave great
encouragement to the workers themselves by telling them
of good conditions which undoubtedly existed in close
proximity with the bad. Sometimes good employers came
with their girls to talk over in a friendly manner possibilities
for increasing the general comfort of the workers. At
our meetings all kinds of opportunities were given to the
delegates to give expression to their views on different
aspects of industry. I remember one evening we
considered the life of the domestic worker, and tried to
remove some of the popular fallacies about the conditions
of her life. One girl, herself a cook, and the leader of a
smail club, told us of her joy in being completely trusted
by her employer, who gave her the keys of all her stores.

In the early days of our Club, there was an exhibition
organised by Dr. J. Mallon (now Warden of Toynbee
Hall) in order to combat the sweating which prevailed in
some of the trades. I persuaded two of my girls who
made cigarettes under revolting conditions to demonstrate
at the exhibition. Unfortunately, I forgot to remind them
not to bring any bags or boxes bearing the name of their
employer. He visited the exhibition, and was much
intrigued to find that he was contributing to its interest.
The girls lost their work, but there were many wealthy
people on the committee who were prepared to assist
these victims. In no other circumstances have I found it
so easy to collect money for any charitable cause. I
received so much that I was able to set these girls up in
a small private business.

In our Club we have realised for about thirty years
the difficulties to which a girl was subjected if she went
from place to place and found herself unwanted. There
is something particularly tragic in a girl’s losing her work
especially if she is not living at home with her family.
Money is so easily obtainable if she does not hold
steadfastly to her moral standards. Happily, our girls have a
fine inheritance of moral fibre, and I cannot recall one
case in which a girl has been tempted through economic
difficulties to lose her self-respect. We have had a
few—a very few—instances of girls who have sacrificed their
honour in order to gratify their lust for excitement, or to
increase their power of self-adornment, but never have
we had girls who felt driven to a bad life because they
could not otherwise support themselves. It is against such
possibilities that the collection made by our Silent Friends
is used.

My friend, Miss Lewis (known as “Father” in Club
life) started her employment bureau about thirty-five
years ago. This was long before the institution of the
Labour Exchanges. Numberless girls have owed their
industrial start to her knowledge of conditions,
and her tact and sympathy. She knows that when girls
are out of work they feel intensely sensitive, and are
generally desperately unhappy. They hate to be
unwanted, They are conscious of their commercial value
and lay great store upon it, because it represents their
life’s capital. No official could allow the girls to talk
about themselves as Miss Lewis can. She gives
them the relief of talking as long as they wish. In the old
days, she would ask her group of unemployed girls to go
with her to interview possible employers. She had many
friends in the neighbourhood and great persuasive power,
and would make an employer think he wanted a particular
girl of a particular age when he had until then thought
quite otherwise. Miss Lewis has again and again bridged
over little differences between girls and their employers
and prevailed on the girls to endure patiently certain
temporary inconveniences in order later to improve their
position. She has a gift of infinite patience for all this
work, and can give the necessary encouragement in
moments of great disappointment and depression. In
recent years, it has not been considered necessary or found
possible to take the older girls to interview employers.

Miss Lewis is now so well known to neighbouring
employers that she can find an opening by letter or
telephone, and then send the girl to fill it.

The work of our bureau has greatly decreased since the
instalment of Labour Exchanges. Miss Lewis and her
assistant work in close touch with the Labour Exchanges
and add that personal note which makes all the difference
to those in need of employment. She attends the local
school conferences and visits the parents of school-leaving
girls, and so helps the children to make the wise industrial
start on which their future lives so much depend. The
prejudice against employing Jewish girls is one of the
unnecessary troubles which we have to endure, and it
causes us naturally a great deal of pain. I do not think it
can ever be justified on grounds of real religion. The
discrimination is so entirely unjust that it is a challenge
to those who profess Christianity that they should eradicate
it. The girls, as I have explained in another chapter,
are forced through economic pressure to work on the
Sabbath. They ask, as a rule, for three special days in
the year for religious observance. There are few workers
who are refused this very moderate concession on any
ground whatever. When they are known as Jews, as they
should be, they are respected for their allegiance to their
religion. They are generally prepared to make up any
time lost, or to relieve their Christian fellow-workers if
any emergency should arise and they in their turn wish
to take a little time off. Our girls have on occasion,
however, been refused work because they are Jews. Sometimes
they have served faithfully and well for a number of years
and have then been discharged to satisfy the wishes of a
new and prejudiced manager, or a certain director who
had to be appeased. It is obvious that this unkindness
creates rebellion and misery in the mind of the Jewish girl.
She suffers through no fault of her own. The firms have
some Jews at least among their customers, and do not
scruple to trade with them. The God whom the Jews seek
to serve is the God of the Christians. The founder of the
Christian religion was himself a Jew, and he expressed
his faith in the “Shema” (Deuteronomy VI, Verses 4-9),
the manifesto of our brotherhood. The employer says to
the bewildered and indignant Jewish applicant that he
must consider English girls first for work. But our girls
are English, or if not born here have for the most part
been educated in England. They wish to identify their
interests with the interests of our country. They do not
regard themselves as anything but English. There may
be some justification for the feeling that English girls must
be taken before foreign girls, but surely there should be
no religious discrimination. Moreover, if the position of
the foreign girls is investigated, and it is realised that they
are bringing to us some gifts of mind and heart with
which they ask leave to enrich our English life, are we
not doing wrong if we refuse employment to such
workers, whatever their origin? We wish to love our
neighbour as ourselves. She needs the opportunities we
value. We allow her to live in our midst. Dare we deny
her the means of living? Surely we shall not suffer
through employing our neighbour who is in need of work.
Is our faith not strong enough to make us believe this?
One girl may be less fitted than another for a particular
job. Then, in justice to the public, we must refuse her,
but if she is the more suitable, then, whatever her faith
or nationality, let us not refuse her, for in so doing we
are turning away from the All Father and creating misery
for which He will make us render an account. He does
not withhold His mercy from any of His children. How
can we discriminate between worker and worker, and, at
the same time, invoke His blessing on our lives and work?
