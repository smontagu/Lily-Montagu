\frontmatter
\pagestyle{empty}

% half title page
\vspace*{2\baselineskip}
\begin{center}
  Lily H. Montagu

  \vspace*{2\baselineskip}
  \textbf{My Club and I}
\end{center}

\clearpage

\begin{center}
  \vspace*{2\baselineskip}
  
  \small Originally published by Neville Spearman Limited \&\\
  Herbert Joseph Limited, 1954

  This edition 2024
  \vspace*{22\baselineskip}

  \includegraphics[width=16mm]{hatafSegolLogoNoText.png}\\
  
  {
    \fontspec{Linux Libertine}\Large\bfseries Hataf \ Segol\\Publications
  }

  \vspace{1\baselineskip}
  \small
  Typeset in \XeLaTeX\ by Simon Montagu
  \end{center}

\clearpage
 
{
  \vspace*{4\baselineskip}

  \centering\Huge\textbf{My Club and I}

  \large\textsl{The Story of the West Central Jewish Club}

  \vspace{2\baselineskip}

  \Large Lily H. Montagu
  
}

\cleardoublepage
\vspace*{10em}

{\centering
  \large DEDICATED

}
{\parindent 0em
with love, gratitude and hope to those who yesterday helped me to build up the West Central Jewish Club and Settlement; to those who are to-day struggling to carry on the work; and to those who to-morrow will realise some of the fruits of our labout, and who will, on the old foundations of faith and mutual service produce in the West Central district something far better and nobler for our Jewish youth.
}

\cleardoublepage
\tableofcontents*

\numberlesschapter{Foreword}
{
The story of the West Central Jewish Club was written
in 1940, before our building was destroyed by enemy
action. I have thought it best to leave the story in its
original form until, in the last chapters, I have explained
that the existence of the old Club in Alfred Place is
ended for ever.

I have described our methods of Club organisation
with some elaboration, because I hope that our experience
may be of value to Club workers and other students
in social service. For that reason also, I have thought it
worthwhile to dwell upon our great hopes and our small
achievements while these are fresh in my memory.

Our original Club members, who are now scattered
all over the world, and their descendants, will, I think,
respond to the interest of the story which, although it has
ended in a calamity, is, when regarded from one angle,
full of joyous hope which will endure.

Throughout my story, I have made only a few references
to my “better half,” my wise and beloved sister
Marian, who has accompanied me step by step in my
travels in Clubland. Our lives are entirely intertwined,
she supplementing me in some of my many deficiencies.
I thank God that I have been able to share with her all
the work and all the joy which our Club has given us in
such abundant measure. By thousands of members, Miss
Marian, our Club “Auntie,” is respected for her wisdom
and unselfishness, and loved for all the tender beauty of
her character.

I owe much to my other sisters, especially Mrs.\ Franklin,
with her readiness to share her possessions with all
and sundry, and her knowledge of educational matters;
Mrs.\ Waley, who became our capable and devoted
Treasurer; Mrs.\ Hart, with her gentle sympathy; and
Mrs.\ Myer, with her downright sincerity.

\begin{tp}{512}
It would have been impossible for my sister and myself
to give whole-time social service had it not been for our
friend, Miss Lewis, who has looked after our home and
relieved us of great responsibility while interesting herself
in all our work. She, as one of the Red Lodge trio, supplied
the happy, peaceful background which has meant
so much to us.
\end{tp}

In so far as my Club is the expression of the faith by
which I have lived, I owe to my beloved friend and
teacher, Mr.\ C.\,G.\,Montefiore, much of my power to
carry on, for the secret of the Lord was with him, and
he shared it with those who like myself went to him con-
tinually for encouragement and advice.

I have mentioned by name very few of the helpers to
whom I owe so much, but it must be understood that to
hundreds of the unnamed I should like to offer my most
sincere thanks. In deep affection, I say to all of them:
“As your Club Mother, I thank you and wish you God-speed.”

{
\raggedleft
Lily H.\,Montagu.

}

\noindent\textsl{London, August 1940—December 1941.}

