\chapter{Club Activities in War-time}

Our Club is so old that it has experience of two wars. In
1914, I remember establishing a workroom in our Club
Hall, under the auspices of the National Organisation of
Girls’ Clubs, and with the assistance of Mrs.\ Arnold
Glover, who founded the organisation with me, sought
to employ girls who suddenly lost their work through the
disorganisation of industrial conditions. The Club Hall
resounded with machines. Our workers included the
roughest of rough girls from all parts of London. We had
our first experience of pulling girls away from soldiers,
into whose arms they literally threw themselves when
they managed to escape into the streets on some pretext.
We had also to prevent violent fights between girls whose
language was often quite lurid. However, the majority
of our workers were of the type to whom we had been
accustomed in our own Club, and they were glad to avail
themselves of the opportunity to earn enough with which
to maintain themselves. I remember that we had one
room set apart for aliens, some charming folk with whom
we had long been acquainted and who were really devoted
to our country and were distressed to be classified as
enemies. Our workroom was carried on and supported
by faith. I don’t think I have ever at any other time been
as tired as I was during this period, but I was very much
impressed by the generosity of those who visited us and
gave us money for materials and wages so that we could
carry on until industry became more normal and our girls
could find again regular employment. Somebody invariably
appeared to help us when our coffers were all but
exhausted.

The raids, to be considered as mild in the light of later
experience, were directed against London only towards
the end of the war, and we were able to carry on with
our classes and socials in the usual way. Club holidays
were not given up, but instead of venturing to the seaside
we made ourselves happy on a country farm near
Windlesham. Our teachers were splendid and worked
on during the raids, comforting and encouraging the girls.
When the warning came a few ran to the Tubes, but the
majority deprecated such timidity and bravely remained
in the Club, singing and playing games often to a late
hour. Sometimes girls would rush into the Club during
a raid, as they felt they would be safe there and wanted
to join us for prayers. Like so many people, they expected
spiritual help to flow to them as soon as they turned on
the right tap, even though, in some instances, they had
allowed rust to accumulate through many years. God
must smile sometimes at our funny little ways.

When the present war began many of our girls were
evacuated with their firms, and others with their famililes.
Some parents were nervous about letting their girls leave
home at all in the evening. Miss Levy kept in touch
with as many as possible through correspondence, and
the young people visited us whenever they had the chance
to come to London. Classes were held twice a week and
were well attended. Our Sunday afternoons and evenings
attracted quite large numbers, so large that some
of us offered special prayers that no warning might sound
until the young people had dispersed. We always advised
about shelter accommodation before we announced any
item on the programme.

We were obliged through the limitation of petrol
greatly to restrict our visiting, but visitors continued to
come to us all the week, and particularly on Sunday
afternoons when there were consultations without interruption.
Miss Levy did as much local visiting as she
could. The anxiety of the war had increased personal
and social problems, and the industrial situation made
our people’s difficulties particularly acute.

Since the Nazi persecution began, we have had dealings
with a vast number of German refugees, with whom
I had become acquainted, in the first instance, through
my international religious work. Judging from the number
of people who claimed to know me through having
heard me preach in the Reform Synagogue in Berlin, I
must have had a mighty congregation on that occasion.
I think that a few extra cousins, aunts and uncles must
have become attached to the original group. In any case,
I am always glad to do what I can to help and give advice
to German and other refugees who find their way to our
doors, and work in conjunction with the Council for
Jewish Refugees, which had its centre at Woburn Place
and later at Bloomsbury House. Our Settlement Secretary,
Miss W. I. Paynter, was responsible for the elaborate
letters which we sent to various government departments,
and the extensive records which it was necessary to keep
of each individual.

With the coming of the \textsl{Blitzkrieg}, we were able to
maintain only our Saturday afternoon services, which
have been continued without a break, and to open the
Club on Sunday afternoons for social purposes. Of course,
our Settlement work went on uninterruptedly during the
day. We lent several of our rooms to the Domestic
Bureau of Bloomsbury House, and organised English
classes for some of the students while they waited to be
interviewed. We also lent the Society of Friends the
gymnasium for refugees to work in, the responsible organisations
paying us out-of-pocket expenses.

During the Blitz, our Club offered shelter at night to
a group of people who included the residents in the flat
adjoining the Club (the remnant of the Emily Harris
home), some refugees, and a few cherished neighbours
whose homes were frail or who had been bombed out. We
explained to our guests that our building was not a real
shelter, but that it was considered reasonably strong, and
we tried to make our friends as comfortable as we could.
They said they felt safe in the Club and preferred to be
there rather than anywhere else. At one time there were
as many as fifty shelterers, but these gradually decreased
until the nightly attendance varied from 25 to 30. A considerable
amount of space was given over to bedroom
accommodation. Most of the shelterers had a meal when
they arrived, and the canteen was open for their benefit
during the evening. A library and class were started, and
a wireless set was obtained. Friends knitted and sewed
and read and talked. Miss Jessie Levy acted as marshall
and held Friday evening services which were followed by
a little reading. It was a large family that created a
nightly home for themselves. One girl said to me: “This
place is guarded. It is full of prayers and happiness.”

We held the Married Members’ Guild meetings as
usual during war-time, and were able to face life’s problems
with our friends and find comfort in the sympathy
we were all ready to give one another. The singing and
dramatic classes found time to meet and gave performances
which were very attractive. A number of classes
were, as time advanced, held on Sunday afternoons in
addition to the social activities. The doctor, dentist and
chiropodist held their clinics. But the. financial position
became more and more difficult as the members could
not be asked for their usual subscription while so few
activities were functioning. The Club’s Savings Bank,
which was always administered by my sister as a branch
of the Post Office, changed its character, for she became
the organiser of a National Savings Group. My monthly
letter on religious subjects was issued as usual, especially
to evacuated members.

Miss Levy, assisted several days a week by Miss Lewis,
ran a canteeen which produced a welcome profit. The
midday meals were much appreciated, as was the happy
atmosphere of the Club kitchen.

Before the evacuation of the children had started, and,
while the London schools were closed, Miss Simonis
organised a morning school. The children came regularly
and profited from the teaching, which was good and
efficient, although sometimes a little unconventional. We
had a delightful set of children, but we were naturally
glad when most of them were removed to the country,
where we hoped their safety would be more certain than
in our danger zone. For the few children who did remain
in London, the Children’s Club was reopened on two
afternoons a week. The work of the religion school
changed its character, the lessons in Scripture, Jewish
principles and Hebrew being sent by correspondence, the
answers being returned by the children for correction.
All through this period members and leaders
managed to keep up their courage and confidence. We
all believed in the ultimate victory of our country, and
hoped to be among those privileged to build up the new
lite which would follow. Many of our young people had
entered the services; all were working, some in very
responsible positions. Some lost their homes in raids, or
disaster had come to their friends. Life was not easy for
anybody. Yet cheerfulness and courage gave the dominant
influence to Club life. Whoever came to Alfred Place
knew that she or he was wanted there. Sympathy and
understanding were basic in our Club in peace time, and
were much stronger in war-time. We were interested in
one another and we cared deeply.

Towards the end of March we decided to take advantage
of the longer evenings and to open the Club for
educational work from 6 to 8 o’clock. The L.C.C. gave
us every encouragement. Our young people were delighted
to return, and so we started evening activities.
By April 16th much of our work had been resumed, and
we were all very happy. Miss Levy looked radiant in
spite of the marks of anxiety the war had already made
in her face. We were all glad that Wednesday night as
we said good-night to our shelterers and residents, and
to dear Miss Paynter and Miss Hetty Lewin, who had
come to do their firewatching. That night of
April 16th brought one of the worst raids of the war.
Soon after 2 o’clock our building was struck by a land-mine
and completely demolished. With it went the lives
of the twenty-seven friends who were there, some of them
very close friends, and all our possessions and treasures
and the records of over forty-seven years. The terrible
tidings were brought to me at 6 o'clock in the morning.
The clergy of Whitefields Tabernacle lent us their
beautiful church on the following and successive Saturday
afternoons, and on April 19th a congregation numbering
over 500 people—Club leaders, members and
friends—prayed together. Helped by precious memories
which no enemy action could affect, we dedicated ourselves
to work for the reconstruction of the West Central
Club, Settlement and Synagogue, in a new and better
world on the old foundations of justice, love, truth and
religious faith.
