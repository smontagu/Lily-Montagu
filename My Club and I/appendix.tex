\numberlesschapter{Appendix}
\renewcommand*{\precisfont}{\normalfont\center}
\renewcommand*{\postchapterprecis}{\end{quote}\vspace*{\baselineskip}}
\chapterprecishere{A Message of Thanks to\\
THE HON. LILY H. MONTAGU\\
from her “Children ”\\
On the occasion of the opening of the New Club,\\
31, Alfred Place, Tottenham Court Road, W.C.1.\\
Sunday, November 16th, 1913.

}

\noindent Miss Montagu,

I have been asked to say a few words to you and other
Club friends on behalf of the Representative Committee
and the members of the Club for having made it possible
for us to meet together this evening in such a beautiful
building. This evening means much to us. It is the beginning
of a new Club life with greater opportunities for
education, self-improvement and development. We were
always saying in the old Club, “Oh, if only we had larger
premises, we could do so much.” Through your efforts,
we have these larger premises, but we must now realise
that the larger premises without the personal effort of the
members are useless, and I know I speak for the majority
of the members when I assure you, Miss Lily, that we
mean to make our Club a success in every sense of the
word; we mean to give of ourselves unselfishly and
lovingly.

To you, Miss Lily, this evening means much also. We
know it is the fulfilment of one of your most earnest
desires. You have always wanted to give your children a
second home, a home away from home, as the pageant
song suggested, and to-night you must be proud of your
achievement. You have planted the roots of the Club of
the future, and it is for us members to make them
blossom and bear good fruit, and is this not an opportune
moment, as we have reached the coming of age of your
work among us.

So many people have said to me at various times, “Oh,
your Club is so different. You all seem so fond of each
other, as if you belonged to one family.” We members
know that this is the feeling Miss Lily wishes to instil into
each one of us. Her very presence is love, and we try
in our small way to carry her message, and in our new
Club we must even strengthen this spirit. We are one
family; we have our Club “Mother” and “Father” in
Miss Lily and Miss Lewis, and our devoted and loving
“Auntie” in Miss Marian, and other aunties in many of
our workers. Can any family boast of more loving relations?
Can any mother do more for her children than
our dear friend, Miss Lily, and girls, if we wish to prove
to her, as I know we do, our gratitude for the opportunities
she tries to give us, we now have the chance. The
Club will require every member’s help, and that help
must be given without thought of self; given for the sake
of each other, and the ultimate good of our Club, and if
every member tries even in the smallest way, what a
help we will be to Miss Lily, and how much lighter will
her strenuous task become.

We must all feel especially grateful to Lady Swaythling
for her untiring efforts on our behalf. She has worked
very hard to get the Club ready for this evening, and we
know she has done much in assisting in all the work which
the new Club has involved. We hope she will visit our
Club whenever she possibly can, and we are going to-night
to include her as one of our family, our “Grandmother,”
if she will allow us, for we fully realise that
without her loving sympathy Miss Lily would be unable
to work for us as she does.

We must be specially grateful to Miss Delgado, who
has practically lived at the Club for the last fortnight,
assisting in all the work of moving and arranging. We
thank all the friends for the help they have given Miss
Lily for our sakes, and we know that without their help
she would have been unable to accomplish all she has
done. We must express our special gratitude to Miss
Marian and Miss Lewis, who, we know, work side by
side with Miss Lily, forgetting themselves in their
endeavours to be of real help to us.
