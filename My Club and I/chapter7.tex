\chapter{Social Life in the Club}

We have always found the educational side of our work
easier than the social. This is due partly to the fact that
our members are of such different ages, and the entertainment
which appeals to one set does not always appeal
to another. Also we have girls from every possible type
of home, from the most refined and cultured to the most
elementary. I am thankful to say that we have been able
to bridge the social differences with great success and
friendships have been formed which have no relation at
all to the social origin of the members.

\begin{tp}{256}
Housing problems have, however, affected the social
life of the Club. The district of Soho in which our Club
functioned was among the worst in London as regards
housing. There was no possibility of privacy or of
entertaining at home. The tenements were overcrowded,
and in many of the courts the houses seemed to bend
over and shut out both light and air. Our members had
to depend on outside agencies for their social life. Soon
after Miss Lewis started her flower guild, through which
seeds and plants were distributed for home use, a little
girl was seen standing in the street clutching her plant.
“I thought I had better take it outside,” she said, “because
it wanted air and light.” It has taken some generations
of thinking men and women to realise fully that
human beings also want air and light.
\end{tp}

With the changes which have been effected in recent
years in our district, the houses have in many cases been
transformed into business premises, and the living
accommodation has been still more restricted. The
domestic workshops are now fast disappearing. Our local
members come from St.\ Marylebone, St.\ Pancras and
Holborn, as well as from Soho, and in these districts also
we have seen every example of housing discomfort,
although each borough has its distinctive peculiarities in
housing, and deals with them in its own way. It has
happened that during recent years, the Club, although still
catering for the needs of local groups, has become more a
central Club than a West Central Club. We are easy of
access, and the girls of to-day are accustomed to keep
quite late hours. So we have members from all parts of
London. Some come from comfortable homes, and their
families and nearby friends supply them with social
outlets; they are not so dependent on Club entertainment.
There is something, however, which the Club gives which
cannot be supplied elsewhere, and which counts for a good
deal in the girls’ lives. This something is made up of
personal sympathy and understanding, and an opportunity
for corporate service which stimulates the individual
girl’s self-respect. Through belonging to the Club, she
becomes of greater importance to herself, and feels
that she may be needed in the great world outside.
She must cultivate her powers in order to be of use. There
are people, even outside her home, who care very much
that she should be good and happy. In the Club, and
especially through our visiting, we have established
relations with our girls, and from this intimacy it has been
quite easy to create a delightful community of feeling.
We have succeeded in securing a really happy tone, which
is quickly noticed by everybody entering our building.
When our Club was purely local, everybody knew everybody
else. Many came from the same school. Fathers
belonged to the same friendly societies or benefit societies
or private societies. To-day girls are more isolated. One
girl recommends another, and groups are represented.
A new girl feels rather alone, but this, in spite of all
welcoming efforts, is inevitable, seeing that we have had a
membership for the last fifteen years of between six
hundred and eight hundred girls.

In the early days at Dean Street, the sitting-room
attracted those girls who did not belong to classes. There
are many old and middle-aged women who remember
the appearance of that room with its few chairs and its
large crowd of young people chatting loudly, sitting on
the table, or on the floor, on the sides of the chairs, and
telling of their experiences and giving vent to their
opinions.

We had Sunday concerts and dances. These concerts
were not always entirely enjoyable, at any rate to the
organisers. In the early days of our Club, there were
undoubtedly a certain number of girls who came to these
concerts for no other purpose than to meet one another
and to be troublesome. The concert, however good, had
no appeal for them. On the other hand, baiting those in
charge was distinctly amusing. I was firmly and rightly
convinced that only the best concerts were good enough
for my girls, and the majority responded to what
was given them. They liked best a variety concert, which
did not make too severe a demand on their attention, but
they were also interested in classical music, if well played.
Some of my friends, who were good enough to give us
the entertainment we desired, made no allowances for
bad behaviour. They said this was due entirely to my
propensity to spoil the girls, to let them take everything
for granted, and to show no appreciation whatever for
those who were good enough to give their time and talent
without asking anything in return except a good and
attentive audience.. In vain I pointed out that the
majority were good and attentive, that if they were all
perfect we should not need a Club at all, and that I, as
honorary secretary, and they, as entertainers, would
certainly be considered superfluous, even if not criticised for
failing to maintain the Club standard. I fear I was
always rather impatient with those who talked of their
goodness in giving free concerts. All through my Club
life, I have held the view that as we workers receive from
our work far more happiness than we can hope to give,
we have no right to talk after the manner of benefactors.
Nevertheless, I did suffer agonies when one wicked group
misbehaved, and I am afraid they did so very frequently.
Like a mother who cannot bear her children to be talked
about by people who probably do not understand
children, I hated the thought that the West Central
audiences were severely criticised in many corners of the
community, where people would be quite indifferent to
the undoubted virtues of the majority. My naughty girls
would sometimes recognise my distress as I walked up
and down beside them, trying to cover their giggling.
‘They promised amendment after hearing threats of
suspension, and, for a time, they looked angelic and
always declared that I had made a mistake and had not
admonished the real delinquents. It was always
somebody else who was at fault, and though they were sure
I meant to be just, I made really terrible mistakes and
brought unfair reproach on the innocent! The
well-behaved members were often roused to fierce anger
because the few brought their Club into disrepute. If
they tried to quiet the noisy groups, they were bluntly
told to mind their own business. One of my most
devoted and single-minded colleagues did her very best
to assist in keeping order, and won the title of policeman.
While she spoke severely to one little section, another
started their giggling, and she was woefully misled
through shortsightedness. In spite, however, of our
difficulties, our Sunday concerts played a very important part
in Club life, and attracted large numbers of friends
outside the actual Club membership. The behaviour
gradually improved as the members themselves took greater
responsibility. To-day, new problems have arisen through
the introduction of men associates, but I will refer to
these later on.

Some of the most delightful entertainments have been
given by the members themselves, and their performances
have become more and more “professional” as the work
of the dramatic and operatic classes improved.

The concerts always concluded with a few words of
prayer, which made a heavy demand on the courage of
the Club leader, since she had to draw men and women
of every kind and condition into a fellowship to which
they were unaccustomed.

In addition to concerts, we had Saturday evening
socials and frequent dances. These gave opportunities for
workers and girls to have personal conversations, and for
the members to get to know one another well. At first,
we were satisfied with a pianist who played dance music.
Duets were most acceptable. Gradually, it was agreed
that dances in which girls could dance only with girls
was as dull as sandwiches made of bread and bread. Boys
were necessary, and before long they came.

We gave a party to our Club children at Chanukah
time,\footnote{Chanukah is the festival of courage,
observed in memory of the heroic
struggle of the Maccabees for liberty.}
and these parties, a source of infinite pleasure
to the children themselves, gave only a little less
pleasure to the grown-ups. They were elaborate affairs,
at first reserved for the little brothers and sisters of our
two hundred oldest members. That was in the days
when our children’s party was unique in the
lives of the children. Entertainments of all kinds have
increased since those days, and now the children have a
great many treats. To-day, all members are allowed to
buy sixpenny tickets. One worker member arranges tea
and produces a present for each guest. Her assistants are
very numerous, and there is no doubt that the greatest
pleasure is given. The smallest and most attractive guests
are taken round to be introduced to all the admiring
workers. The majority play games in perfect freedom,
and all the older boys and girls take part in the fun. The
appreciation of the children for the dancers and for the
conjurors who are always included in the programme is
unquestionable, and the little guests go off after the long
afternoon with sweets, very tired, but eager to tell their
mothers all about the party.

The children’s party as a social event is perhaps
surpassed for interest by the Old People’s party. For about
twenty-five years the inmates of neighbouring institutes
and almshouses and of the Jewish homes for old people
have come together to be entertained by our members.
The party is always held on a Sunday afternoon in
summer, and tickets are eagerly awaited. Our friends
arrive very early, and it is moving to see how they
appreciate having their names remembered by those who greet
them. The tea tables look beautiful with masses of
glorious garden flowers, which are afterwards made into
bunches and given to the guests to take home. The old
ladies and gentlemen are very gentle and grateful for
the excellent tea, and for the entertainment produced by
members of both the adult and the children’s clubs. One
of our workers has been able year after year to amuse
the old people by the jokes with which she makes her
announcements, and induces her aged guests to join in
singing the songs of long ago. The old people at these
parties form a sad group. They are in no hurry. Their
expression is one of patient expectation, but we know they
are only waiting for one big event, the event of
deliverance. In the meantime, they enjoy all that is done for
their amusement. They are glad to watch the young
people, who are most gently and sweetly attentive to
them. The party comes to an end as the day gets cooler,
and as the old ladies and gentlemen go out, they are each
given a little gift in addition to the flowers, for which they
give thanks in words of sincerity and affection. “God
bless you. See you next year.” Here and there we notice
a look of great refinement on a face, and a girl whispers
in sad wonder: “He was a lawyer once,” or “ Did you
know she was an actress?” We feel that there is so little
time in which to restore some of the joy and peace which
have been lost, but at least the party has given a delightful
change to our guests, and a number of our members
have had a rare opportunity for sympathy and genuine
assistance.

Before our Club became altogether democratic in
character, we had a ramblers’ club organised by an
outside worker. The rambles included country walks, as well
as visits to concerts and picture galleries, always followed
by tea at the house of one of my friends. During the early
years, we had our annual Club excursion, when a
procession of coaches, and subsequently of motors, went out
for a long day into the country. To save expense, I
borrowed some of these vehicles from friends; the rest we
hired. Each member bought a ticket beforehand. We
were usually entertained in a country or suburban garden
by some generous hostess, who was glad to welcome a
party running into several hundreds. Being a thoroughgoing
Cockney, I confess that this form of excursion
appealed to me very strongly, and though I became
somewhat weary of the long drive with the interminable waits,
while the drivers refreshed themselves at public-houses, I
thoroughly enjoyed myself. I liked being in the open air
with hundreds of the young folk, and to hear them
singing as we drove along the country lanes, a singing which
expressed much simple gaiety and jolly friendliness.
I enjoyed being photographed in large groups in spite of
the annoying way in which we had to delay the
actual photographing because people would move or
make a joke at the last second. Each person was
immensely happy and good tempered, and said she would
not have missed the outing “for worlds” even though she
had lost half a day’s work in order to come. The night
before the excursion was the most exhausting of my whole
year. It was all-important that friends should drive in
the same brake, and some of the most popular girls had
friends far too numerous for one vehicle to contain them.
We made lists beforehand, and read them out, and
changed them again and again, until we had to cry
“finis”; no more changes, and if anybody was dissatisfied
with her place she had better remain away. But
everybody was found to be satisfied, and if one or two
were not, they generally managed to secure a change. The
Club leader was criticised as being weak to give in, but on
such a vital point as a place in the brake, nobody could
possibly have refused to be accommodating, whatever
effort was involved. The coaches started very early the
next morning, for their progress was exceedingly slow. We
were therefore anxious that there should be no delay in
starting, but we always had to wait a little as all the
places were paid for and could not be wasted. The early
comers had the much-coveted box seats, and nobody was
very late.

Gradually the excursions ceased to be popular. Girls’
work became more important; some could not get the
time off; others did not care to go with such a crowd.
Worker members took charge of the rambles, which were
organised in smaller groups. The modern girl and boy
prefer to walk rather than to drive in a char-a-banc. The
social element is. of primary importance. While they
lasted, the big Club excursions were greatly enjoyed and
helped to integrate Club life. They have ceased with the
advent of something better—sport and rambles organised
by the people taking part in them, and quite independent
of outside favour.

In our Club we held games evenings and competitions
of all sorts, but although our members were interested in
sport and outdoor games, they have never been very
enthusiastic about parlour games, barring table tennis,
which is a latter-day amusement. We had a small
orchestra. The first was a team of mandoline players
who (surprising as it may seem) were asked to entertain
other clubs as well as our own on frequent occasions. We
had Conversaziones of all kinds.

The social life of our Club has developed and improved
amazingly under our present Club leader. In the old days
we had simpler pleasures and greater excitement over
commonplace events. Our members’ lives were altogether
much simpler. Popular outside entertainment was more
restricted. Our dances competed with low, cheap, dancing
halls where girls went for a sixpenny hop. These
halls were badly lighted and badly ventilated, and the
company was most objectionable. Our Club floor was
full of lumps, the music was unambitious, but the social
tone and friendliness were so marked that our Club held
its own. To-day, our members have the entry to very
different dancing places, to which I shall refer later in
the chapter dealing with mixed activities. I will only
say now that our Club is upheld always on the social
side by the spirit of friendliness among its members, and
the pleasure of being together. If girls drift away, they
cannot forget that they were once Club members, and
when they come back to their old friends they are never
turned away.
