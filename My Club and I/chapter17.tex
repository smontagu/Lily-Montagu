\chapter{Epilogue}

I have been asked to bring “My Club and I” up to date
and to continue the story from the time of the tragedy in
1941 when our beautiful building used for Club, Settlement
and Synagogue was completely destroyed by enemy
action, and we lost 27 friends in the ruins, for they had
sheltered in the Club believing that it would always
be safe.

At the end of my little book, I described how our Club
might rise again and develop many of its activities. I
believed that we had a future, and this faith has been
justified in wonderful ways. In some directions, we have
diverged from the methods of the past, and my predictions have been falsified. In other directions, I can
gratefully acknowledge that we have advanced further
and more successfully than I ever dared to hope.

When, as I have described, we worshipped together
in Whitefields Tabernacle which, after the destruction of
our building, was lent to us for the purpose, we all felt
deeply conscious of the presence of God, and through
our contact with Him, we found the courage to go on.
Two years following our disaster, the great church was
destroyed even as was our Club and Settlement.

Of course, one can never completely recover what is
lost, but, looking back as I do now, I see that bigger
enterprises have replaced many of the smaller activities
which had been part of the old Club and Settlement in
Alfred Place. Details have been altered in our Club
administration, but its basic principles have not been
changed. The advance that we have made has been due
mainly to the vision and courage of our Club Leader,
Miss Nellie G. Levy.

In these last 12 years, there have naturally been many
changes in our ranks and we have found successors to
the friends from whom we had to part, but we cannot
expect to replace them. The Red Lodge Trio can no
longer function under that precious name, for in
February, 1951, our dear friend and colleague, Miss C.
P. Lewis passed away.

To go back to 1941. New problems and unexpected
difficulties had to be overcome, but the Club work
continued without a pause. For a few days, we were the
guests of the Royal Air Force in a small building in
Alfred Place. Then for a year and a half we occupied
the building used for the Day Nursery of Whitefields
Tabernacle, where the accommodation was quite inadequate.
Then we were allowed to use a Clergy House in
Fitzroy Square which we had to vacate after two years
as the house was needed for the owners’ use. The next
move was to rent two houses in Charlotte Street, which
seemed at the time the only vacant buildings in the
West Central district at all suitable for Club purposes.
We had to spend a considerable sum in adapting these
houses for our use, but at last we were able in a hopeful
spirit to commence our classes and social activities. Our
members were full of happiness and enthusiasm, for they
were making a new start, even though our walls were in
a shabby state, and our surroundings were very dull, and
we had quite insufficient space for all we wanted to do.
No enemy action could deprive us of the faith and
determination which were an integral part of Club life,
and we thought our building would certainly last our
own time and perhaps our successors could sell the lease
and find something better.

\begin{tp}{512}
In 1951, we received, to our astonishment and dismay,
the surprising information that our building was condemned
as unsafe, and would have to be destroyed and
we were required to vacate it within a week. After many
protests and innumerable lamentations, the period
of warning was extended, but we were told that the
safety of our young people was threatened and we were
obliged to leave in May of that year.
\end{tp}

No alternative Club home was in view, although our
Club friends had searched assiduously. Club work is in
itself full of surprises and adventures, but, in our case,
we discovered new depths of kindness among friends and
acquaintances interested in social work. Official bodies
cut through some of their red tape in order to assist
us in a kindly spirit. So it was that we were offered by
the West Central Jewish Lads’ Club the use of one of
their rooms in Hanway Place as an office in the daytime.
The London County Council allowed us to use the
Infants’ School in Netley Street, Hampstead Road, for
our Club evening classes, but not for activities during
the weekends. It needed the courage and ingenuity of
our unconquerable Club Leader, Miss Nellie Levy, and
a few devoted colleagues, as well as the unfaltering
loyalty of our members, to carry on the Club under these
most difficult conditions.

Our young men and women could not easily fit into
infant school equipment, and they could regard nothing
as their own to serve for Club amenities. Everything had
to be packed away very punctually every night, and to
be brought back in the afternoon of the next day. Even
the suppers were cooked in Hanway Place and conveyed
to Netley Street, so that the young people could have
a hot meal after business and before starting the Club
evening.

At last, in 1952, after five moves in ten years, deliverance
came. A committee of men and women, with Miss
Levy at their head, considered the building in Hand
Court, Holborn, which had been used as a Club for
ex-service men, and found that it was available, partly
as freehold and partly through the purchase of a long
lease. The cost was great, too great for our resources,
but nothing else at all suitable could be found.

Facing a bombed site, the premises looked very dreary,
drab and forsaken when we first took possession.
Moreover, the adaptation of the building to our needs
was a difficult and costly affair. Throughout the business
transactions, our young people, through their duly
elected committee of responsible members or ex-member
workers, assured my sister and myself that they wouid
shoulder the financial cares and we need have no worry.

They also promised to carry out the promise made
years before by the Settlement Committee: to give a
part of the war damage compensation towards the cost
of building a Synagogue for the West Central Liberal
Jewish Congregation. The Congregation had, until the
tragedy in 1941, been housed in Alfred Place, using the
beautiful hall for its services, and the other rooms for
the religious school classes and other activities. All the
property of the Congregation, the Reading Desk, Ark
and Scrolls of the Law, the Perpetual Lamp, the prayer
books, organ and much else was completely destroyed.
The Congregation too had continued its full activities,
meeting in different places, but progress was difficult in
consequence. Our Club friends promised to help the
Congregation to re-establish itself, especially since the
Club had its foundation in Jewish worship. Now, after
waiting 12 years, the moment has come when we are
about to build the Synagogue at the corner of Whitfield
and Maple Streets, near Tottenham Court Road, and it
will satisfy the religious aspirations not only of its
congregants, but also of many people living in the
neighbourhood, and also of at least a section of Club
members and Settlement associates.

At the beginning of our Club reconstruction, I do not
think our friends themselves realised the size of the
burden they had so willingly accepted. The adapting
of the building would, if satisfactorily done, involve an
expenditure of several thousand pounds. For the first
time in our history, we had to launch an appeal committee
composed of people outside our membership.
With Mr.\ Leon Rees as Chairman, Mrs.\ Lewson and
Mrs.\ Rubinstein as Hon.\ Secretaries, we applied ourselves
to raising funds, no easy task, and as I write at the end
of 1953, we have only collected a quarter of the sum
required.

The Ministry of Works refused for many months to
allow us to spend a maximum of one thousand pounds.
They not unreasonably feared that we would regard this
sum as an instalment and before long ask for more!
They recognised the vastness of our needs.

A number of our members, seeing the desperate
plight of their Club, enrolled themselves into a voluntary
corps of cleaners, decorators and painters, and spent
much of their leisure—and in one or two instances
sacrificed their holidays wholly or in part—cleaning and
decorating the building as far as they could. The
Minister of Works was impressed with this evidence of
the young people’s sincerity and enthusiasm and
granted the permit to us to spend up to £1,000 on the
building of the Club and Settlement. In September,
1952, the members resolved that they would use the
building in Hand Court in its unfinished state, for above
all they wanted a place of their own.

All classes were restarted with enthusiasm, and we
waited impatiently for the work to be completed. Some
parts of the building were soon let as offices, but our
financial position is still extremely unsatisfactory, especially
as an important part of the property, which was
meant to help us to meet our financial obligations,
remains unsold.

Since we started at Hand Court, many new members
have flocked to the central Club from all parts of London
and Greater London, and club life continues merrily.
Through the strong affection felt by our older members
for the founders of the Club, and passed on with great
loyalty to the Club grandchildren and even
great-grandchildren, our new Club building was named
Montagu House.

It is not difficult for my readers who have understanding
and sympathy for Club work to recognise that
the present Club and Settlement is very different in
character from the organisation we instituted over sixty
years ago. In the first place, our country has changed in
its sense of responsibility and has become a Welfare
State, and so the calls on our Settlement are very much
reduced. We no longer need a medical or dental clinic;
our associates obtain specialist and hospital and convalescent
help through the National Health organisation.
Better wages are earned and unemployment is infrequent
among our young people. All sorts of “extras” in the
way of amenities need no longer be provided by
voluntary agencies. The State is responsible, or the
people can obtain what they want for themselves. They
are better housed and better fed, and their sense of
independence has increased their appreciation of human
rights and human dignity. Partly as a result of the new
responsibility assumed by the State, and partly because
the West Central district has changed and become much
less residential, we have far fewer callers at our Settlement
who require the services of our welfare workers.
Ours is now a central organisation rather than one
supplying local needs. And yet the Settlement work is
still immensely important in spite of its limitations. Our
people have many and complicated problems which
cannot be solved in an official way, even though such
help is efficiently administered. They come to us
because they need sympathy and personal advice from
friends; they may occasionally come for our advice as
to how best to use the opportunities offered by the State
for their advantage. Our workers are more than ever
needed to assist the aged who are lonely and have many
personal difficulties, for their children, not lacking in
filial affection, cannot house them because of the housing
shortage, and having so little leisure cannot give the
small attentions which are as much needed to-day as they
were in the past. Our Darby and Joan Club is now one
of our most useful Settlement activities; our married Club
members whose devotion to the Club has covered many
decades respond with great joy to the activities of the
Married Members’ Guild, with Miss E. Bloom as our
faithful Hon.\ Secretary. The children need their Club
as much as their grandmothers did, for in the Club the
interval between school and bedtime can be happily filled.
Through home visiting, whether it affects the lives
of the aged or the children or the young men and girls
who are associated with us, we hear the Call to Service.
We adhere to our principles which have guided us ever
since our Club and Settlement were founded, in as much
as we recognise the independence of our people and visit
them as friends, anxious to share knowledge and
experience and to create a happy acquaintanceship.
We are made aware of the background of our Club
members and associates who may call upon us in times
of difficulty. Thus we may hope to render the Settlement
and Club adequate by encouraging those who use it to
find in our building, happiness and self-realisation, as
well as encouragement.

But if the demands on the Settlement are far lighter
than in former days, those on the Club are much
heavier. Miss Levy, who knew the Club when she was
herself a very young girl and has grown up with it, first
as a member and then as a leader, supports me in
thinking that the character of the work has greatly
changed. To begin with, the Club, under Miss Levy’s
direction, has become a Mixed Club in the full sense.
We described in the main part of this book how in the
early days we invited men to our Club with a certain
degree of timidity. Then they became Club associates.
There was a tendency on the part of the girls to make
heroes of the boys, and for them to accept the adoration
offered them and to find it amusing.
Their great superiority in the world of industry
encouraged them to exercise a certain patronage. But
gradually, as the economic status of women improved,
came a change in social relations. Miss Levy has been
able to establish a happy sense of comradeship between
the men and girl members. The rule that every member
should belong to some educational class is accepted by
all members alike, and they all observe the smaller
regulations instituted for the general good. The evening
Register is signed by everybody; and as a matter of
course all take part in the general Club interests. They
are asked to attend the short end-of-evening Services
which are arranged once a month on Progressive
Jewish lines. Every member is also expected to take part
in the fellowship of prayer which ends every night’s
main activities. The Sub-Committees and their chairmen
are elected without sex discrimination, and the
work of administration which can be delegated by our
Club leader, is shared by all members.

The most significant development in Club life lies
perhaps in the holiday organisation. While in the old
days, as I have explamed, our members revelled in their
Club holidays which they spent for two weeks, year after
year, at the Green Lady Hostel, Littlehampton, our
present members insist on a complete change. They
feel they have the right, after their year’s hard work,
to spend what they can afford on expensive holidays.
The days when we expected 10s.\ a week as holiday
payment from a girl who earned on the average 15s.\ per
week vanished, happily, fifty years ago. During many
successive summers we rented schools and did our own
catering. But to supplement the holidays “on their
own”, to which I referred in the early part of my book,
the record of holidays organised by our Club leader
shows that the love of complete freedom has yielded,
among several of our groups, to the fine corporate spirit
which Miss Levy has been able to inculcate. This year
—1953—Miss Levy took one party of men and girls to
a hotel in Italy, and another to Belgium. She has
succeeded in awakening a sense of appreciation for
beautiful scenery, and is very gradually inducing
holiday-makers to visit buildings of historical interest
weaning them gradually from the counter-attraction of
“shopping” for the folk at home. They and their
parents must understand that the chance to see these
buildings may never recur, and that they are of the
greatest educational value. The tone of our mixed
holiday parties leaves very little to be desired, and the
men and girls enjoy their mutual companionship,
through their leader’s influence, in a healthy, cheerful
way which leaves behind no vain regrets.

In this same year, 1953, the younger girls stayed at a
Youth Centre in Torquay under the care of our own
responsible leaders. The children from our Junior
Club were able to visit the home of delight at Bracklesham
Bay belonging to the Association for Jewish Youth.

In all Club parties the tradition is upheld of Sabbath
Eve celebrations which, under Miss Levy’s leadership,
our young people thoroughly enjoy. We trust that our
members will in the future, as they have done in the past,
imitate our way of sanctifying the Sabbath Eve in their
own homes.

In May of 1953 we had our Diamond Jubilee, and the
event celebrated with so much kindness, love, and
ability by our young people, will to the end serve as a
joyous landmark in the lives of both my sister Marian
and myself. The Jubilee Committee, with the help of
our Club leader, arranged that a Service of Thanksgiving
should be held at the Liberal Jewish Synagogue, and to
our joy and encouragement the address was given by the
leader of the Liberal Jewish Movement in England,
Rabbi Dr.\ I. I. Mattuck. The Choir, led by Miss Mary
Bonin, consisted of present and past Club members, who
gave a pleasant rendering of the music of thanksgiving.
The large Synagogue was full, and the note of praise and
thankfulness revealed in the prayers and music had an
unmistakable response. It was a time of true worship.
Afterwards a reception was held and congratulations
were offered in speeches and in friendly conversation.
We all shared the spirit of dedication which we carried
with us when we left the Synagogue.

On May 16th a Jubilee Play was performed at the
Scala Theatre, and a very large audience assembled.
The play was based on the lives of my sister and myself,
so far as these were closely connected with the development
of the Club. The author of the play was Jeffrey
Siegal, his wife Rose was the producer, and they, as well
as all the cast, were present or former Club members.
The title of the play was “If I strive not”, and it told
the story of the West Central. John Slater, a former
Club member, appealed for donations towards the Club.
We saw the beginnings of our work, and some of its
difficulties and tragic moments. The characters of our
members were depicted with great and tender skill. The
story carried us from the Club in Dean Street right up
to the establishment of Montagu House in Holborn as
our Community Centre.

We printed a brochure which contained many
precious messages from individuals, committees, and
organisations. I will quote only one which was sent by
the National Association of Girls’ Clubs and Mixed
Clubs, and ran as follows: “It gives me great pleasure
on behalf of the National Association of Mixed Clubs
and Girls’ Clubs to send all good wishes to the West
Central Jewish Club on the occasion of their Diamond
Jubilee. Congratulations to all concerned on sixty years
of devoted work which has brought joy and happiness
into the lives of hundreds of young people and rendered
invaluable service to the whole community. The
National Association is grateful for the inspiration and
encouragement afforded to it by such a record of service
and achievement, and to one and all sends its best
wishes for all success and progress in the coming years.”

One more red-letter day must be recorded since our
Diamond Jubilee. On October 28th, 1953, H.R.H. the
Duchess of Gloucester visited our Club while in working
order, and showed interest in our classes and activities.
After our Club leader had taken her round the Club
with the help of the Reception Committee headed by
the Dowager Lady Swaythling, H.R.H. took tea with
many of the Club workers. She accepted flowers from
the Club members and I thanked her for her most
encouraging visit. The Senior Dramatic Class then
acted a short play exalting the peace ideal. The
memorable evening ended, as usual, with a short
extempore prayer and the recital of the Shema.

Before coming to the final section of my story, I want
to write something under the title of \textsc{Personalia}.
Hundreds of people have contributed by personal
service to the progress of our Club and Settlement. Our
aim, which we have realised to a considerable degree
has always been to achieve our results by democratic
methods. We have had no “managers”. Our members
have, from the first, been members of our Government
and been responsible to their fellow members for their
work. A few members of my family and several intimate
friends came into the work at the beginning and shared
my enthusiasm. But gradually our authority ceased to
be sought or required. My sister Marian has been my
alter-ego, my constant companion and wise colleague;
my sister Mrs.\ Franklin has with unvarying generosity
given us hospitality when we needed it. Miss Lewis
devoted most of her special gifts to the service of our
Club, Miss H. Schlesinger has to the present day assisted
in our councils, following in the way of her Mother, our
former President. In 1947 I was raised from Hon.\ Sec.\ to
the proud position of President.

I could mention many friends who have contributed
to making our work fruitful, apart from our Club leader
Miss N. G. Levy, whose achievements are reflected in
almost every part of my book, and who stands in a class
by herself. There is my colleague Miss Lazarus, whose
work and personality have been described in the first
part of my book, in the chapter called “Settlement”.
But I must again refer to her work, which has continued
for about fifty years. It is characterised by complete
selflessness and. consecrated devotion. She has worked
hard in the field of religious education and has a real
understanding of children and also of old people—by all
of whom she is loved. Once she took children in parties
for holidays: this year she took a party of old folk for a
week in Brighton and they were ecstatic in their description
of their happiness. Indeed her name is venerated
in many homes in which her skilled advice has brought
renewed health and hope. Her devotion to old people
has earned for her the title of “a good girl”, or “a
wonderful girl”—this even to-day when she looks tired
and worn. Her eyes light up with faith, and her sense
of humour carries her over innumerable difficulties.
When man, woman or child calls, O.L. appears, does
her bit, and then fades away.

Then there is my friend and colleague, Dora Isaacs,
the Secretary of our Settlement, who works with the
greatest devotion. Her wonderful memory has made her
most helpful in producing from her brain many of the
records which were destroyed in the Club fire in 1941.
Her orderliness and care for detail, her excellent clerical
qualifications, and, above all, her power of loyal friendship
have made her an invaluable colleague.

I must mention Hannah Feldman, with her strong,
radiant personality and infinite sympathy and understanding,
who has been an excellent worker and teacher;
Lyddie Tasch, with her integrity and extraordinary
accuracy in regard to the dates and sequence of events
has endeared herself to all Club workers; Rita Rosenthal,
with her gift of friendship and her courage in overcoming
personal difficulties, has been a splendid teacher,
revealing in her work her power of love; Fanny
Abrahams, with her indefatigable readiness to assist in
the domestic side of Club work and in introducing
economies which are undoubtedly most important but
which require expert organisation.

Long before we were a Mixed Club I received the
greatest help from some of my men colleagues, especially
Meyer Taylor, Frank Austin and Sam Lyons, who still
give us unstinted support. But with an eye to our hopes
for the future I must, as the last of those I mention with
gratitude from our host of trained workers, write of my
nephew Bryan Montagu. We corresponded during the
war years when he was on active service, and I wrote
about the Club and he promised his co-operation when
he came home. With his wife Elcie, he has splendidly
fulfilled this promise. Together they work strenuously
to support all our Club interests, and have won for
themselves everybody’s affection and esteem.

In bringing my little book, \textsl{My Club and I}, up to date,
I should like to examine rather fully the changes in the
religious outlook of the young people we have sought
to serve during these sixty years. To my deep sorrow I
must confess that the number of professing Liberal Jews
has not greatly increased. The young people have
benefited from our teaching and share our faith in the
close connection between religion and life. They
themselves are not affiliated to any Synagogue—and
affiliation means something more than the payment of
subscriptions either oneself or through parental assistance.
After each discussion meeting, or Club service we
hold, we are told by our young people that they are in
entire sympathy with our point of view, but nevertheless
they prefer to call themselves Orthodox. They are, they
say, not interested enough to make any change which
would involve heart-searching for themselves; and they
have so many real interests to which they give their time;
and a change would mean family explanations, and they
don’t care enough to make this effort. They prefer to
drift on and so they are unjust to the Orthodox way of
life. Only a small percentage are found among. our
young people who adhere sincerely to the type of
Orthodoxy we knew and respected in our early life.
For the rest, they are uninterested in Judaism, and on
account of that indifference they are not prepared to
make their religion an important issue between themselves
and their parents in their daily life. They expect
so little from religion that, without giving it any thought,
they are not disappointed that it offers them so little.
They will occasionally drop into a Synagogue on the
High Holydays. But there is something more. Occasionally
a boy or girl tells us that he has found in his Holy of
Holies a growing conception of the existence of God, and
he is prepared to search for the spiritual relation between
this discovery and daity life. He does not deny the
necessity to interpret this spiritual discovery in modern
life. I feel that this fact gives us the greatest hope and
we must accept the challenge and build upon the
teaching which is universalistic and based on our
ancestral faith and strengthened by actual communion
with God. We cannot be satisfied any longer that our
men members, in order to give evidence of their loyalty,
will wear hats or a substitute for a hat, when Hebrew
is recited at public functions. They don’t think about
their Judaism, and would affirm that in this one
particular (although of course in no other) what was
good enough for their fathers is good enough for them.
When they contemplate marriage and the girl or boy
of their choice is non-Jewish, they appeal to an authority
on Liberal Judaism. They have been told (quite
inaccurately) that it is easier to have a Synagogue
wedding if the non-Jewish party chooses to be accepted
by the Liberal Community. Certainly our interpretation
of the Bible, and the avoidance of the language difficulty,
makes the conversion easier. The non-Jew or Jewess
begins to study, and vicarious interest is awakened in the
Jewish party. The boy or girl has never had occasion
before to think about religion; and when the differences
between the Orthodox and the Liberal presentment of
the ancient faith are explained, he or she is deeply
impressed. He or she could not base modern home life
on outworn principles and observances. Teaching which
is in harmony with modern thought is necessary.
Here is our opportunity, and indeed we must fulfil
our sacred trust. We can and must save Judaism
from one of its most dangerous menaces—the mixed
marriage.

My connection with the Club must necessarily be
severed before very long. It is not easy to turn over a
leaf in my book of life, but I thank God that the story
of Club life reveals extremely happy experiences. When
we celebrated the Diamond Jubilee, while our young
people showed such amazing love, my sister and I heard
the signal to give up our activity. The Club assured us
in all sincerity of their lasting friendship, and we were
glad and thankful on this account. We have reduced our
service very considerably, although our Club leader and
our people as a whole, in their generosity, ask me to
remain President; and my sister shares with me all my
joy in the progress of our Club. We watch its evolution
with complete confidence.

To sum up the impressions I have recorded elsewhere:
We see the educational work growing in quality and
undertaken with great enthusiasm. The Club members
have proved a strong self-governing group and many,
through developing their own personality, have become
quite efficient in social service and hold responsible positions.
Our Club leader and her close associates insist on
having a Club in which religion is regarded as of the
utmost importance. We are grateful to God that He has
allowed this point of view to remain unchanged.

We have to end with hope darkened by.a shade of
anxiety. Our beloved leader has told us she cannot
continue much longer with the work which always is full
of problems, and through its growth becomes increasingly
exhausting, although it must always give happiness to
those who love it. So far we have not found the Workers
to carry on the work. When candidates \textsl{do} appear, will
they have the necessary faith in Judaism? As I have said,
they will find a few sincerely Orthodox young men and
women to whom they must show the greatest sympathy
and respect. They will also find a very much larger
number of Liberals who have imbibed our teaching but
not thoroughly assimilated it. They are somewhat
bewildered, but so far are not altogether ready or
willing to clarify their thoughts. Then our wardens, when
they do come, will find a great section who are indifferent
to religion but wish always to call themselves Jews, and to
turn, as their ancestors did, to the God of their fathers
when in trouble. Finally, they will find a small group of
enthusiastic young men and women who are eager to
spread the light of truth which is illuminating their
lives. All these young people have, we believe, a considerable
endowment of moral grit and their code,
whether they are aware of it or no, has come to them on
the authority of God. They are living on the heritage
handed down by their parents, even though they are not
themselves conscious of its great, spiritual potentiality.
They do not deny God, but only the minority want
to know Him well. But the awakening must come.

For the Club itself, the right people, to carry on the
work in the future, who are absolutely needed, will
appear when our present leader, who cannot be excelled
in greatness and power, feels compelled to take some
rest. God will surely not allow the West Central and its
ideals to perish.

Moreover, the Congregation which was established
in the Club is about to build its own Synagogue and its
influence will, we trust, form a background for our
Social Centre. It will be a life-giving institution which
will, through prayer, fellowship, and instruction give
our young people that joy, and hope which, however
self-dependent they may be, they will find satisfactory.
In that Synagogue, as well as in the Club, we pray they
may hear God saying: “Ho! every one that thirsteth,
come ye to the waters”, and they will drink and find
the waters full of life, love, righteousness, beauty, truth,
justice and peace.
