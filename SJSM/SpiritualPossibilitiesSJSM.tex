\documentclass[12pt, extrafontsizes, twopage, a5paper]{memoir}

\settypeblocksize{17cm}{11cm}{*}
\setlrmargins{0.8in}{*}{*}
\setulmargins{0.8in}{*}{*}
\setheadfoot{\onelineskip}{\onelineskip}
%\setlength{\topskip}{1.6\topskip}
\checkandfixthelayout
%\sloppybottom
\fixpdflayout

\setlength{\footmarkwidth}{1em}
\setlength{\footmarksep}{0em}
\footmarkstyle{#1\hfill}

\usepackage[all]{nowidow}

\usepackage{polyglossia}
\setmainlanguage{english}
\setmainfont{Brill Roman}

\parskip 1.25em
\parindent 1.25em

\begin{document}
{
  \centering\LARGE\bfseries SPIRITUAL POSSIBILITIES\\OF JUDAISM TODAY

  \Large Simon Montagu

  }

\begin{quotation}
``The history of a community, like the history of an 
individual, is marked by the recurrence of periods of self- 
consciousness and self-analysis. At such times its members 
consider their aggregate achievements and failures, and 
mark the tendencies of their corporate life. Perhaps even, 
the sudden recognition of facts which have been uncon- 
sciously suppressed may lead to regeneration.''
\end{quotation}

The first paragraph of this article, as well as its title, are taken from an article by Lily Montagu which appeared in the January, 1899 issue of the \textsl{Jewish Quarterly Review}, when the author was 26 years old. Her name and career as a Jewish leader will be familiar to many, but many fewer will have heard the title of the article and fewer still will be familiar with its content. A recent online project, with funding from the UK Heritage Lottery Fund, \textsl{Lily's Legacy}, declares as its objective to examine ``how Liberal Judaism embodies the vision of its founders -- Lily Montagu, Claude Montefiore and Rabbi Dr Israel Mattuck -- both today and throughout its history.''\footnote{From https://lilyslegacyproject.com, retrieved 4.9.2023} The beautifully produced material on the project website contains many personal memories of Lily Montagu and accounts of her work for the community in general and in particular her involvement in the history of the Liberal Jewish movement that she founded, but rarely if at all quotes from her writings on Jewish and spiritual topics.

This also reflects my experience as a member of Lily Montagu's family (she was my great-grandfather Louis Samuel Montagu's sister) who grew up in the Liberal movement. The movement's offices were in a building called the Montagu Centre; the social hall of more than one of the movement's synagogues was the Montagu Hall, with a picture of her hanging on the wall; and her name was frequently invoked from the pulpit, but her own writings were rarely if ever quoted.

I wish to begin to address this absence by examining her article on the ``Spiritual Possibilities of Judaism Today'' and consider to what extent her analysis is still valid and relevant a century and a quarter after it was written and what Jews of all denominations can learn from it on the spiritual possibilities of Judaism in 2024.

The article begins on an apparently pessimistic note: ``Until Jews are honest enough to recognize that 
the majority of them are either devoted to ceremonialism at 
the expense of religion, or indifferent both to ceremonialism 
and to religion; until they have energy to examine their 
religious needs and courage to formulate them, they are 
courting comfort at the expense of truth, and they must fail 
to restore to Judaism its life and the endless possibilities 
inherent in life,'' but this is almost immediately dispelled: ``these facts prove on examination to be stimulating rather than terrifying, fraught with hope 
rather than with negation.''

Lily Montagu's analysis of the \textsl{status quo} of Judaism in England features a division between two types of Jew: the ``East End'' Jew and the ``West End'' Jew. The terms of this division, based on the social structure of the London Jewish community at the end of the nineteenth century, are no longer relevant, and were already losing relevance at the time of writing, as she points out: ``These two forms of religion do not prevail exclusively
in any particular district of London ... but these epithets are 
intended to convey the idea of two sets of people, differing 
less in dogmatic belief than in the tone and temper of 
their minds, and especially in their view of the proper 
relations between religion and life.''

\end{document}
