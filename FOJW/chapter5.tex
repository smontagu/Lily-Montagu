\chapter[Progress of the Jewish Religious Union
and the Foundation of the World Union
for Progressive Judaism]{Progress of the Jewish Religious Union\\
and the Foundation of the World Union\\
for Progressive Judaism}

During the sad period of public controversy about
Liberal Judaism, which, of course, synchronised with my
experience of personal grief at being unable to meet my
father’s wishes, the correspondence in the Jewish Chronicle
(the Communal organ) was often very acrimonious.
Dr. Montefiore very frequently sent me a note on Friday
mornings to assure me that I must not worry about the
abuse which was being hurled upon him, as nobody
seriously thought he deserved all the blame he was
receiving. But I, knowing that my sex and my
unimportance in the community were shielding me from
being that which I deserved to be — the central target in
the attacks — felt that my colleagues were being unfairly
dealt with. My mother was unhappy because of the
grief in our home. She understood and sympathised with
all the parties concerned, and instead of family talk
being free and unrestrained there were now many points
about which silence was necessary, as discussions must be
avoided, for they would necessarily lead to pain. One
of my sisters, who has for years now been a keenly interested
member of the Jewish Religious Union, thought us
guilty of disloyalty in the early years, and our progress
could not be mentioned in her presence.

My sister, Mrs.\ Franklin, helped me from the very
start and the initial meetings were held in her house.
My sister Marian, to whom I was bound by ties of the
closest and deepest friendship, shared from the beginning
my interest in the cause of Liberal Judaism. She never
desired to be a front line protagonist. She worked in the
field of Liberal Judaism, as in all other fields with which
we were associated, in the background. Her serenity
and quiet confidence were the outcome of a deep religious
faith. All through her life she has seemed to us to
satisfy the requirements of her God as explained by the
prophet Micah, for she is just in all her judgements,
patient and loving, and she has always walked humbly
with her God. Her contacts with the perfect spirit of
righteousness are evident in the conduct of her life and
render her gentleness and unselfishness constant, and
worthy of the deepest respect and admiration.

Changes affecting my position in the Liberal Synagogue
were frequent and generally progressive. From
the beginning it was determined that in our Synagogue
men and women must be absolutely equal in their
congregational privileges. Boys and girls were confirmed
together, and men and women sat together as they chose
in any part of the Synagogue. There was no women’s
gallery, such as we find in Orthodox Synagogues. Women
had, as a matter of course, their seats on the Council,
and took their share as voters in the shaping of Synagogue
policy and in the responsibility of maintaining
and developing its religious influence.

It was, I recall, Dr.\ Harry Lewis who, in the very
early days of the establishment of the Jewish Religious
Union, recommended that women should be allowed to
take part in the conduct of the service as leaders and
preachers. The majority on our Council, however,
although mostly in sympathy with the idea, felt that the
shock to the community which such a radical change
would involve might prove injurious to our cause. My
sister (Mrs.\ Franklin) and I were honestly in agreement
with this point of view, and quite satisfied with the
privileges accorded to us as ordinary members of the
Council and congregation.

For myself, I was receiving some training as a leader
of other and less well known congregations. When the
Liberal Jewish Synagogue was founded, and by its
means the new movement firmly established, I was asked
by my eldest brother to resign from leading the children’s
services at the New West End Synagogue. He was a
Warden, and he knew that the feeling was so high against
the Liberal movement that unless I resigned on my own
account I should very soon be asked to do so. Such a
necessity he very much wanted to avoid, for, even
though a heretic, I was still my father’s daughter, and
he at least should be spared this disgrace. Also, after my
ten years’ regular service, the Council, although sincerely
aggrieved, felt real regret that the impasse had arisen.
My brother was sincerely sorry when he was obliged to
convey the message to me, and indeed we were both very
much distressed and, in a measure, apologetic to each
other.

I had in the meantime started services for members
of our Girls’ Club who were obliged to work on the
Sabbath and were thus debarred from any Jewish public
worship. These services, in their initial stages, were
not well attended, but the numbers grew steadily. The
enterprise was very unpopular in our home; not so much,
I am bound to admit, because the services were established
on a Liberal basis, and I used our Jewish Religious
Union prayerbooks, but because this involved the
sacrifice of my Saturday, my only leisured afternoon.
Also, in order to comply with Orthodox tradition, my
sister and I walked to the service and back, if the time
was still before sunset, and this walk made rather a
heavy demand on our strength. To us, however, the
service was a great joy. After a time we were given a
Minister by the Liberal Synagogue, and were able to
have a reading from the Pentateuchal Scroll, and our
West Central Group was able to lead a full congregational
life, though we met in our Club hall and not in a Synagogue
of our own. I still attended every service and
preached monthly instead of fortnightly. My sister and
I knew we were held in great affection by our congregants,
and our example encouraged them to attend regularly.
Our people’s attendance improved, and we had our
regular children’s services, religion school and social
activities. Men as well as girls took part in the worship
in our Club hall, and we felt that we were a representative
congregation of industrial workers.

Being so familiar with the lives of my people,
I was able to speak of matters of the greatest
importance to them and have no fear of being misunderstood.
It was a disappointment to me that the numbers
did not increase more rapidly. It seemed so hard for
our people to break with tradition. They, especially
the men, told me that, although they appreciated and
entirely understood our form of service, they were, in
some instances, literally afraid to break with the past,
for fear that some retribution would come upon them.
The girls, who had probably not attended Synagogue on
the Fast Day for years, and felt a great desire to get
some spiritual help on this important day, dared not
offend their parents by riding by train or bus, and they
could not walk, while fasting, to our Club hall, for they
lived too far away. I was more and more pained by the
complete failure of some of our Congregation to judge of
values. The emphasis was all wrong, and the spiritual
values of the religion seemed to be lost in an overgrowth
of superstition and prejudice.

Meanwhile, during the time that my own Club
Congregation was changing into a Constituent Synagogue
of the Jewish Religious Union, the opinions of our
Central Council on the position of women in the Synagogue
seemed gradually to change. It so happened that
on one occasion one of our lay leaders grew husky just
before the service for which he was responsible began,
and he asked me to stand by and be prepared to read if
the necessity arose. It arose before the prophetical
reading, and I had the privilege of reading the section,
which happened to be Isaiah 55, and which I have
regarded as my favourite throughout my life. This chapter
seems to me to carry within itself the essence of pure
religion. It contains a call to man to seek God, and an
assurance that if that search is undertaken with sincerity
and faith, all other of life’s activities will fit in according
to a correct measure of values. The chapter gives
glorious assurance that God will cause goodness to
triumph, and that, as He rules by law, we can
count on His law to lead to the establishment of righteousness.
Moreover, we find in these verses the wonderful
comfort for all seekers after truth, who, in spite of their
love and faith, must ever remain to some degree perplexed
and bewildered. “God’s thoughts are not our thoughts,
and His ways are not our ways.” We have no power to
explain God. If we could, we should be Gods ourselves.
Our minds can conceive only a part of His activity. The
perfect whole is beyond us. So we must give rest to our
souls. We must make active effort to reach nearer to
God: we can be sure He is waiting for us and helping us:
we can be sure that He is Love and Goodness, Justice,
Truth and Beauty, all the good things for which we
hunger; but we can receive only that which our human
hearts and minds can contain. Let us make ourselves
as receptive as possible; let us prepare and strive, and
with God be the rest.

My position was accepted by the congregation
without any stir, and my sister (Mrs.\ Franklin) and I
were often invited to assist as readers at the various
Synagogue Services. In June, 1915, our leader,
Dr.\ Mattuck, suggested that I should be invited to preach.
I was ready and glad, and my Council was unanimous
except for the vote of one honoured friend and
colleague, who thought my preaching would be against
the best interests of the congregation. I spoke on
June 15th, on “Kinship with God.” A few extra
people came out of curiosity, but, on the whole, the
innovation passed off with very little comment. My
sermon was printed, as it represented a development in
the story of Liberal Jewish progress. To my great joy,
the friend who had opposed my preaching generously
told me after my sermon that he withdrew his objections,
and thought they were not justified. Since that day,
Dr.\ Mattuck has invited me to preach about once a
quarter, and I have read frequently in the Sabbath and
Festival services, and, although these occasions have
brought, and still bring, to me an agony of nervousness,
I am always happy that I have these opportunities, and
my fellow members have been very kind and encouraging.
I have also been allowed to preach at the Liberal Jewish
Synagogue’s overflow congregation in the Montefiore
Hall on the Day of Atonement, and to read in the big
Synagogue on that day. The unity of feeling experienced
by the immense crowds gathered together. on these
important days conveys itself to those who read or
preach. It is wonderful to feel part of a great movement
seeking an approach to God, and the inspiration given
by mere numbers cannot be denied.

Apart from actual services, I, as Honorary Organising
Secretary of the Jewish Religious Union, entered into
all the activities of the Synagogue, and travelled to
several provincial towns as well as to all parts of London,
explaining the teaching of Liberal Judaism. It was
curious to find the same questions reiterated by men and
women of different types. It seemed sometimes that we
made no impression at all on the intelligence of our
fellow Jews, for the questioners spoke as if they had
made a new discovery when they referred to the problems
which we thought we had thoroughly thrashed out in
Synagogue, in lecture halls, and in the Jewish public
press. “If,” it was said, “you pick and choose pieces
out of the Bible, and discard some laws and customs as
no longer of ethical use, and retain others, don’t you see
that chaos must arise?” In vain we assured our audiences
that chaos did not arise if the test of retention was
sincerely made, and was based on the ethical usefulness
of the custom or of the observance. We pointed out how
selection was already made by the observer who rejected
what did not fit into his life. It was surely better to
impose some principle and to insist that ours was not a
religion of convenience. It is hard to be a good Liberal
Jew, for if one conscientiously feels that observances are
of real value, one cannot give them up whatever burdens
loyalty may inflict. One had to be rigidly exacting with
oneself if the law was felt to be within one’s self. It had
the divine stamp far more surely than if spoken on the
authority of a human lawgiver, however venerable. We
were asked again and again why Liberal Jews did not
wear hats when they worshipped. We had to explain
how earnestly and sincerely we had discussed the value
of the hat observance, and how endless had been the
arguments for and against before we decided to allow
the individual worshipper to do as he chose. It would
seem as if in the minds of our questioners the future of
Judaism depended on the wearing of the hat. The
tradition was so deeply ingrained that, instead of being
an ancient custom, created in part by weather conditions
in Eastern lands, it loomed so large that it seemed as if
the hat contained all the religious treasures of the Jews.
In vain we pointed out that the hat was in itself of no
account; what went on under the hat was of supreme
importance. Many bitter quarrels grew out of the refusal
to wear the hat.

Ever since we started the Jewish Religious Union,
I have felt that if we could only get people to listen to
our message, the possibilities of founding congregations
were endless. We needed itinerant preachers and
organisers, men and money, for in every part of the
country, besides those who were devoted to the Orthodox
presentment, there were many who chafed at the old
teaching and found the observances meaningless.

It is good to recall the start in North London, where
Mr.\ Arthur Joseph and I addressed a group in the house
of a Jew who, for the last thirty years, had found all his
spiritual nourishment in meetings in Hyde Park. In
South London we founded a section and heard, on the
one side, the voice of the convinced materialists, and,
on the other, a large family group, headed by
Mr.\ S.\,M.\,Rich, who in varying ways showed by their zeal and
determination that they were ready to form the nucleus
of a congregation, since they were thoroughly dissatisfied
with the religious assistance then within reach. In
Birmingham, the first meeting was so small that our
host went out into the street and looked appealingly into
the windows of neighbours’ houses, to see if any were
likely to join us in our efforts to revitalise our faith.

Again and again we held promising meetings in all
parts of the country. Sometimes there were definite
results, and new centres were formed, and the preparatory
work started. In some towns we secured a delightful
amount of enthusiasm, and were persuaded that a great
movement would forthwith be initiated. I should like
my readers to understand the joy and encouragement
which such experiences produced in me. I felt Reform
to be necessary, and when people seemed ready for
sacrifice in its cause, my own convictions became strengthened.

Our thoughts are always clarified when we try to
clear away the doubts and misunderstandings of other
people. While doing propaganda work, I found vast
numbers of Jews absolutely uninterested in Judaism.
They seemed to ignore its vital relation to right living.
They were unconcerned with their responsibility to their
children. Most people’s lives are full of small duties.
Outside the work by which we live, we have very little
attachment to the fine things of life. Hard thinking
takes up too much time for most of us. We can get
along satisfactorily, we think, even if our thoughts are
somewhat confused. We have so little time and are too
tired for intellectual effort. So, unaware of spiritual
aspiration, uninspired by religious tradition of which
we are ignorant, unaffected by prayer, we drift along
year after year, clinging to our various cherished
opinions and mistaken theories until these take such
firm root that they are simply immovable.

It was with people of this kind that we often came in
contact in our J.R.U. work. Sometimes, however, the
glow of our message in its novelty and sincerity effected
achange. \textsl{Then} we got the enthusiasm which I mentioned
before. The promise of Liberal Judaism seemed to come
as fresh water to a thirsty land. Unfortunately, however,
after the meetings were over and much kindness had been
expended on the organisers, and we had returned home,
our friends had time to consider more closely the need
for effort and sacrifice and possible unpopularity which
our programme entailed. Then I would not seldom
receive a letter couched in very polite, but very definite,
terms to the effect that no leader could be found with
sufficient time and initiative to undertake the work I
had outlined on my visit, and nothing could be achieved
for the present. The residents were deeply interested
and hoped sometime to avail themselves of our suggestions.
These suggestions, which were really serious and
important, not merely of transitory interest but affecting
man’s relations with God, would receive attention
\textsl{perhaps} some day. In the meantime, unfortunately,
other claims had to take precedence.

In periods of discouragement which such
disappointments produced, our leader, Dr.\ Mattuck, could
help me to see my difficulties in their right proportion.
Indeed, all through our long connection, I have never
come away from a talk with Dr.\ Mattuck without feeling
my confused lines of thought disentangled, and experiencing
new hope and courage.

In the year 1926, a former Club member of ours, who
was doing excellent work in the cause of pacifism, and for
this purpose visiting groups in every country in the
world, asked me if I did not think the cause of Liberal
Judaism would be greatly advanced if its adherents could
come in contact with one another when travelling in
foreign countries. We could give strength to our purpose,
and to the bond of sympathy which bound us together in
trying to accomplish it. In thinking about this suggestion,
I felt how much I wanted to gather together all our
progressive thinkers to stem the tide of materialism
which was threatening the life of our community, and to
make a positive contribution to the spiritual life of the
world. I spoke to Dr.\ Montefiore and to Dr.\ Mattuck,
and once again received from these two friends sympathetic
encouragement and helpful practical suggestions.
Dr.\ Montefiore told me how just before the great war
Jewish leaders in Germany were planning an international
conference to discuss religious problems from
the progressive angle. He thought that we might pick
up the threads again and weave a new pattern from the
“after war” angle of thought. Dr.\ Mattuck advised me
to ask the two American Jewish religious organisations
(Rabbinical and lay) to give their approval to the idea
of international co-operation on Jewish religious lines,
and to agree to send representatives to discuss a possible
union.

I shall never forget my thrill of joy when I received
kindly approval and great encouragement from America.
The Liberal organisations in Germany were naturally
pleased to revive their plan, even in an altered form. So
the first Conference was held in July, 1926, and our World
Union for Progressive Judaism came into being. It was
because men and women were ready for an organisation
which should give dynamic power to their spiritual influence
in the world that the World Union came to birth.
God allowed me to strike a match at the right moment,
and great thinkers in every country showed interest in
fanning the flame. My own share in the work was
important only because it was the beginning. The great
adventure would have proved in its progress far too big
for me to be really effective in it had it not been that I
had such wonderful assistance and support. All through
my life I have been blessed in this direction. I \textsl{have} had
large ideas, but they have been generally rather nebulous.
Then men with the personality of Dr.\ Montefiore, and
with the insight of Dr.\ Mattuck, with the vision of
Dr.\ Baeck, with the ability of Mr.\ Ludwig Vogelstein,
and the religious approach of Dr.\ Morgenstern, and very
many others, gave these ideas shape and form and
enriched them a thousandfold and gave them new
grandeur. The work for the World Union has been in
itself full of joy for me. I know that we have only
covered the first stages of its possible advance; but each
stage has been sheer delight for me. I don’t ignore the
fact that each conference was prepared by dint of very
hard work and fraught for the secretariat with unspeakable
anxiety and great fatigue, but we \textsl{lived} with minds
concentrated on the object which I hold most sacred in
life, namely, the vitalisation of religion. The conferences
gave me the opportunity of meeting men and women
who have given me great interest in life and enlarged my
horizon in every direction. The thrill of starting new
groups and meeting the pioneers who are making them
live gave me intense happiness which remains with me
for always and is renewed every night when I pray that
the World Union may spread through Europe, Asia,
Africa, America and Australia.

From the beginning I was impressed with the fact
that so many of our Jewish religious leaders in America
and in Germany were successful professional and business
men. They spoke of the actualities of life when we met
round the board of the Governing Body or in conference,
and they spoke with spiritual insight and revealed the
fact that religion had entered into their inmost lives.

I have had the most delightful co-operation from
my secretaries, and owe a great measure of gratitude to
Mrs.\ Stonestreet, who, though not a Jewess, entered with
enthusiasm into the initial work of the Union and
encouraged me before and after our first meetings when
we launched our great enterprise. My present secretary,
Miss Jessie Levy, has identified her interest with every
aspect of our work, and will, I trust, carry it further
long after I have been obliged to lay down my responsibility.
We have depended in a very large degree on our
Chairman, Rabbi Dr.\ Mattuck, who guides us all the time,
and teaches us how to place first things first.

I cannot enter into the intricate progress of the
Union. Certain red letter days stand out and they are
like beacons of light in my own life. I shall never forget
the spontaneous singing of the hymn, Ein Keilohenu
(There is none like our God) at the end of our first meeting,
when with so much difficulty we had decided on our
name and our objects. Our American friends led the
hymn, having told us that they were prepared to go back
to their great organisations and obtain wholehearted
co-operation from them. The well-known hymn really
seemed to be made into a declaration of faith, and we
went away with a sense of dedication to a magnificent
idea.

I felt so proud and happy when our Governing Body
decided to send an investigator into Poland, to try to gather
the mass of our Jewish co-religionists who had Liberal tendencies,
and to help the hitherto inarticulate to break their
silence. That effort remained \textsl{apparently} abortive, but its
difficulty was so great, its possibilities so vast, that I feel
even to-day, in spite of the terrible conditions which have
overwhelmed the country as a whole and our co-religionists
in particular, that that effort held within itself a
measure of success which will one day bear fruit. When
we fail in our work for God, it may be that He takes our
failure into His hands and transfigures it.

Our Governing Body discussed the need for work in
Palestine at many meetings, and here again the difficulties
seemed overwhelming and altogether insuperable.
But we had to deal in Palestine with large groups who
had no religious affiliations. The spiritual needs of the
immigrants had to be satisfied. The condition of the Jews
in Palestine focussed the attention of interested people
all over the world who rightly or wrongly judged our
whole brotherhood by the Palestinian development.
After countless fears had been overcome and innumerable
objections silenced and great positive faith had been
expressed, we asked Dr.\ Dienemann to go as our representative
to investigate, and, if possible, to prepare some
organisation. The three congregations functioning at
the present day had their birth at that meeting when we
gave Dr.\ Dienemann his charge.

We had a red letter day when we heard that the
congregation in Melbourne was founded, for we felt
confident that Liberal Judaism would in time spread
throughout the continent of Australia. We rejoiced
when it was told us that Rabbi Weller, a young Rabbi
from the U.S.A. whom we had sent to Johannesburg,
had established one of the biggest Synagogues in the
world and was engaged in many social and religious
activities, of which his Synagogue was the centre. We
rejoiced when an Indian woman represented a small
group at our first Conference of Liberal Jews from her
country, in which countless religions find their home.
We experienced a feeling of wonder and hope when our
President addressed numbers of Liberal Jews in the
“Herren Haus” in Berlin, lent to us for a conference,
and when he replied to the greeting of the Government
representatives who came to do us honour. During this
conference I had the honour of preaching in the Reform
Synagogue in Berlin, being the first woman, I was told,
to occupy a pulpit in Germany. My sermon was on
“Personal Religion.”

Many red letter days were spent in the U.S.A.,
when, in 1930, I visited Philadelphia as World Union
delegate to the conference of the Union of American
Hebrew Congregations. I had recently received
from the Hebrew Union College in Cincinnati the
honorary degree of Doctor of Hebrew Law, and
was grateful for the opportunity of paying my
respects to that College and acknowledging their kindness.
I went from city to city, speaking to all
sorts of congregations and at innumerable luncheon
parties given in my honour. I have never made so many
speeches or received so much kindness and encouragement
in so short a space of time. I was deeply impressed by the
fact that whereas at home it was the habit not to go to
Synagogue, in the cities in the U.S.A. which I visited it
seemed to be the habit to go. I don’t suppose I realised
the size of the public from whom the worshippers were
drawn, but I certainly was not mistaken in my impression
of the dynamic power of religion shown everywhere. It
was delightful to see real unity existing between all
sections of the community in religious matters, and the
close relation between Synagogue and social service. I
was impressed by the number of women in each congregation
who shared enthusiastically in congregational life,
but I rather regretted that separate organisations
for men and women should be considered necessary. I
was grateful to find that, in spite of the rush of work
apparent everywhere, some important and efficient
person was always ready in every organisation to explain
its work and methods, and would give all the time that
was necessary for the purpose.

My hopes were raised high again when in July, 1939,
just before the war, we met to formulate our programme
for the World Union and to appoint a Field Secretary in
the U.S.A. Such limitless spheres of activity seemed to
open themselves out before us. Then came the war.
Now all our work is restricted, and financial difficulties
loom largely before us. We feel that the World Union
must make a struggle for its life. Through the Red Cross
there comes a message that both the congregations in
Amsterdam and The Hague are carrying on, and their
Rabbis are leading them. These congregations were
formed at the suggestion of the World Union, and their
emergence into independence has been the cause of great
encouragement to us.

Can we take this message from the Red Cross as
symbolic of the recuperative power of the World Union?
There can be no permanent defeat while we work for the
the faith in which we believe with such complete
confidence. Our life may be threatened again and again,
but we are assured of ultimate victory.

What do we seek? We seek to help the Jewish
community to fulfil its life; to work with and for its
God. As individuals, we ourselves renew our strength
and quicken our hopes because of the fine fellowship in
which we work. We are conscious all the while of our
own inadequacy, as also of the privileges which our
opportunity offers to us.
