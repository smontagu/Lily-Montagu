\chapter{Faith and Social Service}

I do not propose to give any autobiographical
account of my share in Club work. These experiences are
so numerous that they need a volume to themselves,
and this I hope one day to produce. I woud try
now to interest my readers in the religious aspect as it
affected me and assisted my religious development.

I must explain that I had no recognised training for
Club work. Such training was not considered important
in my day. But in my amateurish way I rather stumbled
on my methods and objects through realising the needs
of my girls by means of personal friendship with them.
I was allowed to share their moods with them. There was
no wall between us. My approach to them was s0 very
simple, just the kind of friendliness given by one girl to
others like herself, and they gave me affection in generous
measure.

I learned about many home tragedies, and especially
of the tragedy of unfulfilled aspiration. So many of my
girls had wanted to be something different, and to achieve
certain purposes which seemed to be denied them just
through the hardness of circumstances. But they very
seldom whined. They were full of vitality and an
exuberance of spirits which was sometimes a little
puzzling. They lacked so much that I thought essential
to happiness — yet they could appear so joyous, and
revelled in making a noise. It seemed that noise itself
gave pleasure.

Very early in my Club career I shared country
holidays with my girls, and we could talk confidentially.
I found that their Judaism was of the weakest, flimsiest
kind, and had hardly any recognised connection with
goodness. I remember being asked quite seriously by an
intelligent girl, while we walked together arm in arm
along the sea front at Littlehampton, “But has religion
anything to do with goodness? I didn’t know,” she
added thoughtfully, as new possibilities presented
themselves in her mind. At that time the question,
“Can a good Jew be a bad man?” was exercising older
heads in the community, and I had often been told that
a good Jew was an observant Jew, and if he kept
the laws with regard to Sabbaths, Holydays and Diet
he could be a good Jew even if he were cruel to his
wife and neglected his children. I was surprised to find
that Judaism was quite external in my girls’ lives:
it did not seem connected with conduct.

My girls were as a whole warmhearted, generous,
courageous and kind. They worked hard and had great
capacity for enjoyment. They laughed and cried easily,
and took a keen interest in all things good, from reading
to eating, from music to telling each other jokes.
There was little planning in life; it was just lived from
day to day. The sex interest played a very important
part in the girls’ lives, and it was controlled by a large
store of moral grit which was in a great measure an
inherited possession. They did not consciously let religion
enter into life.

I was greatly surprised at all this, and felt I must
share my Judaism with these girls for whom I was
learning to care so much. I tried and found that the
religion did not fit the conditions of their lives. Then I
saw that the religion was more to blame than the life,
and the religious questioning began.

Through my training as a leader of children’s
services, I did not find it hard to “make up” prayers
for my girls. They responded most readily to this effort,
and constantly told me that these “made up” prayers
meant a great deal to them. They had been accustomed
to say a word of set prayer, especially at night, but,
except when people they loved were ill, it had never
occurred to them that they might approach God in
prayer on their own account and He would listen. I
tried.to show my girls how prayer came spontaneously
to normal people of our Jewish faith, as seen in the
“Shema,”\footnote{Shema is the Hebrew word for Hear.
The prayer is taken from
Deut. vi., et seq., and embodies the Commandment of the complete
surrender of the human soul in love to the Creator.}
which in some measure may be regarded as
the proclamation of the faith of the Jewish people. We
believed ourselves always in contact with God, when we
were at home, and when we were away from home;
whatever our occupations, whether in work or play.
The bond which united us with Him was the bond of
love. It was not so hard to feel particularly near to God
when we were reverencing the beauty of nature; it seemed
indeed quite natural to give thanks for the beauty which
was revealed to us.

My girls enjoyed themselves supremely while on
holiday. I have never been so convinced of the happiness
of other people as I was on these Club holidays.
Nevertheless, they had to combat moments of genuine home
sickness. The girls, as a whole, loved their homes
passionately, even though, in some instances, the homes
might be very drab and unattractive, and life often grim
and disappointing. When letters were not forthcoming
every day, the girls’ spirits sank with alarming rapidity,
although enjoyments were offered to them. Prayer for
the dear ones at home was generally appreciated. If
these prayers produced tears, it was all the better. The
girls liked to cry when they were moved, and they were
easily moved to sadness as well as to joy.

In the evenings it was our custom to gather round a
tree in our hostel garden and talk about religion with
complete frankness. It was then that I learned most
about this strange, pseudo Orthodoxy which I was myself
beginning to question fundamentally. I say “pseudo,”
for it was only rarely that I came across genuine
believers, after the style of my own parents. When these
did appear, they roused my deep respect, but all through
my Club career I could have counted them on the fingers
of both hands, and during the early years of which I am
now writing, one hand would have been amply sufficient.

The girls did not attend Synagogue, even if they did
not work on Saturdays, and the vast majority \textsl{did} yield
to economic pressure. This was obviously inevitable,
but they were not conscious of any loss. They had not
seen the Sabbath properly honoured in their homes.
They used to tell me that in Russia and Poland, whence
most of the parents came, unmarried girls were not
expected to join in public worship. They were told to
obey certain laws and customs, and to keep clearly in
their minds that they were not Christians, and beyond
that not to bother their elders with too many questions.
The mothers of my girls inculcated teaching about the
old Jewish customs with the utmost faithfulness, and the
girls felt these laws must be altogether good, and were
inclined to hold to them tenaciously even though they
had come to England, a country in which it was said
nobody cared about religion. I used to listen with
horror, the skin of my face very cold, my hands hot,
when I was told about some of the customs, especially as
they affected women and children. Innocent babies had
to be guarded from the evil eye of certain people; their
future might be jeopardised if excessive admiration
were offered to any of their physical charms. During
menstruation, girls were considered unclean, and
were not allowed then to touch any vessel used for
ritualistic observance. They were not entitled to pray
at these times, even if they were in great agony of spirit
through the loss of some near relative.

Even before I began to reason about the meaning of
religion, I knew that all this was wrong and a travesty
of real faith. A loving God, who had given woman her
body, could not hold it to be defiled by the working of
the laws of her nature. I could not see that any \textsl{work} was
involved in my turning out the light on the Sabbath,
when I said Good-night. I could not hurt God by
carrying out my routine duty. It was not thus that I
separated myself from Him. Again and again, I was
told that after my turning out the light the girls would
say among themselves: “Funny that Miss Lily is so
religious, yet she does not mind turning out the light on
Friday nights.” But we none of us realised that Miss
Lily’s sense of religious values was undergoing very
important changes.

\textsl{The Bible for Home Reading} was published in 1907,
and I had got to know Mr.\ Montefiore intimately after
the publication of my “Prayers for Working Girls.”
He was well acquainted with my father long before this
time through co-operation in philanthropic and communal
enterprises, and both men had a sincere liking and
respect for each other, in spite of the fundamental
differences in their faith. Each recognised the sincerity
of the other, and respected him accordingly. My
father was quite as pleased as I when he received a
note from Mr.\ Montefiore saying that my little printed
prayers were remarkable. We did not notice, as we
should to-day, the repulsive nature of the title of my
little booklet. But, indeed, the prayers were the
outcome of a sharing between my girls and me of our
religious faith, and the desire to join together in the
search after God.

I had the opportunity of taking part in family
prayers arranged to meet the special needs of the home
at Round Oak, Englefield Green, the home of my aunt,
Mrs. Nathaniel Cohen. Here I stayed often, as my
cousin, Miss H.\,W.\,Cohen, was one of my closest friends.
I was deeply impressed by this family worship, by the
keen interest, and by the reverence shown by all the
children. It was at my aunt’s home that I first met
Mr.\ Montefiore, and I remember how easily I talked with
him. He seemed deeply interested in all the work I was
trying to do, and from that hour to the time of his last
illness I was accustomed to tell him all about my
activities and the problems which underlay them, and
to receive his encouragement and guidance. Nothing
which interested me seemed too trivial for him, and by
means of his wonderful sympathy he could make me feel
that my work in all its ramifications was important in
his eyes, and he was \textsl{glad} to help in every possible way.
He was so absolutely simple in all human relations that
the joys of friendship could be experienced without any
sense of dread or anxiety.

Mr.\ Montefiore, through his books and his personal
talk, gave me an insight into the meaning of Liberal
Judaism, and that teaching seemed to set me free from
my spiritual fetters. My girls had been so troubled by
the inconsistencies and contradictions they found in
the Bible. I now learned that every word contained in
the Pentateuch was not verbally inspired, even though
the five books contained the highest truth. I learned
that the authority for my faith lay in the conscience of
the individual, confirmed by the traditions of my faith.
Moreover, I learned that the festivals and ceremonies
were given us not as ends in themselves, but as aids to
right living. They were not meant to endure just on
account of their age, but because they directed the way
to holiness. So many of my girls’ problems seemed to
vanish in the light of this new revelation. Moreover,
I could tell them that it was not necessary to juggle with
one’s brain, or with one’s conscience, in order to preserve
the authority of Judaism. The God of the Jews was the God
of truth, and we need not deny truth in order to serve
God. Rather, the better equipped, intellectually as well
as spiritually, a Jew could become, the better servant he
was likely to prove. Judaism did not depend on learning
and intellect. The simplest thought and effort directed
towards God were acceptable in His sight, provided they
revealed the best that man had to offer. Truth was
progressive, and more truth could be realised from
generation to generation. Mr.\ Montefiore impressed on us
that we did not have greater merit because we knew more
than our fathers. The world moved on and we received
enrichment of intellectual and moral experiences. We
stood on our ancestors’ shoulders and saw further because
of our higher position. It was the duty of all men and
women to look for the religious formula of their generation,
and to add their discovery to the spiritual possessions
of the world.

Living among my girls, I soon realised that for many
of them the festivals were of little value from a spiritual
point of view. They brought heavy burdens and responsibility
to the mother in every home. If conscientiously
adhered to, their observance meant often a drop in the
family income. It was difficult in the face of economic
pressure to take days off. There was, of course, joy in
the festival feeling, in the new clothes which seemed
essential to any festival observance, and the little
feeling of excitement which the festival celebrations
produced, but the soul remained untouched.

\begin{tp}{256}
I had a great opportunity to make the Sabbath a
reality of importance when I spent the Sabbath evening
with my holiday parties. We had our little service
before supper. Everybody wore her freshest clothes,
and there prevailed a lovely feeling of friendliness and
peace. In lighting the Sabbath lights, we burned up
any ugly remains of little misunderstandings and
dissensions. The emblem of purity suggested self-control
for the individual and the realisation of family purity.
We had pleasant reading after the service, and talks on
fundamentals, and extempore prayer, which revealed
our strivings and hopes, and brought unity between us
all and the homes we had left in London. We felt the
spirit of our beautiful Sabbath Bride. I have hundreds
of letters from girls who spent the Sabbath evening at
our hostel, and promised to institute its observance in
their own homes if they ever had them. As they lived
with their parents, some few did already observe the
Sabbath in all its beauty; but far more often the want
of privacy, the late work, the absence of the right
atmosphere for prayer, made the evening of little value.
The candles were lighted — a hurried blessing was spoken
— and the family, as often as not, repaired to the pictures.
Family life, which should have been consecrated in the
homes, was left to drift along as well as it could.
\end{tp}

Gradually, through Mr.\ Montefiore’s teaching,
which helped me to satisfy the needs of the young people
with whom I came in contact, I began to realise the value
of my religious inheritance. I had been told that if we
attempted to modify or analyse our inherited religion,
it would fall to pieces, like a house of cards touched
by the slightest of blows. There was no need for such
anxiety. The God who ruled by law, who was changeless
in truth and goodness, deserved a life’s devotion far more
than the erratic Deity of my childhood, whose existence
depended on the truth of miracles. Judaism was a
living, growing, spiritual conception, not a hard and
fast collection of laws. It helped man to live and to
think, to hope and to pray, not merely to obey and to
sacrifice. It proclaimed the possibility of turning to God
at all times, and at every season, and being sure of a
place in His everlasting arms.

I had a number of girl friends who were loyal Jews,
but to whom the new revelation had not been brought.
Judaism was not a valued possession if its observance
seemed to clash with the achievement of personal happiness.
I saw some of my friends give up their allegiance
very early and drift into intermarriage. They did not
see they were giving up anything. The treasures of
Judaism lay hidden from them: the happiness of
marriage was a reality. It was easy to argue that the
man and the girl should each keep the religion which
belonged to them, that labels were unimportant, that
happiness was akin to saintliness, that marriage should
not be refused if people loved one another. I felt that
if these same young people knew the meaning of their
religion, if they had been brought up to value its beauty,
they would have thought themselves to be links in the
chain of testimony. Before Cupid came on the scene,
these young people would have realised their birthright.
They would have heard the call to service, and could
have been determined to give eternal allegiance to their
faith. I longed that my friends should know what
Judaism really meant, and should speak unto the
generations by the fidelity of their lives. The faith of
our fathers could not be easily cast aside. I felt impelled
by a strong desire to found a movement to revitalise
Judaism and rekindle the ancient lights so that these
should cast a glow over the whole of life for all time.
