\chapter{The Expression of My Personal Faith}

I have tried, in a sketchy manner, to show the
influence which helped me to give allegiance to the
Liberal Jewish faith. I will seek, in this final chapter,
to explain further the conception and scope of that faith.

I feel the reality of God. Believing in God as
the God of Love, I believe that His presence in our
midst gives us the power to love, which is of supreme
importance in every individual life. God, whose
attributes we seek in our weak imperfect way to imitate,
is constant in His Love. He neither slumbers nor sleeps.
He is near us when we go out and when we come in. His
Love never changes, for it contains understanding and
consequently the power of forgiveness. His Love is
forever, and He has put eternity in our hearts. So I
believe that the God who endowed me with the power to
love has made that love endless.

It so happens that the closest relation in my life is
with my sister Marian. We have shared our home and all
our interests. Her wisdom and serenity have supplied
the background for all the work I have undertaken and
she is able to supplement all my efforts in social service,
having in good measure those intellectual qualities
which I lack. Our lives are interdependent, and we have
come to rely so completely on each other that with
advancing years we should be depressed with the anticipation
of an early and necessary parting if we did not
cling to the idea that with God there is no death, and
that through His Love we shall always renew our love.
Faith in the God of Love has given me faith in immortality.
It would seem to me that God had mocked me if my
beloved could be withdrawn for ever, when I give her a
devotion which is without end and can never be broken
by any physical change in our development.

I have been singularly blessed in my friendships,
and in my relations with my parents and sisters and
brothers, and in my colleagues in all my Synagogue and
social work. My experience of life has brought me into
contact with more and more love. I have been so happy
for love has seemed to surround me and has given me
a protected place, even while stimulating me to draw
nearer to the source of God Eternal. The love which has
been given me in such generous measure by the young
people among whom I have worked has infinitely enriched
my life and given me the feeling that human understanding
has part in the nature of the Divine. The need for
the infinite which human love produces brings with it at
times an acute consciousness of life’s limitations. We
have a strong sense of insufficiency, the “pain of finite
hearts that yearn,” and since the God of Love is the
author of our lives, this feeling of insufficiency is also,
we think, an intimation of immortality. We reach out
in our desire to grow. We are blocked by the circumstances
of life. That’s what a Heaven’s for!

With a sense of God’s love goes a strong belief in
God’s justice; the belief that no human being endowed
with something of God’s spirit, by which he lives, can be
completely annihilated. The God who gives life cannot
err, and human personality must grow for ever towards
goodness.

In our social work we see daily, indeed hourly,
evidence of undeserved suffering. It is only God who is
all knowing, who sees the whole where we can see only a
part, who knows the meaning of suffering. But God
uses us to fight evil and we must never pause in our
efforts to overcome it. It is in social service particularly
that we can most easily believe in progressive revelation
and act on that belief. If we dedicate our work to
God and seek humbly to co-operate with Him,
we must know we are working for righteousness, and
that only in God’s light can we see light. So
we must be unafraid of change. So much excellent
work becomes abortive because we are afraid to
change our methods. New truth does not appeal to
us when it forces us to uproot our fixed opinions. As
soon as work does not progress it goes backward. It
seldom stands still. We must in social service work for
the end that every human being should have the rights
due to human personality. We fail if we lower his
self-respect or diminish his power for independence and
self-development. In former times we might have sought
to work for others as their benefactors. To-day our
highest service is to give them the opportunity to work
for themselves.

I believe in God as the God of truth. Liberal Jews
have subscribed consistently to the doctrine that we
cannot serve God by juggling with truth, and the stronger
the intellectual grasp of a man, the more likely he is to be
a good servant. It matters infinitely what we think and
believe, for thought and belief do affect conduct. We know
that some of the best Jews the world has produced have
been unlearned, and that learning and the love of truth
are not necessarily the same thing; so we must use all
the powers with which God has endowed us in His service
and be unafraid. The old conception of traditional
authority has passed away. Instead, we are required to
seek and to find and to use the truth God grants us; and
to contribute to the sum total of truth that which in His
mercy we receive from God by our diligent effort.

We reverence God as the God of Beauty. The
reality of God in the universe makes nature herself
beautiful, for it is part of His revelation. We see in the
harmonies of colour and sound the harmony of God’s rule.
We see in the regular procession of night and day, in the
passing of light to darkness and darkness to light, in the
coming and turning of the tides, in the change of the
seasons, and in the law of growth and decay, the working
of God’s changeless laws. The regularity of nature’s
changes gives to her an added beauty and our belief in the
Creator behind creation glorifies the world with the sense
of consecration. Similarly, we feel the at-one-ment of
the artist with God even while we enjoy the human technique
of the artist, and his glory is enhanced if we recognise
that he is God-inspired. There is misery in the world,
there is evil, there is danger, and there is fear, but
because God is, “somewhere something always sings,”
and the song is the song of hope, and the exaltation of
beauty. The Jewish singers have revealed a world which
in its joy and beauty reveal God. The Bible contains
many songs of praise, because the human mind in tune
with the divine is naturally awakened to a feeling of deep
gratitude leading to adoration.

I believe in the God of Righteousness, and kinship
with God makes human righteousness not only possible
but desirable. “Ye shall be holy, for I the Lord Thy
God am holy.” We think that through efforts after
righteousness we can approach our God. The problem
of faith versus works hardly presents itself to us because
faith leads to righteousness. The two aspects of the
religious life are closely related. By contact with God
we can renew our power of righteousness. We refresh
ourselves in His rivers of life. Prayer is to us an effort;
an effort to control our minds and spirits and surrender
them to the influence of the divine. As we ask for assistance,
we render ourselves more capable to serve the
Highest. We ask for all that we desire in the light of
God's light. But in asking we feel assured, in the first
place, of those spiritual gifts which grow through contact
with God. We seek to imitate God Himself, and we feel
that we are empowered to do this because God has made
us in His image. Religion is the binding of the soul to
God, and it is expressed in the life of everyman, who
seeks good and not evil all the days of his life. All
through the ages Judaism has been criticised as a legalistic
religion. It is falsely believed that we consider obedience
to the ceremonial law as the be all and end all of our
religion. Liberal Jews hold that these ceremonies must
be regarded as religious symbols, or as vehicles to hold
ethical teaching. They are not ends in themselves, but
aids to right living. Judaism has morality as its basis,
and only through the practice of righteousness can we
fulfil the Jewish life.

We are not surprised that Judaism has emphasised
the importance of social justice. A community, as well
as an individual, must live in harmony with God’s law.
We cannot believe that God is satisfied with the existence
of extreme poverty and extreme luxury, with the life of
children prevented by circumstances from realising the
best that is in them and even being denied the elementary
physical necessities. The doctrine of the Unity of God
reveals the central teaching of our faith. The whole of
life is holy — body, mind and soul. We have no right to
neglect any part of ourselves, for all parts are necessary
in the creation of the human personality. A man who
leads an unchaste life degrades not only his body but also
the mind and the spirit. There should be no submerged
sections in society for all are united under the guidance
of the One God. The Jewish prophets enunciated the
ideal of Peace, because the Unity of God the Father of
all implies the fellowship of all men.

It is not sufficient, as Dr.\ Montefiore pointed out,
to emphasise the faith in One God in order to
distinguish Judaism from any religions which accept the
divinity of more than one perfect being. We hold on to
our faith in God’s Unity because, through that faith,
life in its entirety is made holy; the whole of humanity
is rendered capable of spiritual perfectibility, for all
partake of God’s spirit. We are drawn to try to harmonise
God’s world with the ideal of righteousness which He
has revealed to us. He has left to His children the
privilege of sharing with Him in the creation of good.

As I grow old, I feel more and more clearly that the
denial of God’s rule leads to international conflict and
social disintegration. There have been attempts in all
ages to turn God out of His world. In recent times that
attempt has been made with peculiar energy and
concentration. The result has been the present chaotic
state of the world, and the tendency to allow hatred and
cruelty to fill the throne which men think blindly is empty
of God. I believe that with renewed faith in the One
God, peace will prevail throughout the world and love
will show itself in the lives of all peoples. I pray that I
may be allowed, with those I love, to share in the work
of reconstruction which will follow the war. I believe
that the new basis will be spiritual, and on that basis a
righteous and just life will be built up.

We Jews have lost some of our influence because the
name Jew is no longer synonymous with a faith in the
vitalising power of religion. We have to come into our
own again, and by the intensity of belief increase the
spiritual life of humanity, through our own conduct and
through the depth of our power of loving. We shall
strengthen the influence of beauty by our reverence for
beauty. We shall stimulate the search for truth through
our love of truth. So we shall move more and more
quickly to the day when God will be One, and all peoples
will call upon His Name.

I hold this belief as the outcome of our traditional
faith that God placed the Jews in the world as witnesses
to the reality of His being. That is our function, the
purpose of our existence as a brotherhood. We must each
contribute by the conduct of our lives to the strengthening
of our communal ideal inherited from the past and
intended for an infinite future. We can serve in this
most holy cause only if we ourselves, besides inheriting
our mission, become convinced of its truth through our
own passionate faith supported by actual communion
with our Divine Father in prayer.
