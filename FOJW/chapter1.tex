\chapter{Faith and Childhood}

I was brought up in a home impregnated with the
Jewish atmosphere. In spite of the fact that my father
had many public duties to perform in connection with
his life as a member of Parliament and subsequently as
a Peer, all the members of his household knew him
primarily as a great Jew. His Jewishness was the
central fact in his life.

As an Englishman, my father was passionately
concerned with the well-being of his country; and he had
a number of other secular interests, being a keen fisherman,
a good whist player, a lover and connoisseur of
great pictures and of silver and porcelain, a clever
business man with a deep knowledge of big financial
issues, a lover of nature, and an inveterate novel reader.
He had travelled widely in both hemispheres.

My father’s Judaism had a setting of its own. It
was something infinitely important, which cast an
influence over the whole of his life. It was like a mountain,
which could not be moved or shaken, and which
was complete in itself. It was altogether separate from
other interests; it dominated them. The mountain could
be scaled only through effort and sacrifice: so with my
father’s Judaism, it was difficult to attain, and it
certainly could not be ignored or even disregarded. It
could not be diminished or tampered with. It was a
controlling power right in the centre of his life. After
the effort of scaling the mountain had been accomplished,
it produced joy and quietness of mind. Life
was made holy because of the beyond which touched it —
the mountain tops on which was eternal light. The
clouds on the mountain were the mystery of faith which
must be accepted, but could not be penetrated. Those
clouds would eventually disappear under the influence
of God's light. The path up the mountain was the path
of undeviating obedience, which produced the disciplined
life of the good Jew. My father was the kindest and most
tolerant of parents. His children (and he had ten)
were not afraid to do as they liked and to say what they
wished in his presence. He was rather silent in the home
circle, although always a genial host when he was
called upon to entertain guests. He had a power of
abstraction, and was never disturbed by our chatter
even when he was writing difficult business letters or
preparing his notes for public speeches.

When I was a child, my religion was definitely
shaped by my father. Up to the age of fifteen, it
never occurred to me to question the authority
for all the small regulations which, in the name
of Judaism, restricted my liberty. I would never
think of picking a flower or a fruit on the
Sabbath. That involved work and destruction, and was
not allowed on the Sabbath. For the same reason, I
cheerfully allowed a gentile maid or governess to open
the envelopes of the letters I received, or to write the
essential letters at my dictation and to sign my name.
Head work was not altogether prohibited so long as it
was limited to matters of vital importance, but manual
work in any form was forbidden. We were not allowed
to travel by train, bus or carriage. Not only must we
not let animals work on our behalf, (our own animals or
those belonging to anybody else,) but it was contrary to
Sabbath observance to take part in any form of travelling.
The question of convenience did not in the slightest
degree affect the situation. We referred to the times
advertised in the official calendar for the beginning and
ending of the Sabbath, and we knew ourselves forbidden
to move about except on foot, between these times.
Invitations which involved journeying to a place outside
our walking strength were just not accepted. We did
not rebel. We believed that we were acting as Jews
must, and there was no court of appeal against the strict
laws. There was no question of degree. All the regulations
were part of the fence built up to defend the
Divine laws as given to Moses, and we accepted them.

When on the festivals we stayed at our country
house in Hampshire, we walked with our father to and
from Synagogue along a hot and dusty road, and we
believed we were doing right, however exhausted we
might feel. We were allowed to have children to tea
on Saturday afternoons, but we had to choose our games
carefully. There must be no breaking of anything. The
piano must not be used, for we were still mourning for
the destruction of the Temple, and even uplifting music
must wait until the Sabbath was over, and then we were
also free to play and dance as we liked. But we were not
in the least unhappy; nor did we chafe in any way against
the Sabbath and festival rules. These were accepted as
essential to our lives and were rigidly obeyed.

As I was a shy child, it was sometimes a little
difficult for me to explain to my hostess when I went out
to a meal that all sorts of restrictions were necessary.
I must not eat any kind of meat, not even chicken,
unless prepared according to “Mosaic” rites, and I must
not eat cakes or cream for at least one and a half hours
after meat. I saw what troubie my father took even on
his travels to deny himself any forbidden food. At a
grand hotel table d’hote, he would contentedly sit and
eat bread and cheese if the manner of cooking vegetables
made him suspicious. I felt sure the laws were good,
that they ensured cleanliness, that the meat we as Jews
ate was produced only by the most humane methods. I
was satisfied to follow the practices of my ancestors.
The law was divine: its hygienic enactments, its ceremonial
regulations, its moral and spiritual commandments,
were all divine in their origin. Of course, one
could not pick and choose in observing these laws. If
second day observance of festivals, which might lower
my place in class at school, made me a little restive, my
father would say that Jews who kept one day were
allowing the thin end of the wedge to be driven into
their Judaism. If all the leaders of the community
rabbis and laymen all over the world, decided together
to follow a modern calendar and abolish second day
observance, he would not stand out: but the leaders were
not inclined to uniformity in traditional interpretation;
they were afraid of any changes or modifications, and so
it was far safer and better cheerfully to obey too much
than too little. My father used to cite a Jewish leader,
an English member of the peerage, who confessed to
racing on the second day of a festival, and, when pressed
closely, admitted that on \textsl{one} occasion, he had raced
on a first day. So through his laxity he had been led to
sin. It was far better to be over careful.

As a young child, I was quite convinced. (Years
later, through the teaching of my dear friend,
Mr.\ C.\,G.\,Montefiore, it was explained to me that the human spirit
must make itself responsible for the conduct of life. It
could not be enslaved by laws which had lost their
meaning. The argument that when you had decided to
give up a meaningless law you were certain to disobey
vital principles was only worthy of a meat jack which
turned automatically and could have no power to resist.
The human personality was on a different plane and must
be responsible.) I followed the dictate of my father,
and tried to live as he did.

What about my mother in all this? She had married
at the age of eighteen, and had herself been educated in
an Orthodox home. The observances in her home were,
I should imagine, as rigid as in the home of my father’s
parents, but there was nevertheless a considerable difference.
My father left school at the age of fourteen, and
is to be reckoned among those real aristocrats of humanity,
the self-educated men. He had no opportunity to
come in contact with cultured men and women as did my
mother. He worked hard, and read enormously during
all the scanty hours of leisure he could spare from
business.

My mother’s father was one of the few leaders of the
Anglo-Jewish community and his home was a centre for
visitors, both English and foreign. She was influenced
by the intellectual activity of her day. It was, I always
think, mainly due to her complete wifely devotion that
my mother remained faithful all her life to the Orthodox
observances, both great and small. She was most
scrupulous about all the dietary laws and in the most
minute details of preparations for the Feast of Unleavened
Bread, the Passover. She helped my father to follow all
the small ritualistic customs beloved of his ancestors,
but she did allow herself a certain degree of elasticity.
She liked the introduction of English prayers into our
home Sabbath morning service, and herself read a
Biblical passage of her own selection every night before
retiring to rest. As we grew to adolescence, and asked
our father rather radical questions, we could count on
our mother’s sympathy and support if our questions, in 
her view, were not \textsl{altogether} unreasonable. My mother
was deeply impressed with the belief that her children
would escape from the paternal nest if held there against
their will. She understood better, I think, than my
father did, that young people could not be expected to
want life to follow exactly the same pattern used by
their grandfathers. My father was true to that most
popular of all fallacies, that what was good enough for
his fathers was good enough for him. My mother would
add, in the depths of her unformulated thoughts, gently
and wistfully to herself, that the life of the past was not
likely in any of its aspects to be good enough for her
children. It \textsl{must} be better. I don’t think her liberalism
went much further than that. She was a woman of the
greatest piety, but she did not think it was part of her
role as wife and mother to seek new presentments of
religious doctrine. Perhaps her children might feel they
had the right to do and think things which their mother
would consider quite improper for herself, but then she
was driven to her conclusions not so much by thought as
by the impulse of love which is divine in origin.

My earliest recollections of Hebrew and religious
education come from my mother, who began when I was
about four years old to teach me my Hebrew letters by
means of a game of word-making. It was she, too, who
taught me my earliest prayers. “God bless my dear
Mamma and God bless my dear Papa, my brothers and
sisters. God bless me and make me a good girl.” My
mother used to read to us Little Miriam’s Bible Stories,
and we delighted in these tales. I began my serious
Bible and religion lessons with Miss Rebecca Aguilar,
but I am afraid that the only clear memory I have of
these lessons was that of learning the “grace after meals,”
and my being helped to produce the words by the teacher’s
silent emphasis with her lips on the initial letter of
each word.

The recital of the grace after meals was an important
observance in our family. We used to say the first
paragraph in turn after mid-day lunch and recite the
long grace in Hebrew in turn on Saturday mornings, in
the presence of our father. This was rather an ordeal,
but he and my mother were perfectly patient and never
found fault with us even if we hesitated and giggled
a little through nervousness. They would just wait until
the recital could be concluded.

On the Sabbath Eve we read with our mother in
Hebrew the prescribed service from the prayerbook,
paragraph by paragraph. It was my mother who lighted
the Sabbath lights at the time of sunset, and my father
who blessed the wine and bread before the Sabbath
evening meal. We rejoiced in the Sabbath evening meals
when we were allowed to stay up late and each in turn
was blessed by both our parents before the actual business
— the eating — was begun.

The Sabbath was, I think, always a day of joy. We
liked walking to Synagogue with our mother and looking
round at the congregation. Our father generally went to
Synagogue at an earlier hour than ourselves. The service
itself, with its long Scriptural portion read in Hebrew,
simply bored me, but not badly enough for me to resent
the experience of having to sit still. \textsl{Until} I began to
think about religion, which was not till I had reached the
age of fifteen, I never expected anything from my observances
except the pleasure of having carried them out
in good company, and done what was expected of me.
I was interested in the various symbols, in the raising
of the Scroll and the opening and closing of the Ark.
I don’t think I ever \textsl{prayed} in Synagogue. I just sat
through the service, quite unimpressed, but pleased with
the familiarity of everything around me. My faith only
entered into my life when, while suffering from night
terrors, I was told by my mother that nothing would
happen to me, for Papa was so good: God would never
hurt him so much as to allow his little girl to suffer.

When I heard of death, I was immensely comforted
by the assurance of one of my parents that the soul
sprouted from the grave after the body was planted, just
as flowers did from the roots put into the ground. God
seemed very near to me, but in the Person of a Lawgiver,
whose laws were prescribed through the parents He in
His love had set over me. I had just to go on with my
life as well as I could and obey all the teaching that was
given me. Religion was almost entirely objective as far
as I was concerned, and I was quite content to be interested
in it as something outside myself.

The festivals were full of interest, again in this
objective way, and had the power of making life romantic
and exciting. This was particularly the case with
Passover. We rejoiced in all the details of the Seder (the
first night of the festival commemorating the Exodus of
the Israelites from Egypt) and in the rigid observance
of all the ancient ritualistic customs. After the house
had been completely cleaned of every form of leaven, and
even the silver dipped in hot water by my mother herself
for fear a crumb might cling to it, my father would go
round the house and seek leaven. And my mother would
sometimes put something forbidden in his path so that
he might have the satisfaction of finding and burning it.

I want to convey to my readers the fact that I
should have been a most insensitive child if I had not
been affected by the spirit of devotion and the complete
faith of my parents. I was not conscious of any personal
spiritual experience stimulated by the Sabbaths and
festivals, but I could become very enthusiastic over
the symbols, and, if asked, should have unhesitatingly
said that their preservation was required by God.

The observance of the Day of Atonement had a
peculiar effect on me. I was deeply impressed by the
satisfaction which both my parents felt when it was
completed, and we were able to take up ordinary life
again. It was certainly not that either of them chafed
at the physical discomfort of the fast day. They both had
complete control over their appetites, and were prepared
cheerfully to suffer any inconvenience for the sake of the
faithful observance of the Law. But the day was so
important. It was such a privilege to observe it. It had
inherent power over men’s lives. It was \textsl{the} day.

I can trace my first questioning of the utility of
observances if pursued as ends in themselves to experiences
connected with Passover and the Day of Atonement.
My father read the Seder service to all his children and a
large family of nephews, nieces and a few stray friends.
He read every word from beginning to end, and many
of his hearers behaved as if nothing was going on which
was even remotely connected with religion. They
joined in the singing without even the slightest reverence;
they joked and laughed; and my father went on reading,
and, at the end, with unquestioning faith asked God to
accept the divine service. I remember rushing up to my
eldest brother after one of these Seders and expostulating.
“I feel ashamed,” I said, “at the behaviour of many
of the people. How dare they think they are praying?
If that is religion, I hate it and would rather take the
religion of ——” (mentioning a rigid Christian of my
acquaintance). “You don’t understand,” he said.
It is the Jews’ bank holiday; they should be jolly.”
“But why \textsl{pretend} to pray? I would gladly join in comic
songs if people want to break loose, but let the service
stop.” He laughed, but my cheeks were very red and.
hot, and I began to wonder about the funny religion
which permitted such crass irreverence.

Then the Day of Atonement ritual was so full of
repetition. People came in masses, but they were not,
apparently, interested in the long service. They thought
it quite right that it should begin early and continue for
the prescribed time, but they really seemed unaffected by
the contents of the prayers. They lounged back in their
seats in Synagogue. They walked in and out. They
chatted and laughed on the threshold of the synagogue.
For the closing section, they seemed all to return. The
phrases were deeply impressive. I was moved by the
closing proclamation of faith in the One God, and
believed all my fellow worshippers were similarly
affected.

But no sooner was the last word spoken than there
was a very hurried exodus. Prayer shawls were thrown
down, seats opened to receive prayer books, and people
literally rushed home to break their fast. The few
remained, for an evening service was in progress. Nobody
wanted it, but it was the right and proper thing to read
the evening prayers at that particular hour, and the
Minister had to stay and read. We who loved our
Synagogue and its leaders stayed also, but we closed our
books. Our bit was over. The reading was accomplished
at a miraculous speed, and we too hurried home. \textsl{Then
everything began as usual}. \textsl{That} was the part which
mystified me, and set me thinking. People were tired —
and a little irritable. They all seemed in a hurry. The
\textsl{solemn day} seemed to have made so little difference.
What did all this piety mean? Why didn’t it get inside
you and change you a little? How could it be \textsl{finished}
so suddenly? I was approaching adolescence and beginning
to ask questions. There were people ready to
answer me and to help me in my bewilderment.
