\chapter{The Relation of Jews to Non-Jews}

I have all my life been in close contact with
non-Jews and have counted them among my closest friends.
I have always felt drawn to those who were greatly
interested in their own religion, for these were in close
sympathy with my devotion to Judaism. We shared the
essential need to feel in contact with God, and were
satisfied that our modes of approach were different and
distinct.

\begin{tp}{512}
I was glad when, at the suggestion of Mrs.\ McArthur,
a member of our Synagogue, our Society of Jews and
Christians was formed, under the leadership of
Dr.\ Mattuck. Through our meetings we were able to get a
better understanding of Judaism and Christianity and a
clearer recognition of our differences. Better
understanding, of course, opened out new vistas for possible
co-operation. We found that while our doctrines and
beliefs were fundamentally different, we could
wholeheartedly endeavour to work together for God our
Father, in the service of our fellow men. The
cooperation between Jews and Christians in work of social
amelioration revealed different aspects of the God-idea
which both communities held in common, and this
tended to increase our faith and to strengthen our hopes.
\end{tp}

In the social work of an undenominational character
in which I was engaged I was always conscious that on
account of my religion I was \textsl{different} from my fellow
workers. A certain self-consciousness is necessary to
every member of a minority group. In the Women’s
Industrial Council and the National Union of Women
Workers (later National Council of Women), and in all
the work which I did with the Central Association for the
Employment of Women during the last war, I was
naturally always accepted on exactly the same footing as
my colleagues. My self-consciousness must not be taken
to mean that I was conscious of any coldness or contempt
because I belonged to the minority group. I did,
nevertheless, feel a certain sensitiveness which is hard to define,
and a considerable degree of responsibility. Both these
feelings were increased as anti-Semitism acquired a
greater hold on the community as a whole and when in
England some degree of prejudice became more and more
apparent. I was always anxious to show that Judaism
demanded a high standard of integrity and required all
its adherents to be scrupulous in thought and in deed.
If Jews did not \textsl{care} to act according to the dictates of
morality, it was certainly not the fault of their faith but
because they were among its unworthy adherents.

I was sensitive to any criticism about our Jewish
community or of any individual Jew, and I often felt
impelled to speak in their defence. I remember once
giving an address on the teaching of Liberal Judaism. to
a Church Society, and being asked at the close: “Why,
with all those fine doctrines, are Jews always twisters?”
I had to explain that here and there Jews were at fault,
not Judaism. Of course, we are always let down by those
who, while giving nominal adherence to our faith, do not
cause it to affect their lives. If we could show the beauty
of our faith in our own lives, we should not need to offer
defence for the minority whom we represent.

I came in close contact with Club leaders and Club
members of every denomination through our Clubs’
Industrial Organisation, and, later, through the National
Organisation of Girls’ Clubs (now the National Council
of Girls’ Clubs), which, with Mrs.\ Arnold Glover, it was
my privilege to found. These contacts gave me the
opportunity to discover the extraordinary ignorance
about Judaism and the Jews which prevailed, particularly
among the industrial section of the population.
In the first place they identified Jews with foreigners.
They had no conception that our patriotism, like every
other fine spiritual emotion, was evolved naturally
from the influence of the Judaism in which we believed.
For myself, I give place to nobody in the degree of my
love for England, and every moment of my life I am
grateful that I am an English woman of the Jewish faith.

Unfortunately, the identification by Jews themselves
of “nationalism” with religion has given some
excuse for the fallacy that Jews do not belong to the
country in which they live, but are enduring a long term
of exile which can only terminate with a return to
Palestine. The Balfour declaration, in spite of its saving
clause, gave further strength to this belief. In the last
generation of Jews, the references to Jerusalem which
exist in the Orthodox liturgy were interpreted only in a
spiritual sense. A return to Palestine would be the
culmination of the Messianic period when the Kingdom
of God would be established on earth, and His law would
go out from Jerusalem. God, as the One God, would be
known throughout the world and all nations would
together call upon Him. But modern Zionism, the
outcome in a great measure, not of a religious ideal, but
of human cruelty, persecution and oppression, gave a
new turn to the relations of the Jews to Jerusalem. The
anti-Semite refused to share his nationality with the
Jew whose claims were really identical with his own, and
who was often quite as good and devoted a patriot. The
Jew became “staatlos” and homeless and turned his
eyes with longing to the land of his fathers. His misery
aroused the compassion of his fellow Jews all over the
world, but unfortunately some of these, even in emancipated
countries, replied to the challenge of the anti-Semite
by accepting the anti-Semitic point of view. The Jew was
prepared to regard Palestine as his country if it could
give deliverance to the oppressed Jews from the hand of
the tyrant rulers. Unfortunately faith shifted from
the spiritual plane to the political and temporal. Instead
of demanding freedom in the name of God because
freedom is necessary to the well-being of humanity
since every man is created in the image of God, the
Zionist urged with passionate earnestness that the Jews
might have a better chance in the world if they had their
own country and the protection afforded by their own
government. They insisted that Palestine alone could
give the exile a home and safeguard his interest. It is
hard to convince our fellow countrymen in countries
where we have been fully emancipated that we exist as a
spiritual brotherhood and not as a political entity. We
do \textsl{not} desire an army and a navy of our own, but we do
desire to give testimony in freedom to our faith in God.
If our co-religionists cannot remain in their own countries
it is because the world is still in certain places Godless.
There need be no refugees from a country in which God’s
rule is recognised. In the meantime, we welcome any
place of refuge offered to the persecuted and the homeless,
but we do not wish to perpetuate these evils of persecution
by sacrificing our great religious inheritance on the altar
of prejudice and oppression. Of course, on historical and
sentimental grounds, Palestine offers an acceptable
home to the homeless. But we who believe in Judaism
as a universal religion cannot identify the acceptance of
its spiritual teaching with any one place or country.

Our people are inclined, in workshop and factory,
to give colour and apparent justification to the errors
of their neighbours. They often call themselves Jews,
and their non-Jewish fellow countrymen English. Thus
they accept the discrimination of their persecutors. So
it happens sometimes that a Jew says that his neighbour
is jealous of him because he is careful of his money and
the English spend it on drink, or that a Jew is more
concerned with his home and family than are the English.
So long as he does use these terms of identification, the
Jew has little cause for indignation when, in a court of
law, a judge, in sentencing a Jew for keeping a brothel
or illicit gambling house, says: “When I have punished
you, probably another member of your nation will take
the premises and carry on in a similar manner.”

We social workers have sought strenuously to get
our Jewish friends to place their emphasis as Jews on
their religion. They must respect their neighbour’s
religion, even while they require respect for Judaism.
They must see that both Jews and Christians are
unfaithful to and even deny the teaching of their respective
religions, when they cast aside the standards of ethical
life. No outward observances and no church-going can
automatically make Jews or Christians true to their
faith. The observances are only vehicles for right living:
the church services give stimulus to good conduct in
everyday life. The feeling of enmity and mutual contempt
which is unfortunately increasing in our country
is due, in a large measure, to the mistake made by the
Christians, and often unchallenged by the Jews, that our
differences are racial rather than theological,

My main activity in undenominational social work
has been during the last decade, and is to-day, mainly
confined to work in the Juvenile Courts. I have throughout
that time been accepted by my colleagues without
the slightest suggestion that through my Jewish allegiance
I am different from them. But my self-consciousness is
always aroused when misdemeanours. among my coreligionists
have to be considered. I feel ashamed when
a Jewish child has been allowed to grow up quite ignorant
of the restraining influences of religion. Unreasonably,
perhaps, I am particularly ashamed when a girl or boy
is considered beyond control, because he is under the
domination of self and is interested only in gratifying
his appetites. The percentage of these cases is very
small, but, somehow, perhaps through my own
self-consciousness, the attitude of the Jewish parent who
has neglected his spiritual obligations seems to me
particularly revolting, especially when he lays stress
on his material well-being, and on the fact that he has
denied his child nothing. His want of understanding,
which must give to non-Jews a poor estimate of Judaism,
increases his culpability, as it reflects adversely on his
community, who, as a minority group, have to bear a
disproportionate amount of attention.

\begin{tp}{256}
The ideal of justice as interpreted by the Hebrew
prophets and teachers gives to those who dispense justice
in any court, juvenile or adult, a fine inspiration. We
cannot altogether despair of human personality created
in the image of God. Given the right opportunity, the
wrongdoer can return to God at any moment, if he will
make the necessary effort at atonement. The community
which denies to an individual freedom to live worthily,
which shuts him out from the enjoyment of his birthright
as a human being, must be blamed for his backsliding.
The human personality is sacred — body, mind and soul
are sacred. We dare not crush them under the indifference
and cruelty of any section of society. We must be
alive to man’s needs and give him the opportunity of
self-redemption.
\end{tp}

Many sayings which occur in our wisdom literature
are helpful still. Our fathers were accustomed
to pray for the power of resistance against sudden
temptation, which, as we know ourselves, constitutes
the great difficulty in the lives of young people. “I don’t
know what came over me; I suppose it was a sudden
temptation,” we hear constantly reiterated by our boys
and girls when trying to excuse dishonesty. Our
ancestors, with their keen knowledge of psychology,
exhorted their disciples to put the desire to serve God
always before their minds, because they knew that we
are more easily saved by positive teaching than by
negative exhortation. In Jewish tradition we find many
valuable lessons about the evil inclination which can,
and must, be controlled. Human appetites and instincts
must, however, be regarded not as evil in themselves.
They can be sublimated and used for the best and highest
purposes. Indeed, those who crush their physical
inclinations instead of directing them towards good are
themselves guilty of wrongdoing.

In my relations with non-Jews I have naturally
considered the missionary problem as it affects both Jews
and Christians. I have always had a considerable degree
of sympathy and respect for those Christian missionaries
who, finding individual Jews without any religion
whatever, feel impelled to teach them the best they
know. I am not now referring to those Christians who
hold that all people who do not believe in the
Christian Saviour are doomed to damnation. I do not
think any such belief can be consistent with belief in the
All-Father. But, in a country recognised as living under
the God-idea, I can understand Christians feeling
impelled to try to influence men and women who attempt to
live without any knowledge whatever of religion. If we
have not done our duty to equip Jewish children with a
knowledge of their faith we cannot blame Christians
who attempt to make good our neglect. The methods of
Christian missionaries are sometimes deplorable, degrading
to the missionary as well as to those whom they
believe themselves to be serving. They make capital of
the misery of the submerged, using spiritual promises as
a kind of blackmail by offering in return for baptism the
food and clothing and medical aid of which the potential
proselytes are in dire need. We know, however, that
vast numbers of our Christian friends disapprove of this
kind of missionary effort as much as we do ourselves, and
believe that the only form of missionary work worthy of
being done in the name of their Master is that by which
an appeal is made to the Jewish mind and heart, and that
its success can only be achieved through conviction. I
should like to explain my own view on Jewish missionary
work so far as it can exist in our community.

I feel that as custodians of the truth of Judaism, we
ought to share our spiritual possessions with our neighbours.
The gifts of the spirit must not be held by
individuals or communities in terms of proprietorship.
Our glory should be in sharing. We hold that the righteous
of all people share in the Kingdom of God, both now
and forever. We cannot believe that those to whom no
revelation has come are less precious in the sight of the
Lord than those for whom truth has dawned. Yet, as
believers in the faith of Judaism, we must hold that our
religion contains a greater measure of truth than any
other. Why else should we be faithful to it? Also we
think that belief and conduct are related, and it does
matter what beliefs a man holds sacred, for these beliefs
affect his life and shape his daily conduct. Of course,
we all know Christians who lead as high a life as any Jew
of our acquaintance; but we should be insincere if, while
proclaiming our faith in the Unity of God, we did not at
the same time affirm that the man who believes wholeheartedly
in the One and Only God to whom he is directly
responsible for the conduct of his life has a better incentive
to righteousness than he who believes himself
redeemed by the Christ who gave his life to atone for the
sins of the world. Is it not a fact that the majority of
Christians are quite ignorant of Jewish doctrine? They
sometimes know a few Jewish customs which are interesting
because of their quaintness, but they know nothing
of our spiritual teaching, perhaps not even of its
existence. It seems to me that if we are to be true to our
responsibilities we must make our point of view known
and understood.

Among the more ignorant section of the population,
the Jews are still held to be the enemies of Christians,
because they are believed to be responsible for the death
of Christ. I have asked, and my request has been
sympathetically received by heads of churches and
religion schools, that the historical account of the trial
and death of Jesus should be accurately given. More
especially I have asked that, even though the Jews’
attitude towards Jesus did nineteen hundred years ago
expedite his death, the Jews of to-day must certainly
not be penalised for the tragedy of the far-off past. It
can and should be shown how contrary it is to the spirit
of Christianity that hatred should be perpetuated and
its fruits allowed to poison the minds and hearts of
generation after generation.

\begin{tp}{512}
Like every other community, the Jews have their
converts. These are usually attracted to Judaism if they
want to marry a member of the Jewish community and
feel that they cannot establish a home on a foundation
of differences. The proselytes, whatever their motive,
are accepted by Liberal Jews only after being carefully
instructed and when there is unequivocal evidence of
sincerity. We have had our community greatly enriched
by the admission of converts who have given whole-hearted
devotion to the teaching of Liberal Judaism.
Sometimes, of course, in spite of all the care taken by
the ministers, we are disappointed in the lapses of those
who seemed at one time to give wholehearted allegiance
to Jewish doctrine. It has always seemed to me to be
of the utmost importance that in giving a new religious
point of view we do not deprive any man or woman of the
support to which they have grown accustomed and which
they cherish. We can do infinite harm if we attempt
conversion without the most careful recognition of the
soil in which we plant our seed. We may risk the
disintegration of the spiritual and moral organisation of
another human being, and so imperil his future wellbeing.
Such missionary efforts are, in our view, inexcusable,
if they are made without necessary care and
integrity, and if pressure is used to upset any particle
of existing faith by means of coercion. But it
is not only legitimate, but part of our religious obligation,
to share the truth which we hold sacred with
those who desire to receive it. Dr.\ Montefiore devoted
a great part of his life as an author to making Judaism
understood by Christians, and in showing his fellow
Jews what the New Testament teaching can offer to
them as sincere and consistent Jews. Through our
services at the Liberal Synagogue, especially the Sunday
services, with the opportunity our leader offers for
questions, we are able to explain Jewish teaching to
numbers of Christians who are desirous to know more of
our religion. These opportunities are fruitful in result,
not only because they dispel ignorance, but because they
show our positive contribution to the spiritual faith of
our generation.
\end{tp}

In my next and final chapter I will try to show what
form our belief takes, and it will be seen that ideas held
sacred by the Jews will be invaluable in building up the
Kingdom of God. We have so much which is universal
in its application, for it is independent of time and
place, and seems to be revealed by the God of the Universe
to the heart of mankind. We stand at the bar of humanity
and ask that this teaching of ours should be understood
and accepted, wherever its appeal is established.
