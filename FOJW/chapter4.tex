\chapter{The Foundation of the Jewish Religious
Union for the Advancement of Liberal Judaism}

When in about the year 1900 I considered the difficulty
of reconciling religious truth with the life with which I
was familiar, I learned that both in Germany and in the
United States of America, over a hundred years ago, a
new presentment of Judaism had been given to the world.
In both these countries, the Reform movement was
strong, and the children in the schools were being saved
for Judaism, for they were spiritually alive. So I wrote
an article in the \textsl{Jewish Quarterly Review} on the spiritual
possibilities of Judaism, and set before the community
the choice between death and life. Mr.\ Montefiore was
not inclined to lead movements as a partisan, even in the
cause of religion. He was too big a scholar for that
position. But when I received letters from my correspondents
to whom I submitted the principles underlying
the spiritual possibilities of my faith, he told me
that he was prepared to accompany me in my adventure
into the unknown. We were going to gather people
together and show them the eternal elements in Judaism.
From that time, Mr.\ Montefiore’s leadership was given
to the formation of the Jewish Religious Union. The
cry was no longer for changed externalities such as were
secured by the Reform Synagogue already established
for seventy years, but for a re-statement of Jewish
doctrine in the light of scientific truth.

\begin{tp}{512}
Economic pressure made Sabbath observance impossible.
The number of people who could work only
five days a week was extremely small. I had noticed that
there was little relation between real piety and outward
conformity to religious observance. It was borne in upon
my mind that unless ceremonials expressed spiritual
aspiration they were not worth preserving. Services
were held at times made impossible for workers by
economic pressure. These people were not wicked
because they did the only work open to them, instead of
being dependent on others. The laws which required
attendance at divine worship at times irreconcilable
with industrial conditions, could never have been
made binding for all times by the Spirit of perfect
wisdom.
\end{tp}

Those people, however, who had leisure to attend
services were unwilling to do so because they were sure
of being bored. Their knowledge of Hebrew was scanty,
and they needed to pray in the language in which they
were accustomed to think. Certainly, Hebrew had great
beauty as well as historical interest, and it was a bond
between different communities that they should have the
same liturgy. But the religious bond could be of real
use only if it expressed a living faith. There was no use
in peoples meeting in various lands and going through
identical services, in order to endure an identical sense
of boredom. Life was \textsl{essential}, and the bond of religion
must be the bond of life.

The Jewish Religious Union offered services mainly
in English at a time when those obliged to work on
Saturday mornings were nevertheless able to attend in
the afternoons. We no longer took the letter of the law
as important. The Sabbath principle was of spiritual
value and had been established by Jews for all times.
All the same, truth and sincerity, freedom and self-respect
were still more important than literal obedience,
and when these were inconsistent with the prescribed
usage, the usage must be adapted to meet the needs of the
hour.

In my relations with the members of my Club, the
problem of Sabbath keeping created an infinite number
of doubts. As children of those who thought on fundamentalist
lines, they actually thought it better to forego
worship altogether than to break the law. The Sabbath
must be maintained from sunset to sunset, and those who
could not accept all the privileges of Sabbath keeping
were outside the pale. They must forego Jewish public
worship altogether. I must remind my readers
that the actual need for contact with God in
public worship had not been sufficiently recognised.
The importance of obedience to the letter of the law
had been accepted by the majority for hundreds of
years. I felt that I must help in altering the emphasis:
I must join those who would say — “The Synagogue needs
your co-operation. Your life must be joined with that
of the other members of the Congregation. You must
join us because we share aspirations and needs. You
must give of your vitality. You must grow in fellowship
and in sharing your experience with your fellow Jews;
you will feel the unity of worship, the divine unity which
unites people through their relation with God.”

My people told me that it was impossible to go to
Synagogue, if, in order to do so, you travelled by bus or
train. If you lived beyond walking distance, you must
remain at home. In vain I told them that the law about
rest could be brought home only to those who possessed
animals and consequently were responsible for their rest.
It was better to make use of the vehicles which travelled
with or without your co-operation than to remain away
from services. They would not mind driving, they said,
in order to go to business or even to the cinema, but it
was hypocritical to go to Synagogue pretending that you
were a faithful Jewess, and breaking at the same time
the sacred law of your fathers. So they would go shopping
or walk up and down much-frequented streets and show
off their pretty Sabbath garments, and I writhed and
grieved because they, in the name of religion, would not
attend Synagogue and had not the slightest conception
of the Sabbath idea.

We tried to make the new services as beautiful as
possible through the choir and instrumental music,
believing that music strengthened the emotional influence
of the service. I could not feel that the traditional view
was acceptable, that since our Temple was destroyed, we
could not have the right setting for music and must
dispense with it. We had resolved to stimulate the soul
of the worshipper, and we could not be recalled from our
task by any legalistic considerations.

For ten years, under Mr.\ Montefiore’s leadership,
we tried to formulate our views in accordance with the
claims of tradition and the religious claims of actual life.
At the beginning, our new presentment of Judaism
attracted vast congregations. We had novelty on our side,
and the curious were attracted. Our success roused fierce
opposition, and as the community is always interested
in discord, our attendances grew still larger. The
task before us was prodigious. People came gradually
to agree with us that new types of services were necessary
to meet the special needs of the time. Even my father
and my teacher, Mr.\ Singer, were in favour of services
for young people, so long as these could be regarded as
supplementary to and in no way substitutes for the
statutory services. The question of authority had not
been publicly challenged. The Law retained its place of
distinction in the Ark or Consecrated Receptacle prepared
to receive the Scrolls. Mr.\ Montefiore showed us
that through the Law the spiritual and ethical teaching
of the ages was focussed and expressed in action and
conduct. The Law brought the teaching into daily
life. Therefore, it was quite consistent with our ideas
to preserve the traditional reverence for the Scroll. But
when, in a manifesto signed by Mr.\ Montefiore as President
and Mr.\ J.\,Abrahams and myself as Vice-Presidents,
we affirmed that we did not believe the Law to have been
given in its entirety on Mount Sinai, nor that every
word was literally true and binding for all time, then
there was terrible consternation in my home and
in the community We, who ventured to place the
authority for truth in the trained conscience of man,
were accused of disruptive tendencies. It seemed
that schism must of necessity occur in the community.
There would, we were told, be chaos
unending, and our ancestral religion would altogether
lose its binding influence. The manifesto declared
that the Pentateuch was built up by many generations
of teachers, and revealed progressive truth, but
that not every word was true. We appealed for a new
presentment of the old teaching, in a form acceptable to
those who had been affected by the scientific spirit of the
age, and who could not think that God was honoured
through the suppression of truth. People who had been
puzzled by the ethical contradictions in the Bible and
by the low standards of morality occasionally expressed,
were introduced to the work of the Higher Critics and
shown how the Bible was built up through the ages, and
the meaning of progressive truth. The teaching of the
prophets was related to the actual conditions of life
through which we were passing. Our preachers no longer
confined themselves to dissertations on dogma, but
applied the teachings of Judaism to the live problems of
the day, whether personal, social or political. Religion
was lifted for us out of the communal record office and
placed in the big thoroughfares of life.

It is not my intention to give the history of the
Jewish Religious Union. Such a history to be at all
adequate would require a volume of its own. In trying
to reveal the development of my faith, I should like to show
how devotion to a cause far bigger than even an age-long
religious tradition seemed to galvanise my own belief
into something very vital. It is, I think, possible to
belong to a place of worship and to feel considerable
interest in it and much real affection, and yet to attend
only as a matter of habit and convention; not in reply
to a soul-stirring appeal.

The teaching of Liberal Judaism had set me free. I
felt I must take every opportunity to testify to my strong
faith. I wanted to think out the logical issues of the
tenets underlying Progressive Judaism. I knew that to
be of any value to my community I must try to shape
my life in accordance with my idea of God. There was
endless soul searching to be accomplished, and actual
work to do, to render Judaism an ever-broadening influence
for the generations yet to come. I was conscious of
my great weakness and my many limitations, but I felt
I could give my aii to the cause. Nothing less was
acceptable; nothing more was required. The services
and meetings and discussions gave opportunity for quiet
meditation, and for the clarification of our thoughts. I
rejoiced in these opportunities with eagerness.

After ten years of preparatory work, we came to the
moment when a congregation must be formed and a
leader appointed who would give all his time to our
work and organise us into a definite congregation.
During these years there had been moments of the
greatest elation when our halls were crammed with eager
worshippers, and when men and women who had been
estranged from Judaism most of their lives told us that
we were showing them the way back to Judaism through
a new presentment of the old truths. There were times
of overwhelming disappointment, when interest flagged
and our numbers dwindled, when we sat round the
Committee table conscious that our material support had
almost vanished, and that the interest in our work,
instead of spreading, seemed to be diminishing before
our eyes. For family reasons, or on account of spiritual
lethargy, some of our enthusiasts dropped off. Cruel
things were said about us by the pseudo-Orthodox, whose
official conscience made them fear our religion lest it
should overwhelm their own flimsy edifice. One or two
of our leaders — brilliant scholars and teachers — were
threatened with the loss of their livelihood if they had
any intercourse with us. But the gloomy moments
passed as our great hopes advanced. As we realised more
and more the advance that Liberal Judaism was making
in America, Germany and France, we in England felt
that we were being challenged to give stronger expression
to our faith and to show that we too had a contribution
to make to the faith of humanity. So we decided to found
our Synagogue and gave the call to Rabbi Israel Mattuck
to come to us as our leader. The life of the Synagogue
was henceforth directed by our leader, and we all felt
the influence of his inspiring personality. Gradually, as
offshoots of the Liberal Jewish Synagogue, small congregations
were founded in different parts of London and in
the provinces.

Even the mother Synagogue at Hill Street was small
and unpretentious, and we were able, especially with
Dr.\ Montefiore as our President, to experience the
advantage of friendliness in our congregational life.
Dr.\ Montefiore, with Dr.\ Mattuck, was helped to
make each one of us realise the very big purpose of our
congregational life, and also the way of simplicity and
sincerity by which our great aims might be secured.
From Hill Street we grew until we were able to establish
in St. John’s Wood Road the largest Anglo-Jewish
Synagogue in the Community : our organisation expanded
and some great attainments became possible. The
intimacy of our early congregational life was in a large
measure lost, but a big price has always to be paid for
big opportunity. It can easily be understood that our
great Synagogue, with fifteen hundred members, and a
large number of activities, could exert influence in the
community through sheer numbers. To this was added
the personal influence of our own Minister, and those who
came to share his pulpit with him. Dr.\ Montefiore, as a
spiritual giant, was revered and loved even by those who
could claim only the very slightest personal acquaintance,
and so we went from strength to strength.

For myself I suffered greatly, even as the greatest
hopes of my life were being realised, and in this book I
am recording my personal experiences, believing that,
unimportant as they are in the history of a great movement,
they show how that movement can give direction
to the individual human spirit.

My father was deeply grieved when we founded our
Synagogue. He had been quite sympathetic to our early
Jewish Religious Union activities. He did not, as I have
said, object to our supplementing statutory services and
holding religious meetings in hotel rooms, and attracting
young people who were not yet under the influence of a
Synagogue. The whole outlook, however, was changed
when we organised our Synagogue with its own Minister
and burial ground, the two essentials of Jewish congregational
life. My dear father was distressed at the thought
of a break in the Anglo-Jewish community. He loved
uniformity. He did not argue with me; he just let it be
known, chiefly through my mother, that he was deeply
pained, and that he would like me to leave my colleagues
and remain faithful to the old traditions. I could not do
that, because I felt that by doing so I should betray the
God-given truth which had been shown me. I could not
desert my colleagues, for we were a small company in a
great army of seekers pledged to work for the faith in
which we believed. I was, however, devoted to my
father and suffered acutely from the silence which grew
up between us. Outwardly, there was no rupture between
us, but our talk was on the surface. I remember one
evening when my mother and sisters were in the country,
and we went to a reception together. On our way home
the silence seemed to give way. We did not speak, but
for a moment nothing seemed to count but the affection
we had for each other. Isms seemed to fade away, and
I still remember the warmth of his embrace as we stepped
into the house and his “Good-night, Lilchen.” But
the silence crept again into our lives. For two years
before his death I cried “inside.” It was only shortly
after his death in 1911 that I saw him in a dream, and he
gave me a document. I am not psychical and have not had
another experience of this kind before or since. With the
presentation of that document I felt a revival of trust,
and with it an absolute certainty that my father knew I
was carrying on \textsl{his} work in preserving the inheritance
he had given me, and reviving and even increasing its
value.
