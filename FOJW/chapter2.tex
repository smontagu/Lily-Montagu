\chapter{Adolescence and Girlhood}

I attribute much of the change in my religious
outlook to the teaching of the Rev. Simeon Singer,
Minister of the New West End Synagogue, who, in his
Sunday morning lessons, given in our home to my sister
and me, first made me realise the need for thinking about
religion. His gift of sympathy made it possible for me
to ask the most elementary questions without fear of
being snubbed or misunderstood.

I had known nothing of the Hebrew prophets before
I began to study with Mr.\ Singer, but he made me
receptive to their influence. We read about them
together and discussed their teaching from every angle.
Moreover, we talked about the ideals of Judaism as
translated into social service of all kinds, and I was soon
immensely enthusiastic about identifying myself with
every form of human amelioration. Having been taken
away trom regular school life at the age of fifteen, I was
allowed by my indulgent parents to choose very much
my own course of study. Tutors were given me and I
read a great many books on social philosophy and all
sorts of papers and pamphlets. Between the ages of
fifteen and nineteen, I collected any amount of
intellectual food and suffered acutely from mental indigestion.
It was the best reading period I have ever had in my life.

I always felt deeply moved by any effort at social
amelioration, especially when based on a definitely
religious foundation. My friendship with Margaret
Gladstone, afterwards Mrs.\ J.\,Ramsay Macdonald, was
a formative influence in my life. She put me in touch
with her industrial interests and work in one of the
poorest districts of London, i.e. North Kensington, and
her work was on a spiritual basis.

The effect of Buckle’s \textsl{History of Civilisation},
combined with \textsl{Robert Ellesmere}, rather bowled me over
for a time. I remember being terribly shocked by reading
in Buckle that it could be computed how many
people were likely to make the same mistakes
every year, how many would stick on stamps the
wrong way up, and how many babies would be
born, how many women die in childbirth. My
world, my beautiful, human world, lighted up by the
flame of God in its midst, was then only a thing evolved
in a mathematician’s brain, and moving mechanically
from one stage to the next. I was depressed by the
cruelty of men, by their intolerance and silly prejudices,
as shown in \textsl{Robert Ellesmere}. Always nervous and
worried by imaginary fears, I began to lose my sense of
security, and my normal belief that life was full of
glorious possibility seemed to be disappearing.

My mother took me to a nerve specialist who was a
kindly old gentleman, but he treated me as if I were very
young and offended me greatly by asking me to name my
favourite pudding and suggesting that it should be given
me to cheer me up. At the moment, I was walking on
stilts in a highly rarified atmosphere and puddings were
not interesting me in the least. However, I was advised
to tour England and interest myself in new experiences.
With a companion, I went from place to place, too
restless to remain anywhere for more than a day or two,
but gradually learning to enjoy myself.

Before I started on my travels, I had a memorable
talk with my dear friend and teacher, Mr.\ Singer, who,
I remember, persuaded me that my miseries were an
introduction into the “weltschmerz” which everybody
had sooner or later to experience, and that the acute
form would disappear as surely as would a little lump
which at the time of the interview was disfiguring my
hand.

My parents, in their concern about me, came to
Clifton to meet me after I had wandered about for three
weeks, and were satisfied with the improvement in my
health. Mr.\ Singer had given me a little prayer which
helped me. I still remember some of the words. “Teach
me to pray. Show me what I need. Thou hast freed my
life from grinding care. Make me grateful for the
opportunities given to me to serve Thee with my whole
heart.”

I was once more taken to the benevolent old
doctor, who assured me that I was wonderfully
better, but must not be surprised or discouraged
if I got bad again before long, as I had been too ill
to be expected to get well so quickly. My spirit of
resistance was at once roused, and I promised myself
that I \textsl{should} surprise everybody and not get ill in the
same way again. Then Sir William Jenner did me one
very great service by suggesting that, as I had so many
thoughts buzzing in my head, it would be a good idea to
start writing stories. I started immediately, and covered
reams of paper, and found great happiness in so doing.

It was the time of great changes in the lives of girls
in the so-called leisured homes. Their education was too
good to allow of their being any longer content with the
small home duties which in another generation satisfied
unmarried girls. They felt the need to justify their
existence by some form of useful effort. My parents led
strong, purposeful lives, and were not opposed to our
having interests outside the home. They applauded our
desire to assist others less fortunately placed than
ourselves, and, through their close concern with the East End
of London, I was constantly visiting there.

I was deeply shocked by the inequalities which
prevailed in large cities, the terrible injustice which
allowed me to have such an easy, happy, protected
girlhood while there was, in some districts, a monotony
of misery. But I felt convinced that God did not desire
such injustice to continue. My faith in His righteousness
was never affected, but I was worried by the apparent
inability of God to stem the tide of injustice. I was
convinced that man, with God’s help, could set things
right if he wished to, but how was he to be made to
realise his obligations? My first attempts at social
service, when I was about seventeen years old, were
divided between teaching girls history and literature
and “happy evenings.” These girls came from an
elementary school and were very intelligent and keen
on learning, but there had been no time for so-called
“extra subjects” — those subjects, which seemed to me
to make the grind of learning worth while. They had to
think of earning their living, although they were several
years younger than I. Here was a small injustice which
I could in a very restricted degree put right. I could
share my advantages with my pupils, and this work,
which gave me great pleasure and started friendships
which lasted very many years, I shared with my sister
Marian, who undertook the subjects with which I was
incapable of dealing.

The “happy evenings” brought me in contact with
young people who in most cases undertook social service
with a big S, in a very light spirit and without worrying
themselves in any degree at all about underlying
problems. They enjoyed going to the East End together,
and making friends with one another on their journeys
to and from the school where they tried to give the
children as good a time as possible. I was a little
worried by the amount of fun all the grown-ups seemed
to get out of these “happy evenings,” and was not quite
sure if the children really enjoyed themselves as much
as they should. The noise and excitement were terrific
in the large school hall. The children were organised all
the time into groups and sets for noisy games. The
entertainments were always exciting, but left the
children’s power of imagination quite unaffected. I
saw that the rich came \textsl{down} to amuse the poor, and did
what they could for the children as a mass, and they went
away at the end of the entertainment with the children's
applause ringing in their ears. The children accompanied
us to the station. Some even clung with their grimy
fingers to our hands and moved round our feet. We
laughed about them afterwards, and were glad they seemed
disinclined to let us go. They had queer ways, but they
were certainly, as a mass, very affectionate. What more
was needed? I felt vaguely that much more was needed
to make the recreation real. The children themselves
had so much to give; indeed, they were \textsl{individuals}, and
it was a shame simply to regard them as a mass to whom
we could be kind for an hour or two and then forget altogether.

I was quite vague in my reflections although they
made me uncomfortable. I am afraid I was regarded as
somewhat priggish and superior by my contemporaries.
In truth, I was very shy and self-conscious, and had the
utmost respect for the social qualities which my friends
possessed and which I was entirely without. I had not
the slightest feeling of superiority—perhaps rather the
opposite—and I felt vaguely culpable when my
companions said to me, “We can’t talk about that before you.
\textsl{You} would be shocked.” It seemed to me that being
shocked was a highly reprehensible feeling, and my head
just whirled with questions and bewilderment. Why
were things so wrong? Surely I ought to be able to put
something right. Why was I so entirely incapable of
doing so? Why did not people bother more? How could
I have so many gifts and pleasures and often feel so
thoroughly happy while there was so much misery among
people quite as young as I? Was I not living a very
wicked life? Was I never going to \textsl{do} anything?

\begin{tp}{128}
The first opportunity I had for taking real initiative
was when I started children’s services in the vestry room
of the New West End Synagogue, when I was between
seventeen and eighteen years of age. The organisation
was very simple at the start. I knew I had the encouragement
and sympathy of Mr.\ Singer, our Minister, but
when I began he was away on a holiday or on sick leave.
So I wrote to the Committee of Management and told them
how utterly boring the long Hebrew services were to the
children of the Congregation, and how entirely
uninfluenced they were by the opportunities for worship.
If they prayed at all, I said, it was that the service should
come to a speedy end. I asked, and quickly received,
permission to hold special services either before or after
the regular service. Indeed, only two conditions were
imposed on me. The time of the service must not clash
with the accredited services, and I must submit my order
of service to the Minister. I circularised the Congregation,
and elicited from them that they would prefer
the service to be held from 12-15 to 1 o'clock
rather than before 10 o’clock. I obtained the
cooperation of Miss F. Lyons and, as her successor
Miss T. Davis, who agreed to be responsible for
the musical portions of the liturgy, and to. lead
the children in singing. Together we produced the
children’s hymn and prayer book, with the support and
assistance of Mr.\ Singer as well as of the Committee of the
Berkeley Street Synagogue. Then the services began
and for ten years Sabbath by Sabbath I led them. The
attendances averaged about fifty, but often greatly
exceeded that number. My mother and my sister Marian
were deeply interested in the children’s services and
never failed to attend. My mother, of course, exaggerated
my success as a children’s preacher, and was never
tired of speaking about my small endeavours with great
enthusiasm. It rather worried her that several mothers
attended, whose sense of duty should really have sent
them to the regular Sabbath service, but they seemed to
prefer the short, easy service, and, unlike herself, never
went to the Synagogue. She had a premonition that the
popularity of the English service might later lead to
more serious dissent in the community, because she saw
around her people who grumbled about the length of the
regular services and their difficulty. She herself attended
the children’s services purely out of interest in the leader,
as an extra Sabbath experience, and in no way desired
as a substitute for the “proper” service.
\end{tp}

I was immensely interested in my part as leader of
the services, and spent a great deal of time preparing my
little talks. We aimed at awakening a spirit of worship;
in getting the children to join in the services and feel
that it was their own. I was delighted when one little
boy demanded that his rather indifferent parents should
let him learn Hebrew, for, although the service was
almost entirely in English, he missed a little bit by not
being able to join in the Shema. I learned to lead my
small Congregation in extempore prayers and found that
these came to me fairly easily. It was always a joy to
me to find that the service seemed complete in the
allotted time, and some machinery in my mind always
gave me the right time without reference to a clock. I
discovered without any attempts at proper reasoning
that every member of a congregation should give
\textsl{thought} to its interests. So I tried sometimes to ask the
children to write me answers to questions meant to
lead them to share my responsibility. For example, in
appealing for contributions for children’s country
holidays, I did not appeal for contributions in bulk, but
rather I explained the circumstances of several real
children and asked my congregants to select the one
whom they most wanted to help. I have still some
charming letters from children — now middle-aged men
and women — telling me which case they wished to
support.

I shall always be grateful for my experience as
leader of the children’s congregation. It helped me to
clarify my thoughts on the Bible and Judaism, and it
gave me some indication of what were to me the essentials
of public worship. I discovered that the leader must
actually pray with the Congregation as one of the group.
She must not be anywhere outside them. Indeed, she
only addresses them when she is actually preaching;
otherwise, she must invite her congregation to seek God
with her, realising the dignity and the difficulty of the
search, never speaking for them, for prayer to be effective
must be wrung from the inmost depths of the human soul.
The congregation must give live testimony throughout
the time of worship. They must never be passive, or
take the character of an audience. They must give of
themselves, their \textsl{whole} selves, in worship.

The cleavage with the Synagogue which forced my
resignation as leader of the children’s services belongs
to another part of my story. Now I would dwell only
on the positive side of my religious experiences, which
were interwoven in my work. These experiences were
enriched by contact with the new friends whom I met
in Club-work when I was about nineteen years old.
These influences helped to shape my life and, about the
same time, I had the opportunity of reading some of
Mr.\ Montefiore’s works and coming under his influence.
