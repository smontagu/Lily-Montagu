\documentclass[14pt, article, extrafontsizes, twopage, a4paper]{memoir}

\settypeblocksize{24cm}{17cm}{*}
\setlrmargins{2cm}{*}{*}
\setulmargins{2cm}{*}{*}
\setheadfoot{\onelineskip}{\onelineskip}
%\setlength{\topskip}{1.6\topskip}
\checkandfixthelayout
%\sloppybottom
\fixpdflayout

\parskip 0.25em
\parindent 1em

\usepackage{relsize}
\usepackage[all]{nowidow}
\usepackage[dvipsnames]{xcolor}
\usepackage{bidihl}
\usepackage{graphicx}
\usepackage{tikz}
\usepackage{caption}

\usepackage{polyglossia}
\setmainlanguage{hebrew}
\setmainfont{David CLM}
\setotherlanguage{english}
\newfontfamily\englishfont[Script=Latin, Scale=0.9]{Times New Roman}

\setlength{\footmarkwidth}{1em}
\setlength{\footmarksep}{0em}
\footmarkstyle{#1\hfill}
\rightfootnoterule

\newcommand{\attr}[1]{
{\raggedright\smaller#1}
}
\newcommand{\hlc}[2][1,1,0]{{\definecolor{bidihlcolor}{rgb}{#1}\bidihl{#2}}}
\newcommand{\todo}[1]{\hlc{#1}}

\begin{document}
{
  \centering
  \LARGE ליליאן מונטגיו (1873–1963)

  \Large דף לימוד לבבלי ירושלמי

}

\chapter{קצת על חייה}

ליליאן (לילי) מונטגיו נולדה ב־22.12.1873 למשפחת יהודית בריטית אמידה. אביה, סמואל מונטגיו, היה בנקאי שפעל רבות למען הקהילה היהודית, במיוחד עבור המהגרים שהגיעו מרוסיה ומזרח אירופה במספרים גדולים לקראת סוף המאה ה־19. לימים הוא היה אחד היהודים הראשונים שנבחר כחבר בפרלמנט הבריטי.

כל חייה פעלה לילי למען הקהילה, הן היהודית והן הכללית. היא עבדה כעובדת סוציאלית והקימה חבורה (\textenglish{Club}) שמטרתה הייתה להעשיר את החיים של נערות יהודיות ממעמד הפועלים העשרה אינטלקואלית, חברתית ודתית. היא הייתה בין הפעילים למען זכות ההצבעה של נשים בתחילת המאה ה־20.

ב־1902 היא הייתה בין המקימים של ה־”איחוד הדתי היהודי“ (\textenglish{Jewish Religious Union}). בתחילה הייתה מטרתו לערוך תפילות בשבת אחרי הצהרים  כתוספת לתפילות הרגילות של בתי הכנסת. לימים התפתח האיחוד לתנועת היהדות הליברלית. ב־1926 היא ייסדה את האיגוד העולמי ליהדות פרוגרסיבית הקיים עד היום כארגון גג לקהילות ליברליות ורפורמיות בכל העולם.

הספרים הרבים שכתבה לילי כוללים רומנים, ספרים על יהדות, איגרת שבועית לחברי החבורה שלה, וגם ספרים של תפילות לצרכי יהודים בני ימינו.

היא מעולם לא התחתנה, ורוב חייה חייה עם אחותה מריאן. לילי נפטרה ב־22.1.1963 בגיל 89.

{
  \centering
\includegraphics[width=4cm]{lilyolder.png}\\
\textsmaller{לילי מונטגיו}

}


\chapter{לילי מספרת על תחילת דרכה}

הילדות שלי הייתה מאוד שמחה. הייתי אחת ממשפחה של עשרה ילדים, שש בנות וארבעה בנים. אני הגעתי באמצע המשפחה, עם ארבע אחיות ואח אחד גדולים ממני, ואחות אחת ושלושה אחים צעירים.

הורינו העניקו לנו, מדוגמא אישית ולא דרך ציווי, את התובנה שהיהדות האורתודוקסית היא ירושה גדולה ומופלאה, שיש לקיים אותה ביראת כבוד מרבית, ואשר תתקיים לכל עת. למרות אדיקותם, היהדות, לדעת הוריי, לא הייתה קשורה ישירות לבעיות היומיום. יש לקיים את המצוות שלפי דעתם היהדות מושתת עליהן במלוא הנוקשות, והציות היא מעשה דתי שאין למעלה ממנו.

אני זוכרת את השאלות הראשונות שעלו לי לגבי תועלתן של מצוות אם רודפים אחריהן כמטרות בפני עצמן לחוויות הקשורות ... לטקס ביום הכיפורים. ... היו מלאות בחזרות. אנשים הגיעו בהמוניהם, אבל הם לא התעניינו, כנראה, בתפילות הארוכות. הם חשבו שנכון להתחיל מוקדם ולהמשיך לזמן שנקבע, אבל הם באמת לא נראו מושפעים מתוכן התפילות. הם התרווחו בחזרה במקומותיהם בבית הכנסת, נכנסו ויצאו, פטפטו וצחקו על סף בית הכנסת. לקראת הנעילה, נראה שכולם חזרו. היו מילים מאוד מרשימות. התרגשתי מההכרזה המסיימת של האמונה באל אחד, והאמנתי שכל חברי המתפללים הושפעו באופן דומה.
אבל ברגע שנאמרה המילה האחרונה, הייתה דהירה החוצה ... יחידי סגולה נשארו לתפילת ערבית. אף אחד לא רצה אותה, אבל היה נכון וראוי לקרוא את תפילת ערבית באותה שעה, והחזן היה צריך להישאר ולקרוא אותה... הקריאה עברה במהירות מדהימה, וגם אנחנו רצנו הביתה. ואז הכל התחיל כרגיל. זה היה החלק שעורר בי תמיהה וגרם לי לחשוב ... איך זה יכול להיגמר בצורה כל כך פתאומית? התקרבתי לגיל ההתבגרות והתחלתי לשאול שאלות.

ההזדמנות הראשונה שהייתה לי ליוזמה אמיתית הייתה כשהתחלתי תפילות ילדים בחדר צדדי של בית הכנסת ניו ווסט אנד, כשהייתי בת שבע עשרה־שמונה עשרה. הארגון היה מאוד פשוט בהתחלה. ידעתי שיש לי את העידוד והאהדה של רב הקהילה, אבל כשהתחלתי הוא היה בחופשה או חולה. כתבתי לוועד המנהל וסיפרתי להם כמה משעממות היו התפילות הארוכות בעברית בשביל ילדי הקהילה... אם הם התפללו בכלל, אמרתי, זה היה שהתפילה תסתיים מהר. ביקשתי, וקיבלתי מהר, אישור לקיים תפילות מיוחדות לפני או אחרי התפילה הרגילה.

האהדה העמוקה של אבי אלי, וההנאה שלו מההישגים הקטנים שלי, גרמו לו להיות מרוצה למדי מהצלחתי. נוסח התפילה שהייתה ברובה באנגלית ... נערכה לחלוטין על ידי המתפללים. שיניתי את התפילה משבוע לשבוע, ושיחות לא רשמיות במקום דרשות התקבלו באהדה. ההצלחה של התפילות הפשוטות האלה עודדה אותי מאוחר יותר לנסות משהו דומה בשביל מבוגרים... בדרך זו, אבי חשב שאוכל לעזור לכמה מהאחים החלשים ולהכין אותם לתפילה אמיתית. רק כשהייתי בין מנהיגיה של תנועה \todo{פלגנית} והכרזתי על אמונתי שהתנ״ך הוא בחלקו אנושי ולא ספר אלוהי לחלוטין, גרמתי לו כאב אמיתי. בשנה שנתיים האחרונות לפני מותו, הרגיש אבי שה”לילצ׳ן“ שלו (כינוי החיבה המיוחדת שלו עבורי) מופרדת ממנו בחומה של אי הסכמה שאפילו אהבה גדולה לא הצליחה לשבור. שנינו ידענו שהאהבה הייתה שם כל הזמן, והיום אני מרגיש שהאהדה המבינה שלו הוחזרה אלי. אולי הוא יודע שכל עבודתי, ובמיוחד מפעל החבורה שלי, נועדה לשמור על היהדות כאמונה חיה בקרב הצעירים שלנו.

האמונה של אמי הייתה פשוטה להפליא... היא האמינה באמונה שלמה באבהותו של אלוהים, ושהמצוות נקבעו על ידו לטובת ילדיו, ושיש לציית לו... התעכבתי על ההבנה האוהבת של אמי ... כי במהלך ארבעים שנות חיי מספר עצום של נערות ונשים קראו לי ”אמא“. אני מרגישה שאני חייבת הרבה מכוחי לקבל את אמונם של ”ילדיי“ בזכות הדוגמא שהייתה לי של אמא שלאורך כל חייה לא הייתה לה הנאה יותר גדולה מזה של שיתוף בתחומי העניין של ילדיה, ושאמונתה באלוהים באה לידי ביטוי בכל פרט ופרט של חייה.

\attr{מתוך \textbf{החבורה שלי ואני}, 1954, ו\textbf{אמונתה של יהודיה}, 1943}

{
  \centering
\vspace*{.5\baselineskip}
\includegraphics[width=4cm]{lily19.png}\\
\textsmaller{לילי, בת 19}

}

\chapter{התרומה הרוחנית של נשים באשר הן נשים}

אני מנסה לדון איתכם היום בנושא
שהוא כרוך בקושי מיוחד. קודם כול,
סביר מאוד להניח שאתן כפמיניסטיות לא תקבלו
את ההנחות הראשונות שלי. הרעיון המרכזי שלי מבוסס על התפיסה
שלמרות שעל הנשים לשתף פעולה עם גברים בכל
הפעילויות הפתוחות בפני בני אדם, ולמרות שהן לא צריכות
להכיר באף מגבלה חוץ ממגבלות פיזיות, ולמרות שמגדר
לא יכול לפסול אף אחת מלעשות את מה שהיא
מרגישה את עצמה מתאימה לו, בכל זאת לנשים יש כישורים מסוימים
שהם שונים משל הגברים. אמנם,
גם אני פמיניסטית ומאמינה בשוויון מוחלט בין
גברים ונשים בתחום החברתי, הפוליטי, הכלכלי והדתי,
אבל אני חושבת שהאנושות מתעשרת
מהגיוון בין שני המינים. מכאן נובע שנשים חייבות
לפתח את הכישורים המיוחדים שלהן, ושהן חייבות לתרום
לאוצר הרוחני של העולם תרומה שלימה
ככל האפשר, אבל אותה תרומה חייבת להיות בעלת אופי משלה,
ולא להיות חיקוי או העתק של תרומת הגברים.

לגברים ונשים יש יכולת יצירתית משותפת, אבל
גברים מסוגלים לפעול בצורה אובייקטיבית יותר מנשים. הרי
דוגמא אחת פשוטה. אבא ואימא רוצים שהילד הקטן שלהם
בן 4 או הילדה הקטנה בת 4 ילמדו להתפלל. הם רוצים
לעורר את היכולת לתפילה. אבא לוקח את הילד
לבית הכנסת בזמן שאין שם תפילה, נושא
אותו מסביב ומראה לו את התכונות השונות, כולל המושב האישי שלו,
וארון הקודש, ומסביר שחמישה חומשי תורה
כלולים בספר הנמצא בארון הקודש.
הוא מעביר מידע ומעניין את הילד. השבת
מגיעה והילד או הילדה מתלבשים במיטב בגדיהם,
מלווים את אבא לבית הכנסת, ומחזיקים לו את הסידור.
הילד יושב בינו לבין  אימא.
מדי פעם אַבָּא מראה לו את המקום בסידור, והוא מפסיק
להתנועע לרגע. הוא שמח ומוכן לחזור
על הניסוי בשבתות הבאות. לאט לאט, לאחר
זמן רב, האווירה של בית הכנסת עושה רושם
על הילד והוא מרגיש את השאיפה להתפלל.
מימוש האפשרות הזו יהיה תלוי במידה מרובה
על תגובתו של האבא עצמו לתפילה
ובמה הוא אומר עליה כשהוא חוזר הביתה.

לאימא יש שיטה אחרת. כשהיא משכיבה את הילדים היא מסוגלת לעשות
הרבה. האלמנט של הודיה הוא
הכנה טובה לתפילה. היא זוכרת פטריה יפה
שהיא וג׳וני התפלאו עליה בזמן שטיילו.
כוח הדמיון שלה חזק. ''ג׳וני`` היא
אומרת, ''אתה זוכר את טיול שלנו היום ואת הפטריה ההיא
שראינו, עם פסי הצבע המקסימים, האדום והצהוב,
והחתיכות הקטנות של ירוק וחום, ואיך שאמרנו
הלוואי שיכולנו למצוא פטריה שאפשר לאכול, אבל אז אמרנו
שהיא לא תהיה יפה באותה מידה? האם נודה לאל
על זה שהוא ברא את הפטרייה ההיא?“ ''כן, נודה,  אימא`` ''תודה לך
אלוהים שבראת את הפטריה המקסימה ההיא, עם האדום והצהוב,
והחתיכות של חום וירוק``  הוסיף ג׳וני, ''ובפעם
הבאה בבקשה תהפוך אותה לפטריה שנוכל לאכול.`` ''אמן``
אומרת  אימא. בלילה אחר  אימא וג׳וני
מכינים רשימה של האנשים ששניהם אוהבים ומבקשים מאלוהים
לברך אותם. הם מחברים יחד את התפילות ואלה הן
התפילות האישיות שלהם.

אישה יוצרת מעניקה את חוויה רוחנית שלה של
כאב ושמחה לתפיסות המתקדמות של היהדות.
גבר מנתח ומסנן ושוקל בזמן שהוא
עולה על הר האלוהים ביחד איתה. שמעתי שאומרים שגבר
מטפס צעד אחר צעד עד שהוא מגיע לרמה שהוא מסוגל
להגיע. דרכו הייתה בטוחה אך איטית למדי. הוא מסתכל
מסביב ורואה אישה לצידו. הוא לא ראה אותה
בזמן שהוא טיפס כי היא זינקה ממדף אל
מדף ולקחה סיכונים רבים.

בדיונים דתיים על מקור הסמכות
ביהדות מתקדמת, תמצא שלגברים יותר אכפת
מסמכות חיצונית מלנשים. גבר אומר: ”יהיה לנו
תוהו ובוהו אלא אם כן בעלי הידע והניסיון מתכנסים
יחד ומחליטים על טכסים ספציפיים. האם תפילה ביום ראשון
מובילה לחוסר נאמנות? האם כדאי שתהיה יותר עברית
בליטורגיה שלנו? האם סיוע החזן חיוני או
אפילו רצוי?“ ”טוב, אני לא רואה שזה משנה מה
האנשים הגדולים חושבים“ אומרת האישה. ”אני יודעת שאני לא יכולה להביא
את הקטנים שלי לבית כנסת בכל יום חוץ מיום ראשון. עִברִית
אולי מתאימה לאנשים מסוימים, אבל היא לא מועילה
לאלה שלא מבינים אותה. אם אני רוצה את השירה המשובחת ביותר,
אני הולכת לאופרה. כשאני בבית כנסת, אני רוצה
לשיר ולהצטרף, ואני יודעת שהילדים שלי גם רוצים. אם הם
יכולים לשיר, עדרבא, ואם הם לא יכולים, שאחרים
ישירו חזק יותר ויכסו את קולות ילדיי, אבל גם הם
שרים.“ יש אלמנט של פרקטיות בכל
זה, אולי קצת פחות תחושת אחריות, שאיפה
לנחת רוח של הצעירים שלה.

לפי דעתי, עלינו הנשים לגשת לשאלת השלום הבינלאומי, מזווית די שונה מזו
שגברים מאמצים. ראשית, אני חושבת שאנחנו צריכות להעלות
את הבעיה מתחום הפוליטיקה אל זירת
הדת, או, כפי שהייתי מאוד מעדיפה, להפעיל
השפעה דתית חזקה על נושא הפוליטיקה. גברים
ונשים מכירים באותה מידה את האומללות ואת
חוסר התוחלת של מלחמה, ואת הייואש הבא בעקבותיה.
אבל נשים בוודאי מבינות בצורה מלאה יותר את ההשפעה
של מלחמה על חיי הבית. נראה לי שעליהן מוטלת האחריות
להתגבר על תחושת התבוסתנות ותסכול את העולם. נקודת המבט של גברים היא
מציאותית יותר מזו של נשים. החזון שלהם
חסום בשל קיומה של פצצת האטום ואיום
הטוטליטריות. הם נתנו כל כך הרבה למען
החירות, והם רואים את כל צורות העריצות פורחות
ומאיימות יותר ויותר. הם עייפים אבל
נחושים לא להפגין חולשה. לכן הם פונים להכנות למלחמה
כדרך היחידה להבטיח שלום. כאן נמצא
חלקה של האישה. עלינו להצהיר בכל הכוח
העומד לרשותנו שבגלל שהאלוהים קיים, ממשלת השלום
והצדקה חייבת לנצח בעולם. אם אנחנו מאמינות בצורה חד־משמעית
ברשעות המלחמה, עלינו להתרחק ממנה
ולמצוא דרכים אחרות ליישב את המחלוקות בינינו,
עם כל הקושי שבחיפוש. אם ניכשל עכשיו, הציוויליזציה שלנו
תכלה ועימה כל היקר \todo{בחיי הבית}.
רק במישור הדתי נוכל למצוא
דרך להחזיר את האמונה שלנו באדם ובעצמנו. אנחנו
יכולות לעזור כי אנחנו מחוץ לזירת הלחימה הממשית.
למרות שאנחנו ביחידת המלחמה, השיטות שלנו אינן
מוגבלות לכוח פיזי.

לפני סיום, אבקש מכן לשקול אם
אתן חושבות שאתן יכולות להתנגד לסחף הקיים המתרחק
\todo{מהַקְדָּשַׁת} חיי הבית. גברים עשויים להתאמץ אפילו יותר
מנשים לשמור על מראה חיצוני, אבל אתן יודעות טוב כמוני
שריקבון משתלט, והגיע הרגע
בשבילכן לקום ולהאיר. על הבמה,
ברומנים ובעיתונים, רעיון הבית \todo{המפולג}
מתקבל כבלתי נמנע. הילדים מוקרבים
לתחושה הכללית של הבלתי נמנע.
חיי בית שמחים, צנועים וטבעיים נחשבים
יוצאי דופן. נדר הנישואין התרפה.
טקס הנישואין דתי רק בצורתו,
ומשמעותו נשכחה. גברים חושבים
ששלב זה הוא בלתי נמנע, שהקוד החדש של המוסר, או
חוסר מוסר, עלול להתגבר, \todo{ושזהו לפי המגמה של הזמן}.
\todo{העסקים תופסים את תשומת הלב}. הישרדותם של החזקים
הוא צו השעה, ואי אפשר לקחת בחשבון אף שיקול אחר
בעוד שמאבק הקיום כה עז. שוב כאן
עלינו להבליט את אמונתנו, ולפני שנוכל לעשות זאת, עלינו \todo{להעריך} אותה,
כל אחת לעצמה. אם הבית והילדים
הם רכושנו היקר ביותר, אם שבח
הערך שלהם הוא יותר מסתם התבטאות, עלינו
למהר לעבודת הגאולה.

כן, חברים, בתחום החינוך הדתי, בבית,
בעבודה למען שלום וצניעות, עלינו לעבוד עם
קנאות חדשה ואמונה חדשה. \todo{התרומה שלנו} חיונית. את האומץ שֶׁלָנוּ
יש להוסיף לזה של הגברים שלנו. עלינו להחזיק
עבורם חלק מהביטחון והתקווה האבודים שלהם. נתנו
להם לאבד הרבה בשל האדישות וחוסר האקטיביות שלנו. עכשיו עלינו
לדעת שאנחנו עומדות לפני האלוהים וחייבות או למות
או להיות מוכנות לשמור את דברו. כל אחת חייבת לומר
בעצמה: ”הִנְנִי שְׁלָחֵנִי.“\footnote{ישעיהו ו ח}

\attr{נאום במרכז חינוך יהודי של שיקגו, 26.11.1948}

\chapter{על התפילה}

”הוֹי כׇּל צָמֵא לְכוּ לַמַּיִם.“
קריאה זו, הלקוחה מישעיהו פרק נ״ה, היא קריאה לתפילה. חלקכם שואלים: ”למה להתפלל? איזה תועלת יש בזה?“

אני חושבת שיש לנו צורך להתפלל. איננו שלמים ללא מגע עם אלוהים חיים - וזאת המשמעות של תפילה. אנו יצורים שנוצרו על ידי הרוח החיה של טוב, אמת, אהבה וצדק, שרוצות לחזור ולשאוב מהמקור שלנו אנרגיה מחודשת שבאמצעותה נמשיך את חיינו. אנו מתפללים, אם כן, ל”הגדלת“ כוחנו הרוחני. אנו זקוקים למזון ולפעילות גופנית לנפשותינו, בדיוק כמו שאנו זקוקים לאוכל והתעמלות לגופנו. אנו הוגים את חוק הצדקנות של אלוהים, והרצון להיות טוב יותר ולעשות טוב יותר ממלא את ליבנו. למה להתפלל? \textbf{דבר ראשון אנחנו מתפללים כדי שנוכל לחיות חיים מלאים יותר}.

כולנו מודעים לחטאים כלשהם בחיינו המרחיקים אותנו מן האלוהים. אנו מתפללים לכוח להתגבר על הרוע בתוכנו, ולהיות מאוחדים עם אלוהים. כל אחד מאיתנו אחראי באופן ישיר להתנהלות חייו. עלינו להרוס את העוול בעזרת המאמצים שלנו. התודעה שאלוהים הוא אמיתי -- שמשהו מרוחו נמצא בליבנו, באותו זמן שהיא, בשלמותה, מהווה כוח החיים העליון ביקום, אמונה זו נותנת לנו כוח להתגבר על החטא, כי היא מחזקת את רצוננו ומכוונת אותו לטוב. \textbf{דבר שני אנו מתפללים, אם כן, כדי שנתקדם ביושר}.

בזמן שאנו מתפללים, אנו מרגישים את עצמנו מאוחדים עם כל בן אדם השואף לטוב; חייו ורווחתו הם חלק מהחיים שלנו. אנו מבינים את צרכיו, כי אנו ממש חולקים אותם, ולכן \textbf{הברכה השלישית שאנו מגלים בתפילה היא אחדות האנושות}.

אבל היינו רוצים לשנות את העולם; היינו רוצים לראות כמה מהרעות והסבל, האכזריות והעוולות נמחקות. \textbf{אז אנחנו מתפללים להסרת הרוע}, למרות שהוא ממשיך להתקיים ולפעמים נראה שהוא מוסיף לגדול. אם כן מה נועיל בתפילתנו?

התועלת נמצאת בכיוון אחר. בתפילה אנו מכירים את הערך והכוח של האישיות האנושית. אנו מתפללים, ואפשרות ההישג נחשפת לנגד עינינו. אנחנו יכולים לבחור בטוב ולדחות את הרע. זו זכותנו האנושית. אותה פריבילגיה שייכת לגברים ולנשים שנמצאים בעמדות כוח ואחריות גדולות. לא היינו מוותרים על חירות האישיות האנושית. \textbf{כאשר אנו מתפללים אנו מדגימים את הקשר בין אמונה ומעשה}.

החירות היא חלק מהמורשת האנושית. התפילה חשפה את כוחו של האדם ליצור עולם טוב יותר באמצעות ציות ונאמנות לחוקי אלוהים. \textbf{מוטל עלינו} לעזור בהקמת מלכותו. אסור לנו לבקש מאלוהים לעשות זאת עבורנו, ובזה לוותר על עצמאותנו האנושית. הוא מציע לנו את הכוח לשתף איתו פעולה. קוּמִי אוֹרִי!\footnote{ישעיהו ס א}

אנו אוהבים להתפלל עבור היקרים לנו, ולפעמים אנו מתפללים והאסון שאנו רוצים למנוע מגיע בדיוק כאילו לא התפללנו כלל. מדוע אם כן, אנו שואלים, אלוהים אינו שומע? אני מאמינה שאלוהים \textbf {כן} שומע, וטוב לנו לחשוב על היקרים לנו כאשר אנו בוחנים את מציאותו של אלוהים בתפילה. הבה נחפש מתוך התגלותו דרכים להגביר את חוכמתנו ואת כוח האהבה שלנו. אולי נאבד חלק מהאנוכיות שלנו ומהיכולת שלנו להכאיב. אבל כשדברים אינם הולכים כשורה, כפי שאנו מאמינים, עבור אהובינו, עלינו לזכור את גבולות הראייה שלנו ושמה שנראה לנו רע עשוי בסופו של דבר להיות טוב. \textbf{באמצעות התפילה אנו לומדים לבטוח באהבת האל העליונה, ושהוא פועל רק באמצעות אהבה}.

החיים מתוקים לכולכם. למרות הקטעים העצובים והקודרים, יש לכם הכוח ללמוד ולאהוב. אתם יכולים לראות קצת יופי בעולם, אם כי בהבלחות מועטות ונדירות. לפעמים אפשר לראות את פלאי הטבע ולשמוע מוזיקה מפוארת. לרובכם יש מאחוריכם הביטחון של חיי הבית והאמון של אלה שאוהבים אתכם. אתם, בגלל שאתם צעירים, יכולים לחוות את הנאות הגוף והנפש. אתם מרגישים אסירי תודה לחיים. \textbf{בתפילה אתם יכולים להודות}.

האוכל להפציר בכם להתפלל כל יום -- להרגיש במודע בנוכחות האלוהים? אם ספקות תוקפים אתכם, התמודדו איתם והיאבקו איתם. בסופו של דבר, אני מאמינה, הם יחזקו את האמונה שלכם. אל תוותרו על התפילה כי היא קשה. למדו ליצור את האווירה הנכונה לתפילה, אווירת יראה וענווה. נקו את לבכם לפני שאתם מתפללים מאנוכיות וחוסר כנות, ואת מוחכם ממחשבות טמאות. אז השליכו את עצמכם אל ”זרועות עולם“\footnote{דברים לג כב} של האלוהים, ובעמקי לבבכם תשמעו את דברו. ”דַּבֵּר י״י כִּי שֹׁמֵעַ עַבְדֶּךָ.“\footnote{שמואל א ג ט} אני מתחננת בפניכם: אל תעכבו את התפילה עד שתהיו חלשים ועייפים מדי, בגלל חוסר תזונה רוחנית, להתפלל בכלל. \textbf{התפללו עכשיו}! מחר עלול להיות מאוחר מדי.

\attr{איגרת לחבורה מס׳ 5, מרץ 1939}

\chapter{על יהדות בחבורה}

היו לנו חילוקי דעות רבים בנושא מחללי שבת. [מיס הריס] לא חשבה שצריך לקבל אותם לחבורה. אני, בהיותי צעירה יותר, הבנתי טוב יותר את הלחץ הכלכלי של התקופה שמנע ממי שרצה לחיות חיים עצמאיים את האפשרות לשמור שבת על כל פרטיה ודקדוקיה. הושפעתי מהמסר של היהדות הליברלית, והאמנתי שאם נעבוד את אלוהינו שהוא האב של כל בני האדם, עלינו לתרגם את דברו לפי נסיבות החיים המשתנות. חשוב מעל הכול שנבקש מכל מי שנמצאים תחת השפעתנו לגלות את דבר אלוהים, ולנסות לחיות על פי דברו בחיי העבודה שלהם. אם הם חיו באמת לפי דברו, הם יכלו לעבוד אותו כל היום, בכל מה שהם עשו, ולא רק בתפילות שבת. יתר על כן, הם יכלו לקדש כל יום דרך תפילה, ויכלו להשתמש לפולחן \todo{באיזה חלק של השבת הרשמית שעמד לרשותם}. במובן זה, קראתי קריאה נמרצה לשמירת ערב שבת בבית. מבחינתי, שמירת השבת הייתה קשורה קשר הדוק לכמה מהחוויות האישיות הכי שמחות שלי בחיי הבית. לנו הילדים היו תפילות ערב שבת שקראנו עם אמא שלנו. ערכנו את ההתכנסויות המשפחתיות שלנו בערב שבת שהתחילו עם קידוש על \todo{יין האחווה ולחם המזון}, והחשוב מכל, כל ילד בתורו קיבל ברכה. שני ההורים בירכו כל ילד, הניחו את ידיהם על ראשו של הילד ואמרו את הברכה בעברית. המנהג הזה השאיר עליי רושם עמוק. בלילות שישי שמחנו להיות ביחד. כשכמה מאחיותיי ואחיי התחתנו, הם חזרו עם בעליהם או נשותיהם לחגוג את השבת בבית, ובהמשך הצטרפו אלינו ילדי המשפחה, והערב המשפחתי נעשה מענג יותר ויותר. חיינו היו שקועים כל כך בפעילויות מגוונות, עד שהשבת נתנה לנו הזדמנות ייחודית לשמר את להבת העניין האוהב בינינו. גם לאחר שהורינו היקרים הלכו לעולמם, נשאר המנהג של ערבי שבת בבית עם התכנסויות משפחתיות ותפילות משפחתיות והוא נשמר על ידי אחותנו הבכורה, גברת פרנקלין\footnote{הנריאטה פרנקלין (1866–1964), אחותה של לילי, היתה מחנכת, מייסדת בית ספר לחינוך ביתי ופעילה למען זכות ההצבעה לנשים.}, שנתנה לנו את ההזדמנות לקיים מפגשים אלו בביתה. צעירים מספרים לנו היום כי פעילויות חשובות מתרחשות לעתים קרובות בליל שישי, ואי אפשר לסרב להן. מבחינתי, מכיוון שהערכתי מאוד את חיי המשפחה המקודשים, מעולם לא הייתה בעיה. פשוט נשארנו בבית בלילות שישי, ואף פעם לא עלתה על לבנו שום אפשרות אחרת. החיים הסתדרו סביב המנהג הקבוע הזה כמו שמי היום זורמים סביב סלע הניצב בעוצמה שאי אפשר לכבוש אותה. הוא שם, ואף אחד לא יזיז אותו.

לפני הקמת הקהילה היהודית הליברלית של מערב מרכז לונדון, צעדו חלק מחברי החבורה (\textenglish{Club}) המסורים שלנו עד \todo{שכונת נוטינג היל} בימים נוראים כדי לאפשר לי לערוך תפילה בת ארבע שעות באולם ליד בית הכנסת שבו התפללתי במשך כמה שנים. בשבת אחר הצהריים, הלכתי עם אחותי האהובה לרחוב דין, וקיימנו את תפילות מנחה של שבת, בהשתתפות אנשים רבים. לפני שהצלחתי להתחיל את התפילות האלה הייתי צריכה להתגבר על קשיים גדולים בבית, בגלל המחשבה שזה עלול לעלות לי ביותר מדי מאמץ אחרי שבוע כבד של עבודה. אבל האמנתי באמונה חזקה אז, כמו שאני מאמינה היום, בכוח התפילה, והייתי משוכנעת שאי הגעה לתפילה נובע nשעמום שמעוררת התפילה המסורתית. שנה או שנתיים של חיי עבודה  במפעל כנראה מחקו את מעט הידע בעברית שרוב הבנות שלנו רכשו בילדותן. הן לא יכלו להבין את התפילה המסורתית והיא שיעממה אותן לגמרי. יתרה מכך, הן לא היו רגילות לקחת חלק כל שהוא בתפילה ולא מצאו בנוסח התפילה ובדבר תורה שום דבר שהיה קשור לחיי היום־יום. התפילות שלנו היו משהו אחר. הן היו באנגלית ושירה ביחד הוסיפה להן אור. נעשה שימוש רק בתפילות כאלה שהייתה להן משמעות עבור יהודים ויהודיות מודרניים בנסיבות חייהם בפועל. הדרשות עסקו בנושאים חיוניים. בסופו של דבר, נוסדה הקהילה היהודית הליברלית מערב מרכז, וגברים וגם נערות השתתפו וישבו ביחד, שלא כמו בבתי הכנסת האורתודוכסים  שבה הייתה ישיבה נפרדת. תפילות ילדים נערכו גם במשך כמה שנים ומשכו רבים. בהמשך הוזמנו הילדים לבוא עם הוריהם ונערכה שיחה מיוחדת בשבילם בחדר נפרד.

בשנים הראשונות נמנענו מלהשתמש בכלי נגינה מכיוון שחששנו להרחיק את המתפללים המעטים שהיו אורתודוכסים באמת ובתמים. להפתעתי, לא מצאתי התנגדות כשעשינו את השינוי. כפי שקורה לעתים קרובות, דתיים באמת מתנגדים רק לעתים רחוקות לרפורמה גם אם הם עצמם אינם רוצים אותה או מעריכים אותה. חברינו האורתודוכסים אמרו בהזדמנות זו: ”אולי המוזיקה עשויה להביא כמה אנשים שאחרת לא היו מגיעים לתפילה“. רק הפסאודו־אורתודוכסים מסבכים את נשמתם בדיקדוקי עניות, ובאים עלינו בטענות כאשר אנחנו מייצרים תפילה יפה ככל האפשר...

לאורך ההיסטוריה של החבורה שלנו, שיחות על דת, כמו גם מעשה התפילה ביחד, היו מאפיינים בחיי החבורה. תמיד התפללנו יחד בחופשות של החבורה, ובאמצעות ההתפעלות מהיופי של הטבע הובלנו את החברות לשבח את בורא העולם. הרגשנו את האחדות הנובעת מטקס קבלת שבת הנערכת באווירה משפחתית. קיימנו תפילות תחת כיפת השמים והרגשנו את האמת של מזמור מ״ב, שכן בתוך אווירת החופשה חווינו בקלות את הערגה שלנו לאלוהים ”כְּאַיָּל תַּעֲרֹג עַל אֲפִיקֵי מָיִם“. \todo{דורות} של חברות החבורה שבו הביתה מהחופשות שלהם באכסניית גרין ליידי, \todo{בעיר ליטלהאמפטון}, עם זיכרונות של ”שיחות מתחת לעץ“ שנערכו מדי ערב למי שהיה חשוב לבוא ולא להאזין לקונצרטים בטיילת. בדיונים אלה, כמו גם בפגישות שהובילו אנשים שונים בתוך החבורה, \todo{עודדו התשאול הכֵנה ביותר}. הייתה לנו הזדמנות להסיר אמונות טפלות מוזרות וללמד את היסודות של יהדות חיה. שוב ושוב נחרדנו מסולם הערכים השגוי בקרב הצעירות שלנו, ומהפסילות השפלות על בסיס מגדר שהבנות עוד האמינו בהן. בגלל הכבוד שרחשנו לחסידות אבותינו, חשבנו בזהירות רבה מאוד לפני שניסינו להחליש את נאמנותן הבלתי מעורערת של הצעירות שלנו למנהגים עתיקים. ידענו שהן שילמו מס שפתיים בלבד, אבל לא רצינו \todo{לכבות את הפשתה הכהה}. עם זאת, לא יכולנו לתמוך עוד במדיניות של סחף בלבד, כי הסחף הזה עלול להפוך בקלות לסחף אל החומרנות. אנו מאמינים שפעלנו לטובת היהדות כשסייענו לצעירים ולצעירות שלנו להשתמש במוחם בחיפוש אחר אלוהים.

בשנים האחרונות כתבתי מכתב חודשי בנושא דתי לחברי החבורה, והזמנתי אותם לשוחח איתי על הבעיות האישיות שלהם. אני מאמינה שהחיים הדתיים שלנו יתחזקו באמת כאשר נשכנע את הצעירים שלנו שמוטל עליהם החובה לגלות את האלוהים מחדש כל אחד ואחת בעצמו ובעצמה. האלוהים גדול דיו כדי להיכנס לכל נשמה, אבל עלינו להתכונן אליו ולהתרגש לקראת בואו.

אבותינו סיפרו לנו על חוויותיהם ומסרו לנו את החוויות הללו. אבל הדת שלנו מתה אם אנחנו עובדים רק בעבודה שבלב של אבותינו. עלינו להדליק את האורות שלנו. אנו מאמינים שהאמת של היהדות מכסה את כל החיים, אבל היא כוח מתחדש, כמו החיים עצמם, והופעה שלה לא יכולה להיות ללא שינוי אם היא תתאים לכל דור של מאמינים. דרך החבורה שלנו עלינו לנסות תמיד לעודד אמונה, ולהתעניין במושגים שלה ובהיסטוריה שלו. אבל, מעל לכל, עלינו להטמיע את רעיון האל עד כמה שהשכל המוגבל שלינו מאפשר. אלוהים חייב להיות נוכח מאוד עבורנו כדי שנוכל לחיות תחת הדרכתו, לעבוד עבורו ואיתו, ולסמוך על כך שהקרבה הזו היא לנצח. עם אמונה כזאת, גם כי נלך בגיא צלמות לא נירא רע.

\attr{מתוך: \textbf{החבורה שלי ואני}, 1954}


\pagebreak
\chapter{תפילה והודאה לרגל לידת בן}
\parskip 0.75em
\parindent 0em


אָהַבְתִּי כִּי־יִשְׁמַע יְיָ אֶת־קוֹלִי תַּחֲנוּנָי׃\\
כִּי־הִטָּה אׇזְנוֹ לִי וּבְיָמַי אֶקְרָא׃\\
אֲפָפוּנִי  חֶבְלֵי־מָוֶת וּמְצָרֵי שְׁאוֹל מְצָאוּנִי צָרָה וְיָגוֹן אֶמְצָא׃\\
וּבְשֵׁם־יְיָ אֶקְרָא אָנָּה יְיָ מַלְּטָה נַפְשִׁי׃\\
חַנּוּן יְיָ וְצַדִּיק וֵאלֹהֵינוּ מְרַחֵם׃\\
שֹׁמֵר פְּתָאיִם יְיָ דַּלֹּתִי וְלִי יְהוֹשִׁיעַ׃\\
שׁוּבִי נַפְשִׁי לִמְנוּחָיְכִי כִּי־יְיָ גָּמַל עָלָיְכִי׃\\
כִּי חִלַּצְתָּ נַפְשִׁי מִמָּוֶת אֶת־עֵינִי מִן־דִּמְעָה אֶת־רַגְלִי מִדֶּחִי׃
\attr{מתוך תהלים קטז}

בָּרוּךְ אַתָּה יְיָ אֱלֹהֵינוּ מֶלֶךְ הַעוֹלָם הַגּוֹמֵל חֲסָדִים טוֹבִים לבָנָיו׃

בָּרוּךְ אַתָּה יְיָ אֱלֹהֵינוּ מֶלֶךְ הַעוֹלָם שֶׁהֶחֱיָנוּ וְקִיְּמָנוּ וְהִגִּיעָנוּ לַזְּמַן הַזֶּה׃

הִנֵּה נַחֲלַת יְיָ בָּנִים שָׂכָר פְּרִי הַבָּטֶן׃
\attr{תהלים קכז ג}

אֵין־קָדוֹשׁ כַּייָ כִּי אֵין בִּלְתֶּךָ וְאֵין צוּר כֵּאלֹהֵינוּ׃
\attr{שמואל א ב ב}

\textsmaller{האם תאמר}\\
אֱלֹהַי,
אֲנִי פּוֹנָה אֵלֶיךָ וּמוֹדָה מִקֶּרֶב לִבִּי,
עַל הַתְּשׁוּעָה שֶׁעָשִׂיתָ לִי.
בְּעֶזְרָתְךָ עָבַרְתִּי תְּקוּפָה שֶׁל מִבְחָן קָשֶׁה,
דֶּרֶךְ כְּאֵב וְחוּלְשָׁה, וְעָמַדְתִּי אֵיתָנָה.
אַתָּה שׁוֹמְרֵנוּ.
הֶעֱשַׁרְתָּ אֶת חַיֵּינוּ בְּמַתָּנוֹת רַבּוֹת.
בַּעֲנָוָה רָאִינוּ אֶת טוּבְךָ בַּיְּלָדִים שֶׁנָּתַתָּ
וּמוֹדִים בְּכָל לִבֵּנוּ.
וַאֲנִי מְבַקֶּשֶׁת מִמְּךָ, אֱלֹהַי,
לַעֲזוֹר לִי וּלְאִישִׁי
לִהְיוֹת רְאוּיִים לַבְּרָכָה
שֶׁנָּתַתָּ לָנוּ בַּיֶּלֶד הַזֶּה,
וּלְהַבִּיעַ אֶת תּוֹדָתֵנוּ אֵלֶיךָ
בַּמַּאֲמָצִים בִּלְתִּי נִלְאִים לְהוֹבִיל אוֹתוֹ
בְּדֶרֶךְ שֶׁל יוֹשֶׁר וּקְדוּשָׁה.
לַמְּדֵנוּ לְהוֹבִיל אוֹתוֹ וּלְהוֹרוֹת לוֹ
כַּךְ שֶׁיִּגְדַּל לִהְיוֹת נֶאֱמָן לְדַת הַיְּהוּדִית 
וְחָבֵר רָאוּי בִּקְהִילַּת יִשְׂרָאֵל.

\textsmaller{הרב יאמר:}\\
אֱלֹהֵינוּ, אָבִינוּ,
אָנוּ מְבַקְּשִׁים אֶת אֲהָבָתְךָ הָאַבָּהִית לַיֶּלֶד הַזֶּה.
יְהִי רָצוֹן שֶׁיִּלְמַד עוֹד וָעוֹד,
כְכָל שֶׁהוּא יִהְיֶה מְסֻגָּל לָדַעַת,
שֶׁאַתָּה תָמִיד קָרוֹב אֵלָיו,
וְהוּא גָדֵל בְּאוֹר פָּנֶיךָ כְּבַשֶּׁמֶשׁ,
וְשֶׁיַּרְגִּישׁ אוֹתוֹ כָּל יְמֵי חַיָּיו.
פְּקַח אֶת עֵינָיו לִרְאוֹת אֶת הַיּוֹפִי וְהַפֶּלֶא שֶׁל הָעוֹלָם סְבִיבוֹ,
כְּדֵי שֶׁיִּהְיֶה בְּלִבּוֹ מַעְיָין נוֹבֵעַ וְרַעֲנָן.
זֹאת וָעוֹד,
תֵּן לוֹ לִלְמוֹד לֶאֱהוֹב אֶת הַטּוֹב בַּגְּבָרִים וּבַנָּשִׁים.
אֵין אָנוּ מְבַקְּשִׁים שֵׁיֵּחָֽסְכוּ מִמֶּנּוּ הַנִּסְיוֹנוֹת וְהַצָּרוֹת
שֶׁאֵין אָדָם בָּאָרֶץ חוֹפְשִׁי מֵהֶם.
אֲבָל אָנוּ מְבַקְּשִׁים בַּעֲנָוָה שֶׁיִּהְיֶה לוֹ כֹחַ שֶׁיִּגְדַּל עִם הַשָּׁנִים,
וְאוֹמֶץ לַעֲמוֹד לִפְנֵי כָל רוֹעַ וּלְהַשְׁקִיטוֹ.
יוֹתֵר מִכֹּל, תֵּן לוֹ לֵב אוֹהֵב כְּדֵי שֶׁיִּחְיֶה לַעֲשׂוֹת רְצוֹנְךָ בְּנֶאֱמָנוּת.
תָּבִיא אֶת הַיֶּלֶד שֶׁיּקָרֵא שְׁמוֹ \textsmaller{פְּלוֹנִי בֶּן פְּלוֹנִי}
לְהַגִּיעַ לַחֲסִידוּת וּלְיֶדַע,
וְלִהיוֹת בְּרָכָה,
וּבָרֵךְ אוֹתוֹ בְּחַיִּים אֲרוּכִּים.

{\centering\larger[2]
  פרשת שמע

 ברכת כהנים

}


\end{document}
