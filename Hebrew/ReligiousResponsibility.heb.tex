\documentclass[12pt, extrafontsizes, twopage, a5paper]{memoir}

\settypeblocksize{17cm}{11cm}{*}
\setlrmargins{0.8in}{*}{*}
\setulmargins{0.8in}{*}{*}
\setheadfoot{\onelineskip}{\onelineskip}
%\setlength{\topskip}{1.6\topskip}
\checkandfixthelayout
%\sloppybottom
\fixpdflayout

\usepackage[all]{nowidow}

\usepackage{polyglossia}
\setmainlanguage{hebrew}
\setmainfont{Frank Ruehl CLM}
\setotherlanguage{english}
\newfontfamily\englishfont[Scale=0.9]{Brill Roman}

\parskip 0.25em
\parindent 1em
\tightlists

\setlength{\footmarkwidth}{1em}
\setlength{\footmarksep}{0em}
\footmarkstyle{#1\hfill}
\rightfootnoterule

\begin{document}
{
  \centering\LARGE\bfseries אחריות דתית בחיים הציבוריים
  
  \large פגישה בין דתית: 28 יוני, 1950

  \Large לילי מונטגיו


  }

אני אסירת תודה לזכות לפנות אליכן, אחיותיי היהודיות והנוצריות. הנושא שלי הוא אחריות דתית של נשים בחיים הציבוריים, ואני מדברת כחברה בקהילה היהודית הליברלית. אני לא מייצג את הקבוצה האורתודוקסית בתוך הקהילה שלנו, למרות שאנו רוכשים כבוד גדול לאורתודוכסים, שכן הפרוגרסיביים הם מי שהשיטת הדתית שלהם משתלבת, אם לצטט את מנהיגנו ד"ר מאטוק,\footnote{ישראל מטוק (\textenglish{Israel Mattuck}), 1883--1954, היה רב והמנהיג של תנועת היהדות הליברלית בבריטניה ב־40 השנים
הראשונות שלה.}
עם המחשבות והאידיאלים של התרבות המערבית.

אנו מאמינים באמיתות נצחיות מסוימות שאנו מבקשים להחל על חיי המדינה. אלוהים, על ידי נביאיו ומוריו, גילה לנו קנה מידה מוחלט של צדק. דרך אמונתנו באחדות האל, אנו תופסים את הרעיון של אחוות האדם. לכל הגברים והנשים יש זכות לחיות, ולכן עלינו לדאוג שמספקים להם את ההזדמנות להשיג את צרכי החיים. אין לשלול מאיש או אישה, כילד של אלוהים, את הזכות לבית עם אוויר ואור; ובאותה מידה את האפשרות לפתח את חייו הגשמיים, האינטלקטואליים והרוחניים. לאור אמונתנו באל אחד, יש לכבד את הגוף, את הנפש את והנשמה כקדושים.

הייתה תקופה שבה אנשים לא פחות טובים מאיתנו חשבו שצריך לתת לילדים קטנים לעבוד במכרות ובמפעלים כדי להגדיל את רווחי התעשייה. אלוהים יודע במה אנו חוטאים היום למען מה שאנו תופסים כרווח הוגן בזמן ששפטסברי\footnote{אנטוני, לורד שפטסברי (\textenglish{Lord Shaftesbury}), 1801--1885, פעיל חברתי בהרבה תחומים, כולל הגבלת העסקת קטינים.}
אחד מתכונן להראות לנו את הפשעים שלנו ולקרוא לנו לשנות את דרכינו. אנו נישפט על ידי הדורות הבאים בדיוק כמו שאנו שופטים את קודמינו.

כיום, מכיוון שאנו מודעים לכך שילדים נבראו בצלם אלוהים, יש לנו צורך לכבד אותם כאישיות. הכרה זו חייבת לדרבן אותנו לנסות בהקדם האפשרי לפסול כיתות צפופות בבתי ספר, כך שנדע את היכולות ואת המגבלות של כל ילד. יתרה מכך, על מנת שילדינו יממשו את עצמם, חייבת להיות להם הזדמנות מדי פעם לראות שטחים נרחבים של ארץ ושל ים; אותו הים שילדה אחת פעם תיארה כדבר היחיד בעולם שנראה שיש מספיק ממנו לכולם, וכשהיא אמרה את זה, צחקה בקול ושמחה שמחה גדולה.

בגלל שאנו מכבדים את האישיות האנושית אנו מדגישים גם את הצורך במימוש עצמי בקרב מבוגרים. אנו יודעים שניתן להבטיח את זה רק אם אפשר כמות מסוימת של פנאי למנוחה, לבילוי, ללימוד ולמחשבה. האישה מסייעת במעשה הבריאה, ולכן קל לה להכיר באחריות לשמירה על ערכם של היכולות האנושיות. היא חייבת להטיל את עצמה לתוך הזירה שבה נלחמים נגד הרוע עד שמתגברים עליו. היא כבר לא יכולה למצוא הגנה על ידי התעקשות על פרישה בזמן שהגברים שלי נלחמים למען הצדק. שחרור האישה, אם יש לה משמעות, חייבת להביא לכך שהיא נקראת לשתף פעולה עם הגבר בכל תחומי החיים. מכיוון שלפי דעתי, הכשרונות שלה, למרות שהם שווי ערך לאלו של הגברים, הם שונים מהם, מתנותיה הייחודיות נחוצות לקידום אנושי. עליה למלא את אחריותה כאזרחית ובעלת זכות הצבעה, ומכיוון שהיא מאמינה בנוכחות המתמדת של האלוהים, היא אינה יכולה לאפשר לשום פעילות להיראות כאילו שהיא מחוץ לתחום השפעתו. אם יש שחיתות בחיים הפוליטיים או האזרחיים שהיא מכירה, היא אחראית בכל הנוגע להשפעתה האישית על הבטחת הנהלה צלולה. היכן שניתן להבחין בדיכוי או חוסר הוגנות בין מעסיק ומועסק, בין מעמד למעמד, עליה להיות מודעת לדבקות שלה בחופש ולעבוד כדי להסיר את העוול. העריץ, באשר הוא ובכל מקום שהוא, חייב לשלח את עמו כדי שיעבדו את אלוהים, אל עליון. האישה לא מכירה בשולט אחר.

חלק מאחריות האישה היא הדאגה לבריאות הקהילה. גדלנו ויצאנו מהעידן שבו נחשב שקבוצות מסויימות של אנשים נהנו מתנאים לא סניטריים, והושארנו להם ללא מחאה או הקלה. שמענו מי ששאמר שאם היו להם אמטיות, לא היו משתמשים בהם, או שחלק מהאוכלוסיה היה כל כך נבער שמידה מסוימת של תמותת תינוקות הייתה בלתי נמנעת. אף עצה טובה, לפי תפיסה זו, היתה מועילה, גם אם ניתנה במלוא אלטרואיזם. היום אנו יודעים שבורות אינה מוערכת כפריבילגיה מעמדית. כל חלק באוכלוסייה מחפש את הידע שיציל אותם ואת בתיהם ממחלות והידרדרות כל עוד הוא ניתן בצורה מקובלת. עלינו להדגיש את הרעיון שיש להתייחס לגברים ולנשים כאל בני אדם, כאל אנשים בעלי יעדים ושאיפות ולא כמכונות המסוגלות רק להתקדם בצורה מכנית. הבוחרת חייבת להכניס למדיניות האומה הרבה אהבה, הרבה דמיון, כמות גדולה של סימפתיה מבינה.

קראתי לאחרונה את סיפורו של ברמונדזי בחייו של אלפרד סלטר מאת פ. ברקווי.\footnote{אלפרד סלטר (\textenglish{Alfred Salter}), 1873-1945, היה פעיל למען שיפור פני החברה, וחבר פרלמנט עבור ברמונדזי (\textenglish{Bermondsey}), שכונה של פועלי נמלים ותעשיית העור בדרום לונדון.} לא צריך לקבל את האידיאלים הפוליטיים הקיצוניים שלו, כדח שחלק אחד מפעילותו יעורר את הערצתן של נשים מכל גווני הקשת הפוליטית. זוהי אמונתו ביופי כאמצעי לגאולה. לא רק שהאזרחים הצפופים והלא בריאים של ברמונדזי הובלו לכפרים והתאפשרו לספוג את השפעת המקומות היפים, אלא גם הובאו פרחים ושיחים נהדרים לרחובות ולסמטאות של שכונת העוני כדי לעורר את רוח התושבים. סלטר הצליח במטרותיו במידה מפתיעה. מראה השכונה עבר תפנית, והתעניינות התושבים בתרבות ובדת התעוררה. מתוך ההכרה ביופי כהתגלות אלוהית, נראה היה שאפשר לקבל את קירבתו, לשאוף להתאים את החיים מעט יותר למציאות הווייתו. כאזרחים יש לנו הזדמנות להגביר את היופי של חיי היומיום, להתעקש על חקיקה שיכולה לתת הזדמנות לגברים ונשים מן השורה להגדיל את הידע שלהם באמנות ואת ההזדמנויות שלהם ליהנות ממנה. אין ספק, שהתגובה ליופי של אמנות ומוזיקה אינה אוניברסלית או אחידה. אבל התגובה מגיעה לפעמים ממקומות לא צפויים ולכל מגיעה ההזדמנות שלו. אני זוכר שלפני שנים נשאתי הרצאה על הנאות החיים בפני קבוצה של גברים עניים מאוד ותהיתי האם הם משתעממים, והאם רק מתוף נימוס הם נמנעו להביע את השתממותם. לאחר ההרצאה איש אחד קם ואמר: ”אני חובב גדול של צפרות, גם את?“ לא הייתי לגמרי בטוח באותו זמן מה המילה הזאת בדיוק אומרת, אבל הרגשתי ענווה כשהוא התחיל לבטא ההרגשה שלו בשפה ציורית.

עלינו הנשים לא לפחד להיכנס לחיים הציבוריים בכל דלת שנראית לנו מתאימה כל עוד האור משמיים מאיר את עבודתנו, ואנו מעיזות לבקש את ההדרכה של האלוהים בהתמודדות עם הבעיות העומדות בפנינו. אנחנו שומעים מכל הצדדים שרמת המוסר יורדת, כי אי־מוסריות קטנה מאוד גדלה וגם מתקבלת בסובלנות. אנשים מהססים בהתנגדותם להידרדרות מוסרית. גם בעודנו בתפקידי אחריות, אנו הנשים מאפשרות להביא רוע כזה למען הצדקה הקדושה. אנו משחדים אנשים באהבת כל מיני הנאות לתרום כסף למטרות נעלות. אנו מאפשרים לדעות קדומות אישיות קטנות לסכן את התקדמות המאמצים הגדולים בתחום השירות החברתי. הגיע הזמן, אני מעזה לחשוב, שעלינו לזכור שהשירות החברתי שלנו לא יכול להועיל הרבה אם הישגיו כרוכים בהחלשת המצפון החברתי. אנחנו שנכנסנו לחיים הציבוריים מכירים את העידוד לרוע שניתן בארבעת המילים: כולם עושים את זה. אחת התרומות הגדולות שאנו הנשים יכולות לתרום לחיים הציבוריים נמצאת בהתעקשותנו שהעבודה שלנו חסרת ערך אם היא פורחת בזכות רמאות או הימורים או פרסום עצמי. שיטות הארגון שלנו חייבות להיות ראויות להערצה באותה מידה כמו המטרה שלשמה אנו עמלים. אנו לוחצים לטיפול טוב יותר בילדים בבתיהם ובמוסדות. אנו יודעים ומכריזים מכל במה שחלק ניכר מהעבריינות של ילדים, ולמעשה מאומללותם, נובע מהבתים השבורים שאליהם משתייכים רבים מהם.

ליידי סטנסגייט\footnote{כנראה מרגרט בן, (\textenglish{Margaret Benn, Viscountess Stansgate}), 1897--1991, פעילה למען זכויות נשים בכלל ובתוך הכנסיה הנוצרית בפרט.} הדגישה את חשיבות ההשפעה שלנו בחיי הבית. לפי דעתי, חייבים להעביר את ההכרה בחוק המוסר מהמשכן הביתי אל החיים הציבוריים. אסור לנו לפחד להראות את יראת כבודנו לעקרונות היסוד של ניקיון וצניעות, הן על ידי דוגמא אישית הן על ידי ציווי. אם העקרונות שלנו מיושנים, אנחנו מצדיעים להם בזכות זה. הם שרדו 3,000 שנה ואבדתם יביא לאבדת האמת והאהבה.

אולי באמת הגיע הזמן שעלינו להחל את הדת שלנו, אם נוצרית אם יהודית, לחיים שאנו מכירים --- חיי היומיום שלנו. אנחנו בהחלט עושים את זה עד גבול מסויים, אבל בדרך כלל בצורה מטושטשת ולא מוחשית. אנשים מאשימים את הדת בחוסר המוסריות הקיים בעולם. ”מה היא הועילה בכלל?“ הם אומרים. ”הרבה אומללות נגרמת על ידי הדת, אבל מתי היא באמת מונעת רוע? למרות הדת אנשים הולכים בדרכם ומנסים לברוח מלפני ה׳“.

מאשימים את הדת, אבל אולי אנו צריכים להאשים את עצמנו שלא שמנו לב לדת? כיהודיה, אני יודעת בוודאות שלימדו אותנו את היהדות במשך אלפי שנים, אבל לעתים קרובות אנחנו לא קולטים ולכן הסטנדרטים שלנו עדיין נמוכים בחיים האמיתיים. אנחנו מוסגלים להצביר אמיתות גדולות על יושר, לרמות את פקידי המכס, ולחייך לנוכח החכמה שלנו. אולי נשמע את הדיבר: לא תנאף, אך לאחר מכן נשכח אותו, ואולי המעידות שלנו תתקבלנה בסלחנות, שכן מרשים לתאווה לנצח. לפעמים אנחנו לא מפחדים לספר שקרים כי אנחנו חושבים שהצבע שלהם ”לבן“. המזמור שאומר אֱמֶת חָפַצְתָּ בַטֻּחוֹת הוא יפה מאוד, אבל ניתן לראות אותו כאילו שהוא קצת מיושן. האם אנו מוצדקים בהסתייגויות הקטנות הללו? מי הרשה לנו?

אני חושבת שאנחנו חייבות באמת להכינס את אמונתנו לחיים הציבוריים ולהעיד שהיא עדיין רלבנטית. האם נפסיק להגדיר כפורע החוק את החבר בוועדה שמאשר עיוותים, ואת האיש המתבלט מטעם עקרוני כמתחסד? אחרי הכל, אם אנחנו זוכים לעסוק בשירות סוציאלי, זה בגלל שאנחנו רוצים לחלוק את המיטב שאנחנו מכירים עם מי שאנחנו מנסים לשרת. הכישרון המוסרי שלהם, המחוזקת ממאבקים קשים עם המציאות, היא לרוב גדולה משלנו. האם כנות דתית לא חייבת להיות אחת מהכישורים הטובים ביותר שלנו לתועלת? דת פעילה חייבת לבטא את האמונה שלנו.

זו הייתה אמונה מעשית בציווי האלוהי: ”עזוב את עמי ויעבדוני“ שהפכה את החירות בשביל המדוכאים -- (חופש אמיתי, לא מתירנות) -- לאחד הנכסים הרוחניים היקרים ביותר של האנושות, והוא היה פרי מוחו של עם של עבדים בשעת שחרורם, מוחל על החיים ככל שהם מתקדמים לאורך הדורות.

יש היום נטייה בחיים הציבוריים להמעיט את האפשרות של שלום על ידי ציניות וחוסר אמונה. גברת מאי קורוון\footnote{אן מאי קורוון (\textenglish{Dame Ann May Curwen}), 1889--1973, פעילה חברתית, מזכ״לית ה־\textenglish{YWCA} באנגליה.} תעסוק באחריות הבינלאומית שלנו, אבל אני רוצה להציע לכם שכאן בארצנו, עלינו להדגיש את האידיאל של שיתוף פעולה במקום יריבות. ארגוני הנשים הרבים שלה מתחילים עם יעדים גבוהים ומתפלגים בגלל קנאה קטנה ויריבויות. ניתן למנוע אסונות כאלה רק על ידי האומץ והאמונה של מי שמעריכים את השלום כהתגלות של אלוהים.

ברשותכן אסכם את הרעיונות שניסיתי להביע במאמר זה בציטוט מאת רמזי שמובא בספר הציטוטים של הלורד סמואל:

\begin{quote}
אומה לא תצליח להישאר באופן קבוע על רמה יותר גבוהה מזו של נשיה.\footnote{\textenglish{From Viscount Samuel's \textsl{Book of
  Quotations} (London, 1947), quoting from \textsl{A Historical Commentary on St. Paul's Epistle to the Galatians} (London, 1899) by William M.\,Ramsay.}}
\end{quote}

אני מאמינה אפוא שאין תחום של חיים ציבוריים שאין בו צורך בנשים, אך הצלחת כניסתן תלויה בנאמנותן לאמות המידה של מוסר ורוחניות שיש לפרש אותן על פי תורת אמתונת אבותיהן. תורה זו יחד עם הפרשנות שלה היא מתקדמת באופייה, ובסופו של דבר יקרבו את כל המחפשים לאלוהי העולם כולו.

\end{document}
