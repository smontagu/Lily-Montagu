\documentclass[12pt, extrafontsizes, twopage, a5paper]{memoir}

\settypeblocksize{17cm}{11cm}{*}
\setlrmargins{0.8in}{*}{*}
\setulmargins{0.8in}{*}{*}
\setheadfoot{\onelineskip}{\onelineskip}
%\setlength{\topskip}{1.6\topskip}
\checkandfixthelayout
%\sloppybottom
\fixpdflayout

\usepackage[all]{nowidow}

\usepackage{polyglossia}
\setmainlanguage{hebrew}
\setmainfont{Frank Ruehl CLM}
\setotherlanguage{english}
\newfontfamily\englishfont[Scale=0.9]{Brill Roman}

\parskip 1em
\parindent 1em

\setlength{\footmarkwidth}{1em}
\setlength{\footmarksep}{0em}
\footmarkstyle{#1\hfill}
\rightfootnoterule

\begin{document}
{
  \centering\LARGE\bfseries על התפילה\\
  \large מכתב לחבורה מס׳ 5, מרץ 1939

  \Large לילי מונטגיו


}

``הוֹי כׇּל צָמֵא לְכוּ לַמַּיִם.''
קריאה זו, הלקוחה מישעיהו פרק נ״ה, היא קריאה לתפילה. חלקכם שואלים: ``למה להתפלל? איזה תועלת יש בזה?''

אני חושבת שיש לנו צורך להתפלל. איננו שלמים ללא מגע עם אלוהים חיים - וזאת המשמעות של תפילה. אנו יצורים שנוצרו על ידי הרוח החיה של טוב, אמת, אהבה וצדק, שרוצות לחזור ולשאוב מהמקור שלנו אנרגיה מחודשת שבאמצעותה נמשיך את חיינו. אנו מתפללים, אם כן, ל"הגדלת" של כוחנו הרוחני. אנו זקוקים למזון ולפעילות גופנית לנפשותינו, בדיוק כמו שאנו זקוקים לאוכל והתעמלות לגופנו. אנו הוגים את חוק הצדקנות של אלוהים, והרצון להיות טוב יותר ולעשות טוב יותר ממלא את ליבנו. למה להתפלל? \textbf{דבר ראשון אנחנו מתפללים כדי שנוכל לחיות חיים מלאים יותר}.

כולנו מודעים לחטאים כלשהם בחיינו המרחיקה אותנו מן האלוהים. אנו מתפללים לכוח להתגבר על הרוע בתוכנו, ולהיות מואחדים עם אלוהים. כל אחד מאיתנו אחראי באופן ישיר להתנהלות חייו. עלינו להרוס את העוול על ידי המאמצים שלנו. התודעה שאלוהים הוא אמיתי -- שמשהו מרוחו נמצא בליבנו, באותו זמן שהיא, בשלמותה, מהווה כוח החיים העליון ביקום, אמונה זו נותנת לנו את הכוח להתגבר על החטא, כי היא מחזקת את רצוננו ומכוונת אותו לטוב. \textbf{אנו מתפללים, אם כן, דבר שני כדי שנתקדם ביושר}.

בזמן שאנו מתפללים, אנו מרגישים את עצמנו מאוחדים עם כל בן אדם השואף לטוב; חייו ורווחתו הם חלק מהחיים שלנו. אנו מבינים את צרכיו, כי אנו ממש חולקים אותם, ולכן \textbf{הברכה השלישית שאנו מגלים בתפילה היא אחדות האנושות}.

אבל היינו רוצים לשנות את העולם; היינו רוצים לראות כמה מהרעות והסבל, האכזריות והעוולות נמחקות. \textbf{אז אנחנו מתפללים להסרת הרוע}, למרות שהוא ממשיך להתקיים ולפעמים נראה שהוא מוסיף לגדול. אם כן מה נועיל בתפילתנו?

התועלת נמצאת בכיוון אחר. בתפילה אנו מכירים את הערך והכוח של האישיות האנושית. אנו מתפללים, ואפשרות ההישג נחשפת לנגד עינינו. אנחנו יכולים לבחור בטוב ולדחות את הרע. זו זכותנו האנושית. אותה פריבילגיה שייכת לגברים ולנשים שנמצאים בעמדות כוח ואחריות גדולות. לא היינו מוותרים על חירות האישיות האנושית. \textbf{כאשר אנו מתפללים אנו מדגימים את הקשר בין אמונה ומעשה}.

החירות היא חלק מהמורשת האנושית. התפילה חשפה את כוחו של האדם ליצור עולם טוב יותר באמצעות ציות ונאמנות לחוקי אלוהים. \textbf{מוטל עלינו} לעזור בהקמת מלכותו. אסור לנו לבקש מאלוהים לעשות זאת עבורנו, ובזה לוותר על עצמאותנו האנושית. הוא מציע לנו את הכוח לשתף איתו פעולה. קוּמִי אוֹרִי!\footnote{ישעיהו ס א}

אנו אוהבים להתפלל עבור היקרים לנו, ולפעמים אנו מתפללים והאסון שאנו רוצים למנוע מגיע בדיוק כאילו לא התפללנו כלל. מדוע אם כן, אנו שואלים, אלוהים אינו שומע? אני מאמינה שאלוהים \textbf {כן} שומע, וטוב לנו לחשוב על היקרים לנו כאשר אנו בוחנים את מציאותו של אלוהים בתפילה. הבה נחפש מתוך התגלותו דרכים להגביר את חוכמתנו ואת כוח האהבה שלנו. אולי נאבד חלק מהאנוכיות שלנו ומהיכולת שלנו להכאיב. אבל כשדברים לא הולכים כשורה, כפי שאנו מאמינים, עבור אהובינו, עלינו לזכור את גבולות הראייה שלנו ושמה שנראה לנו רע עשוי בסופו של דבר להיות טוב. \textbf{באמצעות התפילה אנו לומדים לבטוח באהבת האל העליונה, ושהוא פועל רק באמצעות אהבה}.

החיים מתוקים לכולכם. למרות הקטעים העצובים והקודרים, יש לכם הכוח ללמוד ולאהוב. אתם יכולים לראות קצת יופי בעולם, אם כי בהבלחות מועטות ונדירות. לפעמים אפשר לראות את פלאי הטבע ולשמוע מוזיקה מפוארת. לרובכם יש מאחוריכם הביטחון של חיי הבית והאמון של אלה שאוהבים אתכם. אתם, בגלל שאתם צעירים, יכול לחוות את הנאות הגוף והנפש. אתם מרגישים אסירי תודה להחיים. \textbf{בתפילה אתם יכולים להודות}.

האוכל להפציר בכם להתפלל כל יום -- להרגיש באופן מודע בנוכחות האלוהים? אם ספקות תוקפים אתכם, התמודדו איתם והיאבקו איתם. בסופו של דבר, אני מאמינה, הם יחזקו את האמונה שלכם. אל תוותרו על התפילה כי היא קשה. למדו ליצור את האווירה הנכונה לתפילה, אווירת יראה וענווה. נקו את לבכם לפני שאתם מתפללים מאנוכיות וחוסר כנות, ואת מוחכם ממחשבות טמאות. אז השליכו את עצמכם אל ``זרועות עולם''\footnote{דברים לג כב} של האלוהים, ובעמקי לבבכם תשמעו את דברו. ``דַּבֵּר י״י כִּי שֹׁמֵעַ עַבְדֶּךָ.''\footnote{שמואל א ג ט} אני מתחננת בפניכם: אל תעכבו את התפילה עד שתהיו חלשים ועייפים מדי, בגלל חוסר תזונה רוחנית, להתפלל בכלל. \textbf{התפללו עכשיו}! מחר עלול להיות מאוחר מדי.
\end{document}
