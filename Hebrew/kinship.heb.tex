\documentclass[12pt, extrafontsizes, twopage, a5paper]{memoir}

\settypeblocksize{17cm}{11cm}{*}
\setlrmargins{0.8in}{*}{*}
\setulmargins{0.8in}{*}{*}
\setheadfoot{\onelineskip}{\onelineskip}
%\setlength{\topskip}{1.6\topskip}
\checkandfixthelayout
%\sloppybottom
\fixpdflayout

\usepackage[all]{nowidow}

\usepackage{polyglossia}
\setmainlanguage{hebrew}
\setmainfont{Frank Ruehl CLM}
\setotherlanguage{english}
\newfontfamily\englishfont[Scale=0.9]{Brill Roman}

\parskip 1em
\parindent 1em

\setlength{\footmarkwidth}{1em}
\setlength{\footmarksep}{0em}
\footmarkstyle{#1\hfill}
\rightfootnoterule

\newcommand{\cut}[1]{}

\begin{document}
{
  \centering\LARGE\bfseries קרבה עם אלהים

  \Large לילי מונטגיו

  \large מתוך דבר תורה שניתנה ב־15 ליוני, 1918\\בבית כנסת הליברלי

  }
  \cut{
רבים מכם זוכרים ספר שיצא לאור בשנת 1907,
תחת הכותרת \textbf{אב ובנו}.\footnote{רומן אוטוביוגרפי מאת אדמונד גוס} זהו תיעוד של מאבק בין שני מזגים,
שני מצפונים וכמעט שתי
תקופות. אחרי שנים של מאמץ נוגע ללב להגיע לפיוס,
האב האשים את הבן בהתחמקות
מהשראתם של כתבי הקודש ועם הכחשה לכל מאמר ומאמר של אלוהים, שהעיק לו. הוא
הרגיש שהבן מפליג על גל הזמן המהיר
לקראת הנצח, בלי אף מדריך מוסמך
חוץ ממה שהוא יודע לחשל על הסדן שלו, למעשה חוץ
ממה שהוא מסוגל \textbf{לנחש}. הבן מרד בתיאור הזה של מצבו הרוחני.
הוא סירב להמשיך לראות את
העולם הזה אך ורק כפרוסדור לא נוח לפני טרקלין שאיש לא חקר.
נאמר לו שהוא חייב להפסיק לחשוב לעצמו; התרסה הייתה
מוצע לאינטליגנציה של נער חושב וישר, עם הדחפים הרגילים של בן עשרים ואחת שנים,
והתוצאה היתה הרס. עצמאות דתית הייתה צריכה להיות
מודגש, ושתי אנשים שאהבו זה את זה בעוצמה
הוכרחו להפרד כלפי חוץ בדיוק כמו, דרך הפעולה של
כוח הזמן, הם נפרדו במשך שנים רבות מבפנים.

אני בספק אם מקרים כאלה של עימות רוחני נמצאים
ניתן להבחין בצורתם הקיצונית ביהודי שלנו
אחווה, אפילו במאה הקודמת. האבות אצלנו
באמצע לא ראו את עצמם אחראים לאלוהים
המחשבות הסודיות וההרשעות האינטימיות ביותר של בניהם.
די בכך שההתאמה החיצונית נשמרה, שלא
הפילוג הוכנס לקהילה כולה. ב
כמה מקרים, אכן, יכול היה להיות צער מייסר רב
נמנע אם הקשר ההגיוני בין דתיים
המחשבה והשמירה על הדת היו מהירים יותר
מוּכָּר. ציות שבחיי האב הוביל אליו
קדושה, נראה היה דבר טוב בפני עצמו, שצריך
להיות מאומץ על ידי הילד בנאמנות ובנאמנות. זה היה
הגילוי של חוסר עקביות מאוד ברור ביניהם
מחשבה וטקסית, שנתנה תנופה לליברל
התנועה במדינה הזו. רוח התקופה נתנה חדש
חשיבות לערך האמת - מבוקשת בזכות עצמה
סאקה. אי אפשר היה לסבול עוד קונפורמיות, אם כן
כרוך בהקרבת האמת. המנהיגים שלנו הדריכו אותנו פנימה
עבודת ההסתגלות. היחס בין העקרונות היהודיים
והחיים המודרניים הוכחו מחדש וכקהילתיים
החיים הפכו אפשריים.

היום הפרספקטיבה השתנתה מעט ואנחנו
לעמוד, כל אחד מאיתנו, מודע יותר מתמיד
לפני ה``נשמה התלויה בעצומה`` משלנו. ביחד
עם הנכונות, שכה נפוצה כיום, להקריב
הפרט למען סיבה, מגיע
כבוד עז לאישיות. אנחנו \textbf{יודעים} היום, כמו שיש לנו
מעולם לא ידעתי קודם, שאלוהים הכניס את הנצח לתוכנו
לבבות; ואנו מוחים בכל הכנות שיש
נגזרת מאינטואיציה, בלתי מוסברת ובלתי מוסברת, כי
החיים אינם זולים, אלא יקרים מאין כמותם.
היום אנחנו הרבה פחות עוסקים באמנציפציה של
האחווה מדיכוי הלגליזם הצחיח, מאשר
עם שחרור הנשמה הפרטית מה
עינוי שלילה צחיחה. כקהילה יש לכבלים שלנו
נפלו ואנו חופשיים להיכנס למורשת הדתית שלנו.
צורת השירות שלנו היא בהרמוניה עם תחושתנו
כושר; אנחנו כבר לא מודאגים מטענות המסורת
כאשר אלה מתנגשים עם תפיסת האמת שלנו. אָנוּ
ביטאו באומץ את אמונתנו בהתגלות מתקדמת,
ואמונה זו האישה את תקוותנו לעתיד
והעצימו את יראת הכבוד שלנו לעבר. הקהילה
של היום הוא בחינם, ואם זה כדי לתפוס את הפירות של
האמנציפציה שלו היא חייבת לאסוף את המחשבות
והשאיפה, האופי והרצון, אפילו הספקות ו
התמיהות של כל חבר בנפרד.

 }  %end cut
  

  נראה לי שאפשר לגשת לאידיאל של חיי קהילה רק כאשר כל חבר וחבר נהיה מודע לחייו הדתיים, או, במילים אחרות, ל"קרבתו עם אלוהים". בתודעה זו תלויה ההכרה מאיפה באנו
  ולאן אנו הולכים, היכולת שלנו להתפלל, הדחף שלנו למסור את עצמנו לתפיסה הגבוהה ביותר של אמת וצדקה. ``קדושים תהיו, כי קדוש אני ה' אלוהיכם.''\footnote{ויקרא יט:ב} מתוך ייסורי האי־שקט שאנו חווים היום, אנו נאחזים בכמיהה עזה ברעיון של אלוה נצחי, שאיתו נוכל להתייחד, \textbf{בגלל} שיש לנו חלק מחייו ומטבעו. אולי העיקו לנו מדי פעם המילים שחוזרות על עצמן בנוסח התפילה בציבור העתיקה שלנו: ``אלוהינו ואלוהי אבותינו''. היום אנחנו מבינים טוב יותר, והם מעיקים פחות. \textbf{רעיון האלוהים שלנו} משתנה מדור לדור; אנחנו מאמינים שהוא מתרחב ומתפתח, אבל חייבים אנחנו היום להרגיש בטוחים בקיומו של אלוהים ובקרבתנו אליו. אנו רוצים להכיר אותו כרוח הבלתי משתנה של אהבה מוחלטת, אמת וצדקה. פניותיהם של אבותינו מהדהדות לנו לאורך הדורות; תחושת הקירבה שלהם עם אלוהים מעודדת אותנו היום. אחדות \textbf{הכמיהה} בין ילדי אלוהים היא אחת העדויות לאבהותו; את הצורך הרוחני שלנו, המשותף, כפי שהוא, לאנושות כולה, ניתן להסביר רק בהתייחסות למוצאנו המשותף. אני מציע שתחושת הקרבה שלנו עם אלוהים, המבוססת בעיקר על אינטואיציה, נתמכת בעדות העבר ובאופי הפולחן העתיק שלנו. אבל התפיסה שלנו נתמכת גם בחוויות הממשיות של חיי היום יום.

  הנביא קורא לנו להיות קדושים, \textbf{בגלל} שהאלוהים הוא קדוש. הקריאה מבוססת על היכולת של בני אדם לחקות את האלוהים. הנשמה הטהורה תחזה אלוה. הקדושה, הטהרה, הן יוצרת את הקירבה. אם כן מהי הקדושה האלוהית? היכן נקודת המפגש בין קדושת האדם לקדושה האל?

  אני יכולה רק לגעת בשולי השאלה העצומה הזו. מקדושת האלוהים משתמע שקיים קנה מידה מוחלט של אהבה, אמת וצדקה. האם לא תושג קדושת האדם על ידי מאמציו להתקרב לקנה מידה זה, מתוך משמעת עצמית לציית, גם במחיר הקרבה של קידום עצמי חומרי ונוחיות? תורת היהדות מציגה לפנינו אל קדוש נעלה ונשגב ללא שיעור מעל האידיאל האנושי; אבל היא גם מצביעה על קירבה לאידיאל הזה. ''קדושים תהיו כי קדוש אני ה' אלוהיכם.`` כוח ההתבודדות הישירה מהווה אחת מתפארות הדת שלנו; אך ללא תחושת הקירבה שפת התפילה, בין בצורה מילולית בין בצורה לא מילולית, תחסר לנו, ומה שחשוב יותר, לא יישאר מקום רחב לשאיפה אנושית. ואכן, חלק ניכר מהשמחה והתקווה בחיי האדם טמון ב"כאב האינסופי של כמיהת לבבות סופיים".\footnote{%
  רוברט בראונינג, \textenglish{\textsl{Two in the Campagna}}.}
הכמיהה הזו מותנית באמונה בקירבה לאלוהים. באחת מדרשותיו מסביר פרופסור ג'ווט שקדושת האלוהים היא היא הרוח שהיא לגמרי מעל העולם ובכל זאת יש לה זיקה לטוב ולאמת שבעולם. פירושו גם נבדלות וגם התרוממות רוח, גם הדר וגם תמימות.

האמונה בקירבה שלנו עם האלוהים מסבירה כמה מההיבטים של חיי האדם שמעוררים את הערצתנו הגדולה ביותר. אכן, היא מסבירה את האופטימיות הבלתי ניתנת לכיבוש שצריכה להיות שייכת לכל מאמין נורמלי. אין לסבול שום רוע כמאפיין את הטבע האנושי - שכן קרבה עם אלוהים מרמזת על השלמות האולטימטיבית של האדם. היא מצדיקה את הדוקטרינה היהודית בדבר חיסולו האפשרי של הרוע על ידי החלפתו בטוב.

האוניברסליות של אהבתו של אלוהים כוללת גם את החוטא וגם את צדיק. אהבת האדם על היבטיה היפים משקפת את האלוהי בהכללתו ובכוחו לסלוח. ``ודאי אמא שלו אהבה אותו'', נאמר על פושע, שלא ניתן היה לומר עליו מילה טוב חוץ מזה. היום ילד קטן שואל כשהוא הולך לישון, ``האם אלוהים שמח איתי היום?'' דור אחר דור רצו שילדיהם יישמרו לעליון. בתכניות החינוכיות שלנו אנו מאשרים מחדש, עם כל ילד שנולד, את אמונתנו בקירבתו אל האלוהים. אמונה זו היא הדוחפת אותנו להיות מוכנין להקריב את הכול כדי להבטיח לילד את תפארת הירושה שלו. שוב, האמונה בקירבתו של האדם אל אלוהים היא המעוררת את זעמנו כאשר אנו רואים חיים תמימים נמחצים ומסוכלים על ידי המאבק המשפיל בתנאים סוציאליים לא צודקים. תנאים כלכליים גרועים עלולים לגזול מהאדם את חירותו, והחירות היא חלק מקדושת האדם, הקרובה לקדושת האל. המחשבה על קרבתנו עם אלוהים צריכה לגרום לנו להתעלם מכל מה שחלף וחולף בהערכתנו את מה שהוא נצחי. תקוותנו היא באלוהים, כאשר אנו משתדלים בהכנעה ללכת איתו, עושים צדק ואוהבים רחמים כאשר אנו קולטים קרן אור אלוהי שלו. שבטו ומשענתו המה ינחמונו, כי אנו חשים חלק מהחיים שלו, גם כי נלך בגיא שאנשים מכנים גיא צַלְמָוֶת.

עד כה התעסקנו בזכות של האדם לקרבה עם אלוהים כמצב אשר, אם נחווה באופן מודע, מביא שמחה ושלווה ותקווה, ומניע אותנו למאמץ הגבוה ביותר. אך עלינו לזכור שאלוהים, על פי תורתם של נביאים ומשוררי תהילים יהודים, ברחמיו האינסופיים מגלה את קירבתו לבניו, בני האדם. לאורך כל המקרא האלהים מתואר גם כמלך וגם כאב. ``כְּרַחֵם אָב עַל בָּנִים רִחַם ה׳ עַל יְרֵאָיו.''\footnote{תהלים קג יג.} הרוך המופלג של אלוהים מתבטא עוד יותר: ``כְּאִישׁ אֲשֶׁר אִמּוֹ תְּנַחֲמֶנּוּ כֵּן אָנֹכִי אֲנַחֶמְכֶם.''\footnote{ישעיהו סג יג.} לדעתו של כותב התהילים, בוודאי, אם לא הייתה קרבה בין הרוח האלוהית לרוח האנושית, לא יכלה להיות אהדה קטנה. האם אז הרחמים יהיו מקובלים? האם אז נחמה תהיה אפשרית? אולי עוד יותר נועזת היא תפיסתו של ישעיהו: ``בְּכׇל צָרָתָם לוֹ צָר וּמַלְאַךְ פָּנָיו הוֹשִׁיעָם בְּאַהֲבָתוֹ וּבְחֶמְלָתוֹ הוּא גְאָלָם וַיְנַטְּלֵם וַיְנַשְּׂאֵם כׇּל יְמֵי עוֹלָם''\footnote{ישעיהו סג ט.}

התפיסה הזו של אלוהים שצר לו ממקם אותו במערכת יחסים קרובה עם חיי אדם, וזוהי תחושת היחסים הקרובה אשר, אני מעיז לחשוב, אנו זקוקים לה היום. אלוהים בחירותו בוחר לסובל עם ילדיו הסובלים. האמונה שלנו ברוח העליונה, השולטת על פי חוק, אינה מתערערת בשום פנים ואופן. דרך האינטואיציה, דרך תחושת חוסר היכולת שלנו, המביאה אותונ להושיט יד לשלמות מעבר לנו, דרך החוויות הממשיות של חיי היום יום, למדנו שאפשר לבסס את תחושת הקרבה שלנו עם רוח הקודש, המובילה אל הצדק. האל שעמו אנו טוענים לקרבה, למרות חולשתנו וחוסר השלמות שלנו, הוא אלוהים חיים, והקרבה שלנו עמו מעידה על אפשרות של חיים שלמים. ניכור קבוע הוא בלתי אפשרי בגלל אותה קרבה: גם בנפש החוטא המושבע ישנו כוח הקדושה.

ההכרה של קרבתנו עם אלוהים מגרשת את תחושת הבדידות, המביאה לעתים קרובות כל כך לידי ייאוש. העולם עשוי להיראות, כפי שהוא נראה היום, מפחיד ואפילו נורא. אנו עשויים לחזות בכל שעה בהכחשת אלוהים ושל הטוב; אבל בתוך האדם עצמו, בעצם קרבתו עם אלוהים, נמצאת ההבטחה לניצחון הסופי של האהבה. אידיאל אהבתו של אלוהים ניתן לחיקוי, אם כי בלתי ניתן להשגה, עלי אדמות; האמונה שאנו קשורים אליה דוחפת אותנו לעמל ולתקווה. האלוהים פועל אך ורק דרך צדק בלבד, ורק על ידי הקמת הטוב נוכל לגרש את הרע. ``קדושים תהיו, כי קדוש אני ה' אלוהיכם.'' זה כאילו האב בעדינות וברכות מציע לנו להשליך כל חוסר ערך ולהיוודע לברית שלנו עמו. הוא מציע לנו לחדש את אמונתנו בכל מה שטוב וטהור, צודק וקדוש, כי הטבע שלנו מסוגל לאמונה זו. הוא מציע לנו אהבה ללא גבולות, כי אנחנו יכולים לשאוב אהבה ממקורות האהבה. הוא מציע לנו להיות דוברי אמת, אפילו בתוך תוכנו, כי שום משמעת לא קשה מדי עבורנו שחיינו קשורים לאלוהי, הוא נותן לנו את החירות להשיג - אפילו בזמן שהוא מציב בפנינו את אידיאל הצדק. ``קדושים תהיו כי קדוש אני ה' אלוהיכם''.

בצדק נאמר: ``אלוהים אינו שוכן בעולם אחד ואדם בעולם אחר, \textbf{הבריאה}, שהוא חלק ממנה, היא התגלמות חיי האלוהים. טבעו של אלוהים אינו נבדל באופן קיצוני מטבעו של האדם. חיי אלוהים וחיי אדם אינם סותרים זה את זה; אם חייו של האדם הם חלק מחיי הבריאה וחיי הבריאה הם חיי האל בהתגלמותו, חייבת להיות לכל הפחות האפשרות של יחס מודע בין אלהים לאדם. העובדה המרכזית הגדולה בחיי האדם, בחייך ובחיי, היא ההתהוות של הכרה מודעות וחיה של האחדות שלנו עם החיים האינסופיים, וההתפתחות המלאה שלנו  לשפע האלוהי. במידה שאנו מתפתחים לשפע האלוהי אנו משתנים מבני אדם בלבד לבני אלוהים.''\footnote{מתוך ראלף וולדו טריין, \textsl{\textenglish{In Tune with the Infinite}}.}

הייתי מתחננת בפניכם היום שהתודעה העצמית המפוארת הזו, המצביעה בוודאי על תחושת קרבה עם אלוהים, נגזרת מהחוויה הממשית של התפילה, ומהשאיפה לצדק; שרק הוא מסביר את מגמת העלייה האיטית ולעתים קרובות כואבת, אך בכל זאת מתמדת, של חיי אדם. ככל שאנו מתוודעים ליחסינו עם אלוהים, אנו פחות ופחות נמשכים לקשרים מרושעים. האין זה פירוש הציווי: ``אֹהֲבֵי י״י שִׂנְאוּ רָע''?\footnote{תהלים צז י.} המודעות העצמית שתיארנו מדרבנת את האדם למאמץ הכי גבוה לצדק והאמונה החזקה ביותר בכוחו של אותו מאמץ, \textbf{במידה} שאפשר לקשר אותו עם האלוהי. אבל אנו מזכירים לעצמנו שתחושת הכבוד האנושית תלויה באמונתנו באלוהים כאבא האוניברסלי. מכיוון שיש לנו את הכוח להגיע לקדושה, אנו קרובים ל־``רָם וְנִשָּׂא שֹׁכֵן עַד וְקָדוֹשׁ שְׁמוֹ``\footnote{ישעיהו לז טו.} אבל הכוח הזה קשור בנו לכוח האדם לחטוא. הקרבה שלנו עם אלוהים היא בלתי נמנעת, אבל אנחנו חולקים אותה עם החוטא השחור ביותר ביננו. איננו יכולים להכחיש את תהילת האנושות של החוטא הזה מבלי להתכחש לזו שלנו, שכן האל המתנה את התהילה הוא האל האוניברסלי, השוכן ב\textbf{כל} נפש אנושית כדי לגאול ולהצילה לטובה. נקודת המפגש בין האדם לאלוהים היא הקדושה – ככל שמתגברת קדושת האדם כך מתהדקים היחסים בין האדם לאלוהים. האדם מאוחד לאדם בחולשתו וכן בחוזקו, בכשלונו וכן במצוינותו המוסרית. אוי לו אם יכחיש את שתי צורות הקרבה. \textbf{תודעת} הקרבה שלנו עם אלוהים אינה מצב פסיבי. הוא בא עם המאמץ שלנו אחרי צדקה. נותר לנו רק לנסח את האידיאל שלנו כדי לממש את חסרונותינו המצערים. האמונה בקירבה שלנו עם האלוהים - באפשרויות האנושיות שלנו - ממלאת את ליבנו בתחושה של חוסר ערך מוחלט, בו בזמן שהיא מעוררת אותנו לתקווה מלאה יותר.

אנו מתפללים ליום בו ``יִהְיֶה י״י אֶחָד וּשְׁמוֹ אֶחָד.''\footnote{זכריה יד ט.} היום הזה יתקרב באופן ניכר כאשר תוכר אחדות חיי האדם באמצעות ההכרה של אחדות אלוהית בנפש כל אחד ואחד. הקריאה ברורה לנו היום. הבה נשמע אותה ונציית לה; הבה נהיה מודעים לקרבתנו עם אלוהים דרך יראת הכבוד שלנו לאמת המוחלטת, דרך חיבורנו בענווה עמו בעבודת הגאולה והישועה שלו, באמצעות כוחנו לאהוב ללא קץ, דרך אמונתנו בהמשכיות חיי האדם, אשר מקושר עם האלוהי לצורך יצירת צדקה.
\end{document}
