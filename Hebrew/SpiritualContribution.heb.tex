\documentclass[12pt, extrafontsizes, twopage, a5paper]{memoir}

\settypeblocksize{17cm}{11cm}{*}
\setlrmargins{0.8in}{*}{*}
\setulmargins{0.8in}{*}{*}
\setheadfoot{\onelineskip}{\onelineskip}
%\setlength{\topskip}{1.6\topskip}
\checkandfixthelayout
%\sloppybottom
\fixpdflayout

\usepackage[all]{nowidow}

\usepackage{polyglossia}
\setmainlanguage{hebrew}
\setmainfont{Frank Ruehl CLM}
\setotherlanguage{english}
\newfontfamily\englishfont[Scale=0.9]{Brill Roman}
  
\setlength{\footmarkwidth}{1em}
\setlength{\footmarksep}{0em}
\footmarkstyle{#1\hfill}
\rightfootnoterule

\parskip 1em
\parindent 1em

\begin{document}
{
  \centering\LARGE\bfseries התרומה הרוחנית של נשים באשר הן נשים
  
  \large שיקגו: מרכז חינוך יהודה\\יום שישי 26 נובמבר, 1948

  \Large לילי מונטגיו

  }
אני מנסה לדון איתכם היום בנושא
שהוא כרוך בקושי מיוחד. קודם כול,
סביר מאוד להניח שאתן כפמיניסטיות לא תקבלו
את ההנחות הראשונות שלי. הרעיון המרכזי שלי מבוסס על התפיסה
שלמרות שעל הנשים לשתף פעולה עם גברים בכל
הפעילויות הפתוחות בפני בני אדם, ולמרות שהן לא צריכות
להכיר באף מגבלה חוץ ממגבלות פיזיות, ולמרות שמגדר
לא יכול לפסול אף אחת מלעשות את מה שהיא
מרגישה את עצמה מתאימה לו, בכל זאת לנשים יש כישורים מסוימים
שהם שונים משל הגברים. אמנם,
גם אני פמיניסטית ומאמינה בשוויון מוחלט בין
גברים ונשים בתחום החברתי, הפוליטי, הכלכלי והדתי,
אבל אני חושבת שהאנושות מתעשרת
מהגיוון בין שני המינים. מכאן נובע שנשים חייבות
לפתח את הכישורים המיוחדים שלהן, ושהן חייבות לתרום
לאוצר הרוחני של העולם תרומה שלימה
ככל האפשר, אבל אותה תרומה חייבת להיות בעלת אופי משלה,
ולא להיות חיקוי או העתק התרומתם
של הגברים.

לגברים ונשים יש יכולת יצירתית משותפת, אבל
גברים מסוגלים לפעול בצורה אובייקטיבית יותר מנשים. הרי
דוגמא אחת פשוטה. אבא ו אימא רוצים שהילד הקטן שלהם
בן 4 או הילדה הקטנה בת 4 ילמדו להתפלל. הם רוצים
לעורר את היכולת לתפילה. אבא לוקח את הילד
לבית הכנסת בזמן שאין שם תפילה, נושא
אותו מסביב ומראה לו את התכונות השונות, כולל המושב האישי שלו,
וארון הקודש, ומסביר שחמישה חומשי תורה
כלולים בספר הנמצא בארון הקודש.
הוא מעביר מידע ומעניין את הילד. השבת
מגיעה והילד או הילדה מתלבשים במיטב בגדיהם,
מלווים את אבא לבית הכנסת, ומחזיקים לו את הסידור.
הילד יושב בינו לבין  אימא.
מדי פעם אַבָּא מראה לו את המקום בסידור, והוא מפסיק
להתנועע לרגע. הוא שמח ומוכן לחזור
על הניסוי בשבתות הבאות. לאט לאט, לאחר
זמן רב, האווירה של בית הכנסת עושה רושם
על הילד והוא מרגיש את השאיפה להתפלל.
מימוש האפשרות הזו יהיה תלוי במידה מרובה
על תגובתו של האבא עצמו לתפילה
ובמה הוא אומר עליה כשהוא חוזר הביתה.

לאימא יש שיטה אחרת. כשהיא משכיבה את הילדים היא מוסגלת לעשות
הרבה. האלמנט של הודיה הוא
הכנה טובה לתפילה. היא זוכרת פטריה יפה
שהיא וג'וני התפלאו עליה בזמן שטיילו.
כוח הדמיון שלה חזק. ''ג׳וני`` היא
אומרת, ''אתה זוכר את טיול שלנו היום ואת הפטריה ההיא
שראינו, עם פסי הצבע המקסימים, האדום והצהוב,
והחתיכות הקטנות של ירוק וחום, ואיך שאמרנו
הלוואי שיכולנו למצוא פטריה שאפשר לאכול, אבל אז אמרנו
שהיא לא תהיה יפה באותה מידה? האם נודה לאל
על זה שהוא ברא הפטרייה ההיא?“ ''כן, נודה,  אימא`` ''תודה לך
אלוהים שבראת את הפטריה המקסימה ההיא, עם האדום והצהוב,
והחתיכות של חום וירוק``  הוסיף ג'וני, ''ובפעם
הבאה בבקשה תהפוך אותה לפטריה שנוכל לאכול.`` ''אמן``
אומרת  אימא. בלילה אחר  אימא וג'וני
מכינים רשימה של האנשים ששניהם אוהבים ומבקשים מאלוהים
לברך אותם. הם מחברים יחד את התפילות ואלה הן
התפילות האישיות שלהם.

אישה יוצרת מעניקה את חוויה רוחנית שלה של
כאב ושמחה לתפיסות המתקדמות של היהדות. 
גבר מנתח ומסנן ושוקל בזמן שהוא
עולה על הר האלוהים ביחד איתה. שמעתי שאומרים שגבר
מטפס צעד אחר צעד עד שהוא מגיע לרמה שהוא מסוגל
להגיע. דרכו הייתה בטוחה אך איטית למדי. הוא מסתכל
מסביב ורואה אישה לצידו. הוא לא ראה אותה
בזמן שהוא טיפס כי היא זינקה ממדף אל
מדף ולקחה סיכונים רבים.

בדיונים דתיים על מקור הסמכות
ביהדות מתקדמת, תמצא שלגברים יותר אכפת
מסמכות חיצונית מלנשים. גבר אומר: ”יהיה לנו
תוהו ובוהו אלא אם כן בעלי הידע והניסיון מתכנסים
יחד ומחליטים על טכסים ספציפיים. האם תפילה ביום ראשון
מובילה לחוסר נאמנות? האם כדאי שתהיה יותר עברית
בליטורגיה שלנו? האם סיוע החזן חיוני או
אפילו רצוי?“ ”טוב, אני לא רואה שזה משנה מה
האנשים הגדולים חושבים“ אומרת האישה. ”אני יודעת שאני לא יכולה להביא
את הקטנים שלי לבית כנסת בכל יום חוץ מיום ראשון. עִברִית
אולי מתאימה לאנשים מסוימים, אבל היא לא מועילה
לאלה שלא מבינים אותה. אם אני רוצה את השירה הכי משובחת,
אני הולכת לאופרה. כשאני בבית כנסת, אני רוצה
לשיר ולהצטרף, ואני יודע שהילדים שלי גם רוצים. אם הם
יכולים לשיר, עדרבא, ואם הם לא יכולים, שאחרים
ישירו חזק יותר ויכסו את קולות ילדיי, אבל גם הם
שרים.“ יש אלמנט של פרקטיות בכל
זה, אולי קצת פחות תחושת אחריות, שאיפה
לנחת רוח של הצעירים שלה.

עלינו הנשים לגשת לשאלת השלום הבינלאומי,
לפי דעתי, מזווית די שונה מזה
שמאומצת על ידי גברים. ראשית, אני חושבת שאנחנו צריכות להעלות
את הבעיה מתחום הפוליטיקה אל זירת
הדת, או, כפי שהייתי מאוד מעדיפה, להפעיל
השפעה דתית חזקה על נושא הפוליטיקה. גברים
ונשים כמובן יודעות באותה מידה את האומללות ואת
חוסר התוחלת של מלחמה, ואת היואש שבא בעקבותיה.
אבל נשים בוודאי מבינות בצורה מלאה יותר את ההשפעה
של מלחמה על חיי הבית. נראה לי שעליהם מוטלת האחריות
להתגבר על תחושת התבוסתנות ותסכול
שמציפה את העולם. גברים מסתכלים בצורה
מציאותית יותר מנשים. החזון שלהם
חסום על ידי קיומה של פצצת האטום ואיום
הטוטליטריות. הם נתנו כל כך הרבה למען
החירות, והם רואים את כל צורות העריצות פורחות
ומאיימות יותר ויותר. הם עייפים אבל
נחושים לא להפגין חולשה. לכן הם פונים להכנות למלחמה
כדרך היחידה להבטיח שלום. כאן נמצא
חלקה של האישה. עלינו להצהיר בכל הכוח
העומד לרשותנו שבגלל שהאלוהים קיים, ממשלת השלום
והצדקה חייבת לנצח בעולם. אם אנחנו מאמינות בצורה חד־משמעית
ברשעות המלחמה, עלינו להתרחק ממנה
ולמצוא דרכים אחרות ליישב את המחלוקות בינינו,
עם כל הקושי שבחיפוש. אם ניכשל עכשיו, הציוויליזציה שלנו
תכלה ועמו כל היקר בחיי הבית.
רק במישור הדת נוכל למצוא
הדרך להחזיר את האמונה שלנו באדם ובעצמנו. אנחנו
יכולות לעזור כי אנחנו מחוץ לזירת הלחימה הממשית.
למרות שאנחנו ביחידת המלחמה, השיטות שלנו לא
מוגבלות לכוח פיזי.

לפני סיום, אבקש מכן לשקול אם
אתן חושבות שאתן יכולות להתנגד לסחף הקיים המתרחק
מהקדשת חיי הבית. גברים עשויים להתאמץ אפילו יותר
מנשים לשמור על מראה חיצוני, אבל אתן יודעות טוב כמוני
שריקבון משתלט, והגיע הרגע
בשבילכן לקום ולהאיר. על הבמה,
ברומנים ובעיתונים, רעיון הבית במפולג
מתקבל כבלתי נמנע. הילדים מוקרבים
לתחושת הכללית של הבלתי נמנע.
חיי בית שמחים וצנועים טבעיים נחשבים
יוצאי דופן ובלתי צפויים. נדר הנישואין התרפה.
טקס הנישואין דתי רק בצורתו,
ומשמעותו נשכחה. גברים חושבים
ששלב זה הוא בלתי נמנע, שהקוד החדש של המוסר או
חוסר מוסר עלול להתגבר, הוא לפי המגמה של הזמי.
העסקים תופסים תשומת הלב. הישרדותם של החזקים
הוא צו השעה, ואי אפשר לקחת בחשבון אף שיקול אחר
בעוד שמאבק הקיום כה עז. שוב כאן
עלינו להבליט את אמונתנו, ולפני שנוכל לעשות זאת, עלינו להאריך אותה,
כל אחת לעצמה. אם הבית והילדים
הם רכושנו היקר ביותר, אם שבח
הערך שלהם הוא יותר מסתם התבטאות, עלינו
למהר לעבודת הישועה.

כן, חברים, בתחום החינוך הדתי, בבית,
בעבודה למען שלום וצניעות, עלינו לעבוד עם
קנאות חדשה ואמונה חדשה. התרומת נחוצה. את האומץ שֶׁלָנוּ
יש להוסיף לזה של הגברים שלנו. עלינו להחזיק
להם חלק מהביטחון והתקווה האבודים שלהם. נתנו
להם לאבד הרבה על ידי האדישות והחוסר פעילות שלנו. עכשיו עלינו
לדעת שאנחנו עומדות לפני האלוהים וחייבות או למות
או להיות מוכנות לשמור את דברו. כל אחת חייבת לומר
בעצמה: ``הִנְנִי שְׁלָחֵנִי.''\footnote{ישעיהו ו ח}

\end{document}
